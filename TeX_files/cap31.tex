\chapterimage{chapter_head_2.pdf}

\chapter{Racionais MC's}

Lançado em 1987, trata-se do álbum de rap do principal grupo do gênero no Brasil, surgido no final da década de 80. É o álbum mais representativo do rap no Brasil, com uma potência e qualidade estética de alto valor.

A Unicamp, nos últimos anos, abordara um caráter mais representativo em suas listas de leitura. Outrora, a lista de obras era compartilhada com a Fuvest, o vestibular de ingresso para a Universidade de São Paulo. Posteriormente, no entanto, a Unicamp passa a abordar de maneira mais presente a questão dos negros no Brasil, adotando, a princípio, a obra \textit{A negrinha}, de Monteiro Lobato, assim como os poemas de Jorge Lima. A lista, no entanto, seria alterada, dada a polêmica envolvendo os autores e, por conseguinte, sua relevância para o debate. Assim, temos em uma nova lista as obras \textit{Quarto de despejo} e \textit{Sobrevivendo no inferno}. Por fim, é interessante notar a influência exercida pelos vestibulares do sul, que com certa antecedência inseriram a canção \textit{Construção}, de Chico Buarque, por exemplo.

Contexto do fim da ditadura e promulgação da Constituição Cidadã de 1988, cujos objetivos envolviam a extensão do conceito de cidadania à totalidade da população (uma das conquistas envolve o SUS, o maior sistema de saúde pública do mundo, assim como o ensino universitário público e gratuito), assim como da própria questão da democracia, como o direito dos jovens maiores de 18 anos ao voto. Ainda nesse período, tivemos o fim da ditadura militar, e a sequente instauração de uma nova ordem democrática para a qual é função do Estado fornecer aos seus cidadãos seus direitos básicos.

O Brasil, desde o período da ditadura, vivia a chamada hiperinflação, um processo de desvalorização muito grande da moeda, representada simbolicamente pelo equipamento utilizado para atualizar os preços dos produtos (as pequenas máquinas de correção de preços e seu som característico). É interessante notar, ainda nesse contexto, a criação dos chamados \textit{soldados de Sarney}, e a posterior proibição do aumento nos preços dos produtos, uma tentativa de acabar com a inflação que ocasionara desabastecimento dos produtos. Collor, por sua vez, confiscara o acesso ao patrimônio retido na poupança, em uma tentativa de reduzir o consumo dos produtos.

Período tampão, com o impeachment de Collor e a subida de Itamar Franco. Sucessão do Plano Real, organizado por Fernando Henrique Cardoso, assim como a privatização de bancos estaduais (lembre-se das polêmicas envolvendo a privatização da Vale do Rio Doce), medidas essas que garantiram o fim da hiperinflação e asseguraram a eleição de FHC nas duas eleições que se seguiram.

Tais medidas, no entanto, não solucionaram os problemas sociais do Brasil de forma definitiva. Na prática, um dos direitos mais básicos do mundo moderno, como o saneamento básico, chegam a apenas pouco mais de metade da população brasileira.

Com o crescimento das cidades ocorrido no período — agora não mais limitado aos grandes centros urbanos, há também um processo de favelização observado, em especial, na década de 90 (data, inclusive, próxima ao surgimento do bairro do Pinheirinho, em São José dos Campos). Em paralelo, temos a política de guerra às drogas, iniciada pelos Estados Unidos — que não são grandes produtores de drogas, então provenientes de outras localidades espalhadas pelo globo. Portanto, era necessário combater, sobretudo, o tráfico internacional de drogas, principalmente  em países subdesenvolvidos, periféricos (também as populações das regiões da periferia foram os mais afetados). Cabe, nesse contexto, mencionar dois temas anteriormente estudados: o documentário \textit{A 13ª emenda}, assim como os aparelhos repressivos do Estado, conceito desenvolvido por Althusser.

Na década de 60 e 70, os bixeiros (criminosos, contraventores), eram os principais patrocinadores do samba no Brasil. O jogo do bicho, de maneira geral, possuía uma série de crimes associados. Na década de 90, o Brasil se torna o meio de difusão de drogas para grande parte do globo, e os crimes no país assumem novos contornos relacionados ao tráfico de drogas, que gradativamente se equipa de armas para garantir sua sobrevivência.

Nesse período, as principais facções criminosas eram o \textbf{Comando vermelho} e a \textbf{Facção satânica} (lembre-se que Leonel Brizola chegara a realizar um acordo com facções criminosas, que gerara uma redução na atuação do Estado nas comunidades mais periféricas).

Nessa sociedade demasiadamente desigual surge o rap. Nos Estados Unidos há a apropriação do hip hop, ao passo que no Rio de Janeiro, do funk.

Algumas influências culturais merecem maior atenção. Em primeira análise, é interessante notar a presença da tradição jamaicana da discotecagem (mais barata), então levada para os Estados Unidos por meio da imigração nos bairros negros de Nova Iorque (lembre-se das produções de Lauryn Hill, em especial, \textit{Doo Wop}). Já existia a tradição das festas de quarteirão — referidas como \textit{block parties}. Nesse contexto, surge a figura do MC, responsável por entreter as pessoas durante a troca de DJ's com improvisações. Posteriormente, há o surgimento de aparelhos de edição de som — como o mixer e o sampler —, que possibilitavam a repetição dos trechos das músicas, assim como a gravação de trechos de músicas e inserção em outras.

A cultura hip hop se desenvolve a partir dos pilares dos MC's, DJ's, do break e do grafite. As letras abordavam a realidade dos locais (crítica social e crônicas da vida urbana na periferia), em um contexto de grandes tensões raciais nos Estados Unidos, representadas, sobretudo, pela violência policial. Em vertentes como o \textit{gangsta} ou \textit{proibidão}, ainda há apologia ao uso de drogas, confrontos policiais, etc.

A rima atua como o eixo da composição, apresentada por meio de letras de forte cunho social. 

No Brasil, o rap surge na estação São Bento, no final dos anos 1980, na capital paulista — linha muito movimentada e sujeira às pressões policiais. Os indivíduos passaram a se reunir na praça Roosevelt. Publicação, em 1988, do primeiro volume de \textit{Consciência Black}, que contara com a participação dos Racionais MC's.

Relembre a diferença entre ep, single e álbum.

Em 1980, tivemos o lançamento de \textit{Holocausto urbano} (ep). Em 1992, temos \textit{Escolha o seu caminho}, também um ep. Por fim, em 1993, temos \textit{Fim de semana no parque}, um single.

A capa do álbum faz referência ao salmo preferido do grupo. Dois elementos importantes envolvem a religiosidade e questão da violência, representada por meio do buraco de bala no centro da cruz. De maneira geral, as cores e a fonte utilizada também possuem referências bíblicas e, em especial, para elemento religioso do inferno.

A primeira faixa — e única que não é de autoria do grupo — é nomeada \textit{Jorge da Capadócia}. Refere-se a um soldado romano de mesmo nome que se convertera ao cristianismo. No candomblé, é o orixá da guerra e do ferro. ''Sentar praça`` é um termo utilizado em referência à ação de assumir um cargo no exército — de assumir um cargo no exército romano.

Trata-se de uma oração de proteção (forma de garantir que nenhum inimigo o fará mal). No candomblé, é referida como \textit{corpo fechado}. A imagem de São Jorge, potencialmente apenas uma lenda e, em certo grau, distanciado pela Igreja Católica, é de grande importância para o candomblé.

\textit{Gênesis} é o primeiro livro do Velho Testamento da Bíblia, e também a primeira faixa autoral do álbum. É evidente a decisão de não adequar a linguagem para a norma culta, como visto nos cordéis (emulação da norma culta).

Construção da ideia de que tudo de bom foi feito por Deus. O homem, por sua vez, forneceu tudo de ruim (visão misógina, que transita entre as figuras de Maria e Eva). Contexto da vida periférica, associada às injustiças sociais, que não foram criadas por Deus. A Bíblia velha (fé) e a revolta — a violência como algo do meio, e a violência simbólica representada pelo rap.

Note que não há apologia ao uso de drogas e tampouco ao crime. Talvez, em transformações mais revolucionárias na sociedade.

Sobreviver no inferno como a realidade periférica urbana, e não no sentido teológico do termo — o inferno é a sociedade desigual.

O rap como violência revolucionária, porém simbólica. Organização da obra semelhante a um culto religioso (e a religião que institui um senso de coletividade).

Em \textit{Capítulo 4, versículo 3}, temos referências a um livro da Bíblia — estruturação de um imaginário religioso —, com uma fala inicial de estatísticas que evidencia a marginalização da população negra. O jovem negro que passa dos vinte anos já se encontra na sobrevida (expectativa de vida semelhante à região da Faixa de Gaza).

Constante onomatopeias em referência às armas, e abordagem do rap também como arma — o verso violentamente pacífico, o rap como um instrumento de transformação social que ameaça os privilégios estabelecidos. PT é uma arma automática. Ideia de que, enquanto a população negra estiver sendo ostensivamente reprimida pelas forças do Estado, sempre há a possibilidade de revolta dessa camada.

O jovem negro não oferece mais perigo — contra a lógica do senso-comum. A droga como forma de alienação, um instrumento de adequação dos indivíduos ao sistema, evitando revoltas (conformação social com o sistema, o que distancia o processo de consciência de classe).

Em \textit{Tô ouvindo alguém me chamar}, é narrada a situação de um jovem que, caído no chão e ferido por uma bala, começa a refletir sobre sua vida (relembre, no gênero épico, da utilização da chamada \textit{In medias res}). Em comparação com um culto evangélico, identifica-se notável semelhança com o momento do testemunho. Veja que, ao longo de todo o álbum, o crime é sempre tratado como algo negativo, seja por meio do arrependimento do criminoso ou das consequências que suas ações causaram.

Além disso, o crime também é, frequentemente, discutido do ponto de vista da ordem, e não da perspectiva do próprio criminoso, em 1ª pessoa, como ocorre em parte das canções. Note também, em contexto histórico, a abordagem do crime como um problema de segurança pública, e não problemática social.

Em \textit{Diário de um detento}, temos novamente a continuidade dos ciclos de violência (note que o clipe se inicia com crianças brincando com dominós em preto e branco, e finaliza com crianças brincando com armas). Escrito por Josemir Prado, relata o período em que permaneceu no sistema carcerário. Mano Brown realizara uma adaptação do diário, acrescentando o Massacre do Carandiru\footnote{Em 2 de outubro de 1992, ocorrera uma briga entre os detentos da Casa de Detenção de São Paulo (Carandiru), repreendida pela polícia com extrema violência (corpos com sinais de execução).}, então a grande metáfora para o Brasil.

Em que medida ainda existe uma ditadura militar para as camadas mais vulneráveis da população e, em especial, para a população negra?

O filósofo francês Georges Bataille identifica que a realidade humana é ocultada pelo discurso oficial, em ângulos abordados pela mídia e, em análise geral, pelo Estado. Relembre a política de guerra às drogas dos Estados Unidos, aplicada com o fim do \textit{apartheid}, como uma forma de manutenção das posições de poder e controle social.

O \textit{modess}, descartável como o sangue, e o bombril, em referência aos cabelos dos negros. Recursos poéticos não como forma de embelezar o discurso — o que, potencialmente, poderia naturalizar certas condições da sociedade. O objeto da arte não como o belo da realidade, mas como o indivíduo a observa, em nosso contexto, na criminalidade (metáforas com armas).

Em \textit{Salve}, há a figura de Jesus como indivíduo negro. Identifica-se a exaltação não do indivíduo pacificador ou salvador, mas que caminhava com os excluídos — papel revolucionário do cristianismo em suas origens. O cristianismo, em sua gênese, como oposição às leis estabelecidas e ao próprio judaísmo (religião de indivíduos perseguidos, mas que se transformara ao longo do tempo). Ideia de retorno a esse cristianismo, exaltando a ideia de igualdade. Qual seria, pois, essa religião? Sincrética, com a influência de temas de matriz africana (como o próprio Jorge da Capadócia), católica e protestante (certos salmos).

Há uma concepção religiosa sincrética, com religiões de matriz africana, catolicismo popular (e suas práticas populares e não-canônicas), evangelismo (com ênfase no Velho Testamento e nos profetas). O profeta é o indivíduo que prega o fim da realidade para o surgimento de uma nova realidade mais justa.

O profetismo e a destruição do mundo corrompido pelo capitalismo, para o advento do reino dos justos (sociedade igualitária). Nesse sentido, relembre a obra de Bart D. Ehrman.

Os livros sapienciais da Bíblia (Provérbios, Eclesiastes, Livro da Sabedoria), e as orientações para a vida prática e espiritual (como em ''periferia, periferia``, em conhecimento obtido com a experiência, e não no meio acadêmico). De maneira geral, a Igreja Evangélica fornece maior ênfase ao Velho Testamento, em oposição à Igreja Católica.

A religião é importante para as comunidades periféricas como o ópio do povo (visão marxista da religião). A religião também como obstáculo para a consciência de classe. A Igreja Católica, com a teologia da libertação, em meio à ausência da figura do Estado (relembre o padre Júlio Lancellotti). O para Bento XVI fora o principal contrário à teologia da libertação, em especial em meio à Guerra Fria. O espaço então não preenchido pela Igreja e pelo Estado é então preenchido pelos neopentecostais — e pelas igrejas evangélicas, de maneira geral.

Em \textit{Mágico de Oz}, obra da literatura infantil, é retratada a condição de crianças usuárias de drogas, substitutas para o lúdico e para o processo da infância como um todo (em meio à necessidade de acelerado amadurecimento). A droga é tratada como um caminho para a alienação — na obra de Marx, comparável à abordagem da religião como o ópio do povo.

Em \textit{Qual mentira vou acreditar}, a figura feminina se desenvolve na dicotomia mãe/puta. A mulher que não assume a figura da mãe preocupada com a chegada do filho em casa, por exemplo, é sempre apresentada como egoísta, oportunista, e que apenas deseja adquirir dinheiro (aspecto misógino). \textit{Medieval Misogyny and the Invention of Western Romantic Love}, de Howard Bloch, e a figura da mulher como o principal perigo para o homem e para a sociedade de maneira geral (ou na figura de Maria ou na figura de Eva).

O samba e a idealização da vida no morro (lembre-se de \textit{Opinião}, do compositor Zé Kéti). A perseguição inicial do samba, sempre relacionado à vadiagem e ao candomblé, ambos proibidos. Posterior associação com o samba exaltação e o surgimento de compositores brancos, em especial ao longo do governo de Getúlio Vargas.

O rap e a subversão do imaginário popular sobre a vida no morro. Diferentemente do samba, nunca houve um processo direto de criminalização do rap. No entanto, sobretudo na imprensa, desenvolvera-se um imaginário do gênero associado aos criminosos.

Em \textit{Periferia é periferia}, o morador adquire uma arma para violentar o vizinho que roubara seus pertences dispostos no quintal. Apropriação cultural e a ideia implícita de certa essência cultural (as culturas como elementos imutáveis). A cultura é maleável e baseada em trocas (lembre-se de Bush e o conflito de culturas). Visão essencialista.

Lembre-se de Foucault e a genealogia dos micropoderes.
O malandro e a figura do homem livre pobre (ojeriza ao trabalho, utilização de vestes brancas, em processo de distanciamento com os escravos). Quando conveniente, joga de acordo com as regras da ordem ou da desordem.

No contexto do rap, há o rompimento com a dialética da malandragem e o surgimento dos chamados marginalizados. Lembre-se de \textit{Memórias póstumas de Brás Cubas}, de Machado de Assis, e a dialética da malandragem presente na própria elite, como observado na figura de Brás Cubas e sua moral de conveniência.

Ruptura da dialética da malandragem para a marginalidade (as personagens do álbum, via de regra, não são malandros e tampouco indivíduos de bem).

Lembre de \textit{Teoria estética}, de Theodor Adorno, e a ideia do ''teor de verdade``, nesse caso, presente no discurso dos indivíduos da periferia que idealizam a sua própria condição (tentativa de pacificação das tensões sociais). O lugar de fala não garante uma visão verdadeira ou menos ideológica de sua condição.

Polifonia (multiplicidade de perspectivas).

A presença de três cantores no álbum, por vezes, fornece distintas perspectivas acerca de um mesmo objeto, assegurando uma análise mais complexa da matéria representada (como presente na faixa \textit{Capítulo 4, versículo 3}).

\section{Jorge da Capadócia}

Ainda nada por aqui.
			
\section{Gênesis}

Ainda nada por aqui.
			
\section{Capítulo 4, versículo 3}

Ainda nada por aqui.

\section{Tô ouvindo alguém me chamar}

Ainda nada por aqui.
			
\section{Rapaz comum}

Ainda nada por aqui.
			
\section{...}

Ainda nada por aqui.
			
\section{Diário de um detento}

Ainda nada por aqui.
			
\section{Periferia é periferia}

Ainda nada por aqui.
			
\section{Qual mentira vou acreditar}

Ainda nada por aqui.
			
\section{Mágico de Oz}

Ainda nada por aqui.
			
\section{Fórmula mágica de paz}

Ainda nada por aqui.
			
\section{Salve}

Ainda nada por aqui.