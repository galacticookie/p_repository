\chapterimage{chapter_head_2.pdf}

\chapter{Quincas Borba}

\textit{Quincas Borba}, publicado em 1886 (Joaquim Borba dos Santos), situa-se entre os anos de 1867 e 1871, e narra a história de Pedro Rubião de Alvarenga, um professor do ensimo primário que começara a cuidar de Quincas Borba (de \textit{Memórias póstumas de Brás Cubas}) na cidade de Barbacena. A personagem, mesmo adoecida, transmite os conhecimentos acerca do humanitismo para Rubião.

Uma alegoria central dessa ideia envolve um campo de batatas capaz de servir a apenas uma de duas tribos (note as semelhanças com o darwinismo social de Spencer). A teoria da evolução proposta por Darwin criara um novo modelo de cientificidade (modelo teórico, coerente a um grande número de evidências), e servira de inspiração para outras áreas do conhecimento como a História e a Sociologia em uma tentativa de instituir certa cientificidade. Spencer tentou, por exemplo, explicar a desigualdade social.

As batatas simbolizam os recursos e as tribos, sociedades. ``Ao vencedor as batatas'', assim, é uma grande paródia ao darwinismo social.

Rubião esperava entrar no testamento de Quincas Borba. Em certo dia, ao acordar disposto pela manhã, Borba resolve retornar para o Rio de Janeiro. Rubião, posteriormente, recebe uma carta de Brás Cubas anunciando a morte do amigo. Com isso, o cuidador recebe toda a fortuna, com a prerrogativa de cuidar do cachorro de mesmo nome. Com a quantia, o homem retorna ao Rio de Janeiro como que em busca de suas batatas, e , em última análise, em uma forma de demonstrar a riqueza que adquirira. Rubião se torna um herdeiro do humanitismo de Quincas Borba.

A viagem de trem é interrompida em Vassouras, estação na qual entram Cristiano Palha — ex-fazendeiro que almejava se tornar um banqueiro — e Sofia, sua esposa.. De imediato, Rubião se apaixona pela mulher e, em paralelo, Palha — interesseiro que deliberadamente utilizava a esposa para atrair vítimas — se interessa pela fortuna do homem.

Ao desembarcarem no Rio de Janeiro, os recém-conhecidos trocam endereços (note como ambas as partes possuíam interesses). Outro aspecto interessante está presente no nome do casal: Sofia e a referência à Filosofia, assim como Cristiano Palha e a expressão ``dar palha'', i.e., enganar os outros com palavras difíceis.

Palha, ao tentar atrair Rubião para seus negócios, torna-se o administrador da riqueza do rapaz.

Uma importante personagem inserida nesse momento é Carlos Maria, representação da figura do \textit{dândi} (comparável a um \textit{playboy} da atualidade), sempre presente nas festas e eventos que ocorriam na região. Ocorre que, em momento anterior, durante uma festa, Rubião declarara o seu amor por Sofia, fato que não agradou a jovem e fez com que se distanciassem (a mulher chegou a pedir que o marido também se distanciasse de Rubião, pedido esse negado). Nesse momento, Carlos começa a se interessar pela mulher.

Uma característica importante da obra, e que a difera de outras narrativas de Machado de Assis, é o fato de que \textit{Quincas Borba} possui um narrador onisciente, não-personagem e em 3ª pessoa. Dessa forma, as opiniões são sempre revelas pelos olhos das personagens — mesmo um cachorro em relação ao mordomo da casa. Por essa razão, aos olhos de Sofia — que se mostrara apaixonada —, Carlos Maria havia se declarado naquele momento. No entanto, com o passar dos dias, o rapaz desaparece e, ao retornar repentinamente, começa a tratá-la friamente. Ocorre que Carlos, que tampouco compreendia a própria atitude de se ``declarar'' para Sofia, não desejava seguir adiante com esse relacionamento, e apenas enganara a mulher (de maneira geral, pode-se inferir que o homem possuía uma personalidade narcisista). Com efeito, pouco tempo depois o dândi se casara com uma sobrinha de Sofia.

Em meio a todos esses acontecimentos, Rubião, ainda que se sentisse traído, tenta se aproximar de Sofia. Em dado momento, no entanto, encontra uma carta de Sofia endereçada para Carlos Maria e, sem sequer abri-la, acusa a mulher logo em sua festa de aniversário. Posteriormente, Sofia abre a carta e revela para Rubião que se tratava, na verdade, de um pedido de doação para sua instituição filantrópica. Esse é um momento especialmente importante para se notar o fato de que o narrador da história, nesse momento, aos olhos de Rubião, nem sempre fornecerá informações confiáveis ao leitor (a suposta confirmação da relação entre Sofia Carlos Maria, por exemplo, ocorrera em uma viagem de Rubião em um coche e uma grande distorção dos fatos na mente desse).

A partir de então, Rubião e Sofia começam a desenvolver uma relação razoavelmente curiosa, à medida em que a mulher, que sente pena do rapaz, aos poucos se apaixona. Rubião, contudo, começa a demonstrar sinais de loucura — após se apaixonar por Sofia — em vários momentos da narrativa, sendo o mais célebre o episódio do coche: durante uma viagem com Sofia, tem um momento de delírio no qual acha que é Luís Bonaparte, o então imperador da França, ao lado de Leopoldina. O homem acaba indo à falência (processo, inclusive, que desesperara Cristiano Palha), e se torna um pobre e louco.

Outras duas personagens notáveis da obra são o casal Teófilo (quase Ministro da Fazenda) e Dona Fernanda (tornara-se amiga de Sofia), o \textbf{casal virtuoso}. A família de Sofia tentara cuidar de Rubião, que fora internado em um hospício. O homem, no entanto, tomado pela loucura, foge para Barbacena, onde morre durante uma noite chuvosa (o cachorro, Quincas Borba, acompanhara o dono até a morte, e acabara por morrer também).

É interessante notar que Rubião perdera todas as suas batatas no Rio de Janeiro (a meritocracia não funcionara com o personagem). Também é importante ressaltar que a narrativa induz o leitor a acreditar no caso de Sofia e Carlos Maria, percepção tratada pelo narrador como calúnia por parte do próprio leitor e de Rubião (o narrador induz o leitor ao erro, que se torna culpa do leitor). \textbf{Narrador não confiável}.

\textit{Dom Casmurro}, a obra seguinte, possui uma estrutura semelhante, mas já não há o narrador onisciente presente em \textit{Quincas Borba}.

Em \textit{Quincas Borba}, há um narrador onisciente, não-personagem e em 3ª pessoa, que , utilizando-se de sua onisciência, narra a história da perspectiva das personagens. Dessa forma, ao narrar o suposto caso entre Sofia e Carlos Maria do ponto de vista de Rubião — cego de ciúmes e enlouquecendo —, o narrador induz o leitor ao erro (narrador não confiável). As críticas são identificadas por meio das contradições presentes no discurso do narrador.

O nome de Rubião, Pedro Rubião de Alvarenga, também possui elementos importantes para o entendimento da obra. \textit{Rubiácia} é um sinônimo para o café, presente na bandeira do Império do Brasil. É interessante notar, também, que Dom Pedro II assinava como Pedro de Alcântara — Pedro de Alvarenga, nesse sentido, seria uma referência ao imperador. Lembre-se também que, ao enlouquecer, Rubião pensa ser Luís Bonaparte. Pode-se inferir que Rubião é a representação do império, em crise. Para melhor entender esse detalhe, cabem alguns apontamentos.

Em 1867 houve a primeira fala (pública) do trono abertamente contra a escravidão. O evento, que ocorria em toda abertura do ano legislativo, fora mencionado na conversa com Cristiano Palha durante a viagem (o ex-fazendeiro havia vendido todas as suas propriedades para tentar a vida na sociedade, e era contrário à abolição; Rubião, por outro lado, apoia a posição do trono, mas pretendia vender, e não alforriar, os seus escravos).

Em 1871 foi promulgada a Lei do Ventre Livre, também referida como Lei dos Ingênuos, em especial, aos que acreditavam na completa extinção da escravidão. Anos antes, em 1850, a Lei Eusébio de Queirós apenas incentivara a compra de escravos em outras regiões.

Na sociedade brasileira da época, identificava-se personalidades gradualistas, composta, majoritariamente, pela elite econômica e escravocratas, e que por muito tempo dominou o debate, assim como imediatistas. Com a urbanização, no entanto, surgem camadas mais progressistas na sociedade. Em análise de John Gledson, em \textit{História e ficção}, Machado — descrito como um escritor conciso e um monarquista não apaixonado — identificava o colapso do império com a abolição.