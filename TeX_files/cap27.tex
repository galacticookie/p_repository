\chapterimage{chapter_head_2.pdf}

\chapter{Simbolismo no Brasil}

Em Portugal, tivemos a obra de [pepsida].

Ao contrário do que ocorreu na França, em que o Simbolismo tomou conta do cenário poético, no Brasil foi o Parnasianismo quem imperara. O Simbolismo surge no Brasil de maneira mais periférica — questão geográfica.

\section{Cruz e Souza}

Uma das principais figuras do Simbolismo no Brasil. Nasceu na cidade de Desterro, atual Florianópolis; morreu pobre (existe uma lenda de que o seu cadáver foi transportando por uma charrete que outrora transportava fezes de animais).

Filho de escravos libertos, seu padrinho militar lhe garantiu uma formação intelectual sólida. As portas do mercado editorial, no entanto, não se abriram completamente para sua figura, em partes porque era negro, e também porque era um poeta simbolista.

Ficou conhecido como o ``Cisne negro'' ou ``Dante negro'' da poesia brasileira. A imagem do cisne é muito utilizada no Simbolismo para representar a condição do poeta, indivíduo excepcional cercado por um universo de mediocridade. Também é uma ave associada à beleza (canto do cisne antes de morrer).

Dantesco como algo infernal (lembre-se de que, na \textit{Divina comédia}, a porção da obra que se tornou mais popular foi o inferno). Também uma questão prosaica, já que a maior parte dos leitores sequer terminaram a leitura do livro, mas iniciaram a leitura pelo inferno.

Para Acauam Oliveira, ``o racismo cria a raça''. Em uma sociedade excludente como a brasileira, a questão da cor da pele importa menos do que outros aspectos como a condição socioeconômica do indivíduo. Machado de Assis, por exemplo, era capaz de mimetizar a condição de vida dos indivíduos brancos e ricos — o próprio Brás Cubas representava a elite branca rica da sociedade brasileira, e era o próprio alvo de críticas. Relembre a compreensão de Joaquim Nabuco sobre Machado de Assis. Algo curioso é que o escritor sofria de epilepsia desde a infância.

Para Cruz e Souza, no entanto, a questão da escravidão era muito mais recente em sua vida, o que repercutira muito em sua poesia e em sua crítica literária. O autor leva o processo às últimas consequências (frequente emprego de procedimentos musicais).

\textit{Broquéis}, de Cruz e Souza, é considerada a primeira obra simbolista no Brasil. \textit{O emparedado} trata da posição do negro na sociedade brasileira. Questão ancestral que o empareda, mas não por um sentimento de inferioridade, mas em reconhecimento a todas as restrições que a sociedade impõe sobre ele que a cor da pele proporciona. Frequente análise da questão do indivíduo oprimido, outrora pelo mercado em uma sociedade capitalista, em sua obra assume contornos materiais de uma sociedade escravocrata. Sentimento difuso de opressão.

As críticas literárias são discutíveis. Roger Bastide e a fundação da Universidade de São Paulo em meio à ausência de instituições que servissem de modelo para o conhecimento universal. Os primeiros professores da USP eram, em sua maioria, franceses — como Lévi-Strauss.

Bastide identifica a abordagem do branco como um anseio do escritor em se autoafirmar como branco, leitura essa considerada ultrapassada e, em certa medida, preconceituosa. No Simbolismo, de fato, o branco remete à ideia da pureza do poeta em meio às trevas da mediocridade. No próprio Simbolismo francês também era empregada (o branco associado ao bom e o preto associado ao mal) em processo quase que instintivo (relembre a abordagem histórica das civilizações sobre o sol). Denegrir deriva do latim \textit{denegrine}, utilizado por romanos que escravizavam indivíduos não pela cor da pele, mas por outros motivos. Criação de etimologias como ``criado-mudo'' e ``feito nas coxas''.

Identificamos também a presença de uma linguagem litúrgica, que remete aos rituais, em especial, da Igreja Católica, e esotérica (uma das características do Simbolismo é a tentativa de resgate do cristianismo e mesmo uma aproximação com o ocultismo).

Alphonsus de Guimaraens também buscara a presença de uma religiosidade ligada ao catolicismo (temática cristã católica), além da representação da melancolia (\textit{spleen}).

\poemtitle{Antífona}
\begin{verse}
 Ó Formas alvas, brancas, Formas claras \\
 De luares, de neves, de neblinas!... \\
 Ó Formas vagas, fluidas, cristalinas... \\
 Incensos dos turíbulos\footnote{Incensório.} das aras\footnote{Altar (lembre-se do canto de Inês de Castro, de \textit{Os lusíadas}).}...

 Formas do Amor, constelarmente puras, \\
 De Virgens e de Santas vaporosas... \\
 Brilhos errantes, mádidas frescuras \\
 E dolências de lírios e de rosas...
 
 Indefiníveis músicas supremas, \\
 Harmonias da Cor e do Perfume\footnote{Utilização da sinestesia.}... \\
 Horas do Ocaso\footnote{Momento do pôr do sol.}, trêmulas, extremas, \\
 Réquiem do Sol\footnote{A luz dolorida da morte do sol, em tom melancólico e crepuscular.} que a Dor da Luz resume...
 
 Visões, salmos e cânticos serenos, \\
 Surdinas\footnote{Instrumento utilizado para diminuir o volume de um som, no caso, de um órgão fraco.} de órgãos flébeis, soluçantes... \\
 Dormências de volúpicos\footnote{A forma que instiga o desejo (no caso, venenos que causam sensualidade).} venenos \\
 Sutis e suaves\footnote{Repetição do ``su'' em sutis e suaves. Veneno que causa sensação prazerosa de desfalecimento.}, mórbidos, radiantes...
 
 Infinitos espíritos dispersos, \\
 Inefáveis\footnote{Que não pode ser definido}, edênicos\footnote{Derivado de Éden, o paraíso.}, aéreos, \\
 Fecundai o Mistério destes versos\footnote{O eu lírico deseja que os diversos elementos dos sonhos fecundem os seus versos.} \\
 Com a chama ideal de todos os mistérios.
 
 Do Sonho as mais azuis diafaneidades \\
 Que fuljam, que na Estrofe se levantem \\
 E as emoções, sodas as castidades \\
 Da alma do Verso, pelos versos cantem.
 
 Que o pólen de ouro dos mais finos astros \\
 Fecunde e inflame a rime clara e ardente...\footnote{O eu lírico deseja que a rima do poema seja tão rica e clara quanto as estrelas.} \\
 Que brilhe a correção dos alabastros\footnote{Pedra branca utilizada pelos gregos. Mais frágil e barata do que o mármore.} \\
 Sonoramente, luminosamente.
 
 Forças originais, essência, graça \\
 De carnes de mulher, delicadezas...\footnote{Começa a abordar a carnalidade da mulher (as perversões sexuais e os desejos presentes no Simbolismo).} \\
 Todo esse eflúvio\footnote{Deriva de fluvial.} que por ondas passe \\
 Do Éter\footnote{Como substância etérea, que preenche todos os espaços.} nas róseas e áureas correntezas...
 
 Cristais diluídos de clarões alacres\footnote{Muito luminoso.}, \\
 Desejos, vibrações, ânsias, alentos, \\
 Fulvas\footnote{Em referência à cor vermelha.} vitórias, triunfamentos acres\footnote{Como agridoce. Vitória que proporciona muito mais tristeza do que satisfação.}, \\
 Os mais estranhos estremecimentos...
 
 Flores negras do tédio e flores vagas\footnote{Flores em referência à \textit{Flores do mal}. O tédio é apresentado na seção \textit{O spleen e o ideal}.} \\
 De amores vãos, tantálicos\footnote{Tântalo é uma figura da mitologia grega condenada a uma fome e dese eternas.}, doentios\footnote{Em referência às perversões.}...\footnote{Vazio, efêmero, vácuo.} \\
 Fundas vermelhidões de velhas chagas \\
 Em sangue, abertas, escorrendo em rios.....\footnote{Aspecto mais mórbido, macabro, também presente no Simbolismo.}
 
 Tudo! vivo e nervoso e quente e forte, \\
 Nos turbilhões quiméricos\footnote{Na literatura, de maneira geral, a quimera aparece como metáfora para algo ilusório.} do Sonho, \\
 Passe, cantando, ante o perfil medonho \\
 E o tropel cabalístico\footnote{Em linguagem esotérica.} da Morte... 
 
 Para as Estrelas de cristais gelados \\
 As ânsias e os desejos vão subindo, \\
 Galgando azuis e siderais noivados \\
 De nuvens brancas a amplidão vestindo...

 Num cortejo de cânticos alados \\
 Os arcanjos, as cítaras ferindo, \\
 Passam, das vestes nos troféus prateados, \\
 As asas de ouro finamente abrindo...

 Dos etéreos turíbulos de neve \\
 Claro incenso aromal, límpido e leve, \\
 Ondas nevoentas de Visões levanta...

 E as ânsias e os desejos infinitos \\
 Vão com os arcanjos formulando ritos \\
 Da Eternidade que nos Astros canta... 
\end{verse}

\textit{Antífona} é o poema que abre a obra \textit{Broquéis}; atua como uma profissão de fé, síntese das características do Simbolismo no Brasil. Antífona é o que os fiéis respondem à fala do padre.

Notamos a evocação de entidades que fecundam os versos com a chama ideal, assim como o emprego da letra maiúscula em ``Formas'' em uma questão platônica de universalizá-las (finalidade de emprestar dimensão universal aos termos).

Os dois primeiros versos fazem referência à brancura, pureza e espiritualidade. Como visto em ``Incensos dos turíbulos das aras'', o Simbolismo não busca definir o objeto, mas sugeri-lo (diáfano). Notamos também a sublimação da figura feminina, em que a mulher é apresentada de maneira espiritualizada. Em análise geral, há também uma visão do amor como algo espiritualizado.

Identificamos referências bíblicas.

Trata-se de uma grande celebração da morte, também um aspecto da espiritualidade: o espírito começa quando a carne acaba.

\href{http://www.dominiopublico.gov.br/download/texto/bv000073.pdf}{Domínio público}.

\section{Alphonsus de Guimaraens}

Algum texto aqui.

\poemtitle{A catedral}
\begin{verse}
Entre brumas, ao longe, surge a aurora. \\
O hialino\footnote{Cristalino.} orvalho aos poucos se evapora, \\
Agoniza o arrebol\footnote{A aurora, o nascer do sol.}. \\
A catedral ebúrnea\footnote{Branco, da cor do marfim.} do meu sonho \\
Aparece, na paz do céu risonho, \\
Toda branca de sol.

E o sino canta em lúgubres\footnote{Sinistro, relativo à morte; mortiço, macabro.} responsos: \\
"Pobre Alphonsus! Pobre Alphonsus!"

O astro glorioso segue a eterna estrada\footnote{Percorre o céu.}. \\
Uma áurea seta lhe cintila\footnote{Brilhar.} em cada \\
Refulgente\footnote{Brilhante.} raio de luz. \\
A catedral ebúrnea do meu sonho, \\
Onde os meus olhos tão cansados ponho, \\
Recebe a bênção de Jesus.

E o sino clama em lúgubres responsos: \\
"Pobre Alphonsus! Pobre Alphonsus!"

Por entre lírios e lilases\footnote{Em referência às violetas.} desce \\
A tarde esquiva: amargurada prece \\
Põe-se a lua a rezar. \\
A catedral ebúrnea do meu sonho \\
Aparece, na paz do céu tristonho, \\
Toda branca de luar.

E o sino chora em lúgubres responsos: \\
"Pobre Alphonsus! Pobre Alphonsus!"

O céu é todo trevas: o vento uiva. \\
Do relâmpago a cabeleira ruiva \\
Vem açoitar o rosto meu. \\
E a catedral ebúrnea do meu sonho \\
Afunda-se no caos do céu medonho \\
Como um astro que já morreu.

E o sino geme em lúgubres responsos: \\
"Pobre Alphonsus! Pobre Alphonsus!"
\end{verse}

Alphonsus morava na rua ``direita'', localidade presente nas cidades brasileiras que permanecia à direita da igreja do município. No poema, a primeira estrofe faz referência à manhã, em seguinte, em referência ao dia e, por fim, em referência ao anoitecer (tarde). O aspecto da catedral e do céu se transformam com o passar do dia, à exceção do ``Pobre Alphonsus'', que sempre se sente miserável.

Sobre o estado interior do poeta que permanece inalterado com o passar do dia, é válido mencionar que Alfonso se suicidou; era um alcoólatra e, proibido pela família de ingerir álcool, ingeria solvente de tinta.