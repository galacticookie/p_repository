\chapterimage{chapter_head_2.pdf}

\chapter{Dom Casmurro}

\textit{Dom Casmurro}, de 1900, acompanha a história de Bento Santiago — de família rica. Escritos sobre sua vida, em detrimento de um texto sobre os subúrbios do Rio de Janeiro. Era vizinho de Capitu, de origem humilde, sua família conquistara uma fortuna com um bilhete premiado na loteria.

Bentinho estava destinado a ser padre. José Dias (agregado) e Dona Glória (mãe) e a conversa sobre enviá-lo ao seminário. Paixão entre Capitu e Bentinho, que acaba indo ao seminário. No local, cultiva uma amizade com Escobar, que também não desejava ser padre. José Dias fica ao lado de Bentinho — na esperança de se tornar herdeiro —, e sugere a adoção de uma criança e seu envio ao seminário.

Acontece o casamento — nos moldes dos clássicos folhetins românticos — entre Escobar e Sancha, a melhor amiga de Capitu.

Filho de Capitu e Bentinho, Ezequiel Santiago. Em um jantar organizado por Capitu, Bentinho se interessa por Sancha. Posteriormente, Escobar morre afogado. No velório, ao perceber a reação de Capitu, Bentinho assume que foi traído, e começa a ver no filho a imagem de Escobar.

A peça de Shakespeare, \textit{Otelo}, e os pensamentos suicidas de Bentinho. O homem acaba por desistir e chega a oferecer o veneno ao filho.

Bentinho não lidara bem com o sentimento de traição, e enviara Capitu e o filho para a Suíça — sem prestar qualquer cuidado com a sobrevivência de ambos. A mulher morre e o filho, aos 18 anos, acaba por morrer também.

Os leitores de \textit{Dom Casmurro}, imersos nas convenções e lugares-comuns do Realismo, identificaram em Capitu a figura de uma \textbf{mulher fatal} — \textit{Madame Bovary}, de Gustave Flaubert, constitui exemplo notável. Para Helen Caldwell, autora de \textit{O Otelo brasileiro de Machado de Assis}, os ciúmes não são bons conselheiros. Assim, de maneira semelhante à obra de Shakespeare, Iago estaria na própria figura de Bentinho, um narrador não confiável.

O crítico literário Silviano Santiago, por sua vez, defende uma construção retórica de acusação de Capitu. Com efeito, Bentinho havia se formado em Direito.

Por fim, cabe mencionar que Dom Casmurro é o apelido recebido por Bentinho, que dormira durante a declamação de um poema em uma estação ferroviária.

Existem diferentes ondas de leitura para esse acontecimento de \textit{Dom Casmurro}. Como abordado anteriormente, uma das primeiras interpretações envolvia Capitu na figura de uma mulher fatal, nos moldes do Realismo. Para outros autores, o caso seria um enigma insolúvel, em especial na ideia de Bento Santiago como um narrador não confiável — influência de Helen Caldwell e a relação estabelecida com a personagem Otelo. Para Silviano Santiago, Bentinho está deliberadamente mentindo ao leitor como forma de convencê-lo da traição. Outra leitura, trazida por Roberto Scharcz, dá maior ênfase para a classe social de Bentinho, e a consequente visão de Capitu, pertencente à classe média-baixa. Nesse sentido, a narrativa seria um exemplo de estilo (por muitos anos) como ferramente de sedução do leitor; haveria, então, um narrador honesto, fluido, em oposição com \textit{Memórias póstumas de Brás Cubas}.

Em relação à ideia de Bentinho como um narrador não confiável, é importante analisar em maiores detalhes a forma como Machado de Assis construíra o narrador. Em uma das passagens iniciais, Bentinho nota a semelhança entre a mãe de Sancha e Capitu (um ``retrato''); o rapaz, ao se lembrar do momento, repete as mesmas palavras de Gurgel.

Além disso, é interessante notar que Ezequiel, enquanto crescia, adquirira o costume de imitar as pessoas que conviviam com ele. Em outra passagem importante, Bentinho comenta o fato de que esquecera do rosto de Escobar (relembre, nesse sentido, que Bentinho projetara em Sancha os seus desejos).

Sobre a traição, a construção da narrativa também é relevante. Em certo momento, Capitu tentara trocar o seu dinheiro por libras esterlinas e, para tanto, pedira a ajuda de Escobar. Posteriormente, Escobar passa na casa de Bentinho para entregar a Capitu a quantia, momento identificado como um claro sinal por parte de Bento Santiago. Ocorre que esses acontecimentos foram narrados em ordem invertida, o que intensifica ainda mais a interpretação de Bentinho. Estaria o homem assim tão cego de ciúmes?

O próprio sentimento de ciúmes é uma emoção crescente em Bentinho, desde o período em que permaneceu no seminário. O rapaz também se mostrara relutante nos momentos de contradança dos casais nos bailes, e por isso para de frequentá-los. Também proibira Capitu de esperá-lo na varanda da casa. Soma-se a isso possíveis traições de Bentinho.

Uma consequência relevante para a traição de Capitu, e que desfavorece que o caso realmente tenha ocorrido, envolve a filha de Sancha e Escobar. A garota, tratada como uma possível pretendente para Ezequiel, seria, nesse caso, irmã do menino. As personagens, no entanto, sequer teriam fibra para tamanha maldade, mediocridade; planejamento convicto.

Por fim, uma última leitura curiosa envolve uma possível paixão de Bentinho por Escobar. Quanto estavam no seminário, é memorável o momento no qual Bentinho e Escobar se abraçam — posteriormente, inclusive, comenta-se que já existia certa suposição rondando a localidade. ``Como se fosse invenção minha''; Capitu, em certa medida, também desconfiava.

Nesse sentido, não confessar seus sentimentos seria como castrar a verdade. Existe, ao menos, certa tensão sexual por parte de Bentinho — que então colocara em Capitu os seus desejos sobre Escobar. Assim desejaria também a Sancha, esposa de Escobar.

Note também a sexualidade presente no primeiro beijo, e a masturbação estilística (relembre a primeira de bentinho ainda no seminário, e a abordagem como uma prática religiosa).

Para referência futura: \url{http://antenasdemarfim.blogspot.com}.