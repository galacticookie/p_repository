\chapterimage{chapter_head_2.pdf}

\chapter{Parnasianismo no Brasil}

Em primeira análise, é essencial mencionar o processo de modernização da sociedade brasileira, em paralelo com a profissionalização da literatura. Com isso, houve um distanciamento da massa analfabeta e miscigenada, e o conhecimento era tratado como um instrumento de ascensão social (capital cultural como ferramenta de distinção, ostentação de privilégios).

A Batalha do Parnaso, de 1878, apenas evidenciara o desgaste do Romantismo (quase quatro séculos depois do ocorrido na França). Foi uma polêmica na imprensa da época que reuniu jovens poetas e escritores contra o Romantismo. No ano seguinte, Machado de Assis publicara \textit{A nova geração}, no qual apresenta essa nova geração de poetas.

É curioso notar que, com exceção da França, o Brasil foi o único país que desenvolveu uma escola parnasiana — com certas distinções da francesa.

A poesia realista; social/socialista; científica (lembre-se da transição de Casimiro de Abreu/Alberto de Oliveira para o Parnasianismo).

Olavo Bilac, por sua vez, insere-se não em um cenário de transição do Romantismo e do Parnasianismo como tendência, mas sim como um modelo já consolidado.

Em 1888, publica \textit{Poesias}. Consolidação do Parnasianismo no Brasil.

Absenteísmo: abstenção do debate público. No Brasil, no entanto, há certa conciliação com a tradição lírica existente, e um distanciamento da ``arte pela arte''. Pierre Bordieu e o capital econômico, social e cultural (o conhecimento valorizado pela elite que detém o capital). Continuidade da tradição literária brasileira, e distanciamento da impassibilidade do Parnasianismo francês. Presença de aspectos do Romantismo. A ideia da ``arte pela arte'', em paralelo à arte burguesa, não fora valorizada. Honoré de Balzac e os folhetins românticos, e as possibilidades em enriquecer com a literatura (processo de profissionalização). Como citado anteriormente, no Brasil muitos escritores exerciam funcionarismo público. Em relação aos leitores, lembre-se de Pierre Bordeau e os campos sociais (literatura de arte e literatura de venda; considere o caso de Paulo Coelho e Sidney Sheldon, e os \textit{ghost writers}).

\textit{A elite do atraso}, de Jessé Souza.

O Parnasianismo, no Brasil, contrariando o princípio da ``arte pela arte'', esteve comprometido com o projeto civilizatório da sociedade brasileira (adequação do modelo da modernidade burguesia). Nesse sentido, em \textit{Crítica à razão dualista}, Francisco de Oliveira comenta o baixo custo de produção da mão de obra brasileira e a modernização conservadora da sociedade.
'
Na poesia, Machado de Assis preferira as obras francesas e, em especial, parnasianas; tradução de \textit{O corvo}.

\section{Alberto de Oliveira}

Exibia um rebuscamento excessivo em seus textos, apresentando uso frequente do hipérbato (inversão sintática) e da sínquise (diversos hipérbatos em um mesmo período sintático). Costumava representar a paisagem brasileira, e utilizava ostensivamente as convenções parnasianas — como a écfrase.

\poemtitle{Vaso chinês}
\begin{verse}
Estranho mimo aquele vaso! Vi-o. \\
Casualmente, uma vez, de um perfumado \\
Contador\footnote{Móvel.} sobre o mármor luzidio\footnote{Com ``Vi-o'', é uma rima rara.}, \\
Entre um leque e o começo de um bordado\footnote{Luxo.}.\footnote{A primeira estrofe trata da descrição da localização do objeto.}

Fino artista chinês, enamorado, \\
Nele pusera o coração doentio \\
Em rubras\footnote{Paixão (confimar) ardente do artista.} flores de um sutil lavrado, \\
Na tinta ardente, de um calor sombrio.\footnote{Descrição que busca recriar a impressão.}

Mas, talvez por contraste à desventura, \\
Quem o sabe?... de um velho mandarim\footnote{Classe aristocrática dos escribas e sábios.} \\
Também lá estava a singular figura;

Que arte em pintá-la! a gente acaso vendo-a, \\
Sentia um não sei quê com aquele chim \\
De olhos cortados à feição de amêndoa\footnote{Com ``vendo-a'', é uma rima rara. O termo é utilizado para se referir aos chineses. Cabe mencionar a associação realizada entre o povo mongol, a síndrome de down e a expressão pejorativa ``mongoloide''.}.
\end{verse}

O soneto se utiliza da écfrase para descrever um vaso, como indicado pelo título.

\section{Raimundo Correia}

Escrevera, sobretudo, poesias meditativas (reflexiva, filosófica, existencialista). Carregava uma visão pessimista da existência, refletida em um caráter mais melancólico de seus textos.

\section{Francisca Júlia}

Apresentara a ideia da impassibilidade em seus textos (o não envolvimento do eu lírico com a temática), e seguira quase como uma ortodoxia parnasiana (uso das regras do Parnasianismo francês ao pé da letra). Apresenta elementos simbolistas em suas obras.

Note como a crítica literária brasileira é sujeita às visões do Modernismo. Pré-Modernismo e a ideia de Friedrich Hegel de um fim definido para a história — nesse caso, o Modernismo.

Francisca Júlia morrera no enterro do marido; levara as regras do Parnasianismo francês muito a sério, sobretudo em relação aos tons emocionais, com exceção de uma tintura simbolista observada em seus poemas religiosos.

A impassibilidade (\textit{passum}, de paixão) como reação ao sentimentalismo romântico (distanciamento da perspectiva poética em relação ao tema representado). Não confunda, no entanto, com um desinteresse em emocionar o leitor; apenas entende-se que, para tanto, não é necessário o envolvimento com o poeta.

\poemtitle{Musa impassível}
\begin{verse}
Musa! Um gesto sequer de dor ou de sincero \\
Luto jamais te afeie o cândido\footnote{No sentido de calmo (a expressão da musa não muda mesmo em situação de medo ou luta; não demonstra piedade em relação ao sentimento alheio).} semblante! \\
Diante de um Jó\footnote{Personagem bíblico do Velho Testamento. Perdera todos os seus bens e filhos, mas ainda assim mantivera a sua fé em Deus. Posteriormente, adquirira uma doença de pele e fora acusado de ter cometido algo que justificasse sua punição. Ao passar pela provação, Deus dobra tudo o que Jó possuía antes dos desastres. Representa o sofrimento injustificado, na ideia de que, mesmo diante do sofrimento mais absurdo, a musa não deve se sensibilizar.}, conserva o mesmo orgulho; e diante \\
De um morto, o mesmo olhar e sobrecenho austero.

Em teus olhos não quero a lágrima; não quero \\
Em tua boca o suave e idílico\footnote{Vida amorosa feliz em contato com a natureza — algo que o eu lírico não espera da poesia.} descante. \\
Celebra ora um fantasma anguiforme\footnote{Sem forma definida (remete à água).} de Dante, \\
Ora o vulto marcial\footnote{Luta, guerra; derivado de Marte, o deus da guerra.} de um guerreiro de Homero.

Dá-me o hemistíquio\footnote{Cada uma das metades de um verso alexandrino. Francisca deseja que o verso seja construído perfeitamente, de imagem atrativa e vivacidade (enargeia).} d'ouro, a imagem atrativa; \\
A rima cujo som, de uma harmonia crebra, \\
Cante aos ouvidos\footnote{No sentido de uma leitura silenciosa.} d'alma; a estrofe limpa e viva;

Versos que lembrem, com seus bárbaros ruídos, \\
Ora o áspero rumor de um calhau que se quebra, \\
Ora o surdo rumor de mármores partidos\footnote{Os dois últimos versos indicam ações que possuem uma sonoridade áspera, nada suave, e que causam temor e despertam atenção.}.
\end{verse}

As musas eram as filhas de Zeus com a deusa Memória. Inspiravam, sobretudo, o trabalho poético, e representavam uma força externa, entidades que inspiravam a arte.

Em \textit{Musa impassível}, escrito em versos alexandrinos, a autora analisa o que seria um trabalho artístico ideal. Nota-se também a utilização da metalinguagem.

\section{Olavo Bilac}

Algum texto aqui.

\subsection{Tarde}

\textit{Tarde} (1919), em sua primeira edição, é consideravelmente diferente quando comparada com suas publicações posteriores — muitos poemas foram adicionados pelo autor a cada versão. 

\poemtitle{A um poeta}
\begin{verse}
Longe do estéril\footnote{Incapaz de se reproduzir; nesse sentido, algo supérfluo, que não é produtivo, sem utilidade.} turbilhão\footnote{Em relação ao movimento de redemoinho da água.} da rua, \\
Beneditino\footnote{Como vocativo, refere-se à ordem dos beneditinos (lembre-se da distinção entre o clero regular e o clero secular); o poeta beneditino recai na ideia da Torre de Marfim, que se afasta do estéril turbilhão da rua para escrever como um monge beneditino.}, escreve! No aconchego \\
Do claustro\footnote{Local de descanso do monge. }, na paciência e no sossego, \\
Trabalha e teima, e lima, e sofre, e sua!\footnote{A poesia é tratada como um trabalho custoso, difícil, que envolve esforço; limar o trabalho ao mínimo essencial para que seja perfeito, mas que, uma vez finalizado, o esforço do poeta não deve ficar marcado no poema.}

Mas que na forma se disfarce o emprego\footnote{Aliteração do f e do r na mesma sílaba, detalhe sutil que não torna evidente o trabalho de execução do poema.} \\
Do esforço; e trama viva se construa \\
De tal modo, que a imagem fique nua, \\
Rica mas sóbria, como um templo grego\footnote{O templo grego como exemplo de beleza, caracterizado pela harmonia e pelas formas simples (acreditava-se que as esculturas e a arquitetura, de maneira geral, eram brancas, de tal forma que expressavam um pensamento mais abstrato).}.

Não se mostre na fábrica o suplicio\footnote{Sofrimento intenso.} \\
Do mestre. E, natural, o efeito agrade, \\
Sem lembrar os andaimes do edifício:

Porque a Beleza, gêmea da Verdade, \\
Arte pura, inimiga do artifício\footnote{Artificial, complicado.}, \\
É a força e a graça na simplicidade\footnote{Ideia da arte pela arte (distanciamento das questões políticas e sociais da época. Cabe mencionar, no entanto, que Bilac escrevera sobre temas como o nacionalismo.}.
\end{verse}

Identificamos um nacionalismo e espírito cívico (na segunda edição da obra, em especial, um poema épico sobre Fernão Dias é acrescido). Bilac utiliza uma linguagem culta e erudita, ainda que menos quando comparada a Alberto de Oliveira, por exemplo, o que, potencialmente, assegurara tamanha notoriedade.

\textit{Tarde} foi um livro publicado postumamente. Identificamos um processo de espiritualização da poesia bilaquiana, em produções menos sensuais e sensíveis, e frequentemente com teor mais meditativo (reflexivo), e em tom elegíaco\footnote{Elegia era um canto, lamento por algo que se perdeu. Ao longo do tempo, passa a representar a poesia que abordava a finitude das coisas e da existência (tintas do Barroco).}. 

Bilac, então perseguido por suas posições políticas, estadiou-se em Ouro Preto (no soneto \textit{Vila Rica}, por exemplo, descreve o fim do dia de uma cidade que outrora fora grandiosa, mas agora vivia o fim do ciclo do ouro; melancolia inspirada pelo pôr do sol).

O livro expressa o chamado \textit{espírito de missão} do homem de letras brasileiro. O intelectual brasileiro se ressentia do atraso brasileiro em relação aos países desenvolvidos. A principal explicação, presente por exemplo na obra de Oliveira Viana, é de caráter determinista, e identificava a miscigenação como degeneração. Nesse cenário, aos intelectuais brasileiros caberia disseminar valores civilizatórios para a população, espalhando valores progressistas para uma população ignorante (espírito de missão). Trata-se de uma variação local do \textit{fardo do homem branco}, este no contexto do neocolonialismo do século XIX.

Lembre-se, por fim, da figura de Padre Anchieta e o processo de catequização dos indígenas no Brasil. Bilac era agnóstico ou ateu, e chamara atenção para a imagem do jesuíta que desbrava as matas para catequizar os indivíduos, adentrando a mata e misturando sua voz ao vento. Os indígenas, em sua recepção, comportariam-se ora como aves dóceis, ora como feras selvagens (construção de uma imagem bestializada).

Quimera, nesse contexto, representa uma mistura.

As entradas eram expedições organizadas pelo bandeirantes em busca de metais preciosos ou mesmo indígenas para escravizar. Mais suaves pois Anchieta não desbravava a terra, mas a alma dos indígenas.

Orfeu era o poeta que fora ao mundo dos mortos buscar Eurídice, cujo canto acalmava mesmo as piores feras. Assim, Bilac descreve, ao final do poema, o próprio intelectual brasileiro, que traz a civilização ao povo brasileiro (lembre-se da passagem de São Francisco de Assis e o ensino do catolicismo para as aves).

Olavo Bilac escolhe a figura do Padre Anchieta para representar o papel do intelectual brasileiro em trazer os valores civilizatórios a uma sociedade miscigenada. Há a associação do trabalho dos jesuítas com os bandeirantes, no processo de ampliação dos limites da civilização ocidental (desbravavam a alma dos indígenas; a figura de Orfeu dominando as feras), assim como os bandeirantes expandiram os limites fronteiriços.

Interpretar um poema é realizar perguntas a ele.

\poemtitle{Anchieta}
\begin{verse}
Cavaleiro da mística aventura, \\
Herói cristão! nas provações atrozes \\
Sonhas, casando a tua voz às vozes \\
Dos ventos e dos rios na espessura:

Entrando as brenhas, teu amor procura \\
Os índios, ora filhos, ora algozes, \\
Aves pela inocência, e onças ferozes \\
Pela bruteza, na floresta escura.

Semeador de esperanças e quimeras, \\
Bandeirante de “entradas” mais suaves, \\
Nos espinhos a carne dilaceras:

E, porque as almas e os sertões desbraves, \\
Cantas: Orfeu humanizando as feras, \\
São Francisco de Assis pregando às aves
\end{verse}

\poemtitle{Língua portuguesa}
\begin{verse}
Última flor do Lácio\footnote{Região da Itália na qual Roma foi fundada. Em latim, \textit{latium} (região onde se fala latim).}, inculta\footnote{Pois expressa uma nacionalidade ainda não bem polida.} e bela, \\
Éas, a um tempo, esplendor\footnote{Pois é uma língua que apresenta beleza própria, mas sepultara o Brasil ao isolá-lo do restante do mundo (francês e inglês como línguas dominantes).} e sepultura: \\
Ouro nativo, que na ganga\footnote{Em referência ao ouro de aluvião. Há uma comparação com a questão da mineração, outrora importante para a economia do Brasil. Surgimento de uma língua literária brasileira com o Arcadismo.} impura \\
A bruta mina entre os cascalhos vela...

Amo-te assim, desconhecida e obscura\footnote{Pois ainda não é conhecida por muitos povos.}, \\
Tuba de alto clangor, lira singela\footnote{Pois permite um tom alto, eloquente, como em \textit{Os lusíadas}, mas que também permite cantar o amor, como em \textit{Marília de Dirceu}.} \\
Que tens o trom e o silvo da procela, \\
E o arrolo da saudade e da ternura!

Amo o teu viço agreste\footnote{A língua portuguesa como algo selvagem.} e o teu aroma \\
De virgens selvas\footnote{Em relação ao Brasil.} e de oceano largo\footnote{Em comparação com Portugal.}! \\
Amo-te, ó rude e doloroso idioma,

Em que da voz materna ouvi: “meu filho!”, \\
E em que Camões chorou, no exílio amargo, \\
O gênio sem ventura e o amor sem brilho!
\end{verse}

A língua portuguesa é chamada de ``última flor do Lácio'' pois, na visão de Olavo Bilac, por portugueses eram, em certo grau, vistos como uma sociedade congelada no tempo, que não acompanhara, por exemplo, a Revolução Industrial, e que continuara como nação mercantilista e que consumia os produtos ingleses (relembre Baruch Espinosa, de origem portuguesa, mas que alcançara notoriedade ao publicar suas obras em latim; \textit{Triste Bahia}, de Gregório de Matos).

Relembre, do Romantismo, da ideia do país novo, aqui retomada por Bilac em relação à língua portuguesa ao apresentar a promessa dessa língua no futuro (tal interpretação, em certo grau, fazia certo sentido, sobretudo no início do Segundo Reinado). Olavo Bilac celebra a língua portuguesa, enfatizando o ``vir a ser'' em matriz romântica.

\poemtitle{Música brasileira}
\begin{verse}
Tens, às vezes, o fogo soberano \\
Do amor: encerras na cadência, acesa \\
Em requebros e encantos de impureza, \\
Todo o feitiço do pecado humano.

Mas, sobre essa volúpia, erra a tristeza \\
Dos desertos, das matas e do oceano: \\
Bárbara poracé, banzo africano, \\
E soluços de trova portuguesa.

És samba e jongo\footnote{Músicas de origem africana.}, xiba e fado\footnote{Músicas de origem portuguesa.}, cujos \\
Acordes são desejos e orfandades \\
De selvagens, cativos e marujos\footnote{Respectivamente, indígenas, africanos e portugueses.}:

E em nostalgias e paixões consistes, \\
Lasciva dor, beijo de três saudades, \\
Flor amorosa de três raças tristes.
\end{verse}

Ideia de que a música brasileira é dançante e propõe certa sensualidade. Ainda assim, expressa a tristeza dos desertos (africanos trazidos forçosamente), das matas (indígenas) e dos oceanos (portugueses). \textit{Banzo} é o estado de saudade no qual se encontravam os africanos quando trazidos para o Brasil (casos de suicídio em alto-mar ou ingerindo terra). Sentimento de desterritorialização.

O Brasil é visto não como herdeiro do colonizador, mas uma mistura das três taças, No entanto, um povo que se harmoniza na tristeza em uma visão do bom colonizador (também presente em José de Alencar com Martim, por exemplo). Notamos uma visão edulcorada, domesticada do processo de colonização, e que coloca no mesmo grau de tristeza os portugueses, africanos e indígenas.

A incorporação pela cultura não representa uma incorporação social.

\poemtitle{Pátria}
\begin{verse}
Pátria, latejo em ti, no teu lenho, por onde \\
Círculo! e sou perfume, e sombra, e sol, e orvalho! \\
E, em seiva, ao teu clamor a minha voz responde, \\
E subo do teu cerne ao céu de galho em galho!

Dos teus liquens, dos teus cipós, da tua fronde, \\
Do ninho que gorjeia em teu doce agasalho, \\
Do fruto a amadurar que em teu seio se esconde, \\
De ti, - rebento em luz e em cânticos me espalho!

Vivo, choro em teu pranto; e, em teus dias felizes, \\
No alto, como uma flor, em ti, pompeio e exulto! \\
E eu, morto, - sendo tu cheia de cicatrizes,

Tu golpeada e insultada, - eu tremerei sepulto. \\
E os meus ossos no chão, como as tuas raízes, \\
Se estorcerão de dor, sofrendo o golpe e o insulto!
\end{verse}

Em tom nacionalista, Bilac estabelece a relação da pátria como uma árvore e o eu lírico como a seiva (ideia de que o povo nutre a nação). Existe uma relação simbiótica do povo brasileiro com a pátria. Bilac era como um intelectual orgânico de uma modernização conservadora; o avanço urbano que mantém o contexto social desigual. Educação como a solução final para o avanço nacional.

\subsection{Poesias}

Há uma continuidade com a tradição petrarquista da poesia de língua portuguesa, sobretudo nos sonetos de \textit{Via-Láctea}, uma das seções da antologia \textit{Poesias}. Há também a presença de uma poesia lírico-amorosa, com a sublimação da figura feminina (divinização, espiritualização da mulher, próximo à ideia do amor platônico).

O erotismo era muito mais comum na poesia parnasiana brasileira do que na francesa (pornográfico apenas no caso de Luís Delfino).

Já a seção \textit{Sarças de fogo}, da mesma coletânea, concentra a maior parte dos poemas eróticos. Bilac descreve, por exemplo, o despir de uma mulher ao chegar de uma festa. Comete a ousadia de descrever tais elementos, mas por meio de eufemismos (permanece no limiar).

\href{http://www.dominiopublico.gov.br/download/texto/bv000287.pdf}{Sarças de fogo (domínio público)}.

No poema \textit{De volta ao baile}, por exemplo, Bilac utiliza o termo ``Leve buço dourado sombreia'' em referência aos pelos pubianos.

\poemtitle{Profissão de fé}
\begin{verse}
Le poète est cise1eur, \\
Le ciseleur est poète. \\
Victor Hugo.

Não quero o Zeus Capitolino\footnote{Soberano; local onde as leis são instituídas.} \\
Hercúleo\footnote{Força.} e belo, \\
Talhar no mármore divino \\
Com o camartelo\footnote{Martelo utilizado na confecção de estátuas.}.

Que outro - não eu! - a pedra corte \\
Para, brutal, \\
Erguer de Atene o altivo porte \\
Descomunal.\footnote{Estabelece uma relação entre o trabalho do escritor e do joelheiro (ourives), e não com o escultor.}

Mais que esse vulto extraordinário, \\
Que assombra a vista, \\
Seduz-me um leve relicário\footnote{Colar no qual se coloca como um porta-retrato.} \\
De fino artista.

Invejo o ourives quando escrevo: \\
Imito o amor \\
Com que ele, em ouro, o alto relevo \\
Faz de uma flor.

Imito-o. E, pois, nem de Carrara\footnote{Mármore fino, de alta qualidade.} \\
A pedra firo:\footnote{Novamente, distancia-se da relação com o escultor e, nos versos seguintes, enaltece o trabalho minucioso do ourives, semelhante ao do poeta.} \\
O alvo cristal, a pedra rara, \\
O ônix prefiro.

Por isso, corre, por servir-me, \\
Sobre o papel \\
A pena, como em prata firme \\
Corre o cinzel.

Corre; desenha, enfeita a imagem,\footnote{O poema como o embelezamento da ideia.} \\
A idéia veste: \\
Cinge-lhe ao corpo a ampla roupagem \\
Azul-celeste.\footnote{Associação com Maria: a arte como forma de transcendência.}

Torce, aprimora, alteia, lima \\
A frase; e, enfim,\footnote{Sacralização da poesia, evidenciando o resultado como fruto de árduo trabalho, e não um surto de inspiração indicado pelos escritores românticos.} \\
No verso de ouro engasta a rima, \\
Como um rubim.\footnote{Comparação da rima com uma joia.}

Quero que a estrofe cristalina, \\
Dobrada ao jeito \\
Do ourives, saia da oficina \\
Sem um defeito:

E que o lavor do verso, acaso, \\
Por tão subtil, \\
Possa o lavor lembrar de um vaso \\
De Becerril\footnote{Cidade da Espanha (confirmar), conhecida pela técnica secular de confecção de vasos de cristal.}.

E horas sem conto passo, mudo, \\
O olhar atento, \\
A trabalhar, longe de tudo \\
O pensamento.\footnote{Comparação da poesia como um trabalho.}

Porque o escrever - tanta perícia, \\
Tanta requer, \\
Que oficio tal... nem há notícia \\
De outro qualquer.\footnote{Hipérbole. A poesia é tratada como o trabalho mais difícil.}

Assim procedo. Minha pena \\
Segue esta norma, \\
Por te servir, Deusa serena, \\
Serena Forma!\footnote{Novamente, a sacralização (endeusamento) da forma da poesia.}

Deusa! A onda vil, que se avoluma \\
De um torvo mar, \\
Deixa-a crescer; e o lodo e a espuma \\
Deixa-a rolar!

Blasfemo> em grita surda e horrendo \\
Ímpeto, o bando \\
Venha dos bárbaros crescendo, \\
Vociferando...

Deixa-o: que venha e uivando passe \\
- Bando feroz! \\
Não se te mude a cor da face \\
E o tom da voz!

Olha-os somente, armada e pronta, \\
Radiante e bela: \\
E, ao braço o escudo> a raiva afronta \\
Dessa procela!

Este que à frente vem, e o todo \\
Possui minaz \\
De um vândalo ou de um visigodo, \\
Cruel e audaz;

Este, que, de entre os mais, o vulto \\
Ferrenho alteia, \\
E, em jato, expele o amargo insulto \\
Que te enlameia:

É em vão que as forças cansa, e â luta \\
Se atira; é em vão \\
Que brande no ar a maça bruta \\
A bruta mão.

Não morrerás, Deusa sublime! \\
Do trono egrégio \\
Assistirás intacta ao crime \\
Do sacrilégio.

E, se morreres por ventura, \\
Possa eu morrer \\
Contigo, e a mesma noite escura \\
Nos envolver!

Ah! ver por terra, profanada, \\
A ara partida \\
E a Arte imortal aos pés calcada, \\
Prostituída!...

Ver derribar do eterno sólio \\
O Belo, e o som \\
Ouvir da queda do Acropólio, \\
Do Partenon!...

Sem sacerdote, a Crença morta \\
Sentir, e o susto \\
Ver, e o extermínio, entrando a porta \\
Do templo augusto!...

Ver esta língua, que cultivo, \\
Sem ouropéis, \\
Mirrada ao hálito nocivo \\
Dos infiéis!...

Não! Morra tudo que me é caro, \\
Fique eu sozinho! \\
Que não encontre um só amparo \\
Em meu caminho!

Que a minha dor nem a um amigo \\
Inspire dó... \\
Mas, ah! que eu fique só contigo, \\
Contigo só!

Vive! que eu viverei servindo \\
Teu culto, e, obscuro, \\
Tuas custódias esculpindo \\
No ouro mais puro.

Celebrarei o teu oficio \\
No altar: porém, \\
Se inda é pequeno o sacrifício, \\
Morra eu também!

Caia eu também, sem esperança, \\
Porém tranqüilo, \\
Inda, ao cair, vibrando a lança, \\
Em prol do Estilo!
\end{verse}

O poema atua como um manifesto da poesia parnasiana, e retrata a busca do escritor pela perfeição formal. É um poema regular, com versos de oito e quatro sílabas poéticas.

\href{http://www.dominiopublico.gov.br/download/texto/bi000179.pdf}{Profissão de fé (domínio público)}.

\poemtitle{A sesta\footnote{Descanso após o almoço.} de Nero}
\begin{verse}
Fulge\footnote{Brilha.} de luz banhado, esplêndido e suntuoso\footnote{Luxuoso.}, \\
O palácio imperial de pórfiro\footnote{Pedra.} luzente \\
E mármor da Lacônia\footnote{Região.}. O teto caprichoso \\
Mostra, em prata incrustado, o nácar\footnote{Resina prateada.} do Oriente.

Nero no toro\footnote{Banco maciço.} ebúrneo\footnote{Marfim.} estende-se indolente\footnote{Preguiçoso.}... \\
Gemas em profusão do estrágulo custoso \\
De ouro bordado veem-se. O olhar deslumbra, ardente, \\
Da púrpura\footnote{Tecido da cor púrpura.} da Trácia\footnote{Região.} o brilho esplendoroso.\footnote{Seda.}

Formosa ancila\footnote{Escrava.} canta. A aurilavrada\footnote{Ouro.} lira \\
Em suas mãos soluça. Os ares perfumando, \\
Arde a mirra\footnote{Especiaria.} da Arábia em recendente pira.

Formas quebram, dançando, escravas em coreia. \\
E Neto dorme e sonha, a fronte reclinando \\
Nos alvos seios nus da lúbrica Pompeia\footnote{Mãe de Nero (possível relação incestuosa).}.
\end{verse}

Ao realizar a escansão do poema, note a utilização de versos alexandrinos, com um icto (tônica regular em determinada posição do verso) na sexta sílaba poética, dividindo o verso em dois hemistíquios (cada uma das metades do verso alexandrino). O poema utiliza a hipotipose, descrição em movimento (cena) para retratar um momento solene.

Em \textit{Os lusíadas}, encontramos o verso decassílabo heroico, com icto na sexta sílaba poética (é o verso longo mais utilizado na língua portuguesa). Na França, era comum a utilização de versos dodecassílabos, também com icto na sexta sílaba poética. Por fim, cabe mencionar que a cesura masculina era a oxítona, ao passo que a cesura feminina era a paroxítona (a próxima palavra começa com vogal e junção). No Parnasianismo, há uma preferência do ditongo ao hiato.

\href{http://www.dominiopublico.gov.br/pesquisa/DetalheObraForm.do?select_action=&co_obra=7564}{Panóplias (domínio público)}

Para ler mais tarde: ``O abuso da literatura'', ``Jabuticaba literária''.

https://www.revistas.usp.br/teresa/article/view/127463