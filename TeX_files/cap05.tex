\chapterimage{chapter_head_2.pdf}

\chapter{Introdução}

O Barroco surgira na Europa no século XVII (\textit{El Siglo de Oro} na Espanha, em especial pelo fortalecimento ultramarino do país). Situa-se no contexto da \textbf{Contrarreforma}, organizada pela Igreja Católica em meio ao embate entre católicos e protestantes. Na época, o catolicismo era a religião oficial das nações europeias, e o protestantismo surgia como linha que propunha maior liberdade religiosa. Também nesse período se inicia o Concílio de Trento.

Como citado anteriormente, diversas foram as medidas adotadas pela Igreja Católica com o intuito de barrar o avanço do protestantismo na Europa e ampliar a base de fiéis. Em primeiro lugar, houve uma \textbf{centralização} da Inquisição, instituída já na Idade Média. É interessante notar que, anteriormente, cada bispo possuía o poder de inquisidor, e não haviam procedimentos padrões de tortura, por exemplo. Com tal ação, a partir de um tribunal em Roma, o papel de inquisidor se limitaria aos Familiares do Santo Ofício, que perseguiriam apenas crimes contra a fé (lembre-se de que a apostasia também era considerada um crime). 

O século XVII marcara o ápice da Inquisição. Aos hereges cabia realizar o auto de fé, composto por trinta dias de confissão e penitências. Também nesse período caberia aos indivíduos apresentar ao Santo Ofício culpas ou denúncias que julgavam ser de interesse da instituição (investigações secretas também ocorriam). Indícios suficientes levariam ao interrogatório, este que, por sua vez, poderia levar à tortura (o emprego desse procedimento sem acusações bastava para assegurar a inocência do suspeito). Nesta, os instrumentos a serem utilizados eram primeiramente apresentados para que ocorresse o julgamento de prova (provação divina). Neste processo tão integrado ao sistema jurídico, é interessante o desfecho dos casos de confissão durante os procedimentos: se houvesse arrependimento por parte do indivíduo, este seria levado ao arame e submetido ao sufocamento. Em caso contrário, seria de imediato queimado vivo. A catequização de outros povos também se fez presente nesse período, em especial com o surgimento da \textbf{Companhia de Jesus}, fundada por Santo Inácio de Loiola.

Para melhor entender as razões que levaram a esse cenário, alguns apontamentos são necessários. Em primeira análise, é importante notar que o protestantismo defendia a doutrina da \textit{sola scriptura}, para a qual tudo o que é falado deve possuir algum fundamento bíblico, como o dogma da Imaculada Santíssima e da Trindade presentes no catolicismo. Para tanto, também se exigia a tradução da Bíblia (lembre-se da Bíblia de Lutero, por exemplo, uma tradução alemã) e a alfabetização da população. Ainda no período, as artes eram utilizadas como instrumento da doutrina católica (seja pelo uso de retábulos, quadros, etc.). 

Mas por que, afinal, tais ações eram proibidas pela Igreja? O sociólogo alemão Max Weber (1864-1920), em seu livro \textit{A ética protestante e o espírito do capitalismo}, apresenta alguns esclarecimentos. Em relação à educação da população e tradução dos escritos, lembre-se de que as missas (e textos) eram celebradas em latim, língua conhecida apenas pelo clero. Além disso, a cobrança de juros pelo indivíduo era vista como pecado (usura)  e, pois, o enriquecimento era interpretado como algo negativo. Nota-se, evidentemente, uma oposição entre a doutrina católica e o sistema capitalista, que em paralelo se desenvolvia.

Todos esses acontecimentos levaram a um momento de \textbf{crise da consciência} do europeu, cuja visão naturalista (secular, não religiosa) construída desde o Humanismo entrava em conflito com um modelo rígido de religiosidade.

\begin{table}[h]
\centering
\begin{tabular}{l l}
\toprule
\multicolumn{2}{c}{\textbf{As dicotomias do pensamento barroco}} \\
\midrule
espírito & carne \\
eternidade & vida \\
além & mundo \\
luz & sombra \\
\bottomrule
\end{tabular}
\end{table}

Esse conflito no pensamento também era visto nos lugares-comuns utilizados na época, que transitavam entre:

\begin{theorem}[\textit{Carpe diem}]
No sentido de aproveitar o momento presente (colher o dia).
\end{theorem}

\begin{theorem}[\textit{Memento mori}]
Na lembrança de que os seres humanos são mortais.
\end{theorem}

Um possível questionamento é de que a lembrança da mortalidade pode levar o indivíduo ao desejo de aproveitar ainda mais sua vida e existência, valores compreendidos pelo \textit{carpe diem}. Note, no entanto, que o \textit{memento mori} refere-se, sobretudo, na relação da morte e a sequente preocupação com a salvação da alma. Como síntese, a grande questão do Barroco envolve a contradição do indivíduo que anseia pelos prazeres da vida, mas que ainda se preocupa com a salvação da alma.

No Caderno de Imagens, segue uma compilação de diferentes pinturas e esculturas do período (em alguns casos, comparando-as com outros projetos estéticos com a mesma temática, mas de diferentes períodos), assim como uma breve menção sobre a arquitetura.

O Barroco foi o estilo de época de maior duração. 

Os dois principais estilos dessa época eram o \textbf{cultismo} e o \textbf{conceptismo}. De maneira geral, havia um contraste entre a vida terrena e a vida eterna, e a representação de Jesus era um símbolo para o conflito ideológico barroco, do lado humano que se sobressai durante a crucificação (Jesus frequentemente era representado crucificado, na figura de um mártir). Havia o uso recorrente de antíteses e paradoxos, antecipados pelo \textbf{Maneirismo}, no qual o escritor expressa as contradições de sua consciência.