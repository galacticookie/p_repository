\chapterimage{chapter_head_2.pdf}

\chapter{Arcadismo em Portugal}

Formação do mecenato lusitano. Marquês de Pombal, um dos ministros de Dom João V. Expulsou os jesuítas no contexto da educação ligada à escolástica, e era um símbolo do despotismo esclarecido.

Déspota é um governante autoritário, tirano, que centraliza o poder. Esclarecido deriva de esclarecimento, descrito no capítulo de introdução da seção, aquele que se utiliza do poder de forma racional, buscando a liberdade e a igualdade. Kant os referiu com o monarca da Áustria, que era como um patrono dos filósofos.

A Arcadia lusitana recebia incentivo fiscal de Pombal, o qual também recebera poemas encomiásticos destinados a sua figura.

\section{Manuel du Bocage}

Um poeta árcade português. Produzira poemas líricos, pornográficos e outros, em geral, racistas. É interessante mencionar que, embora possua uma vertente de acordo com os valores do Arcadismo, é possível identificar em parte de suas obras elementos do Romantismo, estilo de época posterior (\textit{pré-romantismo}\footnote{Incorporação de elementos pessoais e informações biográficas - como experiências pessoais e vivências - na poesia lírica, afastando-se, por vezes, dos convencionalismos da poesia árcade.}).

O Classicismo e o Neoclassicismo foram estilos de época muito convencionais. De maneira geral, o poeta lírico era habilidoso em se utilizar dos lugares-comuns e emular os clássicos, além de fazer bom uso do decoro. Tome como exemplo o soneto de Camões \textit{Alma minha gentil que te partiste}, assim como a figura da amante Inês de Castro: não devemos assumir que o autor estava expressando verdadeiramente os seus sentimentos e a sua vida, pois se valorizava convencionalismo, e não autenticidade. A concepção de que a poesia lírica expressava sentimentos honestos surgiria apenas no Romantismo.

É um dos poetas mais representativos do Arcadismo português.

\poemtitle{Soneto do Epitaphio}
\begin{verse}
Lá quando em mim perder a humanidade \\
Mais um daqueles, que não fazem falta, \\
Verbi-gratia — o theologo, o peralta, \\
Algum duque, ou marquez, ou conde, ou frade:
				
Não quero funeral communidade, \\
Que engrole sub-venites\footnote{Canto em intenção dos mortos.} em voz alta; \\
Pingados gatarrões, gente de malta\footnote{Do povo, sem importância.}, \\
Eu também vos dispenso a caridade:
				
Mas quando ferrugenta enchada idosa \\
Sepulchro me cavar em ermo outeiro, \\
Lavre-me este epitaphio\footnote{Escrito colocado na lápide.} mão piedosa:
				
«Aqui dorme Bocage, o putanheiro: \\
Passou vida folgada, e milagrosa; \\
«Comeu, bebeu, fodeu sem ter dinheiro.
\end{verse}

Trata-se de um poema auto-satírico, no qual o autor é o próprio alvo das ironias e críticas. Insinua que, quando morrer, não fará falta para os outros. Revela não desejar nenhum duque, marquez, conde ou frade em seu funeral, tampouco gente de malta.

Fato curioso é que Bocage também escrevera um soneto satirizando o próprio nariz, \textit{Nariz, nariz e nariz}.