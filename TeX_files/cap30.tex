\chapterimage{chapter_head_2.pdf}

\chapter{Modernismo em Portugal}

Primeiramente, é importante mencionar o atraso de Portugal em relação às nações industrializadas, o que gerara uma ambivalência no Modernismo português.

Lembre-se de que o Modernismo é o conjunto das vanguardas europeias (movimentos artísticos de ruptura com a tradição ocidental) e seus desdobramentos nas diversas localidades do globo (não necessariamente um escritor precisa pertencer a uma determinada vanguarda — composição das próprias tendências).

Até esse momento, os estilos de época eram observados primeiramente em Portugal e, posteriormente, no Brasil. Com o Romantismo, por exemplo, tivemos informações que surgiram na França, disseminaram-se para Portugal e, então, para o Brasil.

Nesse período, no entanto, as disparidades entre os países já haviam se amenizado. Posteriormente, a literatura brasileira passa a servir de influência para a literatura portuguesa nas décadas de 40 e 50.

A rigor, o Modernismo no Brasil se iniciara antes de Portugal.

O Modernismo é a expressão de uma sociedade industrial, na qual a tradição perde seu significado (lembre-se de Marinetti que, no \textit{Manifesto futurista}, alegara que todos os museus e bibliotecas deviam ser queimados). Em paralelo, existe um esforço da pintura realista em oposição às facilidades proporcionadas pela fotografia (tivemos o surgimento da Kodak, por exemplo, que trouxera grande praticidade aos instrumentos de registro).

Perceba, no entanto, que o contexto de Portugal ainda não é esse, pois ainda não é um país com um setor industrial responsável pela maior parte da economia. Pelo contrário, sua estrutura econômica estava atrelada a uma fase anterior do capitalismo, o capitalismo mercantil, em oposição ao capitalismo mercantil. Isso gera uma ambivalência (duas coisas vigorando simultaneamente), à medida em que as vanguardas chegam em um país em um cenário diferente do qual se originaram. 

Em Portugal, sobretudo na obra de Fernando Pessoa, há uma tentativa de conciliação entre as vanguardas e a tradição portuguesa. Mensagem é uma obra na qual Fernando Pessoa se propõe a recontar a história de Portugal como uma narrativa mítica (versos livres), em uma técnica muito mais próxima do lírico do que do gênero épico (mesmo um autor do século XIX não escreveria algo como o autor o fizera). As vanguardas europeias em Portugal como uma importação (diferentemente do surrealismo na França, por exemplo). Dialética entre o localismo e o cosmopolitismo.

\section{Fernando Pessoa}

Primeiramente, devemos realizar a diferenciação entre os ortônimos e heterônimos. Fernando Pessoa era de família de diplomatas, e passou parte da vida na África e parte na Inglaterra. Trabalhou a maior parte da vida como tradutor de empresas. Ainda assim, sempre foi muito ativo na imprensa da época, e traduzia livros literários e esotéricos (foi o primeiro a traduzir O homem mais perverso do mundo, por exemplo). Possui interesse pelo misticismo, era astrólogo e se interessava por matemática e escaladas.

Poesia ortônima (Fernando Pessoa ele-mesmo) e heteronímica (outras personalidades, diferente de um pseudônimo). A poesia ortônima recorre, na maioria dos casos, a métricas e rimas, em uma abordagem mais tradicionalista. Pseudônimo significa, literalmente, falso. Trata-se de um nome falso, alternativo. Utilizado, principalmente, por mulheres autores, para que fossem inseridas no mercado editorial.

O heterônimo é uma personalidade com história de vida e visões de mundo próprias, com estilos e formas de escrita individuais. Alberto Caeiro como o primeiro heterônimo de Fernando Pessoa, mestre dos demais (Ricardo Reis e Álvaro de Campos, por exemplo). Bernardo Soares, escritor de prosa, não será tratado em nosso curso.

"O crítico que atribuiu a mim algum tipo de esquizofrenia". Cada um dos heterônimos teve sua morte determinada, com exceção de Ricardo Reis, Décadas depois, José Saramago escreve O ano da morte de Ricardo Reis, indivíduo que sobrevivera a Fernando Pessoa. Retorna para Portugal durante a queda do governo de Salazar, e acaba morrendo. Em uma cena memorável, encontra um homem mascarado nas ruas do Carnaval português. Entrando em um bar é perseguido e descobre que se tratava de Fernando Pessoa, seu fantasma, sombra. Em uma das questões da Fuvest, pedia-se uma relação entre Os lusíadas e Fernando Pessoa (descobre o novo a partir do familiar), na figura dos descobridores (Vasco da Gama e Fernando Pessoa).

Fernando Pessoa se denomina como poeta dramático (drama como gênero teatral), i.e., é um autor/ator interpretando os heterônimos/personagens (em grego, personae significa máscara, utilizada pelos autores nas peças da Grécia antiga, como forma de diferenciar os atores dos outros envolvidos). Por parte de uma crítica portuguesa: Pessoa em drama, o autor que cria outras pessoas.

Mensagem foi o único livro publicado em vida (com exceção de escritos na revista Orpheu, por exemplo). O que Fernando Pessoa deixou foi um baú, escritos mal organizados para o público (pouco metódico e organizado). Quem os organizara fora a sua irmã. Possuía um amor platônico, ainda que questionável em que medida estava apenas se inspirando nos sonetos de Shakespeare, por exemplo.

A revista Orpheu originara o Modernismo em Portugal (muitos dos poemas do autor foram publicados nesse impresso).

\poemtitle{Liberdade}
\begin{verse}
Ai que prazer \\
Não cumprir um dever, \\
Ter um livro para ler \\
E não o fazer! \\
Ler é maçada, \\
Estudar é nada. \\
O sol doira \\
Sem literatura.\footnote{Indicação da literatura como elemento não essencial.} \\
O rio corre, bem ou mal,\footnote{A natureza não se importa com a literatura.} \\
Sem edição original.\footnote{Em referência aos colecionadores e especialistas, que se atentam aos textos originais. Nesse sentido, ao público maior, não há tamanha preciosidade, da qual a natureza também não se importa. Questão das primeiras edições autografadas.} \\
E a brisa, essa, \\
De tão naturalmente matinal, \\
Como tem tempo não tem pressa...\footnote{A natureza tem o seu próprio ritmo, e não se importa se o indivíduo lera todos os clássicos ou não.}

Livros são papéis pintados com tinta.\footnote{Trata-se de uma tautologia, informação óbvia. Os livros como uma espécie de fetiche, que carregam certa aura. No entanto, o livro é apenas o suporte para um texto (distanciamento com a ideia anterior de fetichismo em torno do objeto livro). Não é a quantidade de livros o que torna um indivíduo culto (embora óbvio, é algo que nossa cultura tende a considerar; o livro como capital cultural, símbolo de conhecimento, sabedoria, cultura).} \\
Estudar é uma coisa em que está indistinta\footnote{Estudar é inútil, supérfluo, à medida em que o estudo nos coloca fora de contato com o que é essencial em nossas vidas (afasta o indivíduo da natureza).} \\
A distinção entre nada e coisa nenhuma.

Quanto é melhor, quanto há bruma, \\
Esperar por D. Sebastião,\footnote{Dom Sebastião era o rei de Portugal no período de Camões. O nascimento difícil do garoto não trouxe facilidades ao reino (talvez se interessasse por rapazes), não se casara e não deixara nenhum herdeiro. Desaparecera na batalha de Alcácer-Quibir. Surgimento da União Ibérica (incorporação do reino de Portugal à Espanha), momento da história no qual o país pioneiro das Grandes Navegações, em menos de oitenta anos, perde todo o prestígio anterior. O desaparecimento marca o fundo do poço dessa nação (Espanha, Países Baixos, Inglaterra). Por muito tempo, havia a espera de Dom Sebastião para libertar Portugal e retomar o prestígio. No imaginário português já havia a figura do rei desaparecido (rei Arthur), que lideraria, por exemplo, os ingleses contra os franceses (também presente no norte da França, conflito interessante em meio à Guerra dos Cem Anos). Incorporação também presente na figura de Carlos Magno por parte dos franceses. A figura da bruma associada a figura do rei Arthur (engravidara do homem em relação incestuosa). Combate com próprio filho em campo de batalha. Corpo levado para a ilha de Àlavo, governada pela própria irmã do rei Arthur, Morgana. Com a morte, a mulher se arrepende e recebe o cadáver do rei, o cura, e assim espera para retornar. Associação de Dom Sebastião com a bruma, neblina, lenda vitoriana. Ainda há pessoas que acreditam na volta de Dom Sebastião (também presente no Brasil, como em algumas cidades do Maranhão, e a figura de um grande touro mágico debaixo dos lençóis maranhenses).} \\
Quer venha ou não!\footnote{Melhor do que estudar, pois a imaginação é mais importante do que a literatura, a razão (o lúdico). O que importa é o símbolo. Ideia de que nós temos tradicionalismo pela figura de Dom Sebastião, mas também incorporamos elementos anti-academicistas (ambivalência entre a cultura livresca, anti-literária, e a incorporação de símbolos como a crença no retorno de Dom Sebastião). Não exatamente uma questão de fé ou crença, mas aquilo que Dom Sebastião representa.}

Grande é a poesia, a bondade e as danças...\footnote{Indicação de uma dimensão da existência do maravilhamento, experiência estética da realidade. A suposta poesia que existe nas coisas, diferentes de um poema escrito dentro de um livro. O texto em si não é nada, mas sim o que este contém. Expressão de maravilhamento com a realidade, que também pode se manifestar como dança, por exemplo (Para que a arte de torne vital, e a vida se torne artística).} \\
Mas o melhor do mundo são as crianças, \\
Flores, música, o luar, e o sol, que peca \\
Só quando, em vez de criar, seca.

O mais do que isto \\
É Jesus Cristo, \\
Que não sabia nada de finanças \\
Nem consta que tivesse biblioteca...\footnote{O mais extraordinário de todos os seres humanos que já existiu é Jesus Cristo, na visão do autor, e que não estudara e tampouco conhecia de finanças (aproximação como referência a algo sem vida; aproximação do materialismo burguês com a literatura). Paralelo intelectual semelhante a uma acumulação de bens. Visão contra a instituição Literatura (semelhante à contrariedade da instituição Arte), e por isso um poema vanguardista.}
\end{verse}

Há a incorporação do verso libre como regra (no Simbolismo, o verso libre era utilizado como exceção); as estrofes apresentam diferentes números de versos (ausência de regularidade estrófica), um aspecto inovador, de vanguarda, anti-tradicionalista. As rimas, por sua vez, são utilizadas em expediente tradicional, derivadas do Trovadorismo (há uma conciliação, ambivalência), mas que não seguem uma ordem, são aleatórias.

A linguagem utilizada é muito mais próximo do coloquial (de um português do início da década de 20, cabe mencionar).

A questão do cristianismo é central na história de Portugal (anteriormente em controle dos muçulmanos, se forma no combate a esses indivíduos) e, posteriormente, na disputa contra os protestantes. Figura dos cavaleiros templários, O autor acena para elementos da tradição portuguesa, em um olhar para o futuro que não se esquece dos valores portugueses.

\poemtitle{Autopsicografia}
\begin{verse}
O poeta é um fingidor\footnote{Pois não é necessário, por exemplo, estar apaixonado para falar sobre amor. Não podemos confundir o eu lírico com a pessoa do poeta, ou mesmo o narrador com a pessoa do autor, à medida em que a literatura trabalha com a esfera da imaginação. Ficção no português, possui a mesma raiz etimológica de fingir. Mesmo Bocage quando se refere a ele mesmo.} \\
Finge tão completamente \\
Que chega a fingir que é dor \\
A dor que deveras sente.\footnote{Pois a dor do poema não representa a dor real. O que está presente no poema é uma representação da dor (assim como a pintura de um cachimbo não é um cachimbo). A dor é uma experiência irredutivelmente subjetiva, e que não pode ser compartilhada, apenas representada (por palavras, no caso de um texto, que cria um simulacro da realidade). Discussão entre Platão e Aristóteles sobre simular a realidade.}

E os que lêem o que escreve, \\
Na dor lida sentem bem, \\
Não as duas que ele teve, \\
Mas só a que eles não têm.\footnote{A dor presente no poema e que fornece prazer ao leitor não é a dor do autor, e tampouco a dor representada no texto. Por essa razão é possível gerar prazer, e não criar uma experiência de dor.}

E assim nas calhas de roda \\
Gira, a entreter a razão, \\
Esse comboio de corda \\
Que se chama coração.\footnote{O poema como a água que, na calha de pá, o faz girar (imaginação, razão). Carro inicial que puxa o coração do leitor.}
\end{verse}

Forma poética mais tradicional da língua portuguesa, a trova (rima ABAB, em quadras e redondilha maior). Em inglês, por exemplo, o pé mais utilizado é o ânglico. O título se refere, no contexto do espiritismo, do médium que escreve uma carta orientado pelo espírito. O autor como médium de si mesmo.

Metalinguagem. "Eu não sou um poema de verdade", e devo ser tratado como representação da realidade. Ideia do anti-mimetismo, à medida em que Fernando Pessoa vai contra a concepção do Romantismo de que a obra de arte é uma realização estética da subjetividade do indivíduo. É impossível que o poema faça isso. Apresentação por meio de uma forma tão tradicional. O poeta como um médium de si mesmo, influenciado pela figura inventada referida como eu lírico (criado, por sua vez, pelo próprio indivíduo, por isso o prefixo auto). O eu lírico como uma projeção, no caso, da consciência de Fernando Pessoa.

Para referência futura: \href{http://arquivopessoa.net}{Domínio público}

\subsection{Alberto Caeiro}

Ainda no contexto do poeta dramático (literatura feita para ser encenada), Alberto Caeiro, cronologicamente, é o primeiro heterônimo de fato criado. É o mentor, guru, mestre para os demais. Era um pastor que vivia no campo. Nunca teve acesso a uma educação formal, e aprendeu de maneira autodidata. Morreu ainda na juventude, por complicações da tuberculose.

Uma característica muito presente em suas obras é o \textbf{realismo sensorial}: a ideia de que só existe o que é concreto e pode ser captado pelos cinco sentidos. Para analisarmos esse aspecto, segue o quinto poema de \textit{O guardador de rebanhos}:

\poemtitle{Há metafísica bastante em não pensar em nada}
\begin{verse}
V

Há metafísica bastante em não pensar em nada.\footnote{A Metafísica está além do plano físico (também o ramo da Filosofia).}

O que penso eu do Mundo? \\
Sei lá o que penso do Mundo! \\
Se eu adoecesse pensaria nisso.\footnote{O pensamento é uma doença (pensar é adoecer).}

Que ideia tenho eu das coisas? \\
Que opinião tenho sobre as causas e os efeitos? \\
Que tenho eu meditado sobre Deus e a alma \\
E sobre a criação do Mundo? \\
Não sei. Para mim pensar nisso é fechar os olhos \\
E não pensar. É correr as cortinas \\
Da minha janela (mas ela não tem cortinas).\footnote{Lembre-se da ideia de São Tomás de Aquino, derivada de Aristóteles, de que tudo possui uma causa -- ainda que não causada. A causa e o efeito como um dos argumentos para a existência de Deus. Fechar os olhos é uma forma de perder contato com a realidade, observada por meio da janela -- que não existe na casa de Alberto. Ao observarmos a realidade por meio da janela, também perdemos contato com a realidade.}

O mistério das coisas? Sei lá o que é mistério! \\
O único mistério é haver quem pense no mistério. \\
Quem está ao sol e fecha os olhos, \\
Começa a não saber o que é o Sol \\
E a pensar muitas coisas cheias de calor.\footnote{Ao fecharmos os olhos ao sol, podemos pensar em elementos que nos remetam à ideia de calor, mas que não o são.} \\
Mas abre os olhos e vê o Sol, \\
E já não pode pensar em nada, \\
Porque a luz do Sol vale mais que os pensamentos \\
De todos os filósofos e de todos os poetas. \\
A luz do Sol não sabe o que faz \\
E por isso não erra e é comum e boa.\footnote{Ideia de que o processo de tomar conhecimento da verdade não pode ir além dos cinco sentidos (pensar nas coisas é pensar o contato com elas -- pensamento abstrato que nos distancia da verdade e da realidade), pois os pensamentos não são as coisas em si. A natureza não pensa e é perfeita. O ser humano pensa e é doente.}

Metafísica? Que metafísica têm aquelas árvores \\
A de serem verdes e copadas e de terem ramos \\
E a de dar fruto na sua hora, o que não nos faz pensar, \\
A nós, que não sabemos dar por elas. \\
Mas que melhor metafísica que a delas, \\
Que é a de não saber para que vivem \\
Nem saber que o não sabem?

«Constituição íntima das coisas»... \\
«Sentido íntimo do Universo»... \\
Tudo isto é falso, tudo isto não quer dizer nada. \\
É incrível que se possa pensar em coisas dessas. \\
É como pensar em razões e fins \\
Quando o começo da manhã está raiando, e pelos lados das árvores \\
Um vago ouro lustroso vai perdendo a escuridão.

Pensar no sentido íntimo das coisas \\
É acrescentado, como pensar na saúde \\
Ou levar um copo à água das fontes.

O único sentido íntimo das coisas \\
É elas não terem sentido íntimo nenhum.\footnote{O único enigma da realidade é de que não existe mistério algum.}

Não acredito em Deus porque nunca o vi.\footnote{Apenas acredita no Deus que é capaz de ver, sentir.} \\
Se ele quisesse que eu acreditasse nele, \\
Sem dúvida que viria falar comigo \\
E entraria pela minha porta dentro \\
Dizendo-me, Aqui estou!

(Isto é talvez ridículo aos ouvidos \\
De quem, por não saber o que é olhar para as coisas, \\
Não compreende quem fala delas \\
Com o modo de falar que reparar para elas ensina.)

Mas se Deus é as flores e as árvores \\
E os montes e sol e o luar, \\
Então acredito nele, \\
Então acredito nele a toda a hora \\
E a minha vida é toda uma oração e uma missa, \\
E uma comunhão com os olhos e pelos ouvidos.\footnote{Se Deus é a soma de todas as coisas existentes, então o eu lírico acredita nele. Note que, no realismo sensorial, a própria noção de natureza é sensorial (as árvores, por exemplo). Há uma relação com a ideia do Baruch de Espinosa de que a realidade é a soma de todas as coisas existentes -- referida com o termo \textit{physis}, dos gregos.}

Mas se Deus é as árvores e as flores \\
E os montes e o luar e o sol, \\
Para que lhe chamo eu Deus? \\
Chamo-lhe flores e árvores e montes e sol e luar; \\
Porque, se ele se fez, para eu o ver, \\
Sol e luar e flores e árvores e montes, \\
Se ele me aparece como sendo árvores e montes \\
E luar e sol e flores, \\
É que ele quer que eu o conheça \\
Como árvores e montes e flores e luar e sol.\footnote{Há uma repetição como forma de evidenciar as diferenças inerentes a cada elemento, em uma poesia prosaica cujo critério rítmico apresenta-se em oposição ao semântico.}

E por isso eu obedeço-lhe, \\
(Que mais sei eu de Deus que Deus de si próprio?), \\
Obedeço-lhe a viver, espontaneamente, \\
Como quem abre os olhos e vê, \\
E chamo-lhe luar e sol e flores e árvores e montes, \\
E amo-o sem pensar nele, \\
E penso-o vendo e ouvindo, \\
E ando com ele a toda a hora.
\end{verse}

\poemtitle{Num meio-dia de fim de Primavera}
\begin{verse}
VIII

Num meio-dia de fim de Primavera \\
Tive um sonho como uma fotografia. \\
Vi Jesus Cristo descer à terra. \\
Veio pela encosta de um monte \\
Tornado outra vez menino, \\
A correr e a rolar-se pela erva \\
E a arrancar flores para as deitar fora \\
E a rir de modo a ouvir-se de longe.

Tinha fugido do céu. \\
Era nosso demais para fingir \\
De segunda pessoa da Trindade. \\
No céu era tudo falso, tudo em desacordo \\
Com flores e árvores e pedras. \\
No céu tinha que estar sempre sério \\
E de vez em quando de se tornar outra vez homem \\
E subir para a cruz, e estar sempre a morrer \\
Com uma coroa toda à roda de espinhos \\
E os pés espetados por um prego com cabeça, \\
E até com um trapo à roda da cintura \\
Como os pretos nas ilustrações. \\
Nem sequer o deixavam ter pai e mãe \\
Como as outras crianças. \\
O seu pai era duas pessoas — \\
Um velho chamado José, que era carpinteiro, \\
E que não era pai dele; \\
E o outro pai era uma pomba estúpida, \\
A única pomba feia do mundo \\
Porque não era do mundo nem era pomba. \\
E a sua mãe não tinha amado antes de o ter.

Não era mulher: era uma mala \\
Em que ele tinha vindo do céu. \\
E queriam que ele, que só nascera da mãe, \\
E nunca tivera pai para amar com respeito, \\
Pregasse a bondade e a justiça!

Um dia que Deus estava a dormir \\
E o Espírito Santo andava a voar, \\
Ele foi à caixa dos milagres e roubou três. \\
Com o primeiro fez que ninguém soubesse que ele tinha fugido. \\
Com o segundo criou-se eternamente humano e menino. \\
Com o terceiro criou um Cristo eternamente na cruz \\
E deixou-o pregado na cruz que há no céu \\
E serve de modelo às outras. \\
Depois fugiu para o Sol \\
E desceu pelo primeiro raio que apanhou. \\
Hoje vive na minha aldeia comigo. \\
É uma criança bonita de riso e natural. \\
Limpa o nariz ao braço direito, \\
Chapinha nas poças de água, \\
Colhe as flores e gosta delas e esquece-as. \\
Atira pedras aos burros, \\
Rouba a fruta dos pomares \\
E foge a chorar e a gritar dos cães. \\
E, porque sabe que elas não gostam \\
E que toda a gente acha graça, \\
Corre atrás das raparigas \\
Que vão em ranchos pelas estradas \\
Com as bilhas às cabeças \\
E levanta-lhes as saias.

A mim ensinou-me tudo. \\
Ensinou-me a olhar para as coisas. \\
Aponta-me todas as coisas que há nas flores. \\
Mostra-me como as pedras são engraçadas \\
Quando a gente as tem na mão \\
E olha devagar para elas.

Diz-me muito mal de Deus. \\
Diz que ele é um velho estúpido e doente, \\
Sempre a escarrar no chão \\
E a dizer indecências. \\
A Virgem Maria leva as tardes da eternidade a fazer meia. \\
E o Espírito Santo coça-se com o bico \\
E empoleira-se nas cadeiras e suja-as. \\
Tudo no céu é estúpido como a Igreja Católica. \\
Diz-me que Deus não percebe nada \\
Das coisas que criou — \\
«Se é que ele as criou, do que duvido.» — \\
«Ele diz, por exemplo, que os seres cantam a sua glória, \\
Mas os seres não cantam nada. \\
Se cantassem seriam cantores. \\
Os seres existem e mais nada, \\
E por isso se chamam seres.» \\
E depois, cansado de dizer mal de Deus, \\
O Menino Jesus adormece nos meus braços \\
E eu levo-o ao colo para casa.

……

Ele mora comigo na minha casa a meio do outeiro. \\
Ele é a Eterna Criança, o deus que faltava. \\
Ele é o humano que é natural, \\
Ele é o divino que sorri e que brinca. \\
E por isso é que eu sei com toda a certeza \\
Que ele é o Menino Jesus verdadeiro.

E a criança tão humana que é divina \\
É esta minha quotidiana vida de poeta, \\
E é porque ele anda sempre comigo que eu sou poeta sempre. \\
E que o meu mínimo olhar \\
Me enche de sensação, \\
E o mais pequeno som, seja do que for, \\
Parece falar comigo.

A Criança Nova que habita onde vivo \\
Dá-me uma mão a mim \\
E a outra a tudo que existe \\
E assim vamos os três pelo caminho que houver, \\
Saltando e cantando e rindo \\
E gozando o nosso segredo comum \\
Que é o de saber por toda a parte \\
Que não há mistério no mundo \\
E que tudo vale a pena.

A Criança Eterna acompanha-me sempre. \\
A direcção do meu olhar é o seu dedo apontando. \\
O meu ouvido atento alegremente a todos os sons \\
São as cócegas que ele me faz, brincando, nas orelhas.

Damo-nos tão bem um com o outro \\
Na companhia de tudo \\
Que nunca pensamos um no outro, \\
Mas vivemos juntos e dois \\
Com um acordo íntimo \\
Como a mão direita e a esquerda. \\

Ao anoitecer brincamos as cinco pedrinhas \\
No degrau da porta de casa, \\
Graves como convém a um deus e a um poeta, \\
E como se cada pedra \\
Fosse todo um universo \\
E fosse por isso um grande perigo para ela \\
Deixá-la cair no chão.

Depois eu conto-lhe histórias das coisas só dos homens \\
E ele sorri, porque tudo é incrível. \\
Ri dos reis e dos que não são reis, \\
E tem pena de ouvir falar das guerras, \\
E dos comércios, e dos navios \\
Que ficam fumo no ar dos altos mares. \\
Porque ele sabe que tudo isso falta àquela verdade \\
Que uma flor tem ao florescer \\
E que anda com a luz do Sol \\
A variar os montes e os vales \\
E a fazer doer aos olhos os muros caiados.

Depois ele adormece e eu deito-o. \\
Levo-o ao colo para dentro de casa \\
E deito-o, despindo-o lentamente \\
E como seguindo um ritual muito limpo \\
E todo materno até ele estar nu.

Ele dorme dentro da minha alma \\
E às vezes acorda de noite \\
E brinca com os meus sonhos. \\
Vira uns de pernas para o ar, \\
Põe uns em cima dos outros \\
E bate as palmas sozinho \\
Sorrindo para o meu sono.

……

Quando eu morrer, filhinho, \\
Seja eu a criança, o mais pequeno. \\
Pega-me tu ao colo \\
E leva-me para dentro da tua casa. \\
Despe o meu ser cansado e humano \\
E deita-me na tua cama. \\
E conta-me histórias, caso eu acorde, \\
Para eu tornar a adormecer. \\
E dá-me sonhos teus para eu brincar \\
Até que nasça qualquer dia \\
Que tu sabes qual é.

……

Esta é a história do meu Menino Jesus. \\
Por que razão que se perceba \\
Não há-de ser ela mais verdadeira \\
Que tudo quanto os filósofos pensam \\
E tudo quanto as religiões ensinam?
\end{verse}

É claro o intuito de chocar o leitor. Há grande ênfase na figura de Jesus (Segunda Trindade), então caracterizado pelo lado humano (heresia era o termo utilizado para se referir aos que não acreditavam na natureza divina de Jesus).

Para o eu lírico, o Jesus importante, verdadeiro, é um menino que se comporta como um menino comum. Notamos a ideia do \textit{panteísmo materialista}: crença de que tudo é divino. Note que mesmo no budismo ou no hinduísmo ainda existe certa hierarquia, como observado no xintoísmo (as cidades, próximas ao litoral, e as cidades sagradas). Existe uma projeção do sagrado nas coisas. Panteísmo materialista à medida em que nada é sagrado pois, nesse caso, também deveria existir o profano (hierarquia). Não existe o sagrado, e o essencial das coisas está nas coisas elas mesmas.

Note a oposição entre Fernando Pessoa, que acreditava no misticismo, e as ideias de Alberto Caeiro, extremamente materialista -- desdobramento do realismo sensorial, aplicado na religião. Notamos uma religiosidade antirreligiosa, que subverte a própria ideia de religião.

\poemtitle{Li hoje quase duas páginas}
\begin{verse}
XXVIII

Li hoje quase duas páginas \\
Do livro dum poeta místico, \\
E ri como quem tem chorado muito.

Os poetas místicos são filósofos doentes,\footnote{Os poetas são também homens doidos, assim como doentes (crítica, nas estrofes seguintes, à atribuição de ações ou características a esses seres - prosopopeia). Críticas à utilização da linguagem conotativa, o que faz dos poetas filósofos doentes (descrição da realidade como ela não é).} \\
E os filósofos são homens doidos.\footnote{Filósofos são homens doidos (pensam), e por essa razão se afastam, abstraem a realidade (substituição por uma representação dela).}

Porque os poetas místicos dizem que as flores sentem \\
E dizem que as pedras têm alma \\
E que os rios têm êxtases ao luar.

Mas as flores, se sentissem, não eram flores, \\
Eram gente; \\
E se as pedras tivessem alma, eram coisas vivas, não eram pedras; \\
E se os rios tivessem êxtases ao luar, \\
Os rios seriam homens doentes.

É preciso não saber o que são flores e pedras e rios \\
Para falar dos sentimentos deles. \\
Falar da alma das pedras, das flores, dos rios, \\
É falar de si próprio e dos seus falsos pensamentos. \\
Graças a Deus que as pedras são só pedras, \\
E que os rios não são senão rios, \\
E que as flores são apenas flores.

Por mim, escrevo a prosa dos meus versos \\
E fico contente,\footnote{Sem maiores preocupações com o ritmo (verso prosaico, junto da negação da linguagem conotativa, geralmente associada à linguagem poética).} \\
Porque sei que compreendo a Natureza por fora; \\
E não a compreendo por dentro \\
Porque a Natureza não tem dentro; \\
Senão não era a Natureza.
\end{verse}

Notamos um pensamento contraditório, à medida em que, se não houvessem, seria um poeta racional, e não sensorial. Identificamos também uma crítica à ideia de que a linguagem poética é mais simbólica do que a linguagem do cotidiano (crítica antipoética).

Trata-se de uma poesia antirreligiosa com uma linguagem antipoética, em influência do realismo sensorial, apresentado como uma espécie de filosofia (desdobramentos desse princípio). Construção de uma filosofia antifilosófica, ainda que tenha algo de filosófico por se constituir de um sistema de pensamento.

Identificamos, também, relações com o empirismo e David Hume (lembre-se de Immanuel Kant e \textit{Crítica da razão pura}).

Uma característica presente na obra de Alberto Caeiro, e também presente na obra de Nietzsche, é a utilização de aforismos, um pensamento culto encerrado em si mesmo, diferente do pensamento metódico característico da Filosofia (termo diferente em Portugal). Assim, cada parágrafo representa uma ideia fechada em si mesma, de caráter anti-metódico, com muitas simbologias e linguagem conotativa. Alberto Caeiro leva tais ideias às últimas consequências. Não podemos esquecer que Fernando Pessoa era um homem extremamente culto e, possivelmente, estrutura tais pensamentos de maneira intencional.

\subsection{Ricardo Reis}

Algum texto aqui.

\poemtitle{Ouvi contar que outrora, quando a Pérsia}
\begin{verse}
Ouvi contar que outrora, quando a Pérsia \\
Tinha não sei qual guerra, \\
Quando a invasão ardia na Cidade \\
E as mulheres gritavam, \\
Dois jogadores de xadrez jogavam \\
O seu jogo contínuo.

À sombra de ampla árvore fitavam \\
O tabuleiro antigo, \\
E, ao lado de cada um, esperando os seus \\
Momentos mais folgados, \\
Quando havia movido a pedra, e agora \\
Esperava o adversário, \\
Um púcaro com vinho refrescava \\
Sobriamente a sua sede.

Ardiam casas, saqueadas eram \\
As arcas e as paredes, \\
Violadas, as mulheres eram postas \\
Contra os muros caídos, \\
Traspassadas de lanças, as crianças \\
Eram sangue nas ruas... \\
Mas onde estavam, perto da cidade, \\
E longe do seu ruído, \\
Os jogadores de xadrez jogavam \\
O jogo do xadrez.

Inda que nas mensagens do ermo vento \\
Lhes viessem os gritos, \\
E, ao reflectir, soubessem desde a alma \\
Que por certo as mulheres \\
E as tenras filhas violadas eram \\
Nessa distância próxima, \\
Inda que, no momento que o pensavam, \\
Uma sombra ligeira \\
Lhes passasse na fronte alheada e vaga, \\
Breve seus olhos calmos \\
Volviam sua atenta confiança \\
Ao tabuleiro velho.

Quando o rei de marfim está em perigo, \\
Que importa a carne e o osso \\
Das irmãs e das mães e das crianças?

Quando a torre não cobre \\
A retirada da rainha branca, \\
O saque pouco importa. \\
E quando a mão confiada leva o xeque \\
Ao rei do adversário, \\
Pouco pesa na alma que lá longe \\
Estejam morrendo filhos.

Mesmo que, de repente, sobre o muro \\
Surja a sanhuda face \\
Dum guerreiro invasor, e breve deva \\
Em sangue ali cair \\
O jogador solene de xadrez, \\
O momento antes desse \\
(É ainda dado ao cálculo dum lance \\
Pra a efeito horas depois) \\
É ainda entregue ao jogo predilecto \\
Dos grandes indiferentes.

Caiam cidades, sofram povos, cesse \\
A liberdade e a vida, \\
Os haveres tranquilos e avitos \\
Ardem e que se arranquem, \\
Mas quando a guerra os jogos interrompa, \\
Esteja o rei sem xeque, \\
E o de marfim peão mais avançado \\
Pronto a comprar a torre.

Meus irmãos em amarmos Epicuro \\
E o entendermos mais \\
De acordo com nós-próprios que com ele, \\
Aprendamos na história \\
Dos calmos jogadores de xadrez \\
Como passar a vida.

Tudo o que é sério pouco nos importe, \\
O grave pouco pese, \\
O natural impulsa dos instintos \\
Que ceda ao inútil gozo \\
(Sob a sombra tranquila do arvoredo) \\
De jogar um bom jogo.

O que levamos desta vida inútil \\
Tanto vale se é \\
A glória; a fama, o amor, a ciência, a vida, \\
Como se fosse apenas \\
A memória de um jogo bem jogado \\
E uma partida ganha \\
A um jogador melhor.

A glória pesa como um fardo rico, \\
A fama como a febre, \\
O amor cansa, porque é a sério e busca, \\
A ciência nunca encontra, \\
E a vida passa e dói porque o conhece...

O jogo do xadrez \\
Prende a alma toda, mas, perdido, pouco \\
Pesa, pois não é nada.

Ah! sob as sombras que sem querer nos amam, \\
Com um púcaro de vinho \\
Ao lado, e atentos só à inútil faina \\
Do jogo do xadrez, \\
Mesmo que o jogo seja apenas sonho \\
E não haja parceiro, \\
Imitemos os persas desta história, \\
E, enquanto lá por fora, \\
Ou perto ou longe, a guerra e a pátria e a vida \\
Chamam por nós, deixemos \\
Que em vão nos chamem, cada um de nós \\
Sob as sombras amigas \\
Sonhando, ele os parceiros, e o xadrez \\
A sua indiferença.
\end{verse}

Temos um tema clássico, que remonta ao período da Pérsia, que descreve dois indivíduos que jogam xadrez e, simultaneamente, a cidade é invadida (filhos mortos, esposas violadas). Os enxadristas, no entanto, permanecem despreocupados (note a forma como o xadrez não leva a efeitos práticos, diferentemente da guerra -- mesmo que o jogo seja apenas um sonho, e não haja um parceiro). Ainda assim, precisamos imitar o jogo de xadrez.

Notamos uma inversão de períodos sintáticos (hipérbole), como forma de simular como seria em português a sintaxe latina, a qual não possui ordem sintática normal, padronizada. Existe um aspecto sintático mais plástico.

Os versos são metrificados: os longos em decassílabos heroicos, e os curtos em hexassílabos, que se combinam juntos. A linguagem é mais culta e erudita, quando comparada à linguagem utilizada por Alberto Caeiro. Também está presente a ideia do epicurismo e a concentração no momento presente (no que temos, e não no que gostaríamos de ter).

Existe uma concepção do mundo como espetáculo (\textit{especulo} é o local para onde olhamos -- ideia de algo para ser visto). O mundo como algo para ser contemplado, e não como algo do qual deveríamos participar (pensamento estoico, para o qual - estado de espírito da ataraxia, impassibilidade - de que a felicidade é um estado de paz de espírito, para o qual nada do que ocorre fora do indivíduo o afeta).

O eu lírico propõe que devemos encarar a realidade como um grande tabuleiro de xadrez, cuja finalidade última não envolve aplicações práticas. A paz interior constitui um fim em si mesmo, um caminho para atingir esse fim.

Classicismo: a poesia clássica (grega e romana, sobretudo) tomada como modelo. Não há rimas, ainda que haja metrificação, em consonância com os modelos clássicos (tal aspecto surgiria apenas no Trovadorismo). A rima era associada à poesia popular, do povo. Utilização de hipérbatos (simulação da sintaxe romana, latina), versos metrificados (medido) e brancos. Além disso, há uma temática histórica igualmente classicista e frequentemente mitológica.

\poemtitle{Coroai-me de rosas}
\begin{verse}
Coroai-me de rosas, \\
Coroai-me em verdade \\
De rosas — \\
Rosas que se apagam \\
Em fronte\footnote{Na iminência de.} a apagar-se \\
Tão cedo! \\
Coroai-me de rosas \\
E de folhas breves. \\
E basta.
\end{verse}

Há uma alternância entre versos pentassílabos e dissílabos. O eu lírico deseja ser coroado de rosas pois as flores estão prestes a morrer, murchar, perder o brilho. ``E basta'' se apresenta no sentido do \textit{carpe diem}, pois não importa se as rosas murcharão. Há uma influência das correntes filosóficas gregas do epicurismo e estoicismo.