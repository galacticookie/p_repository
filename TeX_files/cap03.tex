\chapterimage{chapter_head_2.pdf}

\chapter{Introdução}
 
A gênese do Classicismo remonta ao início da Idade Moderna, em meio ao processo de transição do \textbf{feudalismo} para o \textbf{capitalismo mercantil} (primeira fase). O \textbf{Renascimento} é o marco inicial do Classicismo, compreendendo a retomada dos valores estéticos e culturais da Antiguidade Clássica (greco-romana).

A pintura \textit{Primavera} representada na página anterior, do italiano Sandro Botticelli (1445-1510), sintetiza importantes valores do período (no capítulo destinado ao Arcadismo no Brasil encontra-se uma análise mais detalhada do quadro). Entre esses valores, destacam-se:

\begin{itemize}
\item Harmonia
\item Equilíbrio
\item Proporção e simetria (nas artes, refletidas por meio do verismo anatômico)
\item Racionalidade
\item Antropocentrismo
\item Hedodnismo (no sentido da contemplação estética — o corpo humano como algo belo)
\end{itemize}
    
Também é válido mencionar, como curiosidade, a técnica de escorço utilizada na escultura \textit{Davi}, representada na página seguinte, de Michelangelo (1475-1564), como forma de produzir profundidade, extensão na visão do espectador corretamente posicionado (é extremamente difícil, se não impossível, encontrar alguma fotografia que registre apropriadamente tal perspectiva).

\begin{definition}[Clássico]
Em referência à obra que supera a barreira do tempo, mantendo o interesse das gerações posteriores e servindo de modelo dos valores universais.
\end{definition}

\begin{definition}[Emulação]
No sentido de imitação, competição. 
\end{definition}

\begin{definition}[Lugar-comum]
Tema que se repete na tradição literária.
\end{definition}

Uma das características mais presentes no Classicismo é a emulação dos clássicos da cultura greco-romana, incorporando os lugares-comuns presentes nessas produções. Note, nesse sentido, que não necessariamente se buscava \textbf{originalidade}\footnote{Cabe mencionar que a ideia de um texto autoral, reflexo das aspirações e pensamentos de seu autor, assumiria maior importância apenas com o Romantismo. No período do Classicismo, uma obra memorável era aquela capaz de emular os clássicos e se apropriar dos lugares-comuns com excelência — por isso a ideia de competição na definição do conceito acima.}.

Por fim, é importante citar o \textbf{Maneirismo}, uma transição ocorrida dentro do Classicismo identificada apenas em certos autores. É caracterizado pelo abrandamento do racionalismo classicista, à medida em que o amor e a condição humana superam a razão. Nota-se a recorrente utilização de antíteses e paradoxos. Antecipara o Barroco, estilo de época descrito na parte seguinte.

\textbf{Carece de mais informações.}