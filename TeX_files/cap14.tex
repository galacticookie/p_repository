\chapterimage{chapter_head_2.pdf}

\chapter{Realismo em Portugal}

Em primeira análise, é importante ressaltar o atraso de Portugal em relação aos países industrializados — processo que ficará ainda mais evidente com o Modernismo. A Geração de 70, também conhecida como Geração de Coimbra, corresponde a um movimento de jovens reformistas da Universidade de Coimbra. Questão Coimbrã, ou Querela do Bom Senso e do Bom Gosto, em 1868. Conferências do Cassino Lisboeta.

O Romantismo e suas críticas: \textit{Os miseráveis}, de Victor Hugo (críticas e idealizações).

Produto cultural da Segunda Revolução Industrial (surgimento de novas tendências para representar a realidade). Portugal não era um país industrializado (predomínio de indústrias apenas em certas localidades), e ainda possuía muitas estruturas do capitalismo mercantil à deriva do capitalismo industrial (Portugal consumia produtos industrializados, sobretudo ingleses); uma nação ainda presa à ideia do Império Colonial.

Os jovens universitários que consumiam outras culturas começam a considerar Portugal uma nação atrasada, não só em razão de sua economia, mas de sua visão tradicionalista. Destacavam a forte presença da Igreja — posteriormente apoiada por Salazar — e do catolicismo, de maneira geral. Lembre-se de que Portugal se estabelecera como Estado-nação por meio da guerra contra o islamismo, e posterior apoio à Contrarreforma (tal acontecimento também se fará presente na obra de Fernando Pessoa). É curioso notar, no entanto, a simplificação trazida do pensamento dos jovens a toda a problemática. Nesse sentido, citamos, novamente, Max Weber, para o qual a situação de Portugal possui maior relação com o catolicismo do que com a presença da Inglaterra (os jovens davam ênfase ao processo contrário).

Pobreza e atraso de Portugal (graus de barbárie).

Em meio a necessidade de uma literatura mais crítica e atual, os escritores recorrem ao Realismo. Antônio Feliciano de Castilho, por exemplo, poeta oficial de Portugal — de idade avançada e cego — escreve um artigo no qual critica o Realismo pela falta de bom senso e bom gosto, uma vez que retratava de aspectos outrora omitidos pelo Romantismo, pois eram considerados imorais. Em resposta, o líder da Geração de 70, Antero de Quental, ataca tanto o Romantismo quanto Castilho (figura respeitável), acusado de infantilidade. O conflito gerara grande repercussão, e se outrora Castilho desejava sepultar o Realismo, acabou o divulgando por todo o país (o Realismo ainda não era tão conhecido).

Com tal atenção, os realistas realizam uma série de conferências de apresentação do Realismo — por meio de manifestações artísticas como o teatro. Exemplo notável são as Conferências do Cassino Lisbonense.

Francês: edificar. Moral , existencial e pedagógica (literatura). Teatro de costumes. Com o Realismo, a literatura adquire uma função crítica (sociólogos críticos de literatura), então perdida no século XXI (linguagem e códigos de maior importância do que as ciências humanas, necessidade de algo a mais). Concorrência com outras formas de entretenimento.

\section{Eça de Queirós}

Falaremos, agora, de uma importante figura integrante da Geração de 70: Eça de Queirós. Tivera uma primeira fase romântica, na qual publicara, por exemplo, \textit{O esqueleto}, folhetim romântico, assim como outras obras. Segue-se uma segunda fase, que transitava entre o Realismo e o Naturalismo — contra o conservadorismo português, com a ideia do reformismo como movimentação das críticas —, e uma terceira fase, referida como a fase madura. Focaremos nossos estudos na segunda fase.

\subsection{O crime do padre Amaro}

Considerado o primeiro romance realista do autor, apresenta considerável natureza sexual (representada por meio de indivíduos com personalidade sanguínea). No início da narrativa, a personagem é transferida na função de padre para a cidade de Leiria. Conselhos do cônego Dias. Na pensão de Dona Joaneira conhece Amélia, sua filha, e noiva de um jornalista. Dona Joaneira ca cama com o cônego. Confissão: desfazer as amarras morais. Acusações do noivo de Amélia, forçando-o a se mudar. Amélia engravida (o noivo se recusa a assumir o filho) e se transfere para outra cidade sob os cuidados de Josefa (se arrepende do caso com o padre).

Em decorrência de problemas durante o parto, o padre leva o recém-nascido para uma tecedora de anjos (aborto pós-parto). A busca pelo sórdido da vida cotidiana, próximo da realidade. Amélia falece e o bebê é morto. O padre Amaro e cônego Dias, anos depois, ainda continuam com os casos.

\subsection{O primo Basílio}

Novamente pertencente à 2ª fase — realista-naturalista — de Eça de Queirós. No início da obra, somos apresentados a Luísa, de temperamento sanguíneo, recém-casada com Jorge. No contexto de um marido ausente por ocupações de seu trabalho, chega na cidade o primo de Luísa, Basílio, seu primeiro amor e com o qual perdeu a virgindade. Basílio enriquecera no Brasil com golpes e fugira para Portugal. Posteriormente, Jorge retorna, e a empregada do casal, Juliana, ao encontrar as correspondências entre Luísa e Basílio, começa a ameaçar, chantagear a mulher. Há uma tentativa frustrada de fuga dos amantes, pois Basílio desistira. Sebastião, um amigo de Luísa, ameaça Juliana para que parasse com as intimidações. A empregada, então, morre em decorrência de um aneurisma.

Com o acontecimento, Luísa mantivera as cartas, mas adoece como resposta a um sentimento de culpa pelo que ocorrera, e acaba falecendo (ainda que Jorge tivesse descoberto a traição e a perdoado).

É importante ressaltar o fato de que o autor não oculta os momentos e movimentos íntimos entre as personagens. Nesse sentido, é curioso mencionar que o romance retrata a primeira cena de cunilíngua no meio literário, fato que suscitara uma polêmica internacional (demasiadamente cru, explícito para a época).

\subsection{Os Maias}

No início da obra somos apresentados a Afonso de Maia, pai de Pedro da Maia (indivíduo tímido e sensível), que se apaixona por Maria Monforte (filha de um negreiro, fortuna associada ao tráfico). Pedro (suicídio) tivera dois filhos com Maria (inclusive, uma garota levada), que posteriormente fugira com um amante. Jorge da Maia se torna um médico e é o orgulho do avô. Conhece Maria Eduarda, então casada com um brasileiro rico que não mora no país. Maria Eduarda era irmã de Jorge, que tentara ocultar o fato. Afonso descobre e morre. Maria resolve viver a vida com o seu amante, e Jorge segue a sua vida.

\subsection{A relíquia}

Acompanhamos a história de Raposão, um órfão criado por uma tia católica fervorosa (carola). Em casa, é Julião Raposo, e fora, Raposão (famoso nos bordeis de Lisboa; Esmeralda, vida dupla).

Ao se formar, inventa a desculpa para a tia de que viajaria para a Terra Santa com o intuito de buscar uma relíquia (o objetivo, obviamente, era outro). Durante a viagem, conhece um bordel em Alexandria onde conhece Mary, que dá uma camisola ao rapaz.

Em um momento no qual está fraco a alucinando, acha que está no dia da crucificação de Jesus (sugestão de que a ressurreição era uma grande farsa). O núcleo da religião cristã católica. Raposão cai de cama e acorda no dia da volta. Prepara uma coroa para entregar à tia, mas dá a camisola no lugar. É expulso de casa e perde toda a herança.

O clássico \textit{A vida de Jesus}, do francês Ernest Renan, tornara-se uma grande influência para Eça de Queirós, então o escritor mais renomado da literatura portuguesa (tornou-se aquilo que criticava — lembre-se da Geração de 70).