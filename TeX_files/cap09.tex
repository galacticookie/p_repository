\chapterimage{chapter_head_2.pdf}

\chapter{Arcadismo no Brasil}

A Arcadia mineira era localizada em Vila Rica (atual Ouro Preto), sobretudo em razão do Ciclo do Ouro e, mais especificamente, durante sua decadência (nesse período, a prata chegou a ser mais valiosa do que o próprio ouro, dada a tamanha oferta). Surgimento de uma elite extremamente rica, cujos filhos estudavam na Universidade de Coimbra e, posteriormente, retornavam ao Brasil, onde atuavam como juízes e em outros cargos. Sofriam influência de ideias iluministas, refletidas na Conjuração Mineira.

A Inconfidência Mineira tivera claras influências iluministas, e contara com a participação de diversos poetas árcades. Ocorrerera no contexto do declínio da produção de ouro e da criação das casas de fundição - as quais proporcionaram uma maior cobrança de impostos. O grupo fora delatado por José Silvério dos Reis. É curioso mencionar que o atual Museu da Inconfidência era uma prisão, onde aqueles que cometeram delitos esperavam o julgamento (destaque para penas mais duras, rígidas, após a Conjuração).

Ouro Preto foi o principal centro difusor do Neoclassicismo, por meio de jovens que estudavam em Coimbra e retornavam com ideias iluministas (muitos participaram da Inconfidência Mineira).

\section{Cláudio Manuel da Costa}

Foi um dos poetas árcades mais velhos. Interessou-se pela poesia por meio do Barroco, fato presente em sua obra por meio do \textbf{barroquismo} - sobrevivência de elementos barrocos na poesia árcade o autor.

\poemtitle{Leia a posteridade, ó pátrio Rio}
\begin{verse}
Leia a posteridade\footnote{Gerações futuras.}, ó pátrio Rio\footnote{Ribeirão do Carmo, que corta a cidade de Mariana.}, \\
Em meus versos teu nome celebrado, \\
Porque vejas uma hora despertado \\
O sono vil do esquecimento frio:
				
Não vês nas tuas margens o sombrio, \\
Fresco assento de um álamo\footnote{Árvore encontrada no hemisfério norte.} copado; \\
Não vês Ninfa cantar, pastar o gado \\
Na tarde clara do calmoso estio.
				
Turvo banhando as pálidas areias \\
Nas porções do riquíssimo tesouro \\
O vasto campo da ambição recreias.
				
Que de seus raios o Planeta louro\footnote{Sol. No sistema astronômico geocêntrico, o planeta dourado. Utilização de perífrase, característica comum ao Barroco.}, \\
Enriquecendo o influxo em tuas veias, \\
Quanto em chamas fecunda, brota em ouro.
\end{verse}

No poema, o autor descreve o pátrio Rio em referência ao Ribeirão do Carmo, que corta a cidade de Mariana - a primeira cidade, capital e arquidiocese de Minas Gerais. Menciona o ouro de aluvião tão procurado em suas margens, então degradadas e sem a proteção de matas ciliares.

Descreve que, para que o rio não seja esquecido pela posteridade, deve ser registrado. Lugar-comum da \textbf{imortalidade da memória} (herói de Homero) e \textit{kleos} (a poesia permitia que os indivíduos se lembrassem dos acontecimentos).
Cláudio descreve que, nas margens do Ribeirão do Carmo, não há álamos que forneçam sombra e descanso. Não há ninfas cantando ou gado pastando da tarde clara do inverno. Também não há pasto, justamente pela atividade econômica desenvolvida na região. \textbf{As convenções da poesia árcade europeia não se enquadram no cenário de Mariana/Vila Rica}.

Há a negação do lugar-comum \textit{locus amoneus}, à medida em que é necessário se afastar das convenções do Arcadismo (clima mediterrâneo) para descrever a região. Tal processo, referido como \textbf{nativismo}, ocorre quando o poeta fala sobre a realidade local em vez de, por exemplo, de um campo idealizado (cenário bucólico). Incorporação da cor local. O Ribeirção do Carmo não pode ser descrito por meio do \textit{locus amoenus}, em paralelo com o Tejo de Bocage. Águas cristalinas em paralelo com tons turvos.

Note que, embora o Ribeirão do Carmo não seja tão cristalino ou amplo quanto o rio europeu, oculta um riquíssimo tesouro - ouro - que instiga a ambição humana. As veias do rio são então fertilizadas pelos raios de sol, gerando como o fruto o ouro (semelhante ao trigo).

Valorização dos elementos locais (poesia bucólica diferente da europeia).

Revoltas nativistas: após décadas de colonização, surge gradativamente um sentimento de identificação ao contexto social, político e cultural do Brasil, distanciando-se de Portugal e suscitando os desejos de autonomia. Em paralelo, na Literatura, surge o nativismo, expressão de um desejo de poesias que reflitam a realidade local (nativismo político e literário).

\section{Tomás Antônio Gonzaga}

Neto de portugueses, estudou leis em Coimbra. Ao retornar para Vila Rica, começa a atuar como juiz e, posteriormente, ainda participaria da Inconfidência mineira. Com a Devassa ocorrida após tal incidente, descobriu-se um grande número de bens investidos em roupas, perucas e maquiagem.

Apaixonara-se por Maria Doroteia. Posteriormente, com seus envolvimentos na Inconfidência, é exilado na África e enriquece com o tráfico de escravos.

\subsection{Marília de Dirceu}

Dirceu fora o pseudônimo latino incorporado por Gonzaga, o qual se apaixonara por uma \textit{nise}. Note que parte dos poemas fora escrita ainda em Portugal, quando o autor ainda não havia conhecido sua amada. Nesse sentido, há importante pontuar algumas diferenças entre Doroteia, morena, e Marília, loira - esta baseada em Laura, de Petrarca.

A obra é dividia em duas partes. Após a morte do escritor, foram descobertos poemas inéditos, anexados à \textit{Marília de Dirceu} e constituindo a terceira parte. De maneira geral, não eram dados títulos aos poemas (lembre-se do Barroco, com títulos explicativos dados pelos indivíduos envolvidos em sua compilação).

\poemtitle{Lira I (Parte I)}
\begin{verse}[\versewidth]
Eu, Marília, não sou algum vaqueiro, \\
Que viva de guardar alheio gado\footnote{O eu lírico é vaqueiro de seu próprio gado.}; \\
De tosco trato\footnote{Pelo contrário, é educado e sabe se expressar de forma elegante e fina.}, d’ expressões grosseiro, \\
Dos frios gelos, e dos sóis queimado\footnote{Não trabalha exposto ao calor ou frio extremo.}. \\
Tenho próprio casal\footnote{Pequena propriedade rual. Neste caso, lhe fornece descanso, e produtos como vinhos, legumes, frutas, azeite e leite. Assim, embora não seja rico, tem todas as suas necessidades atendidas (\textit{aurea mediocritas}).}, e nele assisto\footnote{Tem condições de decidir quando trabalhará.}; \\
Dá-me vinho, legume, fruta, azeite; \\
Das brancas ovelhinhas tiro o leite, \\
E mais as finas lãs, de que me visto. \\
\hspace{2em} Graças, Marília bela, \\
\hspace{2em} Graças à minha Estrela!
					
Eu vi o meu semblante numa fonte, \\
Dos anos inda não está cortado\footnote{Ainda não está velho. Além disso, se destaca entre os pastores (respeito ao poder do cajado).}: \\
Os pastores, que habitam este monte, \\
Com tal destreza toco a sanfoninha, \\
Que inveja até me tem o próprio Alceste\footnote{Apelido de Apolo, que então sente inveja da destreza com a qual o eu lírico toca sua sanfoninha, e de sua voz celeste, que enuncia apenas canções de própria autoria.}: \\
Ao som dela concerto a voz celeste; \\
Nem canto letra, que não seja minha, \\
\hspace{2em} Graças, Marília bela, \\
\hspace{2em} Graças à minha Estrela\footnote{Concepção astrológica.}!
					
Mas tendo tantos dotes da ventura, \\
Só apreço lhes dou, gentil Pastora, \\
Depois que teu afeto me segura, \\
Que queres do que tenho ser senhora. \\
É bom, minha Marília, é bom ser dono \\
De um rebanho, que cubra monte, e prado; \\
Porém, gentil Pastora, o teu agrado \\
Vale mais q’um rebanho, e mais q’um trono. \\
\hspace{2em} Graças, Marília bela, \\
\hspace{2em} Graças à minha Estrela!
					
Os teus olhos espalham luz divina, \\
A quem a luz do Sol em vão se atreve: \\
Papoula, ou rosa delicada, e fina, \\
Te cobre as faces, que são cor de neve. \\
Os teus cabelos são uns fios d’ouro; \\
Teu lindo corpo bálsamos vapora. \\
Ah! Não, não fez o Céu, gentil Pastora, \\
Para glória de Amor igual tesouro. \\
\hspace{2em} Graças, Marília bela, \\
\hspace{2em} Graças à minha Estrela!
					
Leve-me a sementeira muito embora \\
O rio sobre os campos levantado: \\
Acabe, acabe a peste matadora, \\
Sem deixar uma rês, o nédio gado. \\
Já destes bens, Marília, não preciso: \\
Nem me cega a paixão, que o mundo arrasta; \\
Para viver feliz, Marília, basta \\
Que os olhos movas, e me dês um riso. \\
\hspace{2em} Graças, Marília bela, \\
\hspace{2em} Graças à minha Estrela!
					
Irás a divertir-te na floresta, \\
Sustentada, Marília, no meu braço; \\
Ali descansarei a quente sesta, \\
Dormindo um leve sono em teu regaço: \\
Enquanto a luta jogam os Pastores, \\
E emparelhados correm nas campinas, \\
Toucarei teus cabelos de boninas, \\
Nos troncos gravarei os teus louvores. \\
\hspace{2em} Graças, Marília bela, \\
\hspace{2em} Graças à minha Estrela!
					
Depois de nos ferir a mão da morte, \\
Ou seja neste monte, ou noutra serra, \\
Nossos corpos terão, terão a sorte \\
De consumir os dois a mesma terra. \\
Na campa, rodeada de ciprestes, \\
Lerão estas palavras os Pastores: \\
“Quem quiser ser feliz nos seus amores, \\
Siga os exemplos, que nos deram estes.” \\
\hspace{2em} Graças, Marília bela, \\
\hspace{2em} Graças à minha Estrela! \\
\end{verse}

Lira é uma forma do gênero lírico que descreve os sentimentos do eu lírico. Nesse caso, trata-se de um vaqueiro que afirma que nenhum de seus bens ou dons possuem sentido se não compartilhados com a nise. Dessa forma, o amor de Marília vale mais do que qualquer gado ou trono (poder) no mundo. Seu olhar é mais brilhante do que o próprio Sol. As estrofes ainda apresentas caráter petrarquista.

Retornando ao pré-romantismo, temos uma exploração de experiência pessoal do autor como matéria da poesia. Em certas liras descreve, por exemplo, o caminho que realizava desde sua casa até a de Doroteia (influência direta de elementos biográficos na composição da obra).

Um aspecto interessante a ser notado é que a primeira parte da obra fora escrita antes da condenação de Gonzaga pelo envolvimento na Inconfidência, e a segunda, após (uma curiosidade é de que, na época, a prisão era o local de espera pelo julgamento). Assim sendo, temos as seguintes diferenças:

\begin{enumerate}
\item \textbf{Parte I}: mais convencionalmente árcade, predominância de um tom idílico. Em um primeiro momento, o eu lírico percebe a família presente na natureza, imaginando ainda o seu futuro ao lado de Marília. Exemplo notável é reproduzido abaixo.
\poemtitle{Lira XIX (Parte I)}
\begin{verse}
Enquanto pasta alegre o manso gado, \\
Minha bela Marília, nos sentemos \\
À sombra deste cedro levantado. \\
\hspace{2em} Um pouco meditemos \\
\hspace{2em} Na regular beleza, \\
Que em tudo quanto vive, nos descobre \\
\hspace{2em} A sábia natureza.
						
Atende, como aquela vaca preta \\
O novilhinho seu dos mais separa, \\
E o lambe, enquanto chupa a lisa teta. \\
\hspace{2em} Atende mais, ó cara, \\
\hspace{2em} Como a ruiva cadela \\
Suporta que lhe morda o filho o corpo, \\
\hspace{2em} E salte em cima dela.
						
Repara, como cheia de ternura \\
Entre as asas ao filho essa ave aquenta, \\
Como aquela esgravata a terra dura, \\
\hspace{2em} E os seus assim sustenta; \\
\hspace{2em} Como se encoleriza, \\
E salta sem receio a todo o vulto, \\
\hspace{2em} Que junto deles pisa.
						
Que gosto não terá a esposa amante, \\
Quando der ao filhinho o peito brando, \\
E refletir então no seu semblante! \\
\hspace{2em} Quando, Marília, quando \\
\hspace{2em} Disser consigo: “É esta \\
“De teu querido pai a mesma barba, \\
\hspace{2em} “A mesma boca, e testa.”
						
Que gosto não terá a mãe, que toca, \\
Quando o tem nos seus braços, c’o dedinho \\
Nas faces graciosas, e na boca \\
\hspace{2em} Do inocente filhinho! \\
\hspace{2em} Quando, Marília bela, \\
O tenro infante já com risos mudos \\
\hspace{2em} Começa a conhecê-la!
						
Que prazer não terão os pais ao verem \\
Com as mães um dos filhos abraçados; \\
Jogar outros luta, outros correrem \\
\hspace{2em}Nos cordeiros montados! \\
\hspace{2em}Que estado de ventura! \\
Que até naquilo, que de peso serve, \\
\hspace{2em}Inspira Amor, doçura. \\
\end{verse}
\item \textbf{Parte II}: a poesia já não possui o tom melodioso do passado. O eu lírico já não é mais inspirado por Apolo, e teve a sua lira quebrada (perda da beleza existente no passado). Ainda assim, o eu lírico insiste em cantar o seu amor para Marília, desta vez por meio de um tom elegíaco, melancólico. Presença do \textit{locus horrendus} ou \textit{locus horribilis}, o lugar terrível, de paisagem assustadora. A Natureza se mostra ameaçadora e trai o eu lírico, pois é indiferente ao seu sofrimento (distanciamento físico de Marília). Ruptura da situação de harmonia entre o eu lírico e a Natureza. Exemplo notável é reproduzido abaixo.
\poemtitle{Lira I (Parte II)}
\begin{verse}[\versewidth]
Já não cinjo de louro a minha testa; \\
Nem sonoras canções o Deus me inspira: \\
\hspace{2em} Ah! que nem me resta \\
\hspace{2em} Uma já quebrada, \\
\hspace{2em} Mal sonora Lira!
						
Mas neste mesmo estado, em que me vejo, \\
Pede, Marília, Amor que vá cantar-te: \\
\hspace{2em} Cumpro o seu desejo; \\
\hspace{2em} E ao que resta supra \\
\hspace{2em} A paixão, e a arte.
						
A fumaça, Marília, da candeia, \\
Que a molhada parede ou suja, ou pinta, \\
\hspace{2em} Bem que tosca, e feia, \\
\hspace{2em} Agora me pode \\
\hspace{2em} Ministrar a tinta.
						
Aos mais preparos o discurso apronta: \\
Ele me diz, que faça do pé de uma \\
\hspace{2em} Má laranja ponta, \\
\hspace{2em} E dele me sirva \\
\hspace{2em} Em lugar de pluma.
						
Perder as úteis horas não, não devo; \\
Verás, Marília, uma ideia nova: \\
\hspace{2em} Sim, eu já te escrevo, \\
\hspace{2em} Do que esta alma dita \\
\hspace{2em} Quando amor aprova.
						
Quem vive no regaço da ventura \\
Nada obra em te adorar, que assombro faça: \\
\hspace{2em} Mostra mais ternura \\
\hspace{2em} Quem te ensina, e morre \\
\hspace{2em} Nas mãos da desgraça.
						
Nesta cruel masmorra tenebrosa \\
Ainda vendo estou teus olhos belos, \\
\hspace{2em} A testa formosa, \\
\hspace{2em} Os dentes nevados, \\
\hspace{2em} Os negros cabelos.
						
Vejo, Marília, sim, e vejo ainda \\
A chusma dos Cupidos, que pendentes \\
\hspace{2em} Dessa boca linda, \\
\hspace{2em} Nos ares espalham \\
\hspace{2em} Suspiros ardentes.
						
Se alguém me perguntar onde eu te vejo, \\
Responderei: No peito, que uns Amores \\
\hspace{2em} De casto desejo \\
\hspace{2em} Aqui te pintaram, \\
\hspace{2em} E são bons Pintores.
						
Mal meus olhos te riam, ah! nessa hora \\
Teu retrato fizeram, e tão forte, \\
\hspace{2em} Que entendo, que agora \\
\hspace{2em} Só pode apagá-lo \\
\hspace{2em} O pulso da Morte.
						
Isto escrevia, quando, ó Céus, que vejo! \\
Descubro a ler-me os versos o Deus louro: \\
\hspace{2em} Ah! dá-lhes um beijo, \\
\hspace{2em} E diz-me que valem \\
\hspace{2em} Mais que letras de ouro.
\end{verse}
\end{enumerate}

\subsection{Cartas chilenas}

Utilização do humor como forma de ridicularização (lembre-se de Gregório de Matos). É importante ressaltar que o humor ou a crítica, isolados, não constituem uma sátira.

\textbf{Carece de mais informações.}

\section{Basílio da Gama}

\textbf{Carece de mais informações.}

\subsection{O Uruguai}

Trata-se de um poema épico, dividido em cinco cantos e escrito em versos decassílabos brancos, sem rimas e sem estrofação. Narra a guerra ocorrida entre o exército luso-espanhol e os jesuítas.

\textbf{Carece de mais informações.}