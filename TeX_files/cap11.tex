\chapterimage{chapter_head_2.pdf}

\chapter{Romantismo em Portugal}

Alexandre Herculano fora o introdutor do romance histórico, medievalista, em Portugal. \textit{Eurico, o presbítero}, é uma de suas obras mais notáveis.

Camilo Castelo Branco, por sua vez, popularizara o folhetim (de criação francesa), e também escrevera diversas novelas passionais, narrativas em prosa de média extensão que tratam principalmente de relações amorosas\footnote{Algo que cabe ser mencionado é a etimologia do termo paixão. Derivado do latim \textit{passio}, significava, a princípio, sofrimento, o ato de suportar alguma emoção, em geral, negativa. Nesse sentido, é interessante notar a relação entre o termo e a obra \textit{Amor de perdição}.}. É autor de \textit{Amor de perdição}, a sua obra mais famosa.

Por fim, temos Almeida Garrett, o principal poeta do Romantismo em Portugal, e autor de \textit{Viagem na minha terra}, um romance digressivo.