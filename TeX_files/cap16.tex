\chapterimage{chapter_head_2.pdf}

\chapter{Introdução}

Como ficará mais claro nas páginas seguintes, há grande discussão em relação a classificar Machado como um autor realista. Com efeito, diversos aspectos de suas obras possuem pouca ou nenhuma relação com os valores presentes no Realismo. Dessa forma, utilizarei uma parte exclusiva para descrever o autor.

Como descrito na seção anterior, o Realismo — no contexto da Revolução Industrial — foi uma resposta ao subjetivismo e à idealização presentes no Romantismo. O movimento buscava, sobretudo, a representação da realidade de uma maneira direta, e por isso houve uma maior ênfase na descrição minuciosa do que em relação à narrativa propriamente. Nas obras realistas também era notória uma visão crítica da sociedade — vale mencionar que, nesse período, surgiram diferentes propostas de transformação da sociedade, como o socialismo e o anarquismo —, ao estilo de vida fútil da burguesia, assim como ao moralismo em paralelo com a hipocrisia presente do discurso dessa classe. Nesse sentido, relembre a ideia romântica da burguesia como um \textit{farol moral} para a sociedade, e às contradições dessa afirmação presentes em obras como \textit{O crime do Padre Amaro} e \textit{A relíquia}.

Machado de Assis era de origem humilde, neto de escravos. Por essa razão, não frequentara a universidade, mas ainda assim aprendera francês e inglês. É interessante notar, nesse sentido, que Machado foi um dos primeiros autores a apresentar uma influência sistemática\footnote{Para melhor compreender esse detalhe cabem alguns apontamentos. Lord Byron, por exemplo, citado na seção sobre o Romantismo, foi um autor britânico de grande influência na literatura brasileira. Ainda assim, a maior influência para as produções do Brasil até então era a literatura francesa, seguida pela portuguesa. Com efeito, mesmo as traduções realizadas por Castro Alves de algumas obras de Lord Byron se alinhavam muito mais à tradição francesa do que britânica.} da literatura inglesa em suas obras. O autor também possuía expectativas para traduzir alguns cantos de \textit{A divina comédia}.

Na imprensa, Machado de Assis se tornara um crítico literário de respeito: era um funcionário público\footnote{O fato de que muitos escritores desse e de posterior período trabalhavam para o governo é uma constante interessante presente entre os intelectuais brasileiros. Essa questão será melhor explicada no seção sobre o Modernismo, mas cabe citar o fato de que esses indivíduos, então com considerável capital cultural e intelectual, não possuíam formação direcionada aos trabalhos braçais, mas tampouco detinham meios de produção que lhe fornecessem subsistência.} e cronista, ao mesmo tempo em que publicava folhetins; foi também o fundador e presidente da Academia Brasileira de Letras. \textbf{Machado de Assis não romantizara a sua ascensão social.}

Machado de Assis iniciara sua trajetória como um escritor romântico, e frequentemente publicava suas obras por meio de folhetins (\textit{Quincas Borba}, por exemplo, fora publicado por meio de folhetins lançados seriadamente na revista feminina \textit{A Estação}). Nessa primeira fase, incluem-se obras como \textit{A mão e a luva}, \textit{Helena}, \textit{Ressurreição} e \textit{Iaiá Garcia}, os típicos folhetins românticos do século XIX, frequentemente com protagonistas femininas que se apaixonam por um indivíduo de classe mais alta, em uma relação condenada pela família do jovem.

No ano de publicação do último romance machadiano, \textit{Iaiá Garcia}, no ano de 1878, também era publicado \textit{O primo Basílio}, o qual recebera duras críticas de um Machado de Assis ainda romântico (``É fora de dúvida de que...'') apontando, em especial, defeitos no Realismo, e não aspectos particulares de Eça de Queirós. O autor português, por sua vez, respondera à crítica na segunda edição de \textit{O crime do padre Amaro} (``O esscritor provinciano...'').

% Nesse parte, tente futuramente adicionar um texto que melhor explique a polêmica envolvendo Machado de Assis e Eça de Queirós!

Não obstante, no ano de 1878 também houve a chamada \textbf{Batalha do Parnaso}, caracterizada pela publicação de poemas satíricos escritos por jovens realistas que frequentemente envolviam a figura de Machado de Assis. É curioso notar, no entanto, que tais conflitos tornaram o autor — então já conhecido e odiado — no crítico mais famoso do Brasil (relembre, nesse sentido, da fama de Eça de Queirós e a Geração de 70).

No ano seguinte, Machado publicaria o ensaio \textit{A nova geração}, no qual, com o prestígio então conquistado, apresenta ao público os jovens escritores que surgiam na época, incluindo mesmo aqueles que outrora o criticaram — torna-se, em certa medida, um mentor para esses autores.

A segunda fase de Machado de Assis, reconhecida como realista ou madura, é representada pela publicação de \textit{Memórias póstumas de Brás Cubas} — considerado, por alguns autores, como o primeiro romance do Realismo no Brasil (para alguns estudiosos, no entanto, não caberia tal designação, afinal, o que há de tão real em um livro narrado em 1ª pessoa por um defunto?). Obras como \textit{Quincas Borba}, \textit{Dom Casmurro}, \textit{Esaú e Jacó} e \textit{Memorial de Aires}, também se incluem nesse período.