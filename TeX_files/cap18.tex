\chapterimage{chapter_head_2.pdf}

\chapter{Memórias Póstumas de Brás Cubas}

No excerto abaixo, segue a nota ao leitor de \textit{Memórias póstumas de Brás Cubas}, de Machado de Assis.

\begin{corollary}[Memórias póstumas de Brás Cubas]
\textsc{Ao verme que primeiro roeu as frias carnes do meu cadáver dedico como saudosa lembrança estas memórias póstumas} \\

Ao leitor % Centralizar o texto

Que, no alto do principal de seus livros, confessasse Stendhal havê-lo escrito para cem leitores, coisa é que admira e consterna. O que não admira, nem provavelmente consternará é se este outro livro não tiver os cem leitores de Stendhal, nem cinqüenta, nem vinte, e quando muito, dez, Dez? Talvez cinco. Trata-se, na verdade, de uma obra difusa\footnote{No sentido de desvio do tema central, da trama. Nota-se também um estilo digressivo na prosa romântica machadiana, em especial como influência inglesa, presente em obras de Laurence Sterne como \textit{A vida e as opiniões do cavalheiro Tristam Shandy}.}, na qual eu, Brás Cubas, se adotei a forma livre de um Stern de um Lamb ou de um de Maistre\footnote{Autor de \textit{Viagem em torno do meu quarto}.}, não sei se lhe meti algumas rabugens de pessimismo\footnote{Machado de Assis possuía, de maneira geral, uma visão pessimista (ceticismo) acerca da natureza e da existência. Nesse sentido, construíra personagens sem grande maldade — os graves e frívolos. A grande exceção está presente em \textit{A causa secreta}, que representa a tortura de roedores por Fortunato.}. Pode ser. Obra de finado. Escrevia-a com a pena da galhofa\footnote{Humor.} e a tinta da melancolia\footnote{No sentido de tristeza. Note a mistura do humor inglês com a sétira francesa para expressar o pessimismo da realidade.}; e não é difícil antever o que poderá sair desse conúbio. Acresce que a gente grave achará no livro umas aparências de puro romance, ao passo que a gente frívola não achará nele o seu romance usual; e ei-lo aí fica privado da estima dos graves e do amor dos frívolos, que são as duas colunas máximas da opinião.

Mas eu ainda espero angariar as simpatias da opinião, e o meio eficaz para isto é fugir a um prólogo explícito e longo. O melhor prólogo é o que contém menos coisas, ou o que as diz de um jeito obscuro e truncado. Conseguintemente, evito contar o processo extraordinário que empreguei na composição destas Memórias, trabalhadas cá no outro mundo. Seria curioso, mas nimiamente extenso, e aliás desnecessário ao entendimento da obra. A obra em si mesma é tudo: se te agradar, fino leitor, pago-me da tarefa; se te não agradar, pago-te com um piparote\footnote{Peteleco. note a relação provocativa com o leitor por meio de insultos. Ideia de que o maior defeito é o próprio autor.}, e adeus.

Brás Cubas % Alinhar à direita
\end{corollary}

Em ``Acresce que a gente...'', é importante ressaltar a literatura como forma de dificação do leitor, para entretenimento desse. Escrita metalinguística que realiza uma reflexão sobre o próprio ato de escrever.

Novamente, nota-se a presença da ironia machadiana, que na obra do autor é mais do que apenas uma figura de linguagem, mas um instrumento de desvelamento da realidade, pela qual o autor expõe as reais motivações por trás das ações das personagens.

Brás Cubas e a cortesã Marcela, com a qual pretendera fugir. Brás Cubas e Virgília. Trama tênue, mas com muitas digressões ao longo da narrativa.

Na literatura romântica, era comum a conversa com o leitor. Nesse caso, há uma subversão por parte do autor desse processo, com um recuo crítico em relação a Brás Cubas (cínico), algo oculto em Bento Santiago, de \textit{Dom Casmurro}, por exemplo.

\begin{corollary}[Capítulo I]
\textsc{Óbito do autor} \\

Algum tempo hesitei se devia abrir estas memórias pelo princípio ou pelo fim, isto é, se poria em primeiro lugar o meu nascimento ou a minha morte. Suposto o uso vulgar seja começar pelo nascimento, duas considerações me levaram a adotar diferente método: a primeira é que eu não sou propriamente um autor defunto, mas um defunto autor\footnote{Não um escritor que morreu, mas um defunto que começou a escrever.}, para quem a campa foi outro berço; a segunda é que o escrito ficaria assim mais galante e mais novo. Moisés, que também contou a sua morte, não a pôs no intróito, mas no cabo; diferença radical entre este livro e o Pentateuco\footnote{Narração da morte de Moisés (autor} no Deuteronômio..

Dito isto, expirei às duas horas da tarde de uma sexta-feira do mês de agosto de 1869, na minha bela chácara de Catumbi. Tinha uns sessenta e quatro anos, rijos e prósperos, era solteiro, possuía cerca de trezentos contos e fui acompanhado ao cemitério por onze amigos. Onze amigos! Verdade é que não houve cartas nem anúncios. Acresce que chovia - peneirava - uma chuvinha miúda, triste e constante, tão constante e tão triste, que levou um daqueles fiéis da última hora a intercalar esta engenhosa idéia no discurso que proferiu à beira de minha cova: — "Vós, que o conhecestes, meus senhores, vós podeis dizer comigo que a natureza parece estar chorando a perda irreparável de um dos mais belos caracteres que tem honrado a humanidade. Este ar sombrio, estas gotas do céu, aquelas nuvens escuras que cobrem o azul como um crepe funéreo, tudo isso é a dor crua e má que lhe rói à natureza as mais íntimas entranhas; tudo isso é um sublime louvor ao nosso ilustre finado."

Bom e fiel amigo! Não, não me arrependo das vinte apólices que lhe deixei\footnote{Note, novamente, a presença da ironia machadiana.}. E foi assim que cheguei à cláusula dos meus dias; foi assim que me encaminhei para o undiscovered country de Hamlet, sem as ânsias nem as dúvidas do moçopríncipe, mas pausado e trôpego, como quem se retira tarde do espetáculo. Tarde e aborrecido. Viram-me ir umas nove ou dez pessoas, entre elas três senhoras, — minha irmã Sabina, casada com o Cotrim, — a filha, um lírio-do-vale, — e... Tenham paciência! daqui a pouco lhes direi quem era a terceira senhora. Contentem-se de saber que essa anônima, ainda que não parenta, padeceu mais do que as parentas. É verdade, padeceu mais. Não digo que se carpisse, não digo que se deixasse rolar pelo chão, epiléptica. Nem o meu óbito era coisa altamente dramática... Um solteirão que expira aos sessenta e quatro anos, não parece que reúna em si todos os elementos de uma tragédia. E dado que sim, o que menos convinha a essa anônima era aparentá-lo. De pé, à cabeceira da cama, com os olhos estúpidos, a boca entreaberta, a triste senhora mal podia crer na minha extinção.

- Morto! morto! dizia consigo.

E a imaginação dela, como as cegonhas que um ilustre viajante viu desferirem o vôo desde o Ilisso às ribas africanas, sem embargo das ruínas e dos tempos, — a imaginação dessa senhora também voou por sobre os destroços presentes até às ribas de uma África juvenil... Deixá-la ir; lá iremos mais tarde; lá iremos quando eu me restituir aos primeiros anos. Agora, quero morrer tranqüilamente, metodicamente, ouvindo os soluços das damas, as falas baixas dos homens, a chuva que tamborila nas folhas de tinhorão da chácara, e o som estrídulo de uma navalha que um amolador está afiando lá fora, à porta de um correeiro. Juro-lhes que essa orquestra da morte foi muito menos triste do que podia parecer. De certo ponto em diante chegou a ser deliciosa. A vida estrebuchava-me no peito, com uns ímpetos de vaga marinha, esvaía-se-me a consciência, eu descia à imobilidade física e moral, e o corpo fazia-se-me planta, e pedra, e lodo, e coisa nenhuma.

Morri de uma pneumonia; mas se lhe disser que foi menos a pneumonia, do que uma idéia grandiosa e útil, a causa da minha morte, é possível que o leitor me não creia, e todavia é verdade. Vou expor-lhe sumariamente o caso. Julgue-o por si mesmo.
\end{corollary}

No primeiro parágrago, há a narração da morte de Moisés. Veja a provocação realizada pelo narrador — arrogante a sincero — à Bíblia, e os trezentos contos que lhe renderam onze amigos.

Brás Cubas foi uma figura histórica que fundara a capitania de São Vicente, de família de posses. Ao narrador da história, da família Cubas, fora dado o nome Brás como indicativo não de uma origem humilde, mas pela descendência dessa família. Como legitimar a ascensão social da família?

Brás Cubas adota uma vida medíocre. Em dado momento, inventa um adesivo canforado que, em teoria, era capaz de curar hipocondria (tentativa de imortalizar o nome de sua família. Acaba desenvolvendo pneumonia, mas alega ter morrido de ``ideia fixa'' como forma de se distanciar desse acontecimento.

No excerto abaixo, segue a reprodução do capítulo \textit{O delírio}, mais extenso, porém igualmente interessante.

\begin{corollary}[Capítulo VII]
\\
\textsc{O delírio}

Que me conste, ainda ninguém relatou o seu próprio delírio; faço-o eu, e a ciência mo agradecerá. Se o leitor não é dado à contemplação destes fenômenos mentais, pode saltar o capítulo; vá direito à narração. Mas, por menos curioso que seja, sempre lhe digo que é interessante saber o que se passou na minha cabeça durante uns vinte a trinta minutos,

Primeiramente, tomei a figura de um barbeiro chinês, bojudo, destro, escanhoando um mandarim, que me pagava o trabalho com beliscões e confeitos: caprichos de mandarim.

Logo depois, senti-me transformado na Summa Theologica e São Tomás, impressa num volume, e encadernada em
 marroquim, com fechos de prata e stampas; idéia esta que me deu ao corpo a mais completa imobilidade; e ainda agora me lembra que, sendo as minhas mãos os fechos do livro, e cruzando-as eu sobre o ventre, alguém as descruzava (Virgília decerto), porque a atitude lhe dava a imagem de um defunto,

Ultimamente, restituído à forma humana, vi chegar um hipopótamo, que me arrebatou. Deixei-me ir, calado, não sei se por medo ou confiança; mas, dentro em pouco, a carreira de tal modo se tomou vertiginosa, que me atrevia interrogá-lo, e com alguma arte lhe disse que a viagem me parecia sem destino.

- Engana-se, replicou o animal, nós vamos à origem dos séculos.

Insinuei que deveria ser muitíssimo longe; mas o hipopótamo não me entendeu ou não me ouviu, se é que não fingiu uma dessas coisas; e, perguntando-lhe, visto que ele falava, se era descendente do cavalo de Aquiles ou da asna de Balaão, retorquiu-me com um gesto peculiar a estes dois quadrúpedes: abanou as orelhas. Pela minha parte fechei os olhos e deixei-me ir à ventura. Já agora não se me dá de confessar que sentia umas tais ou quais cócegas de curiosidade, por saber onde ficava a origem dos séculos, se era tão misteriosa como a origem do Nilo, e sobretudo se valia alguma coisa mais ou menos do que a consumação dos mesmos séculos, tudo isto reflexões de um cérebro enfermo. Como ia de olhos fechados, não via o caminho; lembrame só que a sensação de frio aumentava com a jornada, e que chegou uma ocasião em que me pareceu entrar na região dos gelos eternos. Com efeito, abri os olhos e vi que o meu animal galopava numa planície branca de neve, com uma ou outra montanha de neve, vegetação de neve, e vários animais grandes e de neve. Tudo neve; chegava a gelar-nos um sol de neve. Tentei falar, mas apenas pude grunhir esta pergunta ansiosa:

- Onde estamos?

- Já passamos o Éden.

- Bem; paremos na tenda de Abraão.

- Mas se nós caminhamos para trás! redargüiu motejando a minha cavalgadura.

Fiquei vexado e aturdido. A jornada entrou a parecer-me enfadonha e extravagante, o frio incômodo, a condução violenta, e o resultado impalpável. E depois -- cogitações de enfermo -- dado que chegássemos ao fim indicado, não era impossível que os séculos, irritados com lhes devassarem a origem, me esmagassem entre as unhas que deviam ser tão seculares como eles. Enquanto assim pensava, íamos devorando caminho, e a planície voava debaixo dos nossos pés, até que o animal estacou, e pude olhar mais tranqüilamente em tomo de mim. Olhar somente; nada vi, além da imensa brancura da neve, que desta vez invadira o próprio céu, até ali azul. Talvez, a espaços, me aparecia uma ou outra planta, enorme, brutesca, meneando ao vento as suas largas folhas. O silêncio daquela região era igual ao do sepulcro: dissera-se que a vida das coisas ficara estúpida diante do homem.

Caiu do ar? destacou-se da terra? não sei; sei que um vulto imenso, uma figura de mulher me apareceu então, fitando-me uns olhos rutilantes como o sol. Tudo nessa figura tinha a vastidão das formas selváticas, e tudo escapava à compreensão do olhar humano, porque os contornos perdiam-se no ambiente, e o que parecia espesso era muita vez diáfano. Estupefato, não disse nada, não cheguei sequer a soltar um grito; mas, ao cabo de algum tempo, que foi breve, perguntei quem era e como se chamava: curiosidade de delírio.

- Chama-me Natureza ou Pandora; sou tua mãe e tua inimiga.

Ao ouvir esta última palavra, recuei um pouco, tomado de susto. A figura soltou uma gargalhada, que produziu em torno de nós o efeito de um tufão; as plantas torceram-se e um longo gemido quebrou a mudez das coisas externas.

- Não te assustes, disse ela, minha inimizade não mata; é sobretudo pela vida que se afirma. Vives: não quero outro flagelo.

- Vivo? perguntei eu, enterrando as unhas nas mãos, como para certificar-me da existência.

- Sim, verme, tu vives. Não receies perder esse andrajo que é teu orgulho; provarás ainda, por algumas horas, o pão da dor e o vinho da miséria. Vives: agora mesmo que ensandeceste, vives; e se a tua consciência reouver um instante de sagacidade, tu dirás que queres viver.

Dizendo isto, a visão estendeu o braço, segurou-me pelos cabelos e levantou-me ao ar, como se fora uma simples pluma. Só então, pude ver-lhe de perto o rosto, que era enorme. Nada mais quieto; nenhuma contorção violenta, nenhuma expressão de ódio ou ferocidade; a feição única, geral, completa, era a da impassibilidade egoísta, a da eterna surdez, a da vontade imóvel. Raivas, se as tinha, ficavam encerradas no coração. Ao mesmo tempo, nesse rosto de expressão glacial, havia um ar de juventude, mescla de força e viço, diante do qual me sentia eu o mais débil e decrépito dos seres.

- Entendeste-me? disse ela, no fim de algum tempo de mútua contemplação.

- Não, respondi; nem quero entender-te; tu és absurda, tu és uma fábula. Estou sonhando, decerto, ou, se é verdade que enlouqueci, tu não 
 passas de uma concepção de alienado,isto é, uma coisa vã, que a razão ausente não pode reger nem palpar. Natureza, tu? a Natureza que eu conheço é só mãe e não inimiga; não faz da vida um flagelo, nem, como tu, traz esse rosto indiferente, como o sepulcro. E por que Pandora?

- Porque levo na minha bolsa os bens e os males, e o maior de todos, a esperança, consolação dos homens. Tremes?

- Sim; o teu olhar fascina-me.

- Creio; eu não sou somente a vida; sou também a morte, e tu estás prestes a devolver-me o que te emprestei. Grande lascivo, espera-te a voluptuosidade do nada.

Quando esta palavra ecoou, como um trovão, naquele imenso vale, afigurou-se-me que era o último som que chegava a meus ouvidos; pareceu-me sentir a decomposição súbita de mim mesmo. Então, encarei-a com olhos súplices, e pedi mais alguns anos.

- Pobre minuto! exclamou. Para que queres tu mais alguns instantes de vida! Para devorar e seres devorado depois! Não estás farto do espetáculo e da luta? Conheces de sobejo tudo o que eu te deparei menos torpe ou menos aflitivo: o alvor do dia, a melancolia da tarde, a quietação da noite, os aspectos da terra, o sono, enfim, o maior benefício das minhas mãos. Que mais queres tu, sublime idiota?

- Viver somente, não te peço mais nada. Quem me pôs no coração este amor da vida, se não tu? e, se eu amo a vida, por que te hás de golpear a ti mesma, matando-me?

- Porque já não preciso de ti. Não importa ao tempo o minuto que passa, mas o minuto que vem. O minuto que vem é forte, jocundo, supõe trazer em si a eternidade, e traz a morte, e perece como o outro, mas o tempo subsiste. Egoísmo, dizes tu? Sim, egoísmo, não tenho outra lei. Egoísmo, conservação. A onça mata o novilho porque o raciocínio da onça é que ela deve viver, e se o novilho é tenro tanto melhor: eis o estatuto universal. Sobe e olha.

Isto dizendo, arrebatou-me ao alto de uma montanha. Inclinei os olhos a uma das vertentes, e contemplei, durante um tempo largo, ao longe, através de um nevoeiro, uma coisa única. Imagina tu, leitor, uma redução dos séculos, e um desfilar de todos eles, as raças todas, todas as paixões, o tumulto dos impérios, a guerra dos apetites e dos ódios, a destruição recíproca dos seres e das coisas. Tal era o espetáculo, acerbo e curioso espetáculo. A história do homem e da terra tinha assim uma intensidade que lhe não podiam dar nem a imaginação nem a ciência, porque a ciência é mais lenta e a imaginação mais vaga, enquanto que o que eu ali via era a condensação viva de todos os tempos. Para descrevê-la seria preciso fixar o relâmpago. Os séculos desfilavam num turbilhão, e, não obstante, porque os olhos do delírio são outros, eu via tudo o que passava diante de mim, -- flagelos e delícias, -- desde essa coisa que se chama glória até essa outra que se chama miséria, e via o amor multiplicando a miséria, e via a miséria agravando a debilidade. Aí vinham a cobiça que devora, a cólera que inflama, a inveja que baba, e a enxada e a pena, úmidas de suor, e a ambição, a fome, a vaidade, a melancolia, a riqueza, o amor, e todos agitavam o homem, como um chocalho, até destruí-lo, como um farrapo. Eram as formas várias de um mal, que ora mordia a víscera, ora mordia o pensamento, e passeava eternamente as suas vestes de arlequim, em derredor da espécie humana. A dor cedia alguma vez, mas cedia à indiferença, que era um sono sem sonhos, ou ao prazer, que era uma dor bastarda. Então o homem, flagelado e rebelde, corria diante da fatalidade das coisas, atrás de uma figura nebulosa e esquiva, feita de retalhos, um retalho de impalpável, outro de improvável, outro de invisível, cosidos todos a ponto precário, com a agulha da imaginação; e essa figura, -- nada menos que a quimera da felicidade, -- ou lhe fugia perpetuamente, ou deixava-se apanhar pela fralda, e o homem a cingia ao peito, e então ela ria, como um escárnio, e sumia-se, como uma ilusão.

Ao contemplar tanta calamidade, não pude reter um grito de angústia, que Natureza ou Pandora escutou sem protestar nem rir; e não sei por que lei de transtorno cerebral, fui eu que me pus a rir, -- de um riso descompassado e idiota.

-- Tens razão, disse eu, a coisa é divertida e vale a pena,

-- talvez monótona -- mas vale a pena. Quando Jó amaldiçoava o dia em que fora concebido, é porque lhe davam ganas de ver cá de cima o espetáculo. Vamos lá, Pandora, abre o ventre, e digere-me; a coisa é divertida, mas digere-me.

A resposta foi compelir-me fortemente a olhar para baixo, e a ver os séculos que continuavam a passar, velozes e turbulentos, as gerações que se superpunham às gerações, umas tristes, como os Hebreus do cativeiro, outras alegres, como os devassos de Cômodo, e todas elas pontuais na sepultura. Quis fugir, mas uma força misteriosa me retinha os pés; então disse comigo: -- "Bem, os séculos vão passando, chegará o meu, e passará também, até o último, que me dará a decifração da eternidade." E fixei os olhos, e continuei a ver as idades, que vinham chegando e passando, já então tranqüilo e resoluto, não sei até se alegre. Talvez alegre. Cada século trazia a sua porção de sombra e de luz, de apatia e de combate, de verdade e de erro, e o seu cortejo de sistemas, de idéias novas, de novas ilusões; em cada um deles rebentavam as verduras de uma primavera, e amareleciam depois, para remoçar mais tarde. Ao passo que a vida tinha assim uma regularidade de calendário, faziam-se a história e a civilização, e o homem, nu e desarmado, armava-se e vestia-se, construía o tugúrio e o palácio, a rude aldeia e Tebas de cem portas, criava a ciência, que perscruta, e a arte que enleva, fazia-se orador, mecânico, filósofo, corria a face do globo, descia ao ventre da terra, subia à esfera das nuvens, colaborando assim na obra misteriosa, com que entretinha a necessidade da vida e a melancolia do desamparo. Meu olhar, enfarado e distraído, viu enfim chegar o século presente, e atrás dele os futuros. Aquele vinha ágil, destro, vibrante, cheio de si, um pouco difuso, audaz, sabedor, mas ao cabo tão miserável como os primeiros, e assim passou e assim passaram os outros, com a mesma rapidez e igual monotonia. Redobrei de atenção; fitei a vista; ia enfim ver o último, -- o último!; mas então já a rapidez da marcha era tal, que escapava a toda a compreensão; ao pé dela o relâmpago seria um século. Talvez por isso entraram os objetos a trocarem-se; uns cresceram, outros minguaram, outros perderam-se no ambiente; um nevoeiro cobriu tudo -- menos o hipopótamo que ali me trouxera, e que aliás começou a diminuir, a diminuir, a diminuir, até ficar do tamanho de um gato. Era efetivamente um gato. Encarei-o bem; era o meu gato Sultão, que brincava à porta da alcova, com uma bola de papel...
\end{corollary}

Note a forma como Brás Cubas descreve o início do Universo e da humanidade, com destaque para um sentimento de magalomania. A busca pela felicidade analisada por Schopenhauer. Por fim, perceba o paralelo com Vasco da Gama no canto da Máquina do Mundo, de \textit{Os lusíadas}, assim como as diferenças nos sentidos. Os limites da ilusão mimética.

\section{Um breve resumo}

Brás Cubas era de uma família de riqueza recente, que ascendera economicamente por meio da venda de barris. Nesse sentido, seu pai, em uma tentativa de legitimar a riqueza da família, nomeia o filho como indicativo de uma descendência do fundador de São Vicente — também nomeado Brás Cubas —, história essa que foi desmentida pela família verdadeira. Posteriormente, alegara relação da família com um guerreiro\footnote{No sentido de um cavaleiro romântico, medieval.} que lutara contra os mouros (veja, em todas essas tentativas, a associação entre o nobre e o vilão, e a forma pela qual o trabalho braçal estava associado à inferioridade e desumanidade).

Um aspecto importante da personalidade de Brás Cubas é descrito no Capítulo II, \textit{O emplasto}, no qual o narrador descreve sua ``sede da nomeada'' como indicativo de seu desejo de ser famoso, independentemente dos motivos ou razões, qualidades ou características. O adesivo canforado, citado anteriormente, insere-se como uma crítica ao capitalismo e, em especial, à ideia da mercadoria como causa de fama.

A mãe de Brás Cubas é retratada como uma mulher religiosa, mas fraca e submissa ao marido. Na infância, era uma criança ``arteira'' (eufemismo), sempre repreendido em público pelo pai, mas que o acobertava nas situações.

O Capítulo XI, intitulado \textit{O menino é o pai do homem}, refere-se à ideia da criança como molde para a vida adulta do indivíduo, em uma justificativa para a personalidade de Brás Cubas; vulgaridade como um capricho da vontade). Note a subversão presente em relação ao determinismo (do francês Hippolyte Taine), para o qual os indivíduos são determinados pela raça, meio (físico e social) e momento histórico (em especial, momento de desenvolvimento), e no qual a miscigenação era vista como processo de degeneração (justificativas para o domínio europeu no período do Neocolonialismo). Note a posição crítica de Machado em relação ao determinismo, por meio de uma subersão irônica.

No Capítulo XIV, \textit{O primeiro beijo}, Brás Cubas faz uma comparação com as figuras medievais presentes no Romantismo. Por meio dessa passagem, o autor critica tanto o Realismo com seus temas sórdidos, quanto o Romantismo e seus clichês.

Após o caso que Brás Cubas tivera com Marcela, o jovem é enviado para Coimbra, momento também no qual sua mãe adoece. Relutante, ele volta para o Brasil, onde seu pai havia selecionado Virgília como sua pretendente, assim como seu padrinho político. A mãe acaba falecendo, e Brás Cubas permanece recluso em uma chácara, no qual se torna vizinho de Eugênia\footnote{Vale mencionar que Eugênia é uma das poucas personagens verdadeiramente virtuosas em toda a obra de Machado de Assis (lembre-se do pessimismo do autor em relação à natureza e à existência).} (de origem ilegítima).

Festa passada é utilizado em referência à derrota de Napoleão; atritos com um poeta, pai de Eugênia, que tivera um caso com a mãe, então solteira. Eugênia é tratada como a flor da moita, em meio a elite carioca. Possui uma má formação congênita em sua perna (coxa, nas palavras de Bráas Cubas) que deixara o homem em crise.

Lazeira: mal de Lázaro. Eugênia: bem-nascida

Espantara a borboleta preta e retorna ao Rio de Janeiro. Retoma a relação com Virgília, mas a mulher casara com Lobo Neves. O pai de Brás Cubas falece e, posteriormente, a personagem desenvolve um caso com a antiga pretendente e uma curiosa amizade com Lobo Neves.

Plácida era a agregada de Virgília. Era uma costureira que fora descartada pela família que cuidara da casa na qual os amantes se encontravam (conflito moral da mulher, que permite a traição). Lobo Neves descobre a traição, mas é indicado para administrar a província do Maranhão (convite de Virgília a Brás Cubas, que não pretendia ir). Destaque para o número 13 do decreto, fatífico para lobo Neves. Virgília acompanha o marido, e coloca um fim em seu relacionamento com Brás Cubas.

Quincas Borba é um amigo de infância de Brás Cubas. De origens em uma família rica de Barbacena, perdera a sua riqueza e se tornara um mendigo — permanecia no 13º degrau da Catedral da Sé. Ao se encontrarem, furta o relógio de Brás Cubas, e apresenta sinais de desequilíbrio mental.

Posteriormente, Quincas Borba se descobre herdeiro de uma fortuna de sua tia, e também vende o relógio roubado.

O humanitismo seria o princípio universal de toda a existência. O humanitas seria disseminado por guerras e roubos. Publicação de um jornal narrando acontecimentos sob a visão do humanitismo. Justificativa para o sistema escravista. Brás Cubas, junção de Brasil e Cuba, representa a elite escravocrata brasileira, um cadáver.

O humanitismo é uma paródia machadiana de um sistema filosófico, semelhante à ideia do direito natural abordada em \textit{Bons dias!}, utilizado para justificar os privilégios de uma elite que se sustenta por meio da exploração do trabalho escravo. Serve, para Brás Cubas, como uma \textbf{moral de conveniência}.

Nessa lógica, as epidemias são vistas como um fenômeno destituído de sentido, uma vez que ceifava a vida de todos, sem exceção. A gripe espanhola afetara mais as classes favorecidas. Morte da noiva, Nhá-Loló (note a moral de conveniência nesse último caso). O dinheiro encontrado na rua e, posteriormente, na carruagem.

Também há inúmeros ataques à escravidão em segundo plano, mas que são um tema central da obra.

Irmã Sabina e seu marido Cotrim, rico mas masquinho (cobiça). A partilha da herança do pai dos irmãos como motivo de separação e distanciamento da relação. Posterior aproximação com a gravidez da irmã.

Um episódio curioso da obra envolve Brás Cubas e Cotrim. Para justificar as ações reprováveis do cunhado, escreve um texto no qual justifica as atitudes avarentas como excesso de providência. É válido mencionar que Cotrim enviava seus escravos ao calabouço (estatização da violência aos escravos), o que lhe gerara má fama (sua fortuna se originara com o contrabando de escravos, origens ilícitas dado o momento histórico dos anos de 1831-1850). Note que a riqueza de Brás Cubas também se se originara a partir da exploração do trabalho escravo.

Note, novamente, a utilização da ironia machadiana pela qual o verdadeiro Cotrim é revelado um indivíduo ainda pior do que se pensava. O determinismo da personagem em uma moral cínica, que justifica a violência contra os escravos. Com isso, Brás Cubas apenas confirma as acusações. Com isso, em primeiro plano, temos Brás Cubas alfinetando Cotrim e, em segundo plano, o próprio Machado de Assis criticando Brás Cubas e toda essa camada da população como um todo.

Outro momento que merece maior atenção é o Capítulo LXVIII, \textit{O vergalho}. Ao andar pelo Cais do Valongo (desembarque de escravizados), Brás Cubas se depara com Prudêncio, negro, chicoteando outro negro com um vergalho. Prudêncio fora alforriado pelo pai de Brás Cubas em seu leito de morte. Note o pensamento escravista intrínseco à sociedade, e presente no próprio escravizado — Prudêncio, mesmo livre, se referira a Brás Cubas como ``sinhô'' —, nesse caso, reflexo do passado de Prudêncio sob as determinações de Brás Cubas. Ainda nesse sentido, é importante ressaltar que Brás Cubas narrara o acontecimento como um episódio alegre, uma piada. Veja também a violência simbólica cometida pela elite brasileira, por meio de termos como \textit{gaiato} e \textit{maroto}.

Para o crítico literário brasileiro Roberto Schwarz, no Brasil o liberalismo foi utilizado como forma de justificar a exploração do trabalho escravo, em uma moral de conveniência (ideia fora do lugar), de forma semelhante ao humanitismo de Quincas Borba — representavam, em especial, a forma pela qual a elite brasileira recebia as ideias advindas da Europa.

Como síntese, Brás Cubas é o alvo e o ``autor'' das críticas. Com efeito, seu estilo de vida vadio só era possível em razão do trabalho escravo. Por fim, a única vantagem, saldo positivo de sua vida, foram seus privilégios e sua falta de descendentes.

Desde criança, Brás Cubas apresentava sinais de megalomania. \textit{Humanitas} seria o princípio fundamental, adotado por Brás Cubas como uma moral de conveniência — o direito de propriedade acima da liberdade dos outros indivíduos.