\chapterimage{chapter_head_2.pdf}

\chapter{Realismo no Brasil}

\section{Júlia Lopes de Almeida}\index{Júlia Lopes de Almeida}

Júlia Lopes fora considerada a herdeira de Machado de Assis — atenção para um momento no qual autoras femininas, usualmente, recaíam em certos nichos.  Júlia era filha de portugueses com uma visão mais liberal e moderna. Em uma viagem para Portugal, escreve livros infantis e se casa com Filinto de Almeida. Retorna ao Brasil. A escritora foi uma das idealizadoras da Academia Brasileira de Letras — instituição da qual, ironicamente, seu marido ganhara uma cadeira. Feminismo diferente aos olhos de hoje. Diferentes feminismos  ao longo da história e do globo (relembre a própria abolição tardia).

\subsection{A falência}\index{A falência}

Obra publicada em 1901, pertencente à escola realista, possui um narrador discreto, em 3ª pessoa que, por meio do discurso indireto livre, expõe sua visão por meio das personagens.

Em relação à lista de leituras do vestibular da Unicamp, da qual o livro é integrante, cabem alguns comentários. Monteiro Lobato e Jorge Filho, outrora autores com obras também pertencentes à coletânea, eram filhos de senhores de escravos. Essas obras seriam ,posteriormente, substituídas por \textit{Quarto de despejo}, de Carolina Maria de Jesus, e pelo álbum \textit{Sobrevivendo no inferno}, dos Racionais MC's. Linha de pesquisa USP/Unicamp semelhantes.

Acompanhamos a trajetória da família de Francisco Deodoro, que enriquecera com o comércio de café no Brasil no final do século XIX. Em um dos momentos iniciais da leitura, é válido mencionar uma reunião que ocorreu com outros portugueses ricos que se depararam com um jovem que enriquecera por meio de investimentos da bolsa de valores em pouco tempo — e que potencialmente se tornaria mais rico do que os demais. Deodoro até então recusava os investimentos de maior risco.

Com sua esposa Camila, Deodoro tivera quatro filhos: Mário, o primogênito; Ruth, de notória sensibilidade artística; e as gêmeas Lia e Raquel. Camila, já há muitos anos, possuía um caso com o Doutor Gervásio — inclusive, homem da confiança de Deodoro —, indivíduo refinado e muito presente na vida da família.

Nina representa a figura do agregado — citado anteriormente, indica a parcela da sociedade brasileira dos brancos livres e pobres —, e realizava os trabalhos domésticos da casa. Era apaixonada por Mário. Noca, por sua vez, era a governante.

É curioso notar que Mário possuía uma acompanhante de luxo que, em dado momento, se torna pública. Camila repreende o filho, que revela saber da traição de sua mãe.

Capitão Rino é como um competidor com Doutor Gervásio, pois também é apaixonado por Camila. Em dado momento, convida a família de Deodoro para um jantar em sua embarcação, local no qual se sente confiante e seguro.

Catarina é irmã de Rino. No jantar citado anteriormente, servira o salmão feito pelo irmão, e se envolve em uma discussão acerca da emancipação da mulher (Catarina a defendia).

O Doutor é frio com Camila, e na volta encontra-se com uma mulher vestida de preto.

Mário noiva-se com Paquita, de família rica e tradicional — lembre-se das famílias quatrocentonas, como a Álvares Penteado. Com isso, mostra-se capaz de manter um relacionamento e com alguém rico.

Deodoro é convencido a investir em uma casa de crédito. A bolsa, no entanto, cai dia a dia; o valor continua a cair e a bolsa quebra. Deodoro vai à falência, da qual comunica primeiramente o Doutor (um dia antes do casamento de Mário). A família acaba por acolher Deodoro nesse momento de fragilidade. Mário estava em lua de mel e Paquita possuía uma gravidez de risco.

A família desamparada se muda para a casa com a qual havia presenteado Nina. Camila presencia o suicídio de Deodoro, e fica desestruturada. Nina assume a liderança da família. Gervásio, aos poucos, se aproxima de Camila, e acaba sendo expulso por Mário. As gêmeas, por sugestão do irmão, se mudam para a casa de uma tia.

Descobrimos que Gervásio era um homem casado — lembre-se de que não havia a ideia de divórcio até então. Camila, ao descobrir, recusa continuar com o relacionamento. A mulher, então, começa a trabalhar dando aulas para as crianças do bairro.

Em análise da obra, alguns pontos da trama merecem nossa atenção. Veja que o adultério feminino, cometido por Camila, não levara a um final trágico (diferentes compreensões acerca do adultério feminino; na visão de uma mulher). Há também uma crítica à dupla moralidade patriarcal e, em especial, à tolerância em relação ao comportamento masculino e rigor em relação ao feminino. Notamos a presença da metáfora das duas honestidades, sobretudo na ``fala'' de Camila sobre seu adultério, apaixonada, em comparação com a traição descompromissada do marido. Metáfora das roupas de Gervásio: o casado preto aos homens e o vestido branco destinado às mulheres (veja a facilidade em manchar a moralidade feminina).

Júlia desejava menor liberdade sexual aos homens, e não maior às mulheres.

Camila concluiu que preferia servir a um só homem do que cometer adultério (culpabilização para redenção por meio do trabalho). O trabalho doméstico como fator que constrói a mulher, a qual deve educar os filhos; a mulher ideal seria Nina, que em crise assume a família quando Camila permanecia abalada.

Ao falir, Deodoro perde o seu papel como provedor, o que garantia que Camila, de origem na classe média-baixa, não precisasse trabalhar.

A falência econômica ou moral?

Discussão sobre o mesmo grau de responsabilidade entre os gêneros, e não de liberdade. Reforço do modelo patriarcal da família.

Defesa ao reforço do modelo patriarcal; a ruína moral da família de Deodoro. A punição da mulher adúltera, mas a redenção pelo trabalho.

Contexto da crise do encilhamento (o mercado como um jogo de sorte e azar) no ministério de Ruy Barbosa, e a impressão de papel moeda pelas casas de crédito (grande especulação financeira e volatividade do capital). Difícil transição do capitalismo industrial para o capitalismo financeiro — a sensação de instabilidade desse modelo sendo refletida nas visões de Deodoro. Transição, no Brasil, de modelos econômicos após três séculos de um modelo agroexportador; sequente declínio na produção de café.

Sancha era uma jovem negra que trabalhava na casa da tia de Ruth (possuíam idades semelhantes). Era maltratada, sobretudo, pela tia fervorosamente católica. Em reflexão, por meio do discurso indireto livre, Ruth. Fica a reflexão: a fala representaria uma visão racista da personagem ou da própria autora e, por extensão, da sociedade da época? Os abolicionistas não necessariamente assumiam os negros como iguais (a existência do ``quarto da empregada'' é um exemplo claro).