\chapterimage{chapter_head_2.pdf}

\chapter{Introdução}

O \textbf{Iluminismo} foi um movimento filosófico de contestação dos modelos ideológicos vigentes (tradicionais), pautado no \textbf{racionalismo}. Contestação pois, até o chamado Século das Luzes, o conhecimento ocidental estava baseado no tripé \textbf{fé}, \textbf{autoridade} e \textbf{tradição}.

Para a fé, existem questões que o indivíduo não precisa entender para assumir como verdade, pois são da ordem do divino (relevadas pela Bíblia, e não encontradas pela razão - estão além da compreensão humana). Assim sendo, resta aos mortais apenas aceitá-las (lembre-se da frase de Santo Agostinho, em relação à Santíssima Trindade: \textit{Credo quia absurdum}, creio porque é absurdo.). Note que fé é diferente de crença: esta representa o conjunto de ideias e dados que falam sobre a realidade e que o indivíduo assume como possíveis ou prováveis, ao passo que a fé é uma crença no que não pode ser provado, por definição, mesmo sem nenhuma comprovação e mesmo no impossível. Há coisas na Bíblia que precisam ser aceitas pois são revelações de Deus, ainda que não sigam a razão humana.

Autoridades são figuras recobertas de tradição e fontes privilegiadas da verdade (relembre de Aristóteles que, no livro \textit{Historia animalium}, classificara as aranhas como insetos).

Para a tradição, algo é verdade pois os indivíduos sempre acreditaram nisso. Envolve práticas e costumes que se repetem ao longo da história, como o casamento entre homens e mulheres (tais ideias possuem mais credibilidade do que aquelas que surgem recentemente). Um exemplo notável é a história de Galileu Galilei, que demonstrara que a Terra não era o centro do Universo e fora não só questionado por diversos outros astrônomos do período, como também investigado pela Inquisição. Veja que, para o Iluminismo, as ideias devem ser validadas por meio da razão (comprovação da veracidade), i.e., a razão empregada individualmente é o critério máximo de validação da verdade, subordinando até mesmo a fé, a autoridade e a tradição.

Para o filósofo Immanuel Kant, o \textbf{esclarecimento} se baseia na saída da minoridade intelectual - quando outros indivíduos determinam no que se deve acreditar - e a entrada na maioridade - formação das próprias ideias pelo uso da razão.

O Iluminismo se contrapõe aos fundamentos ideológicos do antigo regime, pois estes se baseavam, especialmente, na tradição e religião (o monarca representa o Estado e pode estabelecer a lei como bem desejar, tendo ainda o poder de decisão da vida e morte dos súditos). O que garantiria que determinado indivíduo seria o melhor governante, ou por que se deveria depositar confiança em sua autoridade? Até então, a resposta para tais questionamentos era simples: Deus o colocou ali (a religião justifica a existência do Absolutismo). Com a dúvida da providência divina, tais respostas também começam a ser questionadas.

Em relação ao aspecto supracitado, é válida a análise do Baruch Espinoza: Deus criou o mundo, é onisciente, onipotente e onipresente. Não podemos, pois, esperar menos do que a perfeição das leis da natureza. No entanto, dessa forma os milagres não poderiam existir, pois o mundo não seria perfeito. A conclusão é de que ou Deus é onisciente, onipresente e onipotente, ou os milagres existem. Também o filósofo francês René Descartes, que, pensando em seus pensamentos em frente a uma fogueira, tinha a única certeza de que ``Penso, logo existo'': todas as ideias devem ser questionadas e provadas racionalmente (falsidade da tradição).

Nesse período, também se deflagraram as revoluções liberais, que alcançaram o ápice com a Revolução Francesa, na Europa, a qual iniciara um regime moderno simbolizado pela decapitação do monarca.

É nesse contexto em que surge o Neoclassicismo, uma retomada dos valores estéticos e mitológicos do Classicismo (este, por sua vez, uma retomada dos valores estéticos da Antiguidade Clássica, a cultura greco-romana), como o equilíbrio, proporção, simetria, racionalidade, hedonismo, etc.

O compositor austríaco Wolfgang Amadeus Mozart foi o maior representante da música neoclássica, referida brevemente como classicista. De maneira geral, Neoclassicismo é o termo que o estilo de época vai adquirir em todas as áreas, ao passo que Arcadismo é o termo que o Neoclassicismo assume na Literatura.

Arcadia era um região montanhosa localizada na Grécia que, na Antiguidade, era conhecida pela atividade pastoril (pecuária). Por muito tempo, os gregos acreditavam que era a região mais antiga da Grécia, e de onde se originou (\textit{arcadia}, origem).
		
A principal caraterística do Arcadismo é a emulação de um tipo de poesia pastoril realizada na Roma Antiga - as \textbf{éclogas} -, uma forma do gênero lírico no qual o eu lírico se passa por um pastor (poesia bucólica\footnote{Bucolismo é a abordagem do campo como um espaço de beleza e tranquilidade, onde é possível entrar em harmonia com a natureza e encontrar felicidade.}, de temática pastoril). Écloga, ou égloga, é uma forma da poesia lírica romana da Antiguidade de temática pastoril e amorosa. Novamente, retornamos à emulação: a cópia dos clássicos, que possuem os valores universais, neste caso, oriundos da Antiguidade.

Um aspecto comum dessas produções é a \textbf{ambientação campestre}, na qual o eu lírico, geralmente um pastor com pseudônimo latino (como o caso de Tomás Antônio Gonzaga, que adotara o pseudônimo de Dirceu), que se declara apaixonado por uma \textit{nise} (pastora). A \textbf{idealização da vida no campo} também se faz presente. Lembre-se, afinal, de que a vida dos camponeses na Europa - e também em Roma, na Antiguidade - do século XVIII era difícil. Com efeito, as éclogas refletiam, sobretudo a visão dos senhores que as escreviam ou as encomendavam.

Por fim, outra característica importante a ser mencionada é o \textbf{idílio}: a representação de uma vida amorosa feliz em contato com a natureza. No Trovadorismo português havia o coito amoroso e, no Classicismo, o amor tirano (como o caso de Inês de Castro), na poesia romana havia a visão de uma vida amorosa feliz). Cabe apenas pontuar que identificamos, em alguns casos, a representação da saudade, da expectativa da volta da amada e, em raras ocasiões, poemas que representam o amor como algo negativo, mas a visão é predominantemente idílica.

Alguns lugares-comuns merecem maior atenção. São eles:

\begin{enumerate}
\item \textit{Carpe diem}: colha o dia, aproveite o momento presente.
\item \textit{Locus amoenus}: lugar aprazível, descrição da paisagem do campo com ênfase para a beleza da natureza e para a tranquilidade.
\item \textit{Fugere urbem}: fugir da cidade, a cidade como o palco das paixões humanas, onde o indivíduo vive umam vida tribulada e ilusória, baseada no luxo e na vaidade (pronúncia incerta do latim).
\item \textit{Aurea mediocritas}: medida de ouro, o indivíduo, para ser feliz, deve buscar o caminho do meio, o equilíbrio: não deve ser muito pobre nem rico, deve ter uma vida simples.
\item \textit{Inutilia truncat}: cortar o inútil, tudo aquilo que não é essencial deve ser dispensado. Ater-se ao que for essencial e dispensar o que for supérfluo. Aplicado na expressão poética, sugere que o estilo deve ser simples, sem excesso de figuras de linguagem (uma crítica aos excessos da literatura barroca, que entre os séculos XVII e XVIII, passou a ser considerada algo exacerbado).
\end{enumerate}
  
A poesia árcade era escrita para a elite aristocrática, em um período de transição da aristocracia para a burguesia que, à medida em que é reprimida, deseja se afugentar nos campos, fugir das cidades onde a burguesia enriquece e a população se revolta. É a negação de uma sociedade muito mais dinâmica que surge para formar a burguesia do século XVIII.

\poemtitle{Convite a Marília}
\begin{verse}
Já se afastou de nós o inverno agreste\footnote{Sem chuva.} \\
Envolto nos seus úmidos vapores\footnote{Névoa, neblina formada em situações de baixa temperatura.}; \\
A fértil primavera, a mãe das flores, \\
O prado\footnote{Forma de relevo.} ameno de boninas\footnote{Flor do campo.} veste.
			
Varrendo os ares, o sutil Nordeste\footnote{Vento que sopra do nordeste.} \\
Os torna azuis\footnote{Céu azul, sem nuvens.}; as aves de mil cores \\
Adejam\footnote{Bater as asas.} entre Zéfiros\footnote{Relembre a pintura de Botticelli. Zéfiros representa o vento oriental, e Bóres, o ocidental.} e Amores\footnote{Cupido.}, \\
E toma o fresco Tejo\footnote{Localizado em Portugal.} a cor celeste\footnote{Por reflexão da superfície.}.
			
Vem, ó Marília, vem lograr comigo \\
Destes alegres campos a beleza, \\
Destas copadas árvores o abrigo.
			
Deixa louvar da corte\footnote{Espaço onde os nobres se reúnem em torno do rei (Corte Real). Na poesia, torna-se sinônimo da sede do reinado de Portugal, localizada em Lisboa, a maior cidade. De maneira geral, serve de metonímia para uma grande cidade, principal.} a vã\footnote{Algo passageiro, ilusório, que não levará a nada.} grandeza: \\
Quanto me agrada mais estar contigo \\
Notando as perfeições da Natureza!
\end{verse}

Há a presença do lugar-comum \textit{locus amoenus}, em especial na descrição das árvores com copa como abrigo e sombra. Além disso, A vida de luxo das cidades é tratada como vã, em oposição com a vida no campo, permeada pelas perfeições da natureza (\textit{fugere urbem}).