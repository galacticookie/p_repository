\chapterimage{chapter_head_2.pdf}

\chapter{Introdução}

O Naturalismo surge inserido no âmbito realista, em especial com o escritor francês Émile Zola (1840-1902). Uma das principais características dessa vertente literária é o \textbf{objetivismo}, observado por meio do descritivismo — até então, a descrição minuciosa de elementos não atuantes na narrativa era considerada irrelevante — e da neutralidade do foco narrativo. O principal objetivo nessa maior ênfase na descrição e representação era causar um efeito de \textbf{ilusão mimética}; além disso, tornara-se comum a utilização de um narrador invisível, sem diálogos entre as personagens ou julgamentos por parte do narrador.

Outro aspecto importante das obras naturalistas é a presença de um olhar cínico sobre as mazelas sociais — influência das correntes cientificistas que circulavam na época. Em relação a essas ideias, é válido relembrar:

\begin{description}
    \item[Darwinismo social] Trata-se da noção de que a vida social se resume a uma luta pela sobrevivência, em que os indivíduos mais aptos superam os menos aptos — em um processo de naturalização das desigualdades sociais e valorização da meritocracia. O pensamento recebera considerável influência da teoria evolucionista de Charles Darwin, que analisara a forma pela qual os animais lutam pelos recursos naturais e as consequências dessa competição para as espécies. Relembre a ideia do \textit{espaço vital} do nazismo. Crescente ideia de raça (anteriormente, o povo), a qual uniu a Prússia no século XIX. Fascismo: \textit{fascio}. Os arianos não possuem comprovações de sua existência.
    \item[Determinismo] Ideia de que o comportamento e a personalidade humanos são o produto de três fatores: racial, mesológico (social e/ou ambiental) e histórico (como as diferentes temporalidades de um mesmo período vistas na industrialização; a vida moderna como mais complexa). O indivíduo nasce com a herança de seus antepassados (atavismo e a ideia da miscigenação como processo degenerativo), e sofre influência do ambiente natural e social que o determinam.. Justificativas para o neocolonialismo pelos europeus (relembre a dominação inglesa na Índia e a proibição da manufaturas têxteis) por uma suposta superioridade (o fardo do homem branco e o combate ao primitivismo).
\end{description}

Como citado anteriormente, o Naturalismo foi um desdobramento do Realismo francês. Caracterizado pelo objetivismo (descritivismo e neutralidade do foco narrativo) e por um olhar clínico sobre a sociedade — tratada como um organismo doente. Sofrera grande influência das correntes de pensamento cientificistas da época.

O intelectual brasileiro, de formação europeia, não identificava possibilidades para o desenvolvimento do país.

Outra questão muito presente na sociedade brasileira era o Positivismo. no Brasil, especialmente presente entre as forças armadas, ainda que Auguste Comte fosse pacifista. A história da humanidade como um progresso de um estado mais primitivo para outro mais evoluído. Auguste Comte desenvolvera a ideia do estágio teológico, no qual o principal sistema de pensamento é a religião, mitologia, refletidas nas estruturas de poder; o metafísico, baseado na filosofia, no qual o filósofo acaba por ignorar as particularidades; e o positivo, baseado na ciência (ditadura dos indivíduos mais aptos, clara influência platônica com a figura do rei filósofo. É importante ressaltar que, nesse contexto, ``ditadura'' representava tão somente uma forma de governo.

Nesse período, tornaram-se comuns os chamados romances experimentais, ou de tese, que se utilizavam da ficção para demonstrar determinada tese científica. Buscavam determinar o comportamento humano. Possuíam demasiada ênfase nos aspectos mais sórdidos e repugnantes da existência humana (o que foge à norma, como desvios sexuais e problemas físicos). Veja que o Realismo, por outro lado, possui um olhar crítico sobre a norma, ao passo que no Naturalismo há grande ênfase nos desvios dessa (patologias sociais). O escritor naturalista, mais normativo e conservador, seria o responsável pelo diagnóstico de uma sociedade enferma. O olhar orgânico do escritor sobre os indivíduos leva a uma maior descrição (relembre Eugênia, indivíduo virtuoso, com uma perna mais curta do que a outra; direção contrária ao Naturalismo).

Por fim, cabe reafirmar a distinção entre o pornográfico e o erótico. No Naturalismo há cenas mais explícitas, mas não necessariamente com um caráter erótico.

\begin{table}[h]
\centering
\begin{tabular}{l l}
\toprule
\textbf{Realismo} & \textbf{Naturalismo} \\
\midrule
foco na burguesia & foco nas párias sociais (excluídos, doentes) \\
estilo mais sugestivo e contido & estilo mais cru e explícito \\
ênfase psicológica & ênfase no orgânico (psicológico subordinado ao corporal) \\
\bottomrule
\end{tabular}
\caption{Legenda aqui.}
\end{table}