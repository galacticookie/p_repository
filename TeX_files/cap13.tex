\chapterimage{chapter_head_2.pdf}

\chapter{Introdução}

O Realismo remonta aos meados do século XIX, mais especificamente, no final da década de 1850 na França (o termo fora utilizado pela primeira vez no ano de 1855, por Gustave Courbet). Ocorrera em paralelo com a Segunda Revolução Industrial\footnote{O modelo industrial se expande pela Europa, Estados Unidos e Japão, e há intensa utilização de energia elétrica e carvão.}, momento de profundas transformações em toda a Europa, em especial pois não apenas a Inglaterra fora afetada, mas países de grande parte do globo. Como consequência, diversos foram os impactos no estilo de vida e nas cidades. Paris figura como a cidade referência de toda essa revolução (posteriormente, seria substituída por Londres), e passara por uma intensa reforma urbanística do centro-histórico\footnote{Inclusive, nesse período, ganhou a alcunha de Cidade Luz, pois possuía estruturas que garantiam iluminação pública ao longo de todo o dia.}, na transição de uma Paris medieval em direção à modernidade, conduzida pelo Barão de Haussmann.

Outro processo, no entanto, ocorria paralelamente a tais transformações: a classe trabalhadora passava a viver nos subúrbios. Além disso, não haviam direitos trabalhistas, e era comum que os operários dormissem em celas. As indústrias serviam como polo de atração da população europeia (urbanização), e os camponeses têm o seu estilo de vida profundamente alterado (transição da subsistência para o mercado). Soma-se a tais processo a ausência de direitos trabalhistas, e surgem movimentos políticos que buscam organizar as camadas na transformação da realidade social.

Então no contexto do capitalismo industrial, também nesse período surgem ideais como o anarquismo, o socialismo, assim como outros movimentos políticos revolucionários e reformistas\footnote{Cabe realizar uma pequena distinção. Por revolucionário entendemos a destruição da ordem social vigente, com o intuito de transformar a realidade social. Por reformista entendemos um processo gradual, aprimoramento gradativo. Em nosso contexto, trataremos o comunismo e o anarquismo como movimentos reformistas, à medida em que penetravam a classe trabalhadora.}.

Um aspecto muito importante é o questionamento da burguesia, que reconsidera a forma pela qual a sociedade se construiu (momento de reflexão). Marx identifica no ano de 1858, com o massacre do proletariado, como a data na qual a burguesia assume a posição de poder e abandona o anterior caráter revolucionário, que conduzira as transformações contra a nobreza.

Émile Zola e \textit{O germinal}.

Há uma reação ao idealismo romântico, em especial aos clichês e formas repetidas frequentemente utilizadas (o Romantismo se desgasta, torna-se clichê ao perder o poder criativo). Assim, há um foco maior na própria realidade.

O termo foi utilizado pela primeira vez em 1855 por Gustave Courbet, que em suas obras representava as pessoas da vida cotidiana em quadros de grandes dimensões (então reservados para as pinturas históricas). No Salão de Artes de Paris, local de grande exposição de obras da Europa, tentara expor o quadro \textit{O ateliê do artista}, com mais de $24 m^2$, mas que fora rejeitado. Ainda em 1855 teríamos o Pavilhão de Pintura Realista. No ano seguinte, também teríamos uma revista de obras realistas, na qual fora publicada a obra \textit{Madame Bovary}, de Gustave Flaubert. É interessante destacar que o tipo de amor romântico não corresponde à realidade, e muitas das obras se encarregarão de temas como os acontecimentos seguintes ao casamento principal da obra, mesmo descrevendo situações de adultério feminino.

Surgimento de uma literatura antiburguesa, autocrítica, de uma classe que outrora de força revolucionária contra o antigo regime, executara o massacre do proletariado (manifestação de uma força conservadora, para a manutenção do \textit{status quo}). Revolta da burguesia contra a própria burguesia.

Como citado anteriormente, um dos temas diletos do Realismo é o adultério feminino (lembre-se do Romantismo e a frequente sublimação da figura feminina), intensamente presente em \textit{Madame Bovary}, por exemplo, de Gustave Flaubert. A burguesia busca se destacar como um modelo de moralidade.

A reação ao idealismo romântico originara uma literatura crítica, marcada pelas seguintes características:

\begin{enumerate}
\item \textit{Objetivismo}: ideia de que a visão correta dos fatos seria a visão objetiva, que dá atenção aos fenômenos sensíveis e à representação da realidade social. Exibe as seguintes ramificações:
\begin{enumerate}
\item \textbf{Descritivismo}: o narrador descreve minuciosamente as personagens, objetos e lugares (anteriormente, um processo subordinado ao enredo, agora se torna autônomo), com o intuito de formar a ideia mais concreta possível na mente do leitor. Criação de uma ilusão mimética total — efeito do real —, e abordagem da leitura como representação da realidade. É válido mencionar, nesse sentido, \textit{O senhor dos anéis}, do século XX, e o processo de garantir concretude a um universo fantástico.
\item Neutralidade do foco narrativo (lembre-se da interferência do narrador nos romances anteriores). O narrador realista não comenta ou julga as ações das personagens. Há um predomínio do narrador não-personagem, em 3ª pessoa e onisciente (narrador invisível), ao menos na maior parte dos casos. Também citamos uma visão cientificista (sacralização), capaz de produzir um mundo perfeito e responder a todas as perguntas (garantia de um texto real e objetivo).
\end{enumerate}
\end{enumerate}

\par Figura da mulher fatal. 