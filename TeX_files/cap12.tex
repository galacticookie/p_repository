\chapterimage{chapter_head_2.pdf}

\chapter{Romantismo no Brasil}

O século XIX foi um período de grandes mudanças dentro do território brasileiro, e a situação do país no contexto geopolítico internacional mudara consideravelmente. Em 1822 ocorrera a independência do Brasil em relação a Portugal, tornando assim a região uma nação independente. Após um período marcado por turbulências políticas e um breve momento de regência, iniciaria-se o Segundo Reinado, com o monarca Dom Pedro II, que perduraria por quase meio século. O Romantismo chega ao país justamente nesse período, e teve papel fundamental no processo de construção de uma \textbf{identidade nacional brasileira}. A revista \textit{Nitheroy}, publicada pela primeira vez em 1836 por estudantes brasileiros em Paris, na França, é considerada a primeira publicação do Romantismo brasileiro. De autoria de Gonçalves Magalhães, \textit{Suspiros poéticos e saudades}, por sua vez, é o primeiro livro de poesias do Romantismo no país. Nota-se, no entanto, que as primeiras obras desse estilo de época ainda possuíam diversos elementos árcades, neoclássicos, identificados no contexto de um \textit{Romantismo de transição} (na própria revista citada anteriormente encontramos muitas dessas características).

Em um primeiro momento, o Romantismo esteve intimamente relacionado com os projetos políticos do Segundo Reinado. Estava inserido em um período anterior de grandes turbulências, instabilidades políticas que seguiram da regência, com inúmeros indivíduos contrários com a forma pela qual a independência fora conquistada. Em relação a esse processo, a própria figura de Dom Pedro I também atuara como elemento de instabilidade, em paralelo com a situação da União Ibérica, à medida em que o rei de Portugal não poderia assumir essa mesma posição em outro reino, em nosso caso, o Brasil. Todo esse cenário de confusão política culminaria, em 1840, na execução do Golpe da Maioridade, marco inicial para o Segundo Reinado.

Assumindo o trono pouco tempo antes de completar 15 anos, diversos eram os problemas a serem enfrentados pelo monarca. Em especial, Dom Pedro II desejava a criação de um sentimento de pertencimento ao Brasil, a adoção de uma identidade nacional, em especial com o intuito de acabar com as revoltas que assolaram o território anos antes. Para tanto, elaborara um projeto político de construção de uma identidade nacional como instrumento ideológico de manutenção da integridade territorial brasileira. Uma das principais formas de fomento a essa identidade fora a criação de uma cultura autônoma em relação à europeia e, em especial, portuguesa. É interessante notar, nesse sentido, o teor da frase ``Eu nasci e cresci no Brasil''; o processo é tratado quase como um acidente geográfico, em oposição com a frase ``Eu sou brasileiro''. Veja a forma como esse sentimento de identidade é fabricado, fruto de decisões políticas dentro de determinado contexto. Lembre-se, também, do processo de construção do samba como um ritmo nacional, que ainda que possuísse origens no Rio de Janeiro, a capital do país na época, era um gênero musical tão regional quanto o forró. Ao longo do regime militar a questão da identidade nacional também se fez fortemente presente.

A fim de melhor compreender os primórdios do Romantismo no Brasil, outras características da sociedade brasileira merecem maior atenção. Em primeira análise, lembre-se de que uma taxa inferior a 10\% da população brasileira era alfabetizada. Dessa forma, o processo de construção de uma identidade nacional por meio da literatura concernia apenas a elite brasileira, a classe social que possuía recursos, influência regional, capacidade de mobilização da sociedade e, sobretudo, controle político dos outros níveis de poder (não era necessário desenvolver tal sentimento entre as classes menos favorecidas, que não detinham poder econômico e tampouco político) — o processo ocorrera de cima para baixo. Na visão de Dom Pedro II, era necessário subordinar a elite ao poder do Estado e, para tanto, era necessária uma identidade a esses indivíduos, para que todos os que nascessem no país se sentissem parte deste.

Além disso, cabe mencionar a importância da literatura como difusora da identidade nacional e de conhecimentos sobre o Brasil — tratada tanto como fonte de entretenimento como meio de transmissão de determinado conhecimento ou ideia, especialmente em uma sociedade, na maior parte, analfabeta. Com efeito, nesse período, o ambiente intelectual europeu possuía uma considerável especialização: o meio acadêmico havia criado a figura dos especialistas (médicos, biólogos, matemáticos, etc.). No Brasil, em contrapartida, o acadêmico possuía uma vida incipiente e muito precária. Nesse contexto, durante o período colonial, era comum que os jovens da elite estudassem em Coimbra, Portugal (frequentemente, cursavam Direito), para então retornar ao país, uma vez que ainda não existiam instituições de ensino superior no Brasil (criadas posteriormente por Dom Pedro I, por meio do curso de Direito de Recife, em Pernambuco, assim como o curso de Direito do Largo de São Francisco - integrado à Universidade de São Paulo (USP) anos depois). Por essa razão, o intelectual brasileiro era, de maneira geral, um homem de letras, com uma formação humanística e mais genérica, de baixa especialização; o leitor brasileiro do século XIX não era um consumidor de monografias históricas, por exemplo, mas de produções mais gerais, menos especializadas. Por conseguinte, a identidade nacional não fora construída por textos históricos ou antropológicos, mas por meio da literatura e, em especial, de narrativas de amor, permeadas por elementos da ficção (tanto na prosa quanto na poesia), contextualizadas em determinada localidade do Brasil. A educação também teve importante papel nesse processo, e a disciplina de História, por exemplo, fora acrescida ao currículo das universidades como forma de divulgação da história nacional (que interessava ao Estado) e, em análise mais geral, de fomento à identidade brasileira. \textbf{No Brasil, os esforços se concentraram na literatura, e não nas escolas ou outros meios acadêmicos}. É interessante traçar um paralelo com a Europa, na qual a literatura também fora utilizada como forma de criação de uma identidade nacional, mas para um público reduzido.

Alguns acontecimentos também merecem ser citados. Primeiramente, elencamos a criação do Instituto Histórico Geográfico Brasileiro (IHGB) em outubro de 1838, o qual reunia o grupo dos homens de letras citados anteriormente. Ademais, é válido mencionar o financiamento pessoal de Dom Pedro II para a elaboração, por parte de Gonçalves Magalhães, da epopeia \textit{A Confederação dos Tamoios}, em 1856. Posteriormente, José de Alencar ainda escreveria uma nota de crítica sobre a obra, que seria rebatida pelo monarca, oculto por um pseudônimo.

Como elemento principal das produções literárias desse período, figura o questionamento \textbf{Como é viver no Brasil?}, à medida em que se buscava a criação de um mosaico da cultura brasileira.

Com o panorama histórico completo, iniciaremos o estudo da poesia romântica no Brasil, tradicionalmente dividida em três gerações: Indianismo, Ultrarromantismo e Condoreirismo.

\section{Indianismo}

A primeira geração do Romantismo no Brasil fora fortemente marcada pelo nacionalismo. Na Europa, sua principal manifestação se deu por meio do medievalismo, a idealização da Idade Média; no Brasil o processo é, naturalmente, impossível. Caracterizado pela exaltação e idealização da natureza brasileira (noção de \textbf{país novo}, caracterizado pela abundância de recursos naturais como promessa de um futuro grandioso). O índio como símbolo da nacionalidade e figura síntese das virtudes nacionais. Desenvolvimento de um país: colonização por povoamento ou exploração, independentemente se essa ocorrera ou não tardiamente. Revolução Industrial, em contraponto com o valor dos recursos naturais do Brasil (esta muda a lógica capitalista, de forma que o Brasil se encontra preso no capitalismo mercantil, do sistema colonial). Manifesta-se por meio da exaltação da natureza brasileira, sempre relacionada à noção de \textbf{país novo} (riqueza natural como promessa de uma futura grandeza nacional), tratada como característica que diferenciava o Brasil dos países europeus, que já sentiam os efeitos ambientais decorrentes da Revolução Industrial. Nesse sentido, o Brasil, apesar de ser um país muito novo, possui muitas riquezas naturais, de forma que um dia o país será uma grande potência. Todavia, note como a idade em pouco ou nada influencia (como o exemplo dos Estados Unidos e da Austrália), e a riqueza natural pode, muitas vezes, significar dependência econômica, em especial na lógica do capitalismo industrial. Idealização da figura do índio, símbolo da nacionalidade brasileira (relembre que o homem branco era descendente dos europeus; neste momento, o objetivo era o distanciamento destes; por outro lado, haviam também os negros, mas que eram relacionados à escravidão; restara o índio). O programa de extermínio indígena levado a cabo pela Coroa Portuguesa havia sido bem sucedido, de forma que o contato com indígenas era incomum no cotidiano da população (indianismo de gabinete). As fontes utilizadas pelos escritores era, além de sua própria imaginação, a literatura de viajantes, relatos de viagens desenvolvidos ao longo do período colonial, principalmente por jesuítas, e destinadas sobretudo ao público europeu. Tal processo fora iniciado com a carta de Pero Vaz de Caminha. O grande problema de se basear em tais textos se deve ao fato que possuem uma visão pautada eurocêntrica. Na própria carta do descobrimento, por exemplo, lembre-se de que a primeira providência tomada pelos portuguesas fora a celebração de uma missa. O erro está em acreditar, por exemplo, que necessariamente uma religião se baseia em símbolos externos. Os nativos perceberam o rito dos portugueses, mas estes não o fizeram (literatura etnocêntrica). Outra fonte da qual os escritores indianistas utilizavam era o indianismo romântico francês (François-René de Chateaubriand), que utilizara do índio como uma representação do mito do bom selvagem (lembre-se de Rousseau, ``o homem nasce puro, inocente, mas a sociedade o corrompe''). O índio é visto como uma síntese das virtudes nacionais, i.e., todas as qualidades que se deseja atribuir ao povo brasileiro são projetadas na figura do índio. Este, ao mesmo tempo em que é corajoso, também é generoso, altruísta (adaptação da figura do cavaleiro medieval da literatura romântica europeia). O brasil não teve uma Idade Média. O período correspondente seria a colonização, em que o cavaleiro é substituído pela figura do índio (idealizado, sem precisão antropológica e etnográfica).

\subsection{Gonçalves Dias}

Expoente da primeira geração romântica no Brasil (não confundir com Gonçalves Magalhães, autor do primeiro livro de poesias), era um mulato que estudou em Portugal. Assim como outros poetas desta geração (Álvares de Azevedo e Castro Alves), contraíra tuberculose, ainda que não tivesse morrido diretamente pelos efeitos da doença (veio a óbito em um naufrágio).

\subsubsection{I-Juca-Pirama}

Narra a história de um índio de mesmo nome, filho de um cacique cuja tribo está em conflito com outra, os aimorés - praticavam atos de antropofagia, com o intuito de, entre outros fatores, absorver a força do inimigo. Em dado momento da história, I-Juca-Pirama é capturado e, no momento de sua morte, chora e implora pelo perdão de seus inimigos. Por essa razão, os aimorés desistem de matar e se alimentar do jovem, à medida em que não desejam ingerir a carne de um índio fraco que temeu a morte. Ao retornar à tribo, descreve os acontecimentos para seu pai, o cacique, que o expulsa da tribo alegando não ser digno. Posteriormente, e outro conflito entre os tupis e os aimorés, I-Juca-Pirama aparece e liberta a sua tribo, mas acaba morrendo no processo (redenção em batalha). Destacamos o processo de nacionalização da ética guerreira, que o Romantismo europeu explorara por meio da figura do cavaleiro medieval. Um das partes mais importantes da obra é o Canto IV, reproduzido abaixo; escrito em redondilha menor (cinco sílabas poéticas, com icto - tônica obrigatória em determinada sílaba - na segunda e quinta sílabas poéticas), possui um ritmo marcante, utilizando-se de um canto de morte como um canto de batalha (como observado pelo emprego de \textit{guerreiros} como vocativo). Trata dos cruéis conflitos disputados pelas tribos, o cansaço proveniente de tais acontecimentos, e lembra de momentos de quando ainda estava na terra no natal.

\poemtitle{Canto IV}
\begin{verse}
Meu canto de morte, \\
Guerreiros, ouvi: \\
Sou filho das selvas, \\
Nas selvas cresci; \\
Guerreiros, descendo \\
Da tribo Tupi.
						
Da tribo pujante, \\
Que agora anda errante \\
Por fado inconstante, \\
Guerreiros, nasci: \\
Sou bravo, sou forte, \\
Sou filho do Norte; \\
Meu canto de morte, \\
Guerreiros, ouvi.
						
Já vi cruas brigas, \\
De tribos imigas, \\
E as duras fadigas \\
Da guerra provei; \\
Nas ondas mendaces \\
Senti pelas faces \\
Os silvos fugaces \\
Dos ventos que amei.
						
Andei longes terras, \\
Lidei cruas guerras, \\
Vaguei pelas serras \\
Dos vis Aimorés; \\
Vi lutas de bravos, \\
Vi fortes — escravos! \\
De estranhos ignavos \\
Calcados aos pés.
						
E os campos talados, \\
E os arcos quebrados, \\
E os piagas coitados \\
Sem seus maracás; \\
E os meigos cantores, \\
Servindo a senhores, \\
Que vinham traidores, \\
Com mostras de paz.
						
Aos golpes do imigo \\
Meu último amigo, \\
Sem lar, sem abrigo \\
Caiu junto a mi! \\
Com plácido rosto, \\
Sereno e composto, \\
O acerbo desgosto \\
Comigo sofri.
						
Meu pai a meu lado \\
Já cego e quebrado, \\
De penas ralado, \\
Firmava-se em mi: \\
Nós ambos, mesquinhos, \\
Por ínvios caminhos, \\
Cobertos d’espinhos \\
Chegamos aqui!
						
O velho no entanto \\
Sofrendo já tanto \\
De fome e quebranto, \\
Só queria morrer! \\
Não mais me contenho, \\
Nas matas me embrenho, \\
Das frechas que tenho \\
Me quero valer.
						
Então, forasteiro, \\
Caí prisioneiro \\
De um troço guerreiro \\
Com que me encontrei: \\
O cru dessossego \\
Do pai fraco e cego, \\
Enquanto não chego, \\
Qual seja — dizei!
						
Eu era o seu guia \\
Na noite sombria, \\
A só alegria \\
Que Deus lhe deixou: \\
Em mim se apoiava, \\
Em mim se firmava, \\
Em mim descansava, \\
Que filho lhe sou.
						
Ao velho coitado \\
De penas ralado, \\
Já cego e quebrado, \\
Que resta? - Morrer. \\
Enquanto descreve \\
O giro tão breve \\
Da vida que teve, \\
Deixa-me viver!
						
Não vil, não ignavo, \\
Mas forte, mas bravo, \\
Serei vosso escravo: \\
Aqui virei ter. \\
Guerreiros, não coro \\
Do pranto que choro; \\
Se a vida deploro, \\
Também sei morrer.
\end{verse}

Por fim, citamos \textit{Canção do exílio}, potencialmente a obra mais famosa do autor. Novamente, há a exploração de temas que permeiam um sentimento de nacionalismo.

\poemtitle{Canção do exílio}
\begin{verse}[\versewidth]
Minha terra tem palmeiras, \\
Onde canta o sabiá; \\
As aves, que aqui gorjeiam, \\
Não gorjeiam como lá.
						
Nosso céu tem mais estrelas, \\
Nossas várzeas têm mais flores, \\
Nossos bosques têm mais vida, \\
Nossa vida mais amores.
					
Em cismar, sozinho, à noite, \\
Mais prazer eu encontro lá; \\
Minha terra tem palmeiras, \\
Onde canta o sabiá.
					
Minha terra tem primores, \\
Que tais não encontro eu cá; \\
Em cismar - sozinho, à noite - \\
Mais prazer eu encontro lá; \\
Minha terra tem palmeiras, \\
Onde canta o sabiá.
					
Não permita Deus que eu morra, \\
Sem que eu volte para lá; \\
Sem que desfrute os primores \\
Que não encontro por cá; \\
Sem qu'inda aviste as palmeiras, \\
Onde canta o sabiá.
\end{verse}

\section{Ultrarromantismo}

A segunda geração romântica, frequentemente referida também como byronismo. Trata-se de um termo pejorativo utilizado na Europa em referência à vertente sentimentalista do Romantismo português. De maneira geral, fora caracterizada pelo sentimentalismo, subjetivismo (a poesia dedicada aos sentimentos e emoções do indivíduo), egotismo (o poeta romântico fala preferencialmente de si mesmo) e  evasionismo (o indivíduo em conflito com a realidade foge para a sua vida interior; fuga para o mundo interior ou para um passado idealizado, como a infância). Compreende, sobretudo, uma poesia que não possui caráter social, i.e. não tratará das principais pautas sociais e políticas de sua época.

Outra característica muito presente nessa vertente é a idealização da figura feminina, citada no capítulo de introdução. Comumente era representada a cena da virgem adormecida e o poeta tímido. A mulher é tratada como um ser valorizado por sua pureza e castidade (em paralelo com a figura da mulher corrompida pela prostituição, por exemplo). A temática do \textit{medo do amor}, expressão utilizada por Mário de Andrade para se referir do niilismo do amor dessa segunda geração. A figura do adolescente tímido, mesmo que o eu lírico não se apresente como adolescente, este revela demasiada insegurança em relação às mulheres, e não sabe como se portar. Do latim \textit{timere}, aquele que tem medo, no caso, de não ser correspondido e por isso sofrer muito. Motivo (uma cena recorrente, que se repete) da virgem adormecida. Eram muito famosos os poemas nos quais o eu lírico apenas observa a amada enquanto dorme, sem se aproximar ou se declarar (satisfação do interesse erótico). O imaginário a partir da vida priva, reclusa das mulheres, em meio à valorização da pureza e virgindade femininas, e o prestígio romântico da mulher, sempre chamada de virgem, anjo (distanciamento da carnalidade), criança (ênfase na pureza e inocência), irmã (não um amor sexualizado, carnal ou satânico, mas um amor puro), etc. Lembre-se também da questão da melancolia e a relação estabelecida com o \textit{spleen}, bile negra — fluido produzido pelo baço.

Também se faz presente o satanismo, comumente referido como byronismo, e a utilização de temas macabros (relacionados à morte), perversões sexuais e \textit{dom juanismo} (libertinagem). Lembre-se de que Lord Byron transformara a figura de Dom Juan em um herói.

\subsection{Casimiro de Abreu}

Um dos principais autores dessa geração, também conhecido como o \textbf{poeta da infância}, uma vez que em muitos poemas, em um processo de idealização da vida infantil e, em última análise, do campo, retorna para sua casa de origem.

\textit{Carece de mais informações.}

\subsection{Álvares de Azevedo}

Álvares de Azevedo, em paralelo com Gonçalves Dias, é o principal autor da segunda geração romântica (Ultrarromantismo); divide espaço com Casimiro de Abreu, o poeta da infância. Bernardo Guimarães (escritor de \textit{A escrava Isaura}), por sua vez, foi um dos maiores expoentes da corrente satanista e sentimentalista da segunda geração.

\subsubsection{Noite na taverna}

Exemplo notável da vertente satânica do Ultrarromantismo. Acompanhamos um grupo de libertinos (sem freios morais) que, após uma noite de orgia, se reúnem uma taverna, onde passam a noite contando histórias uns aos outros como forma de entretenimento. O primeiro conto, por exemplo, possui elementos relacionados diretamente com a necrofilia.

\subsubsection{Lira dos vinte anos}

É interessante notar a diferença existente entre a primeira e a segunda parte da obra. Na primeira parte estão todas as convenções da vertente sentimentalista do Romantismo (os clichês da literatura sentimentalista romântica, como a observação da amada dormindo — ou mesmo a virgem morta\footnote{Na vertente satanista, \textit{aquilo que não era necrofilia, torna-se} (lembre-se de \textit{Noite na taverna}).}, que não é corrompida pelo pecado —, a tristeza pela vida, etc.).

A segunda parte, por sua vez, é caracterizada por um realismo satírico e pela ironia. As convenções da primeira parte reaparecem rebaixadas ao plano da vida ordinária. Os mesmos lugares-comuns e convenções da primeira parte reaparecem em uma perspectiva irônica, tomadas pelo realismo satírico (note que a abordagem anterior é marcada pelo evasionismo), trazendo-os para o cotidiano. Lembre-se do conceito de decoro: para cada tema, há um estilo e tom apropriado. Nesse sentido, o amor é abordado no estilo alto, na medida em que no estilo baixo se aborda a vida cotidiana, ordinária, e a vida do homem comum. Álvares de Azevedo toma os temas da poesia lírica e os transfere para a vida cotidiana, de forma a receber uma abordagem humorística (convenções rebaixadas ao plano da vida ordinária).

Para entender esse processo, Marx descreve que a burguesia da qual o Romantismo é expressão é muito crítica consigo mesma, e está em conflito com o antigo regime. Constrói-se a ideia de uma burguesia anti-burguesa: esta imagina que a sociedade oprime o indivíduo, mas quem a controla é a própria burguesia, de forma que a burguesia oprime o próprio burguês. Há uma sátira a partir dos valores que a burguesia reflete na literatura, o fenômeno da literatura anti-burguesa (vertente reformista do Romantismo), cujo instrumento de crítica é a ironia. \textbf{A repetição se torna cansativa, e por isso se busca uma nova abordagem dos lugares-comuns}.

Analisaremos com maior cuidado cada uma das partes.

\poemtitle{No mar (Parte I)}
\begin{verse}
Era de noite — dormias, \\
Do sonho nas melodias, \\
Ao fresco da viração, \\
Embalada na falua, \\
Ao frio clarão da lua, \\
Aos ais do meu coração!
						
Ah! que véu de palidez \\
Da langue face na tez! \\
Como teus seios revoltos \\
Te palpitavam sonhando! \\
Como eu cismava beijando \\
Teus negros cabelos soltos!
						
Sonhavas? — eu não dormia; \\
A minh'alma se embebia \\
Em tua alma pensativa! \\
E tremias, bela amante, \\
A meus beijos, semelhante \\
Às folhas da sensitivas!
						
E que noite! que luar! \\
E que ardentias no mar! \\
E que perfumes no vento! \\
Que vida que se bebia \\
Na noite que parecia \\
Suspirar de sentimento!
						
Minha rola, ó minha flor, \\
Ó madresilva de amor, \\
Como eras saudosa então! \\
Como pálica sorrias \\
E no meu peito dormias \\
Aos ais do meu coração!

E que noite! que luar! \\
Como a brisa a soluçar \\
Se desmaiava de amor! \\
Como toda evaporava \\
Perfumes que respirava \\
Nas laranjeiras em flor!
						
Suspiravas? que suspiro! \\
Ai que ainda me deliro \\
Entrevendo a imagem tua \\
Ao fresco da viração, \\
Aos ais do meu coração, \\
Embalada na falua!
						
Como virgem que desmaia, \\
Dormia a onda na praia! \\
Tua alma de sonhos cheia \\
Era tão pura, dormente, \\
Como a vaga transparente \\
Sobre seu leito de areia!
						
Era de noite — dormias, \\
Do sonho nas melodias, \\
Ao fresco da viração; \\
Embalada na falua, \\
Ao frio clarão da lua, \\
Aos ais do meu coração \\
\end{verse}

De maiores convenções, o eu lírico observa a mulher amada dormindo sob um leito de flores (ou a estampa do lençol), na luz de lâmpadas. A amada é como se fosse a lua embalsamada (envolta) pela noite (referência ao processo de embalsamento, proporcionando um aspecto mórbido, que acrescenta sensualidade ao poema), em valorização de uma poesia mais doentia. Comparação da virgem adormecida com a virgem surgida no mar (relembre o nascimento da virgem Afrodite). Comparação com um anjo em meio as nuvens (prestígio romântico da mulher, de forma a torná-la mais espiritualizada). Velar um cadáver, com a utilizar de velas (os velórios atravessavam a noite). A maior prova de amor que o eu lírico oferece é morrer pela amada. Utilização do clichê da virgem adormecida, da qual o eu lírico não se aproxima.

\poemtitle{É ela! É ela! (Parte II)}
\begin{verse}
É ela! é ela! — murmurei tremendo, \\
E o eco ao longe murmurou — é ela!... \\
Eu a vi... minha fada aérea e pura, \\
A minha lavadeira na janela!
						
Dessas águas-furtadas onde eu moro \\
Eu a vejo estendendo no telhado \\
Os vestidos de chita, as saias brancas... \\
Eu a vejo e suspiro enamorado!
						
Esta noite eu ousei mais atrevido \\
Nas telhas que estalavam nos meus passos \\
Ir espiar seu venturoso sono, \\
Vê-la mais bela de Morfeu nos braços!
						
Como dormia! que profundo sono!... \\
Tinha na mão o ferro do engomado... \\
Como roncava maviosa e pura! \\
Quase caí na rua desmaiado!
						
Afastei a janela, entrei medroso: \\
Palpitava-lhe o seio adormecido... \\
Fui beijá-la... roubei do seio dela \\
Um bilhete que estava ali metido...
						
Oh! De certo ... (pensei) é doce página \\
Onde a alma derramou gentis amores!... \\
São versos dela... que amanhã decerto \\
Ela me enviará cheios de flores...

Trem de febre! Venturosa folha! \\
Quem pousasse contigo neste seio! \\
Como Otelo beijando a sua esposa, \\
Eu beijei-a a tremer de devaneio...

É ela! é ela! — repeti tremendo, \\
Mas cantou nesse instante uma coruja... \\
Abri cioso a página secreta... \\
Oh! meu Deus! era um rol de roupa suja!
						
Mas se Werther morreu por ver Carlota \\
Dando pão com manteiga às criancinhas, \\
Se achou-a assim mais bela... eu mais te adoro \\
Sonhando-te a lavar as camisinhas!
						
É ela! é ela! meu amor, minh’alma, \\
A Laura, a Beatriz que o céu revela... \\
É ela! é ela! — murmurei tremendo, \\
E o eco ao longe suspirou — é ela!
\end{verse}

Do realismo satírico, aborda o mesmo tema da cena da virgem adormecida, mas com algumas notórias diferenças. A utilização de fada aérea e pura nos remete ao plano da mulher idealizada, mas ao identificá-la como lavadeira, o autor traz o tema para a realidade cotidiana, com uma mulher pertencente às classes menos abastadas. O eu lírico mora no último andar da residência. Chita é um tecido barato. Morfeu é o deus dos sonhos. O eu lírico invade o quarto da amada no meio da noite para observá-la. Maviosa como sinônimo de delicada, ironia com a situação da amada roncando, que dormiu em meio ao trabalho. Quando o eu lírico se aproxima para beijá-la, encontra um bilhete. Pensando que se tratava de um poema para sua pessoa, o pega. Processo de inversão, na medida em que é o eu lírico quem espera as flores por parte da amada. O bilhete era, na verdade, uma lista de roupas dos clientes. Referência com a obra \textit{O sofrimento do jovem Werther}, que se apaixonara por Carlota enquanto a jovem distribuía pão para as crianças da casa. Camisinha como a vestimenta que permanecia por baixo das roupas. Referências à Beatriz de Dante Alighieri, e Laura de Petrarca.

\textbf{A palavra de ordem do Romantismo é a idealização}.

\section{Condoreirismo}

A terceira geração do Romantismo no Brasil, caracterizada, sobretudo, pelo reformismo. Trata-se de uma das manifestações do conflito eu e o mundo, no qual o indivíduo percebe que o mundo não corresponde aos seus ideias. Assim, ao invés de se refugiar em seu interior ou mesmo nas lembranças de um passado idealizado, como ocorrera com os sentimentalistas, busca transformar a realidade. Ideia da literatura como um instrumento de transformação das consciências, em um projeto de mudança social (a literatura, por si ó, não é capaz de transformar a realidade, mas sim despertar a consciência dos indivíduos, estes sim aptos para tal). Dois dos valores fundamentais para essa vertente são o liberalismo e o combate à tirania, para o qual a liberdade pressupõe ganho social. O ser humano alcança a sua dignidade com o exercício da liberdade (econômica e política). Nesse sentido, é interessante citar a análise de Marx para o qual a burguesia das revoluções, a qual abandona se caráter revolucionário com o massacre do proletariado nas ruas de Paris (data definida pelo próprio autor). Em alguns países, o combate à tirania se descreve da deposição do antigo regime, em outros, no combate contra a Igreja (combate à tirania, a depender da conjuntura política).

Condoreirismo deriva de condor, uma ave de grande envergadura (maior do hemisfério sul), a qual habita a Cordilheira dos Andes. Símbolo para o \textit{americanismo} (diferenciação dos países para com aqueles que o colonizaram), e metáfora para a figura do \textit{gênio}, o indivíduo que é naturalmente dotado de inteligência, sensibilidade e imaginação superiores a dos indivíduos comuns. Este, ao ser capaz de ver além, é capaz de guiar a humanidade para o caminho correto (farol para o caminho do progresso). Em paralelo, o condor voa alto, referência ao pensar além do que as pessoas são capazes (natureza diferente).

De maneira geral, o maior compromisso com as causas políticas da época, e que mobilizara os liberais no século XIX, era o abolicionismo, discutido desde a década de 1820 (o republicanismo não era um consenso), assim como o compromisso com a democracia. O brasil, independentemente do sistema político, sempre fora um país pouco democrático (limitação dos direitos políticos, como o voto censitário). A participação feminina (acesso ao voto e ao trabalho) nunca foi uma pauta amplamente defendida pelos liberais. Ainda assim, o feminismo é um movimento que surgiu em meio ao liberalismo e, posteriormente, assumiu outros caminhos.

\subsection{Castro Alves}

Conhecido como o poeta dos escravos, foi o expoente da terceira geração. Mulato, morreu ainda jovem com um tiro no pé. Identificamos, em sua obra, um processo de erotização mais explícita, e a superação da ideia do jovem tímido.

\poemtitle{O Povo ao poder}
\begin{verse}
Quando nas praças se eleva \\
Do Povo a sublime voz… \\
Um raio ilumina a treva \\
O Cristo assombra o algoz…
					
Que o gigante da calçada \\
De pé sobre a barrica \\
Desgrenhado, enorme, nu \\
Em Roma é catão ou Mário, 
					
É Jesus sobre o Calvário, \\
É Garibaldi ou Kosshut.
					
A praça! A praça é do povo \\
Como o céu é do condor\\
É o antro onde a liberdade \\
Cria águias em seu calor! 
					
Senhor!… pois quereis a praça? \\
Desgraçada a populaça \\
Só tem a rua seu… \\
Ninguém vos rouba os castelos
					
Tendes palácios tão belos… \\
Deixai a terra ao Anteu.
					
Na tortura, na fogueira… \\
Nas tocas da inquisição \\
Chiava o ferro na carne \\
Porém gritava a aflição. \\
Pois bem … nesta hora poluta 
					
Nós bebemos a cicuta \\
Sufocados no estertor; \\
Deixai-nos soltar um grito\\
Que topando no infinito 
					
Talvez desperte o Senhor. 
					
A palavra! Vós roubais-la \\
Aos lábios da multidão \\
Dizeis, senhores, à lava \\
Que não rompa do vulcão.\\
Mas qu’infâmia! Ai, velha Roma, \\
Ai cidade de Vendoma, \\
Ai mundos de cem heróis,\\
Dizei, cidades de pedra,\\
Onde a liberdade medra\\
Do porvir aos arrebóis.
					
Dizei, quando a voz dos Gracos \\
Tapou a destra da lei? \\
Onde a toga tribunícia \\
Foi calcada aos pés do rei? \\
Fala, soberba Inglaterra, \\
Do sul ao teu pobre irmão; \\
Dos teus tribunos que é feito? \\
Tu guarda-os no largo peito \\
Não no lodo da prisão. \\
No entanto em sombras tremendas\\
Descansa extinta a nação \\
Fria e treda como o morto. \\
E vós, que sentis-lhes os pulso \\
Apenas tremer convulso \\
Nas extremas contorções… \\
Não deixais que o filho louco \\
Grite “oh! Mãe, descansa um pouco \\
Sobre os nossos corações”. 
						
Mas embalde… Que o direito \\
Não é pasto de punhal. \\
Nem a patas de cavalos \\
Se faz um crime legal… \\
Ah! Não há muitos setembros, \\
Da plebe doem os membros \\
No chicote do poder, \\
E o momento é malfadado \\ 
Quando o povo ensanguentado \\
Diz: já não posso sofrer.
					
Pois bem! Nós que caminhamos \\
Do futuro para a luz, \\
Nós que o Calvário escalamos \\
Levando nos ombros a cruz, \\
Que do presente no escuro \\
Só temos fé no futuro, \\
Como alvorada do bem, \\
Como Laocoonte esmagado \\
Morreremos coroado \\
Erguendo os olhos além. 
					
Irmãos da terra da América, \\
Filhos do solo da cruz, \\
Erguei as frontes altivas, \\
Bebei torrentes de luz..\\
Ai! Soberba populaça, \\
Dos nossos velhos Catões, \\
Lançai um protesto, ó povo, \\
Protesto que o mundo novo \\
Manda aos tronos e às nações.
\end{verse}

A praça simboliza a \textit{ágora}, a qual, por sua vez, simboliza o local onde todas as classes sociais se encontram (metonímia para o espaço público). A imagem de raio está relacionada ao sentido iluminista de razão, afastada da ignorância. Algoz como carrasco, tirano, e o povo como Cristo. Barrica como barricada, um obstáculo desenvolvido de maneira improvisada, e muito relacionado com as revoltas populares (Paris e a designação de \textit{cidade das barricadas}). Gigante pois o povo representa a maioria da população. Desgrenhado, enorme e nu pois o povo é pobre, e tem o direito de se revoltar contra a tirania. Os heróis da república romana. Garibaldi foi um dos heróis da unificação italiana, e Kosshut um líder popular indiano (cada um em seus contextos representam a luta contra a tirania). O espaço público deve ser dominado pelo povo. Gênio é aquele que possui a capacidade de falar em nome do povo (meidação pelo poeta). Antro é um local fechado, na qual o calor da razão gerará águias (pensamento autônomo, com a liberdade de voar conforme a vontade). O senhor representa os detentores do poder econômico, os quais possuem tudo, e também desejam se apoderar da praça, o espaço público (também querer o poder político). Anteu é um gigante que, enquanto em contato com a Terra, é indestrutível. Representa o povo, o qual, enquanto estiver no domínio da praça, jamais será vencido (na democracia, torna-se uma força imperiosa, incapaz de ser barrada). Defesa da ideia de democracia e separação dos interesses econômicos e privados dos interesses públicos.

\poemtitle{O navio negreiro}
\begin{verse}[\versewidth]
Algo aqui.
\end{verse}

É interessante notar a relação estabelecida entre o eu lírico e Deus. Mostra-se horrorizado com o tráfico de escravos. Referência aos elementos da natureza, e questionamentos: por que as ondas do mar permitem que esta mancha na história humana aconteça? Há um pedido para que estes impeçam o tráfico.

Contexto da \textit{Lei para inglês ver}, Eusébio de Queirós.

Os desgraçados são os indivíduos sequestrados na África e lavados na condição de escravos para outras localidades. Algoz são os responsáveis por esse tráfico.

Ideia de que Deus e a natureza não se importam com os escravizados, e são cúmplices desse crime. Libérrima como superlativo de livre, e significa liberdade (aquela que inspira os versos do eu lírico).

A terra é a esposa, casada com a luz, local de onde vieram (na visão de Castro Alves, a África é um local muito iluminado — imagem estereotipada do deserto e do continente africano).

Os tigres manchados, com listra, foram reduzidos à condição de míseros escravos, sem ar e sem luz (referência aos porões dos navios negreiros). Sem razão em uma concepção liberal, para a qual o indivíduo se torna plenamente humano com o exercício da liberdade (na qual se inclui a liberdade de pensar e a razão).

Agar era a escrava de Abraão, a qual deu a luz à Ismael. Ismaelitas como muçulmanos. compromisso com as causas políticas de sua época, sobretudo com a abolição da escravidão. Poesia reformista, na qual o poeta é a representação do gênio (condor).

A prosa romântica no Brasil, por sua vez, seguiu um caminho em muitos aspectos semelhante. Lembre-se de que no século XIX não havia internet, cinema, televisão ou rádio, e a principal forma de entretenimento das classes letradas no ambiente doméstico se dava por meio da literatura e, em especial, dos folhetins. Esses impressos surgiram na França e rapidamente alcançaram grande popularidade: eram baratos e de fácil leitura, além de possuírem uma estrutura narrativa com ganchos e reviravoltas de forte apelo popular. No Brasil o processo não seria diferente, ainda com a especificidade de uma minoria alfabetizada, concentrada nas grandes cidades do país (no Rio de Janeiro, por exemplo, a população chegava ao dobro da média nacional).

Nesse período, tornou-se comum o consumo de folhetins populares (o próprio escritor José de Alencar lia folhetins semanalmente para a família, e outros indivíduos não necessariamente alfabetizados). Uma dessas obras mais famosas foi \textit{O guarani}, também de José de Alencar. Pelas cidades comumente haviam grupos que se reuniam semanalmente para a leitura de folhetins, sob a luz dos postes.

Nesse contexto, como se tratava se uma prosa feita para ser lida em voz alta, não cabia uma linguagem demasiadamente coloquial ou oralizada, mas mais simples para caber na boca das pessoas. É interessante como os próprios autores os escreviam tendo em vista que seriam lidos em voz alta.

Camilo Castelo Branco foi um dos primeiros autores de folhetins do país. Já o primeiro folhetim a fazer grande sucesso, e que serviu de inspiração para outros autores, foi \textit{A moreninha}, de Joaquim Macedo (posteriormente reunido em livros). Também é válido mencionar o papel de José de Alencar no processo de construção de um português brasileiro, enfatizando em sua linguagem diferentes colocações pronominais (ênclise, próclise e mesóclise), por exemplo, em uma tentativa de elaborar uma literatura nacionalista.

De maneira geral, no século XIX, o termo \textit{romance} se referia, sobretudo, às novelas. Isso pois, como instrumento de investigação da realidade nacional, haviam três principais formas do gênero narrativo utilizadas: o conto (não adequado para o folhetim, dada a duração), a novela (a melhor forma narrativa para o folhetim semanal, com uma narrativa de média extensão) e o romance (não convinha, pois os romances do século XIX, frequentemente, eram muitos grandes — \textit{Guerra e paz}, por exemplo), as quais também variavam em número de tramas, complexidade do enredo, etc. Curiosamente, na língua inglesa, ocorrera o processo contrário, com os termos \textit{novel} e \textit{romance}.

Em síntese: os escritores brasileiros tiveram um papel fundamental, por meio da literatura, na construção de uma identidade nacional. Em especial, o Brasil carecia do intelectual especializado (paralelamente, na Europa já havia a figura do matemático, filósofo, etc.) e, consequente, não haviam indivíduos aptos a escrever ou mesmo ler textos acadêmicos, e sim literatura, desenvolvida por esses homens de letras. De maneira geral, as produções eram mais simples e, por conseguinte, mais fáceis para transmitir à população a imagem e sentimento de pertencimento. Os livros circulavam, e a literatura, enquanto entretida, também transmitia um conjunto de valores (dentro da literatura, a principal forma que vai sintetizar isso é o romance).

Ao invés de escrever um texto monográfico sobre o período do colonialismo, por exemplo, narra-se um romance contextualizado nesse período. O romance será utilizado para descrever as diferentes regiões do Brasil, como elo entre os indivíduos do Rio de Janeiro e de outras localidades, seja do Norte ou Sul do Brasil. A linguagem da poesia é mais estilizada, com rimas e um vocabulário mais estilizado, sintetizadas em uma métrica e com inúmeras referências culturais, que requererem um leitor mais especializado. O romance, por outro lado, narra uma história de amor com muitas páginas, possibilitando a explicação de referências (leitura facilitada que se abria a um público maior).

José de Alencar tinha um projeto de escrever um romance para cada região do Brasil, com o objetivo de criar um grande panorama da realidade social e política do Brasil (inspiração em Balzac, escritor de \textit{A comédia humana}, que descrevia a sociedades francesas do século XIX).

Algo importante merece ser mencionado: \textbf{a sociedade brasileira não era tão complexa quanto a francesa (urbanizada e industrializada)}. Por essa razão, (?).

\section{José de Alencar}

José de Alencar é o autor mais importante da prosa romântica no Brasil, e escrevera em todas as vertentes do Romantismo no país. Um dos valores burgueses que inspirou o Romantismo foi o nacionalismo, cuja principal manifestação foi o romance histórico (desenvolvido pelo escocês Arthur Scott). No Romantismo europeu, o supracitado romance possuía características medievalistas, com o cavaleiro atuando como protagonista.

O Brasil não passou pela Idade Média (derivada do colonizador) por uma razão cronológica. Dessa forma, quando o romance histórico, de caráter nacionalista, chegou ao país, transformou-se em um romance indianista (adaptação do romance histórico para o contexto histórico brasileiro). Os romances indianistas se passavam, via de regra, no primeiro século de colonização portuguesa (adaptação da Idade Média para o país).

Tudo o que os escritores sabiam acerca dos indígenas se baseava nos escritos de viajantes dos séculos anteriores (alguns dos personagens de \textit{Iracema} de fato existiram). Esse olhar sobre a cultura indígena possuía um viés etnocêntrico. \textit{Viagens maravilhosas}, um gênero literário presente, por exemplo, nas obras de Marco Polo, mesclando os relatos fantasiosos de viagem com os relatos verídicos no imaginário europeu. José de Alencar chegou a citar que já não existiam indígenas no Brasil (descolamento da realidade; indianismo de gabinete/escritório).

Em ordem cronológica, as três principais obras de José de Alencar são \textit{O guarani}, \textit{Iracema} e \textit{Ubirajara}. A figura do indígena cumpre a mesma função do cavaleiro na literatura romântica: síntese das virtudes nacionais (valores como bravura, generosidade, altruísmo, herdados pelos brasileiros por meio do sangue — atavismo).

\subsection{O guarani}

Narra a história do índio Peri, que se apaixona pela filha de um colonizador português (personalidade histórica, e um dos primeiros colonizadores a se estabelecer no Brasil), Cecília. Um padre almeja expulsar a família de colonizadores para explorar as reservas de prata da região. Peri era um indígena forte e corajoso — em cena memorável, levara uma onça-pintada para que Ceci a observasse. Peri toma veneno e se entrega à tribo inimiga, que desejava invadir os estabelecimento dos colonizadores (no entanto, ingere o antídoto antes). O pai de Ceci, com a invasão dos indígenas da tribo inimiga, explode o sótão com pólvora e mata todos os indígenas. Na última cena, Ceci está viajando com o horizonte de fundo sob a piroga construída por Peri, em uma terra já sem habitantes. O cavaleiro, indígena, provando constantemente o seu valor (matar um dragão ou lutar com uma onça). Com a morte dos indígenas e a posterior enchente, há referências à história de Adão e Eva, assim como ao dilúvio, citados na Bíblia.

José de Alencar busca estruturar um mito fundador, a exemplo de \textit{Ulisses}, de Fernando Pessoa, e a história de Rômulo e Remo, de Roma (cada um, em seu contexto, constituem os mitos fundadores de seu povo). Iracema é uma tentativa de criar um mito fundador para o Brasil.

\subsection{Iracema: a lenda do Ceará}

\textit{Iracema} se passa no primeiro século de colonização do Brasil. Narra o conflito da tribo Tabajara, da qual Iracema é integrante, com a tribo Potiguara (\textit{potiguar}, tribo do camarão). Iracema é filha de Araquém, o pajé e líder espiritual da tribo. Assim, possui a função de preparar o chá da Jurema, feito com frutas fermentadas e considerado uma bebida sagrada para os indígenas, frequentemente utilizado nos rituais. Para preparar a bebida, Iracema precisa ser virgem, sendo assim destinada a não se casar (não pode perder a sua pureza).

Certa vez, enquanto se banhava, percebeu que estava sendo observado por Martim, um homem português aliado da tribo Potiguara (Martim deriva de Marte, o deus da guerra romano). A mulher atira uma flecha contra o intruso, assim o ferindo. Arrependendo-se imediatamente, Iracema o leva para a cabana do Araquém e se dispõe a curar de sua ferida; nesse momento, acabam por se conhecer e começam a desenvolver uma relação mais próxima. Ocorre que Martim tem uma noiva à sua espera em Portugal, e assim permanece dividido entre Iracema e a portuguesa. Com a chegada de Martim na tribo, Irapuã, o líder guerreiro dos Tabajara, fica sabendo e se dirige até a cabana do Araquém (é apaixonado por Iracema, e sabe que Martim é um aliado da tribo inimiga). Araquém, por sua vez, recusa-se a entregar o português, afirmando que era um hóspede que estava em seu espaço, cabendo, pois, ao líder o seu tratamento. É interessante notar a presença de elementos gregos como a questão da hospitalidade, assim como a própria cena de Iracema sendo observada por Martim — presente nos vasos com ninfas e sátiros, como o caso de Diana e Acteon; lembre-se, ainda, do canto da \textit{Ilha dos amores}, em \textit{Os lusíadas}. Por fim, ressalta-se a determinação de poder por parte de Araquém.

Outro momento que cabe ser mencionado ocorre após a recuperação de Martim. Iracema o leva à selva, onde prepara o chá da Jurema para o amado. Ao retornarem para a cabana, o português começa a alucinar e chamar pelo nome de Iracema, que se dirige até a rede no qual estava estendido. Iracema perdera a virgindade (é curioso notar que não fora Martim quem conquistara Iracema, mas esta que se entregara em razão de seu imenso amor).

Com o ocorrido, no entanto, Iracema já não pode exercer a sua função na tribo. Nesse contexto, Martim combina com um de seus melhores amigos, Poti (um guerreiro potiguara), um plano para que Iracema fugisse com o amado para a tribo dos Potiguara, onde viveriam juntos. \textbf{Fuvest e a questão sobre autoridades}. Posteriormente, contudo, estabelece-se um conflito entre as tribos, e Martim se encontra no campo de batalha com Caubi, irmão de Iracema; esta o impede de matar Martim ao ameaçá-lo com um arco, Já no conflito de Martim com Irapuã, que quase vencia a luta, é nocauteado por Iracema. Em trecho memorável, Iracema identifica, nos corpos do campo de batalha, o \textit{sangue de seu sangue e a carne de sua carne}. Uma curiosidade é que, quando José de Alencar escrevera o prefácio da obra, comentara que a análise dessa cena é interessante por sua voz de sangue em paralelo com a do coração (Iracema escolheu um caminho sem retorno ao perceber que seu amor por Martim é maior do que qualquer sentimento de culpa pela morte de sua tribo).

Com o fim do conflito, Martim e Iracema se isolam em uma praia, em localidade distante da tribo dos Potiguara, e sob a segurança de Poti. Em determinado dia, Poti convoca o amigo para uma batalha. No momento, no entanto, Iracema estava grávida e, sem escolhas, é deixada sozinha. Nesse momento, recebe visitas de seu irmão, Caubi, que a perdoa pelos acontecimentos passados. Nota, no entanto, o grande sofrimento em que estava Iracema, por saudades do amado, de tal forma que, após o nascimento da criança — um menino —, não conseguia produzir leite. Para tanto, vai até uma alcateia de lobos, oferecendo o seio para que o dilacerassem e o leite pudesse ser expelido. Por essa razão, a criança é chamada de Moacir, filho da dor (tanto física quanto emocional — saudades). Martim retorna, mas Iracema acaba morrendo. Assim, leva o garoto para Portugal, onde é criado pela noiva do homem, anteriormente citada. Anos depois, Moacir, uma figura histórica, retorna para o Brasil, e é uma importante figura para o surgimento do Ceará.

Muitos são os possíveis questionamentos. Por que Iracema é uma história de um mito fundador? Com efeito, não há um processo tão explícito quanto aquele presente em \textit{O guarani}. O que seria, nesse sentido, uma lenda? Qual é a importância do subtítulo\footnote{Note que nem todas as edições possuem o subtítulo.} da obra?

Além disso, ainda resta análise de elementos importantes da obra. Lembre-se de que o Romantismo se diferencia dos movimentos anteriores à medida em que valoriza a originalidade e as características únicas de cada autor. Ainda assim, nas palavras de Octavio Paz, existem figuras de linguagem frequentes como a analogia, metáfora e ironia, em um processo de disruptura entre o significado e o significante — realidade contraditória que gera dissonância e desarmonia. José de Alencar, nesse sentido, utiliza muitos termos do tupi-guarani.

Para melhor compreender a obra como um todo, alguns aspectos merecem atenção. Veja, inicialmente, o foco narrativo em 3ª pessoa, no qual o narrador e o protagonista não coincidem (narrador observador). Foco pois, logo no primeiro capítulo, há uma passagem em que se lê: ``Uma história que me contaram nas lindas várzeas onde nasci [...]'', de tal forma que o narrador se apresenta; também em ``Não sei eu [...]''. A utilização da 1ª pessoa em tais passagens é uma forma de sugerir que a história não é uma invenção, mas uma narrativa contada ao narrador na terra onde nasceu — o Ceará (clima de narração oral, tradicional). Dessa forma, embora predomine o narrador em 3ª pessoa não-personagem, em algumas poucas ocasiões o narrador se utiliza da 1ª pessoa para sugerir uma narração oral e tradicional. É interessante traçar um paralelo com outras obras, como \textit{A volta do parafuso}, na qual o autor se utiliza de um diário como forma de sugerir que a história não é ficcional. De fato, os autores não esperam que os leitores aceitem a ideia. O principal objetivo é a imersão na narrativa. José de Alencar busca constituir uma \textit{lenda} em uma narração tradicional, oral.

``Não o sei eu [...]'' é significativo, à medida em que relativiza a onisciência do narrador, novamente reforçando o processo citado anteriormente.

Algo válido a ser mencionado é a interpretação de Machado de Assis sobre a obra, o qual identificara a imagem de um bardo\footnote{Poeta épico da Idade Média.} indígena que conta a história para a sua tribo (novamente, reforçando a ideia de narrativa, e não de uma história ficcional). José de Alencar e a sensação relatada por Machado de Assis do bardo indígena.

A linguagem utilizada também merece cuidadosa análise. Em \textit{Iracema}, José de Alencar utilizara um estilo diferente daquele empregado em outras de suas obras. Nesse sentido, o autor buscava a criação de uma versão em português da língua falada pelos índios, por meio de diferentes recursos como forma de criar uma autenticidade (baseada unicamente nas suposições do escritor em como os indígenas de comunicavam) da narração. Destacam-se:

\begin{enumerate}
\item Utilização de termos em tupi, geralmente para designar elementos da flora e fauna.
\item Analogias com elementos da natureza brasileiro, com predomínio do símile sobre a metáfora. Acreditava-se que, para línguas mais antigas, faltavam termos conceituais e mais abstratos, de tal forma que seus falantes se referiam a esses termos utilizando-se de outros termos concretos. \textit{Diké}, do grego, hoje significa justiça, mas em sentido original, representada um caminho reto. No símile existe uma mediação (\textit{como algo}, por exemplo), estabelecida por meio de uma relação de semelhante. Na metáfora, por sua vez, há uma relação de identidade (\textit{é algo}, por exemplo). O símile predomina sobre a metáfora uma vez que esta exige um processo mental mais sofisticado para ser estabelecido (maior preenchimento de aspectos subentendidos). Como a língua falada pelos indígenas é primitiva, na visão de José de Alencar, o símile deveria predominar sobre a metáfora.
\item Períodos mais curtos e simples, nos quais a coordenação predomina sobre a subordinação no processo de construção das orações. Na visão do autor, essa linguagem, ao mesmo tempo poética, musical e de caráter arcaico, representa a essência da cultura indígena. Ainda nesse sentido, \textit{Iracema} seria um poema em prosa, à medida em que possui uma linguagem de ritmo e cadência bem marcados. Também se faz presente o uso da linguagem conotativa, sobretudo por meio de analogias (é a obra de José de Alencar com maior recorrência e metáforas).
\end{enumerate}

Retornando à ideia de uma lenda como mito fundador, é válido mencionar que Moacir, personalidade histórica, é descrito por José de Alencar como o primeiro cearense. Além disso, este representa também o povo brasileiro, o qual nasce a partir de Iracema (anagrama de América; representa a terra brasileira — lembre-se que não fora Martim quem conquistara a índia, mas esta quem se entregar, em paralelo ao processo de colonização, não marcado pela violência dos portugueses, mas pela entrega da terra aos colonizadores, em visão conservadora e oficial do bom colonizador até recentemente predominante) e Martim (Marte, o que realça o caráter guerreiro do indivíduo).

Nesse sentido, os brasileiros seriam tão \textbf{generosos} quanto a terra fértil, e dotados da \textbf{coragem} e \textbf{bravura} do povo português (nível simbólico de interpretação de Moacir).

\textit{Eneida}, de Virgílio, é o principal mito fundador do ocidente. O descendente de Enéias que funda Roma é Rômulo, irmão de Remo. O pai dos irmãos é justamente Marte. Nos tempos do imperador Augusto, Roma era o império mais poderoso do mundo ocidental, e fora criado por um filho de marte. Nesse contexto, Moacir corresponde a Rômulo, e o destino do Brasil, pois, é ser tão grandioso quanto fora Roma na Antiguidade (o Brasil como o Quinto Império — persas, macedônios, romanos, etc.).

Em síntese, \textit{Iracema} pertence a uma vertente indianista da prosa romântica no Brasil. Na Europa, se possuíamos um nacionalismo evidenciado por meio dos romances históricos medievalistas, com a figura do cavaleiro medieval, no Brasil tivemos o indígena (Poti, de \textit{O guarani}, é um notável exemplo). Moacir é tratado como o primeiro cearense, e serve de representação para o próprio povo brasileiro, em paralelo com a figura de Rômulo e Remo. Assim sendo, o Brasil estaria destinado a se tornar a nova Roma.

Um aspecto ainda não citado é a figura de Iracema como modelo para o comportamento feminino. Comumente identificamos, em obras com o caráter de um mito fundador, a presença de modelos, figuras idealizadas, em meio ao processo de construção de uma identidade. José de Alencar descreve Iracema como um indivíduo que sempre coloca os interesses da família acima dos seus. Lembre-se, nesse sentido, do conflito entre a voz do sangue e a voz do coração (durante o combate entre as tribos), do momento no qual permite que Martim a deixe para lutar na guerra (seguido por um período de intensa dor e sofrimento), assim como o sacrifício realizado para alimentar o filho enquanto a figura de Martim está ausente. Identificamos o arquétipo da \textbf{santa mãezinha}, o qual remonta ao Brasil dos séculos XVII e XVIII.

A colonização do Brasil teve início com famílias portuguesas abastadas — agraciadas por Portugal com terras na região —, caracterizadas por um forte modelo patriarcal. Lembre-se da etimologia do termo família, do latim \textit{famulus}, escravo doméstico, servo. Assim, sem a maior presença do Estado ou da Igreja nesse período, não há maiores conflitos entre essas esferas e a figura de autoridade do patriarca. Aos poucos, o brasileiro se acostuma com a figura masculina forte. O ideal supracitado é mais do que o papel da mulher no modelo patriarcal (língua geral e o tupi-guarani), em meio a preocupação de Portugal — e da Igreja Católica, no contexto da Contrarreforma — com o distanciamento cultural dos colonos. Nesse cenário, o elo de ligação, a figura que transmitirá os valores para as futuras gerações, são as mulheres (figura materna). Ana Maria Machado identifica, nesse contexto, um processo de exportação de esposas. A mulher como representação dos valores católicos, forma de transmissão dos valores culturais europeus e cristãos para as gerações futuras. Iracema, assim, representa uma figura crística, o Pelicano Eucarístico, que sacrifica a própria carne (terra) para a sobrevivência de Moacir (brasileiros). A independência do país também é simbolizada com a saída de Moacir e o posterior retorno.

Surge um questionamento: como poderia Iracema, pagã, representar os ideais cristãos? Para responder tal pergunta, devemos retornar ao pensamento de Santo Agostinho, \textit{fiat lux}: cada ser humano recebe o verbo divino, que ecoa em seu coração — a verdade da alma humana é o cristianismo, antes deste próprio. A segunda natureza remonta ao paganismo e às outras religiões. A verdadeira natureza da alma humana (Iracema representa o cristianismo primitivo, natural).

José de Alencar perpetua a visão patriarcal do Brasil, (?), conservadora, na qual a colonização não é interpretada como um processo de invasão.

\section{Manuel Antônio de Almeida}

Ainda nada por aqui.

\subsection{Memórias de um sargento de milícias}

Caio Prado Júnior foi um dos primeiros intelectuais brasileiros a estudar a figura do \textit{homem libre e pobre}. Em sua obra \textit{Formação do Brasil contemporâneo}, descreve a forma pela qual os proprietários burgueses exploravam a mão de obra escrava. Nesse contexto, permaneciam à deriva os indivíduos que não eram escravos, mas tampouco proprietários; ainda assim, eram livres, e garantiam a sua sobrevivência por meio de empregos informais. Essa figura seria, posteriormente, também estudada por Antônio Cândido (retornaremos a esse autor nos próximos capítulos).

É curioso notar a forma como esses indivíduos viviam em realidade muito próxima à escravidão (brancos, pretos e mestiços). Identifique, também, a distinção realizada entre os trabalhos braçais — indignos, realizados pelos escravos — e os ofícios liberais (como professores e médicos). Nesse contexto, o homem livre buscava trabalho nas atividades braçais, mas ainda assim queria ser visto como indivíduo livre. É curioso mencionar, nesse sentido, a simbologia do terno branco, vestimenta facilmente manchada em atividades fisicamente mais exigentes.

Surge, nesse contexto, a figura do malandro, que não é um marginal, mas desliza entre o lícito e o ilícito de acordo com as suas necessidades e interesses, presente em todas as personagens da obra. O homem livre e pobre e a instabilidade social no quadro da sociedade brasileira.

Durante a República Velha, era comum a presença dos [copoeiros] nas eleições, e a utilização dos malandros como forma de controle.

Dialética\footnote{Duas proposições igualmente verdadeiras, mas contraditórias, e que levam a uma síntese.} da malandragem: de um lado, a ordem — lei, trabalho, família e casamento —, e de outro, a desordem — crime, promiscuidade, amassamentos, vadiagem.

Na obra, identificamos em Leonardinho a figura de um malandro. Representam a ordem, por sua vez, as figuras como o barbeiro, a parteira, Dona Maria, Major Vidigal, Luisinha. A desordem é representada por Teotônio e Vidinha. Ainda assim, é interessante notar como todas as personagens da obra transitam entre esses polos.

Um momento que merece maior atenção envolve a figura de Major Vidigal, que ao abrir a porta para atender as mulheres que pediam a libertação de Leonardinho, vestia roupas mescladas entre o oficial e o malandro.

Por fim, é interessante notar como o julgamento do narrador nunca é moral (universo sem culpa). Exclusão da ordem: como exigir uma família em ordem, ou mesmo um emprego estável?

A malandragem é tratada como engraçada, pitoresca, inerente ao próprio estilo de vida dos indivíduos (sem outra escolha).

Machado de Assis e a malandragem das classes dirigentes.

Não há qualquer menção ao trabalho escravo ou ao trabalho por si mesmo, com exceção do barbeiro.

Trata-se de um romance urbano, que se passa na cidade do Rio de Janeiro (dentre as vertentes indianista, urbana e regionalista da prosa romântica no Brasil). É interessante notar, em relação a essas produções, o elevado grau de idealização das personagens, e a característica presença de protagonistas e vilões bem definidos, e frequentemente com finais felizes — ao menos no Brasil, já que em Portugal a preferência era pelo desfecho trágico. Chamamos essas obras de \textit{romances folhetinescos}.

É interessante notar a presença de um \textbf{realismo satírico}, também presente em \textit{Lira dos vinte anos}.

Logo a primeira linha do capítulo introdutório, ``Era no tempo do rei'', nos situa no contexto da vinda da família real para o Brasil (Dom João VI no cargo de rei), no ano de 1808, também limitando o enredo ao ano de 1822. Um dos passageiros de um dos navios que trouxera tais indivíduos, o português Leonardo Pataca, nos é apresentado. Durante a viagem, conhece Maria das Hortaliças, a qual engravida já em território brasileiro — dessa relação nasce um garoto, também Leonardo (duas novas personagens cabem ser mencionadas: os padrinhos do menino, um barbeiro e uma parteira). No Brasil, Leonardo Pataca assume uma posição como meirinho (oficial de justiça). No entanto, é abandonado por Maria, que fugira com outro homem. O filho, também deixado de lado, cresce com um comportamento rebelde, e também não tivera um ambiente confortável para seu desenvolvimento, em razão do comportamento e falas de seu pai (em dado momento, chega a enunciar: ``És filho de uma pisadela e de um beliscão; mereces que um pontapé te acabe a casta''). Nesse ambiente insustentável e que não despertava bons sentimentos, Leonardinho começa a ser criado pelo barbeiro (padrinho), que chega a tentar convencer o garoto a se tornar padre ou advogado, mas cujo pedido é negado (Leonardinho não possuía relações positivas com os estudos).

É curioso notar como Pataca é descrito como um homem que está sempre atrás de alguma mulher. A cigana e o padre que expulsou Leonardinho.

O barbeiro era amigo de Dona Maria, mulher rica e tia de Luisinha, órfã, que possuía um dote demasiadamente atrativo para Leonardinho.

Durante a festa do Espírito Santo, a principal festa do Rio de Janeiro no século XIX, Leonardinho e Luisinha flertam. O momento, no entanto, é interrompido por José Manuel — indivíduo trabalhador com a vida mais bem resolvida quando comparada à de Leonardinho. Assim sendo, Dona Maria trata de afastar Leonardinho de sua sobrinha, uma vez que José Manuel representava uma melhor escolha para o futuro da garota.

Como forma re resolver a situação, a parteira elabora uma mentira em relação a José Manuel, que posteriormente é descoberta. Ainda nesse contexto, o barbeiro falece, deixando a sua herança para Leonardinho.

Posteriormente, o barbeiro morre. Leonardinho retorna para a casa do pai, mas é novamente expulso (briga com a namorada). Já na rua, o jovem ouve uma serenata cantada por um grupo. A dona da voz era Mudinha, mulata pela qual Leonardinho imediatamente se apaixona. Com isso, Leonardinho vai morar com a família da jovem (que vivia com viúvas) como um agregado\footnote{Cabe mencionar que a figura do agregado também se fará presente em outros momentos, como na obra de Machado de Assis e de Aluísio Azevedo.}. Leonardinho então desenvolve uma relação com Vidinha, em certa medida conturbada, uma vez que muitos dos primos da jovem eram apaixonados por ela. Também havia grande pressão para que Leonardinho conseguisse um emprego. Grande parte disso se deve à forma como a vadiagem era vista na época. Somos apresentados à figura de Major Vidigal, chefe de polícia que perseguia\footnote{Processo cartunesco.} os vadios (prática considerada criminosa na época). Enfim, Leonardinho consegue um emprego na despensa do Palácio Real; muda-se para o quarto do amigo que também trabalhava no local, e começa a paquerar a sua esposa. Quando descoberto, foge.

Em dado momento, Leonardinho é capturado por Major Vidigal. O homem, no entanto, resolve torná-lo um policial também responsável por deter os vadios. Leonardinho, contudo, começara a se envolver nas festas que deveria intervir. Em momento cômico, Leonardinho, junto de outros vadios, executa uma encenação do velório de Major Vidigal, o que leva à prisão de todos.

Nesse momento, a parteira, madrinha de Leonardinho, reconcilia-se com Dona Maria e pede por ajuda. As mulheres, então, dirigem-se à Maria Regalada\footnote{Outrora prostituta, disse que moraria com o Major.}, pedindo por ajuda. Em conversa íntima, a mulher convence Major Vidigal a soltar Leonardinho e promovê-lo a sargento de milícias.

Posteriormente, José Manuel morre e Luisinha permanece disponível para Leonardinho. Eles se casam, e Dona Maria e Leonardo Pataca também morrem, deixando as heranças para o casal.

A história do barbeiro ainda esconde mistérios. Era um dentista e médico, que curara acidentalmente um escravo, falhando em tratar do capitão de um navio. O homem roubara o dote da filha do capitão.

Enredo utilizado como contraponto à prosa romântica. Possui um protagonista que não é virtuoso. Também o narrador frequentemente critica o Romantismo (a ideia do romântico relacionado à ingenuidade), em elementos como a constante idealização das personagens (virtuosidade) e o enfoque do sentimentalismo nos diferentes relacionamentos. Leonardinho é o típico malandro (enraizamento social).

Após a encenação do velório, Leonardinho recebe uma segunda chance. Teotônio, malandro, e a imitação do Major, presente na festa de batizado da irmã de Leonardo, deveria ser capturado por Leonardinho, que não o faz (então é preso pelo Major em definitivo).

Presença do realismo satírico e do antisentimentalismo.

Antirromantismo, presente na obra de Álvares de Azevedo.

É interessante notar as referências aos contos de fadas, em especial nas figuras do barbeiro e da parteira, quase como padrinhos mágicos de Leonardinho, e de Major Vidigal como bicho-papão. A diversos personagens não é atribuído um nome, novamente remetendo ao universo dos contos maravilhosos, ainda que, paralelamente, a parteira e o barbeiro desempenhem funções específicas na sociedade brasileira. \textbf{Arquétipos literários}: a parteira como um tipo social (profissão), junto do arquétipo literário da fada madrinha; Vidigal, por sua vez, uma figura histórica junto do arquétipo do bicho-papão. Por fim, é interessante notar os diferentes acontecimentos que quanto favorecem quanto prejudicam Leonardinho.

Ambivalência do registro ficcional, e delimitação histórica, crônica de costumes. Perspectiva histórica, jornalística para com a sociedade brasileira do período representado, presente, por exemplo, na descrição da festa do Espírito Santo, ou do pátio militar da cidade. Representação em minúcias de costumes e tradições, e a convivência simultânea com os arquétipos.

Outro aspecto importante a ser mencionado envolve o \textbf{olhar jornalístico acerca da sociedade carioca}. A crônica, a princípio, é um texto historiográfico. A crônica da semana, por exemplo, resumia os fatos mais importantes que aconteceram. Escrita pelo homem de letras, escritor (lembre-se dos intelectuais de formação humanística), nota-se o surgimento de uma preocupação estilística com os textos.

É possível identificar duas partes diferentes na obra: na primeira parte, notamos uma narração empregada ainda na ideia da crônica de costumes; a segunda parte, por sua vez, após a briga de Leonardinho com a madrasta, possui um maior foco na trajetória do protagonista.

Com o fim da leitura, notamos que a obra é consideravelmente diferente das outras produções românticas do período. Algumas são as possíveis explicações.

Em primeira análise, poderíamos afirmar que se trata de uma obra pré-realista, extemporânea (ideia que, inclusive, é considerada ultrapassada há décadas). Veja, no entanto, que o realismo presente na obra se assemelha muito mais ao realismo satírico presente na \textit{Lira dos vinte anos}, do Romantismo. Cabe, portanto, realizar a seguinte distinção:

\begin{enumerate}
\item O \textbf{realismo} é uma representação da vida cotidiana das pessoas comuns. Referente à comédia, da Grécia.
\item O \textbf{Realismo}, em caixa alta, refere-se a uma escola artística surgida na França nos anos de 1830. Caracterizada por um tratamento sério da vida das pessoas comuns.
\end{enumerate}

Mário de Andrade, por sua vez, identificara na obra um \textbf{romance picaresco}, de origem na literatura espanhola. O termo era inspirado na figura de Pícaro, um garoto, moleque, menino pobre que se utiliza de sua esperteza para sobreviver (como a personagem Chaves, de grande sucesso na televisão brasileira). Leonardinho seria, nessa visão, uma figura picaresca.

Antônio Cândido realizara uma análise diferente. Considerara o fato de que Leonardinho não é um exemplo de esperteza e teimosia, não representando assim uma peça ativa para os acontecimentos do enredo. Ainda nesse sentido, nunca tivera a sua sobrevivência comprometida. Para o autor, assim, a obra bebe da tradição portuguesa com Pedro Malasartes. Nesse sentido, possui um caráter ambíguo, que do imaginário popular português se fixara ao imaginário carioca, originando o \textbf{romance malandro}. Outras obras semelhantes, nesse sentido, incluem \textit{Ópera do malandro}, \textit{A volta do malandro} e \textit{Os 3 malandros in concert}.

Lembre-se da figura do boto cor-de-rosa, que vestina um terno branco, sapatos bicolor e um chapéu panamá. Possuía uma personalidade de namorador, e carregava uma navalha (frequentemente se envolvia em conflitos). Figura também de Teotônio e Chico Juca (contratado pelo pai de Leonardinho) e, para Cândido, presente em todas as personagens.

O carioca se torna o arquétipo do brasileiro — lembre-se de que o Rio de Janeiro era a sede do poder na época — e, na Era Vargas, o samba fora promovido como ritmo nacional.