\chapterimage{chapter_head_2.pdf}

\chapter{Barroco no Brasil}

Ainda no Brasil, no final do século XVIII, tivemos o Barroco mineiro, tardio (enquanto a Europa vivia o Neoclassicismo, descrito na próxima parte). Haviam certas diferenças de material (preferência pela pedra sabão ao mármore), e a técnica de escorço também fora amplamente utilizada.

%Nesse parte, acrescentar o presépio feito por Aleijadinho

\textbf{Carece de mais informações.}

Na literatura, o maior expoente do Barroco no Brasil foi Gregório de Matos, tema principal do presente capítulo.

\section{Gregório de Matos}\index{Gregório de Matos}

Gregório de Matos (1636-1696) foi uma figura muito popular em vida, membro da elite de Salvador e de relações sanguíneas com a família portuguesa. Em uma época na qual os poemas eram passados de boca em boca, a tradição oral era muito presente na sociedade, e por isso existem muitas obras atribuídas\footnote{No século XIX teve início o primeiro esforço em compilar e organizar os seus poemas. Somente no século seguinte, no entanto, tal tarefa adquiriria um caráter mais sistemático e científico.} ao autor. Envolvido em diversas polêmicas (em especial, por suas poesias satíricas, na qual ridicularizava dos poderosos, ganhou a alcunha de \textbf{Boca do inferno}), Gregório teve de se exilar em Portugal.

Em uma época na qual a Coroa proibia a impressão de poesias, e com uma taxa de alfabetização da população inferior a 10\%, a palavra oral é de grande relevância e influência (compare com uma situação de bullying, com linguagens refinadas e populares). Com isso, o autor tivera sua obra incorporada à cultura popular.

Gregório produzira poesias em diversas vertentes:

\begin{enumerate}
\item \textbf{Poesia lírico-amorosa}, de caráter petrarquista\footnote{Lembre-se de Francesco Petrarca, um dos expoentes do Humanismo italiano. Com a obra \textit{Cancioneiro} e sua musa Laura, tornara-se um clássico moderno, estimulando a emulação (petrarquiana) por diversos autores (caracterizadas por elementos como metáforas e exaltação/sublimação da figura feminina). O célebre soneto de Camões, \textit{Alma minha gentil que te partiste}, constitui exemplo notável.}.
\item \textbf{Poesia meditativa}, de temática filosófica.
\item \textbf{Poesia devocional}, de temática religiosa.
\item \textbf{Poesia encomiástica}, de elogio a uma figura pública. Note que, assim como a poesia satírica poderia acabar com a reputação de um indivíduo, a poesia encomiástica possuía o efeito contrário.
\item \textbf{Poesia Satírica}, do gênero satírico (utilização do humor e do ridículo\footnote{Ridículo deriva do latim \textit{rides}, rir.} para criticar uma pessoa, grupo, instituição ou ideia (humor mordaz). Veja que o louvor e a sátira eram frequentemente aplicados a uma mesma figura pública, e constituem gêneros altamente convencionais, que variam de acordo com as circunstâncias. Ainda nesse sentido, note que os poemas nem sempre expressam o que o escritor pensa a respeito dos indivíduos, e que sátira e louvor não são conceitos necessariamente contrários (lembre-se do decoro: cada circunstância exige uma forma apropriada). Esse apontamento ficará mais claro adiante.
\end{enumerate}

Os poemas de Gregório apresentam, frequentemente, títulos longos, autoexplicativos, não necessariamente dados pelo autor. Em sua obra também se identificam tanto elementos conceptistas, como o braço do menino, o pecador sofista e o sol que não dura mais que um dia, assim como elementos cultistas, observados por meio de antíteses.

\poemtitle{Buscando a Cristo}
\begin{verse}
A vós correndo vou, braços sagrados, \\
Nessa cruz sacrossanta descobertos \\
Que, para receber-me, estais abertos, \\
E, por não castigar-me, estais cravados.

A vós, divinos olhos, eclipsados \\
De tanto sangue e lágrimas abertos\footnote{Cobertos, na edição de Sérgio Buarque de Holanda.}, \\
Pois, para perdoar-me, estais despertos, \\
E, por não condenar-me, estais fechados.
				
A vós, pregados pés, por não deixar-me, \\
A vós, sangue vertido, para ungir-me, \\
A vós, cabeça baixa, p'ra chamar-me.
				
A vós, lado patente, quero unir-me, \\
A vós, cravos preciosos\footnote{Pregos (crucificação).}, quero atar-me, \\
Para ficar unido, atado e firme. \\
\end{verse}

Os braços sagrados representam, por metonímia, a figura de Jesus na cruz. Há grande destaque para a misericórdia divina, com braços abertos para acolher o pecador e presos para condená-lo. Os divinos olhos de Jesus também expressam a antítese citada, e os pés pregados garantem que o pecador não será abandonado. Ao longo dos versos, novamente, observa-se o emprego de metonímias.

\poemtitle{Moraliza o poeta nos Ocidentes do Sol a inconstância dos bens do mundo}
\begin{verse}
Nasce o Sol, e não dura mais que um dia, \\
Depois da Luz se segue a noite escura, \\
Em tristes sombras morre a formosura, \\
Em contínuas tristezas a alegria.
				
Porém se acaba o Sol, por que nascia? \\
Se é tão formosa a Luz, por que não dura? \\
Como a beleza assim se transfigura? \\
Como o gosto da pena assim se fia?
				
Mas no Sol, e na Luz falte a firmeza, \\
Na formosura não se dê constância, \\
E na alegria sinta-se tristeza.
				
Começa o mundo enfim pela ignorância, \\
E tem qualquer dos bens por natureza \\
A firmeza somente na inconstância. \\
\end{verse}

Utilização frequente de antíteses, e a construção da beleza como algo que se torna uma sombra à medida em que envelhece. Ocorrência de elipse com o verbo morrer. Não se pode esperar constância na beleza, e um dia a alegria tem um fim. Ideia de que a única coisa que nunca muda é o próprio fato de que as coisas mudam (inconstância como única constância). Identificamos o lugar-comum da \textbf{inconstância}, efemeridade: o indivíduo que aspira a eternidade (ob[rs]era?) a efemeridade da vida.

\poemtitle{A Jesus Cristo nosso senhor}
\begin{verse}
Pequei, Senhor; mas não porque hei pecado, \\
Da vossa alta clemência me despido\footnote{Despeço.}; \\
Porque, quanto mais tenho delinquido, \\
Vos tenho a perdoar mais empenhado.
				
Se basta avos irar tanto pecado, \\
A abrandar-vos sobeja um só gemido: \\
Que a mesma culpa, que vos há ofendido, \\
Vos tem para o perdão lisonjeado.
				
Se uma ovelha perdida e já cobrada\footnote{Recuperada.} \\
Glória tal e prazer tão repentino \\
Vos deu, como afirmais na sacra história, 
				
Eu sou, Senhor, a ovelha desgarrada, \\
Cobrai-a; e não queirais, pastor divino, \\
Perder na vossa ovelha a vossa glória. \\
\end{verse}

Ideia de que, quanto mais peco, mais Jesus se empenha em perdoar. Com um pecado, basta um gemido (dor e sofrimento) para ser perdoado. Construção do \textbf{sofismo} de que Jesus é gratificado com o pecado, em meio aos anseios do indivíduo que quer desfrutar dos prazeres terrenos, mas que ainda deseja a salvação. Se Jesus é capaz de perdoar o soldado de orelha perdida, também tem capacidade de perdoar o eu lírico. Contraste com a salvação em cada alma que não é salva.

\poemtitle{Achando-se um braço perdido do Menino Deus de N. S. das Maravilhas, que desacataram infiéis na Sé da Bahia}
\begin{verse}
O todo sem a parte não é todo; \\
A parte sem o todo não é parte; \\
Mas se a parte o faz todo, sendo parte, \\
Não se diga que é parte, sendo o todo. 
				
Em todo a Sacramento está Deus todo, \\
E todo assiste inteiro em qualquer parte, \\
E feito em partes todo em toda a parte, \\
Em qualquer parte sempre fica todo. 
				
O braço de Jesus não seja parte, \\
Pois que feito Jesus em partes todo, \\
Assiste cada parte em sua parte.
				
Não se sabendo parte deste todo, \\
Um braço que lhe acharam, sendo parte, \\
Nos diz as partes todas deste todo. 
\end{verse}

Construção de uma tautologia a partir de uma imagem quebrada. Deus se faz todo no rito da Eucaristia (Igreja e fiéis). O braço é um todo, e a parte, pois, é um todo (paradoxo).

\poemtitle{À cidade da Bahia}
\begin{verse}
Triste Bahia! ó quão dessemelhante \\
Estás e estou do nosso antigo estado! \\ 
Pobre te vejo a ti, tu a mi empenhado\footnote{Endividado.}, \\
Rica te vi eu já, tu a mi abundante.
				
A ti trocou-te\footnote{Duplo sentido, de comerciar e modificar. Indica a exploração da cidade de Salvador pelo comércio internacional, e a forma pela qual, a partir do Pacto Colonial, a riqueza da cidade é espoliada pelo mercado (inserção desfavorável do Brasil no sistema capitalista). Com efeito, Salvador exporta açúcar, de baixo valor agregado, e importa manufaturas, como tecidos.} a máquina mercante\footnote{Em referência às naus que aportavam para comerciar, símbolo do comércio internacional como um todo (reflexão sobre o capitalismo mercantil).}, \\
Que em tua larga barra\footnote{Litoral, orla, praia (como em ``barra da calça''), limites da terra antes do mar.} tem entrado, \\
A mim foi-me trocando, e tem trocado, \\
Tanto negócio e tanto negociante.
				
Deste em dar tanto açúcar excelente \\
Pelas drogas inúteis, que abelhuda \\
Simples\footnote{Simplório, ingênuo.} aceitas do sagaz\footnote{Esperteza, astúcia.} Brichote\footnote{Designação pejorativa dos ingleses. Lembre-se de que, desde o fim da União Ibérica, Portugal criara forte dependência da Inglaterra (note a relação Inglaterra $\Rightarrow$ Portugal $\Rightarrow$ Brasil).}.
				
Oh se quisera Deus, que de repente \\
Um dia amanheceras tão sisuda\footnote{Sagaz, astuta.} \\
Que fora de algodão o teu capote! \\
\end{verse}

Por metonímia, Bahia é sinônimo para a cidade de Salvador. O eu lírico identifica a si mesmo e a população como muito diferentes do que um dia foram: no passado, a região era rica e o eu lírico, próspero. No entanto, agora a cidade está pobre e o eu lírico endividado, consequência das ações de negócios e negociantes. Lembre-se de que Gregório é ligado à elite produtora de açúcar. Assim, inserido no contexto da emergência de uma nova elite baseada no comércio internacional, ele, da elite tradicional, começa e se endividar. Com efeito, as principais riquezas já não estavam mais relacionadas ao açúcar, mas ao tráfico de escravos. Além disso, os mesmos indivíduos que executavam tal prática também concediam empréstimos de dinheiro.

Por todo o soneto, é evidente o sentimento de \textbf{desqualificação social}, a perda de prestígio social e poder econômico da classe da qual Gregório fazia parte, aliado a um \textbf{ressentimento social}, rancor. Na visão do autor, ele, por ser um fidalgo, deveria ser o detentor do prestígio, e não uma nova elite, muitas vezes de origem mestiça, nas classes trabalhadoras (visão racista acerca da sociedade brasileira). Ao fim, Gregório descreve o fato de que o algodão era menos caro do que o linho inglês (símbolo de status), e espera que um dia a colônia perceba que não precisa do tecido da Inglaterra, pois já possui o algodão.

\poemtitle{Contemplando nas cousas do mundo desde o seu retiro, the atira com o seu ápage, como quem a nado escapou da tormenta}
\begin{verse}
Neste mundo é mais rico o que mais rapa\footnote{Rouba.}: \\
Quem mais limpo se faz, tem mais carepa\footnote{Caspa, sujeira do cabelo. Neste caso, quem se diz honesto é o mais corrupto.}; \\
Com sua língua, ao nobre o vil decepa\footnote{O nobre cai nas mãos dos mal-intencionados. O vil fala mal dos bons, mas deveria ocorrer o contrário. Note a catacrese presente no termo \textit{vil}.}: \\
O velhaco maior sempre tem capa\footnote{Indumentária, símbolo de status social utilizada pelos nobres (velhaco maior). Nesse contexto, quem tem prestígio social são os enganadores, que o conquistam por meio de manipulações.}.
			
Mostra o patife da nobreza o mapa\footnote{Exibe genealogia, pretende-se descendente de linhagem nobre. Mapas genealógicos eram frequentemente falsificados.}: \\
Quem tem mão de agarrar, ligeiro trepa\footnote{Ascende socialmente.}; \\
Quem menos falar pode, mais increpa\footnote{O mais corrupto é o que mais reclama da corrupção.}: \\
Quem dinheiro tiver, pode ser Papa\footnote{Para ser Papa, não é preciso ser virtuoso. Apenas dinheiro é necessário (o cargo de Papa como metonímia para qualquer cargo da sociedade).}. 
			
A flor baixa\footnote{Rasteira, comum.} se inculca por tulipa\footnote{Ornamental, reconhecida pela beleza.}; \\
Bengala hoje na mão, ontem garlopa\footnote{Metonímias da condição social, opostas ironicamente: hoje \textit{bengala}, índice de fidalguia, ontem \textit{garlopa}, ferramenta de marcenaria, para aplainar madeira grossa, índice do trabalho braçal (mal visto, sobretudo em uma nação escravista, que contamina tais ofícios com estigmas).}: \\
Mais isento se mostra o que mais chupa\footnote{Puxar o saco de alguma figura (são as pessoas mais interesseiras e que se proveitam das outras).}.
			
Para a tropa do trapo\footnote{Preconceito com a classe trabalhadora.} vazo a tripa\footnote{Tem o sentido de defecar. Manifestação máxima de desprezo pela ``tropa do trapo'', i.e., a fidalguia baiana sem tradição. Nota-se, também, o emprego da paranomásia, trocadilho com a proximidade sonora dos termos do verso.}, \\
E mais não digo, porque a Musa\footnote{Inspiram o trabalho artístico e poético.} topa \\
Em apa, epa, ipa, opa, upa\footnote{Combinação difícil de palavras. Veja que, nesse verso, Gregório transpassa a ideia de que encerrara o soneto pois já mostrara sua capacidade em rimar com ``apa, epa, ipa, opa, upa'', e poderia continuar até quando quisesse.}. 
\end{verse}

No soneto, Gregório evidencia que aqueles que detêm o \textbf{capital cultural}, o conhecimento, são eles, a elite. Identificamos o lugar-comum (tema recorrente na tradição literária) do \textbf{mundo às avessas}\footnote{Lembre-se, nesse sentido, de falas como ´´bom mesmo era na minha época''.}, em que os valores foram invertidos na sociedade. Veja, no entanto, que o autor não chega a um lugar-comum mais extremo, grotesco, mas se apropria desse tema da tradição para dar-lhe um significado específico relacionado ao seu contexto social, no qual a sua classe de origem, a elite tradicional dos proprietários de terra, perde o prestígio social e a força econômica.

Um apontamento realizado ao longo da análise reside nas diferentes facetas do \textbf{capital}, seja ele social ou econômico, por exemplo. Ao estudarmos o Naturalismo e, em especial, Aluísio Azevedo (1857-1913), tais temas serão aprofundados.

\poemtitle{Ao mesmo assunto}
\begin{verse}
Um calção de pindoba\footnote{Palmeira, coqueiro.}, a meia zorra\footnote{Péricles Eugênio da Silva Ramos supõe caindo.}, \\
Camisa de urucu\footnote{O corpo pintado de vermelho, com a tinta do fruto.}, mantéu de arara, \\
Em lugar de cotó\footnote{Espada curta.}, arco e taquara, \\
Penacho de guarás, em vez de gorra.
			
Furado o beiço, e sem temor que morra \\
O pai, que lho envasou cuma titara\footnote{Nome de palmeira, aqui vareta.}, \\	
Porem a Mãe a pedra lhe aplicara \\
Por reprimir-lhe o sangue que não corra.
			
Alarve\footnote{Tolo, burro, imbecil.} sem razão, bruto sem fé\footnote{Não possui o refinamento da religião, sem leis.}, \\
Sem mais leis que a do gosto, quando erra, \\
De Paiaiá tornou-se em abaité\footnote{Gente feia, repelente.}.
			
Não sei onde acabou, ou em que guerra: \\
Só sei que deste Adão de Massapé\footnote{Forma de construção tipicamente indígena.} \\ 
Procedem os fidalgos desta terra. 
\end{verse}

Um aspecto relevante do estilo poético de Gregório, e que mais o diferencia dos outros autores do período barroco, é a incorporação de termos com origens indígena e africana (língua brasílica). Por essa razão, podemos erroneamente concluir que era um escritor protonacionalista, que talvez demonstrasse a cultura nacional em detrimento da europeia. Diversos são os motivos que invalidam tal apontamento.

Primeiramente, veja que o poema não é muito prestigioso, utilizando-se de superstições acerca da cultura desses povos, além de termos como ``meia-zorra'' e ``taquara'' em contextos depreciativos. Ademais, note que Gregório não utilizara termos de tais culturas fora do gênero satírico. Ocorre que o escritor se apropriara destes termos para rebaixar e ridicularizar, neste caso, a cultura indígena local. Por fim, ao não valorizá-los e, pelo contrário, denegri-los\footnote{Termo antigo, de origem no latim \textit{denigrare}, manchar a reputação.}, insinua que a nova elite brasileira, proveniente desses povos, também é ignorante e selvagem.

Títulos como esse são comuns na obra de Gregório de Matos. Para evitar confusões, considere que o poema reproduzido acima é o segundo seguinte ao soneto \textit{Aos principais da Bahia chamados os caramurus}.

\poemtitle{Disparates na língua brasílica a huma cunhaã, que ali galanteava por vicio}
\begin{verse}
Indo à caça de tatus \\
encontrei Quatimondé \\
na cova de um jacaré \\
tragando trezes Teiús: \\
eis que dous Surucucus \\
como dous Jaratacacas\\
vi vir atrás de umas Pacas,\\
e a não ser um Preá \\
creio, que o Tamanduá \\
não escapa às Gebiracas.
			
De massa um tapiti, \\
um cofo de Sururus, \\
dous puçás de Baiacus, \\
Samburá de Murici: \\
Com uma raiz de aipi \\
vos envio de Passé, \\
e enfiado num imbé \\
Guiamu, e Caiaganga, \\
que são de Jacaracanga \\
Bagre, timbó, Inhapupê.
			
Minha rica Cumari, \\
minha bela Camboatá \\
como assim de Pirajá \\
me desprezas tapiti: \\
não vedes, que murici \\
sou desses olhos timbó \\
amante mais que um cipó \\
desprezado Inhapupê, \\
pois se eu fora Zabelê \\
vos mandara um Miraró.
\end{verse}

Disparate é algo absurdo. Cunhaã, por sua vez, em tupi-guarani, significa moça, mulher jovem. Galanteava (cortejava, namorava) por vício (qualidade imoral, antônimo de virtude, comportamento reprovável). Nesse caso, o namoro imoral de uma jovem refere-se à prostituição com uma indígena.

Logo no início do poema, é notória a extensa referência a animais como tamanduás, aves e serpentes. Note, também, que a dualidade (oposição de valores) da preocupação com a vida após a morte e o apego aos prazeres da vida também se fazem presentes nos escritos do autor.

Além disso, também identificamos uma dualidade entre a poesia lírico-amorosa e erótica produzidas por Gregório: de um lado temos uma poesia lírico-amorosa petrarquista, caracterizada pela exaltação e sublimação da figura feminina, destinada às mulheres caucasianas e de classe social elevada; por outro lado, verificamos uma poesia de baixo estilo (palavras de baixo calão, erotização), com apelo fortemente carnal, de caráter até mesmo pornográfico, destinada às mulheres de origem indígena e africana, assim como às mestiças (a \textit{tropa do trapo}).

\poemtitle{Juízo anatômico dos achaques que padecia o corpo da República, em todos os membros, e inteira definição do que em todos os tempos é a Bahia}
\begin{verse}
1 \\
Que falta nessa cidade? \dotfill Verdade. \\
Que mais por sua desonra? \dotfill Honra. \\
Falta mais que se lhe ponha? \dotfill Vergonha.
			
\hspace{5em} O demo a viver se exponha, \\
\hspace{5em} Por mais que a fama a exalta, \\
\hspace{5em} Numa cidade onde falta \\
\hspace{5em} Verdade, honra, vergonha. 
			
2 \\
Quem a pôs neste socrócio\footnote{Afrânio Peixoto grafa \textit{rocrócio}. Em um dos apógrafos vem \textit{socrócio}. Na primeira hipótese, retrocesso. Na segunda hipótese, socrócio, criado por necessidade de eco com negócio, de socrestar, furtar, rapinar.}? \dotfill Negócio\footnote{Colocou Salvador na situação em que se encontra.}. \\
Quem causa tal perdição? \dotfill Ambição\footnote{Ganância, desejo de possuir tudo.}. \\
E a maior desta loucura? \dotfill Usura\footnote{Termo já citado no capítulo de introdução ao Barroco, envolve o processo de empréstimo de dinheiro e sequente cobrança de juros. Era condenada pela Igreja Católica e, em muitas nações, considerada um crime. Max Weber, nesse contexto, descreve que os judeus foram associados ao dinheiro pois realizavam empréstimos com as leis restritivas de aquisição de bens fundiários. Foram descriminados, sobretudo na Europa, pois foram os assassinos de Cristo ao escolherem libertar Barrabás. Relembre \textit{O mercador de Veneza}, de Shakespeare. Gregório associa a usura à elite mercantil emergente (falta de verdade, honra e vergonha causada pelo negócio, ambição e usura).}.
			
\hspace{5em} Notável desaventura \\
\hspace{5em} De um povo néscio, e sandeu, \\
\hspace{5em} Que não sabe que o perdeu \\
\hspace{5em} Negócio, ambição, usura.
			
3 \\
Quais são os seus doces objetos? \dotfill Pretos. \\
Tem outros bens mais maciços? \dotfill Mestiços. \\
Quais destes lhe são mais gratos? \dotfill Mulatos.
			
\hspace{5em} Dou ao demo os insensatos, \\
\hspace{5em} Dou ao demo a gente asnal\footnote{Burro}, \\
\hspace{5em} Que estima por cabedal\footnote{Riqueza.} \\
\hspace{5em} Pretos, mestiços, mulatos.
			
4 \\
Quem faz os círios mesquinhos? \dotfill Meirinhos. \\
Quem faz as farinhas tardas? \dotfill Guardas. \\
Quem as tem nos aposentos? \dotfill Sargentos.
			
\hspace{5em} Os círios lá vêm aos centos, \\
\hspace{5em} E a terra fica esfaimando \\
\hspace{5em} Porque os vão atravessando \\
\hspace{5em} Meirinhos, guardas, sargentos.
			
5 \\
E que justiça a resguarda?  \dotfill Bastarda. \\
É grátis distribuída? \dotfill Vendida. \\
Que tem, que a todos assusta? \dotfill Injusta. 
			
\hspace{5em} Valha-nos Deus, o que custa \\
\hspace{5em} O que El-Rei nos dá de graça, \\
\hspace{5em} Que anda a justiça na praça \\
\hspace{5em} Bastarda, vendida, injusta. 
			
6 \\
Que vai pela clerezia? \dotfill Simonia. \\
E pelos membros da Igreja? \dotfill Inveja. \\
Cuidei que mais se lhe punha? \dotfill Unha\footnote{Nesse contexto, com o sentido de roubalheira.}.
			
\hspace{5em} Sazonada caramunha\footnote{Experimentada lamentação (Amora).} \\
\hspace{5em} Enfim, que na Santa Sé \\
\hspace{5em} O que mais se pratica é \\
\hspace{5em} Simonia, inveja, unha.
			
7 \\
E nos frades há manqueiras\footnote{Claudicação. No texto, deslize moral.}? \dotfill Freiras. \\
Em que ocupam os serões? \dotfill Sermões. \\
Não se ocupam em disputas? \dotfill Putas.
			
\hspace{5em} Com palavras dissolutas \\
\hspace{5em} Me concluís, na verdade, \\
\hspace{5em} Que as lidas todas de um Frade\\
\hspace{5em} São freiras, sermões, e putas.
			
8 \\
O açúcar já se acabou? \dotfill Baixou\footnote{Queda dos preços desse produto no mercado.}. \\
E o dinheiro se extinguiu? \dotfill Subiu\footnote{Aumento dos juros.}. \\
Logo já convalesceu? \dotfill Morreu\footnote{Em relação à economia. Revela preocupação material com sua classe de origem.}.
			
\hspace{5em} À Bahia aconteceu \\
\hspace{5em} O que a um doente acontece, \\
\hspace{5em} Cai na cama, o mal lhe cresce, \\
\hspace{5em} Baixou, subiu, e morreu.
			
9 \\
A Câmara não acode? \dotfill Não pode. \\
Pois não tem todo o poder? \dotfill Não quer. \\
É que o governo a convence? \dotfill Não vence.
			
\hspace{5em} Quem haverá que tal pense, \\
\hspace{5em} Que uma Câmara tão nobre, \\
\hspace{5em} Por ver-se mísera e pobre, \\
\hspace{5em} Não pode, não quer, não vence. 
\end{verse}

Gregório de matos trata a Bahia como um corpo, organismo vivo, atuando, nesse sentido, como um médico que analise o caso. Há uma estrutura fixa de tercetos seguidos de quartetos, com estrofes de três versos em redondilha maior (sete sílabas poéticas). Nos quartetos se retoma as palavras de complemento dos tercetos.

Para o escritor, um dos problemas de Salvador é que se valoriza uma população de pretos, mestiços e mulatos. Além disso, insinua que as ocupações de um padre da cidade são freiras, sermões e prostitutas (hipocrisia).

Baixou (açúcar) para a cama, a febre subiu (juros) e morreu (economia).

\section{Padre Antônio Vieira}

Algum texto aqui.

\subsection{Os sermões}

Algum texto aqui.