\chapterimage{chapter_head_2.pdf}

\chapter{Naturalismo no Brasil}

Algum texto aqui.

\section{Aluísio Azevedo}

Algum texto aqui.

\subsection{O mulato}

Uma das primeiras publicações do autor, remonta ao início do Realismo no Brasil. Trata-se de um romance de protagonista que narra a história de Raimundo ao retornar para a sua cidade natal São Luís, no Maranhão, como um médico formado. Em dado momento, descobre uma linhagem familiar de ancestralidade africana, o que posteriormente o leva ao suicídio.

\href{http://www.dominiopublico.gov.br/download/texto/bn000166.pdf}{Domínio público}.

\subsection{Casa de pensão}

A segunda obra de Aluísio Azevedo, mais alinhada com o Naturalismo. Trata-se de uma narrativa de transição para o romance de espaço, no qual acompanhamos tramas paralelas relevantes para a história.

\href{http://www.dominiopublico.gov.br/download/texto/bv000014.pdf}{Domínio público}.

\subsection{O cortiço}

\textit{O cortiço} é um romance de espaço, sem protagonistas definidos, que trata da história de uma coletividade que compartilha alguma relação — não necessariamente de moradia — com o cortiço, ambiente em se se desenrolam as ações.  Miranda, por exemplo, não mora no local, mas se relaciona diretamente com o ambiente.

Um cortiço é uma moradia popular e barata, composto por casas de reduzidos cômodos; sua existência data da vinda da família real para o Brasil no ano de 1808, e a sequente desocupação de casarões por famílias nobres brasileiras. Os últimos andares, geralmente, apresentavam as piores condições sanitárias de toda a estrutura.

Edificações como essa eram a principal forma de moradia das classes trabalhadora. Em relação à cidade do Rio de Janeiro, em especial, lembre-se do fenômeno de favelização promovido sob as ordens de Francisco Pereira Passos.

Retornando à obra, o cortiço em questão fora construído por João Romão, imigrante português avarento e mesquinho, que tentara enriquecer no Brasil (no país, ganhara uma vendinha de um amigo português). O cortiço São Romão foi construído com materiais roubados e pela receita gerada pela vendinha (mais sobre isso no parágrafo seguinte).

Somos, então, apresentados a Bertoleza, escrava de ganho cujo senhor vivia em Juiz de Fora (válido mencionar que seu antigo namorado morrera por tanto trabalhar), que começa a trabalhar para Romão. O português ficara com todo o dinheiro da mulher — e que deveria ser, em parte, entregue ao senhor — e, não obstante, falsificara uma carta de alforria.

Jerônimo, por sua vez, é um imigrante português competente que trabalha como pedreiro; é casado com Piedade. Com a construção do cortiço, Romão o convida a trabalhar na pedreira do local, e o homem então se muda com a família.

Rita Baiana é uma mulata lavadeira, namorada de Firmo, que retornara recentemente para a cidade — e fora recebida com uma grande festa em comemoração. Jerônimo se apaixona pela mulher e, em cena notável, é tratado pela lavadeira de uma intensa febre com café e pinga.

Ao conhecer e começar a se relacionar com Rita Baiano, Jerônimo passa a faltar ao trabalho e exibir outros comportamentos que outrora seriam impensáveis. Firmo, no entanto, ao descobrir a relação entre o casal, briga e esfaqueia Jerônimo, que quase morre. O homem, por sua vez, ao se recuperar, mata Firmo com a ajuda de seu amigo Pataca; com isso, muda-se do cortiço com Rita, abandonando a filha e Piedade, que se torna alcoólatra junto de Pataca.

Outra personagem importante a ser mencionada é Miranda, outro imigrante português. O homem se apaixonara por Dona Estela, filha do dono de uma venda. Juntos, casam-se e enriquecem, e moram próximos ao cortiço São Romão — fato que o preocupa. Há uma disputa entre Romão e Miranda, representada por mecanismos de identificação por meio das posses, propriedades. Presença de certo grau de xenofobia com os portugueses.

As características do Naturalismo citadas no capítulo anterior também se fazem presentes. Há uma sexualidade razoavelmente explícita entre Miranda e Dona Estela, que mantém um caso com um garoto. Também é importante a figura de Botelho, o agregado da casa, que ao flagrar a relação, começa a chantagear Estela e a assediar o jovem (representação da homossexualidade).

Albino, o homossexual de obra.

Botelho orientava Romão para que o português pudesse ser aceito na sociedade (em paralelo, Miranda recebia o título de barão). Para tanto, sugere o casamento com Zumira, filha de Miranda. Antes, porém, Romão precisaria se livrar de Bertoleza. Botelho assume a responsabilidade, e se corresponde com o senhor de Bertoleza (lembre-se de que Romão desviara o valor que deveria ser enviado), que vai atrás da mulher. A escrava, ao descobrir, se suicida. Nesse momento, é curioso notar que, enquanto o corpo de Bertoleza é carregado, Romão recebe uma medalha como importante figura abolicionista.

Pombinha era uma jovem que levava uma vida confortável até a morte do pai. Com a tragédia, muda-se para o cortiço, onde espera pelo casamento e pela menarca.

Revolta da Vacina (1904)

Léonie é uma prostituta francesa e madrinha de Pombinha. A jovem, ao visitá-la junto de sua mãe, Dona Isabel, tem uma experiência e finalmente menstrua. Casa-se e trai o marido (expandir), e começa a morar com Léonie.

A filha de Jerônimo e Piedade começa a receber visitas de Pombinha (indicação do determinismo).

A bruxa do cortiço, que incendiara a instalação e morrera no processo.

Firmo, que na verdade não morreu, se une com outro cortiço para atacar o cortiço São Romão, que é incendiado. Romão rouba o dinheiro de um morador, e ainda recebe o valor do seguro.

É importante comentar que o capítulo que descreve o despertar do cortiço não necessariamente o trata como organismo vivo, e mais como força de expressão (prosopopeia), figura de linguagem.

\textit{O cortiço} é um romance de tese, que visa comprovar o determinismo em diversas personagens — a exemplo de Pombinha. Assim, o cortiço São Romão, além de ser um elemento que articula as diferentes tramas do enredo, é também uma metonímia para o fator mesológico como concebido pelo determinismo.

É importante notar o processo de ``abrasileiração'' de Jerônimo. Consciência trágica do intelectual brasileiro, de formação europeia. Rita Baiana é a síntese da natureza brasileira. Piedade trata-o com chá; Rita, com café.

As personagens Bertoleza e Rita Baiana representam a ideia da miscigenação como degeneração (a terra brasileira como fator de corrupção dos portugueses). Bertoleza, nesse sentido, seria um indivíduo puro, apto para o trabalho — no caso, na vendinha. Rita Baiana, por outro lado, era um indivíduo miscigenado, degenerado, e que corrompera o português Jerônimo. Em análise mais geral, temos a representação da própria elite brasileira, miscigenada, e em constante processo de degeneração.

Lembre-se de \textit{Iracema}: o que outrora era uma promessa de grandiosidade (recursos), torna-se uma maldição.

É interessante notar também que a obra descreve as diferentes possibilidades para a vida do europeu no Brasil. Se, por um lado, temos Jerônimo, um português branco para o qual o fator mesológico — representado por Rita Baiana — se sobrepõe ao racial, temos também a figura de João Romão, cujo fator racional preponderara.

João Romão é representante do processo de acumulação primitiva de capital, típica de sistemas econômicos pré-capitalistas; enriquece trabalhando, explorando aos outros e roubando. No Brasil, a escravidão se configura como um sistema capitalista atrasado — no sentido de possuir estruturas e práticas de uma etapa anterior —, e apenas nesse Romão poderia se tornar tão rico como acontecera. Ao final do romance, é notória a introdução do português às práticas modernas, como indicado pela aquisição de um seguro para o cortiço e posterior obtenção de retorno financeiro com tal medida.

Outro ponto ainda não explorado é a presença de imigrantes italianos nas instalações do cortiço, indicativos para a transição de um sistema escravista.

Lembre-se também que o Rio de Janeiro possuía um traçado urbanístico que remontava ao período colonial. Posteriormente, houve um processo de favelização do território brasileiro proporcionado, sobretudo, pela industrialização; cortiços eram moradias completamente insalubres.

Algumas características estilísticas da obra merecem maior atenção:

\begin{itemize}
\item Utilização de um vocabulário científico para os padrões da época.
\item Descrições baseadas em analogias com o mundo natural, como o cortiço que cresce como uma colônia de vermes que se desenvolve sob o esterco.
\item Animalização do comportamento das personagens, as quais agem motivadas por instintos básicos.
\item Bestialização da figura das personagens (caracterização das personagens com aspectos animalescos), como o caso de botelho e seu aspecto de abutre.
\item Situação biológica do agregado como um parasita e naturalização das relações sociais (darwinismo social).
\item Narrador onisciente, não-personagem e em 3ª pessoa.
\end{itemize}

Como reflexão final para a obra e para os romances de tese, de maneira geral, veja a \textbf{ilusão da previsibilidade do comportamento humano}.

\href{http://www.dominiopublico.gov.br/download/texto/bv000015.pdf}{Domínio público}.

\section{Raul Pompeia}

Jornalista. Suicidara-se aos 32 anos com um tiro no peito.

\subsection{O Ateneu}

Ao longo da obra acompanhamos Sérgio, narrador e protagonista, que narra suas experiências de quando, ainda jovem, estudava no internato ``O Ateneu''. A personagem, de família rica, até então havia recebido educação doméstica; ingressara na instituição ao final do Ensino Fundamental.

Aristarco era o diretor do colégio, referido informalmente como o ``homem-sanduíche'', pois em cada ação e fala via a possibilidade de promover o internato. Fornecia tratamento diferenciado aos alunos que não mantinham a mensalidade paga.

Logo em sua primeira aula de Matemática, Sérgio percebe estar defasado em relação aos outros colegas. Sanches, então, um aluno mais velho, começa a tutorar o garoto e o impede de sofrer bullying; o estudante, contudo, constantemente assedia Sérgio (em dado momento, Sanches faz uma proposta indecorosa para o protagonista, e que não é revelada para o leitor).

Sérgio busca na religião uma forma de consolo para a sua angústia. Aproxima-se de Franco, um aluno mais pobre e rebelde — quebrara garrafas de vidro na tina de banho que, felizmente, não ocorrera na semana pelas constantes chuvas. Sérgio denuncia o amigo, que fica então semanas na solitária. Bento Alves, o mais popular da escola, começa a proteger, tutorar e cortejar Sérgio, que se sente lisonjeado, mas desconfortável com sua posição passiva.

Aristarco encontra uma carta de amor.

No meio do ano, após as férias, um novo aluno, Egbert,  ingressa no internato. Desenvolve uma relação íntima com Sérgio; ao estudarem juntos para um exame de Matemática, a dupla obtém nota máxima. Com isso, são convidados a jantar na casa de Aristarco; na noite, Sérgio conhece e se interessa pela esposa do diretor.

Sérgio, então na puberdade, interessa-se por Dona Ema, esposa de Aristarco; a mulher representa como uma figura materna para o garoto — ainda que tenha sentido certa atração ao se conhecerem. Ao final do ano, Sérgio adoece e permanece no internato enquanto seus pais viajam; é cuidado por Dona Ema. Outro garoto, que também permanecera na instituição, incendeia a estrutura.

O romance nos traz o questionamento das implicações em confinar indivíduos na puberdade em um ambiente dominado por um único sexo, e também realiza críticas ao regime de internato e ao sistema educacional vigente, e que defende o fator mesológico — o indivíduo abusado que se torna abusador. Indivíduos que reproduziriam os comportamentos de abuso aprendidos no internato.

Raul Pompeia era um republicano de primeira ordem; os critérios para o direito ao voto envolviam renda e nível de alfabetização.

Trata-se de uma obra impressionista, que narra por meio da subjetividade e impressões do indivíduo. Nesse caso, esses serão a futura elite do país, corrompida pelo regime de internato — ambiente corruptor da inocência dos jovens (determinismo). No romance impressionista, os eventos representados são filtrados pela subjetividade do autor; há a presença de lirismo (função poética e emotiva da linguagem), diferentemente do que ocorria no Naturalismo. Há maior pudor do narrador, refletido no desconhecimento da própria personagem de Sérgio. O expressionismo em \textit{O Ateneu} é a ênfase exagerada em determinados aspectos da realidade (deformação), e a caricaturização da aparência das personagens. Há também grande influência do Naturalismo, como visto na bestialização das personagens.

A figura de Aristarco e a promoção do internato. 

Romance de formação interrompido na adolescência do narrador; desilusão com a realidade.

\href{http://www.dominiopublico.gov.br/download/texto/bv000297.pdf}{Domínio público}.