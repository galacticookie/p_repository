\chapterimage{chapter_head_2.pdf}

\chapter{Introdução}

O termo ``vanguarda'' tem origem militar — referia-se aos discursos de confrontação (em manifestos). Posteriormente, surgiriam as vanguardas políticas e, então, as vanguardas artísticas, movimentos de ruptura com a tradição ocidental, como expresso pele \textit{Manifesto futurista}; escrito por Filippo Marinetti, para o qual ``um automóvel é mais belo do que a \textit{Vitória de Samotrácia}''.

Na visão dos artistas, mudanças na sociedade exigiriam mudanças também na arte (``uma nova arte para um novo homem'').

Os movimentos desse período se relacionaram profundamente com as posições políticas vigentes: havia uma relação entre o fascismo e o futurismo, assim como entre o surrealismo e o comunismo; também é válido mencionar que Salvador Dalí possuía uma relação ambígua com Hitler.

Uma grande questão do Modernismo estava na relação entre arte e vida, então separados pelo modelo capitalista. O maior objetivo dos artistas e do movimento como um todo era de que ``a arte se torne vital e a vida se torne artístia''.

Para alguns autores, as vanguardas surgem com o impressionismo alemão, ainda que não houvesse demasiada organização por parte desse grupo. 

Para entender tamanha insatisfação com os modelos artísticos e, por extensão, do contexto social e político, mantidos desde o Renascimento, precisamos analisar o que estava acontecendo no mundo. A Europa vivia a \textit{Belle Époque} e a explosão da Primeira Guerra Mundial (o primeiro termo foi utilizado com o término do conflito para se referir a esse período anterior). Havia um clima de otimismo derivado dos avanços proporcionados pela Revolução Industrial, na ideia da razão humana como forma de encontrar soluções para todos os problemas (crença na razão expressa pelo racionalismo, marcado por um otimismo progressista), desde o Iluminismo (o sujeito cartesiano).

Para melhor entender a ideia do sujeito cartesiano, retomemos o filósofo francês René Descartes que, em \textit{Discurso sobre o Método}, no qual descreve o filtro da razão, descreve que o sujeito está destinado à verdade.

Friedrich Nietzsche, por sua vez, identifica a razão como uma guerra perpétua do homem consigo mesmo.

Sigmund Freud, enquanto tratava de mulheres histéricas (termo derivado de ´´útero'') com a utilização da hipnose e, posteriormente, dos sonhos e atos falhos, identifica uma constante de relatos de abusos cometidos por parentes próximos e algumas fantasias acerca dessas relações — muito semelhantes a um complexo de Édipo. Os pais como objeto de desejo. Com isso em mente, Freud que ideias potencialmente perigosas para o indivíduo são armazenadas no inconsciente (recalque) e se manifestam por meio de comportamentos (neurose). 

A partir desses conceitos, Freud estabelecia o conceito de um indivíduo:

\begin{itemize}
    \item Neurótico. Também referido como histérico ou obsessivo. Por meio do recalque de ideias, encontra-se em conflito com a realidade.
    \item Psicótico. Também referido como esquizofrênico ou paranoico. O princípio do prazer vence a realidade.
    \item Perverso. Tem prazer em burlar o princípio da realidade, ainda que reconhecido o conflito.
\end{itemize}

Veja que, com isso, há uma ruptura com a ideia do sujeito cartesiano, uma vez que todo indivíduo possui neuroses e a maior parte das decisões ocorrem externamente ao ``eu''.

Outros três conceitos relevantes para a discussão são:

\begin{itemize}
    \item id. Impulsos (pulsões) básicas da espécie.
    \item supereu. Normas internalizadas pelo sujeito.
    \item eu. Zona de conflito entre o id e o supereu (o exemplo do garoto em dúvida se deve ou não colar em um exame).
\end{itemize}

Moderno deriva de hodierno, hoje (tempo presente).

Se outrora se buscava representar a realidade de maneira objetiva (arte mimética), a crise do sujeito cartesiano ocasiona uma mudança na representação pela arte (transformação na ciência). Nesse sentido, é relevante mencionar Walter Benjamin, que no ensaio \textit{Experiência a pobreza}, descreve o fato de que os soldados retornavam dos conflitos absolutamente calados, pois a guerra já não gerava ``experiências'', mas apenas escancarava o conflito entre o frágil corpo humano e a máquina. O otimismo acerca da ciência e da razão entrava em crise com a preferência pela destruição. Essa questão também foi abordada por Freud em \textit{O mal-estar da civilização}, no qual comenta que essa era a sociedade mais racional e irracional que já existiu.

O Futurismo abraça a destruição da razão. O Surrealismo adota o inconsciente e tudo o que circunda a falência do modelo da razão e do sujeito. Uma distinção importante a ser feita, proposta por Peter Bürger, envolve o conceito da instituição ``Arte'', composta por museus, galerias, colecionadores e artistas que compõe arte, a qual define o próprio conceito de arte. As vanguardas buscam o fim dessa instituição (para o Dadaísmo, por exemplo, seria substituída por qualquer outra coisa). É interessante citar que a filosofia de Nietzsche precedera as definições de arte.

Assim, se outrora a arte, no sentido clássico, era a expressão de uma individualidade (autoria), temos agora as propostas do Dadaísmo (imagine palavras sortidas que formam um poema) e do Surrealismo com o acaso. Ainda nesse sentido, é importante mencionar José Paulo Paes com o projeto estético \textit{Sick transit}, no qual temos uma placa com a inscrição ``Liberdade interditada'', em especial no contexto da ditadura (poema pelo acaso).

A obra dadaísta \textit{Fonte}, de Marcel Duchamp, figura como um grande expoente do período; evidencia que a arte não precisa, necessariamente, passar pela autoria humana (questionamentos ao índice de autoria e contestação dos critérios de definição da arte). Transformação em índice de artisticidade.

O jogo de arte com tabuleiro de vidro e uma mulher nua e o questionamento da materialização da arte; o artista que contrata um indivíduo para que atire em seu braço; posteriormente apelo para o choque.

Moda como aquilo que está em voga no momento.

Diferenciação do Modernismo com a Modernidade (referência a determinado período da história da humanidade, com início na queda de Constantinopla e término na Revolução Francesa; termo ambíguo e com diferentes concepções, cuja própria noção tornara-se algo do passado, sob o ponto de vista de uma sociedade contemporânea). Veja que já uma notável diferença entre tempo (passagem dos anos) e temporalidade (conjunto de mudanças e transformações dentro de determinada sociedade).

Termo utilizado em referência a um conjunto de tendências de ruptura com a tradição artística ocidental (temas e técnicas semelhantes aos que eram utilizados mesmo na Renascença e no Barroco, em um processo de clara continuidade), no final do século XIX e início do século XX.

Comparação da obra de Bouguereau ("O Nascimento de Vênus") e de Picasso ("Les demoiselles d'Avignon").

Fases iniciais mais realistas e, posteriormente, com técnicas experimentais inovadoras.

A obra de arte não como algo que dialoga com elementos do passado, mas que busca apresentar o novo, a inovação (que representa o valor em si).

Os movimentos artísticos que surgiram nessa época, referenciadas como vanguardas europeias.

O impressionismo é considerado o primeiro movimento artístico das vanguardas (em referência à organização do exército, a porção que abre espaço no território inimigo inexplorado).

No século XIX, o termo começa a ser empregado no contexto político (vanguardas políticas), em referência aos grupos políticos que propunham modelos de sociedade orientados ao futuro, revolucionários, e não para o já existente (anarquismo, socialismo científico, socialismo utópico).

A arte que revoluciona a própria arte.

No século XX, o vocábulo é incorporado pelos grupos artísticos, em movimentos artísticos, estéticos que, em muitas ocasiões, cruzaram seu caminho com questões políticas (como a relação estabelecida entre o fascismo e o futurismo, assim como o socialismo e o surrealismo). O Modernismo nada mais é do que o conjunto das vanguardas europeias e os seus desdobramentos na arte ocidental; trata-se de uma mescla das características das diferentes vanguardas, em especial após o ápice desse período (mistura de tendências no âmbito da arte, assim formando o Modernismo).

O Modernismo está intrinsecamente relacionado com as vanguardas europeias, movimentos de ruptura com a tradição da arte ocidental. Esses movimentos tiveram início com o impressionismo, ainda pouco articulado; posteriormente, surge o futurismo, o surrealismo, e outros.

Nesse momento, identificamos uma crise do modelo de racionalidade vigente, do sujeito cartesiano. Há um processo de aceleração da temporalidade em uma modernidade muito fragmentada, sobretudo com um sistema econômico complexo, que exige diferentes facetas de uma nova sociedade. Na Europa medieval, por exemplo, haviam comerciantes, guerreiros e camponeses; na sociedade moderna, mais complexa, há um processo de fragmentação da vida social, em especial a partir da divisão do trabalho (universo partimentado em visão marxista).

Nesse cenário, observamos a perda de certa unidade ideológica, o que proporciona diferentes formas de enter e representar a arte e o mundo (tendências artísticas). Com isso, existe um elemento em comum das vanguardas: o princípio da ruptura com a tradição ocidental.

Surgimento da fenomenologia como campo da Filosofia, a partir de Kant.

Anteriormente ao período das vanguardas europeias, vigorava o chamado academicismo (inscrição em uma academia de arte e ensino de formas tradicionais; acúmulo de conhecimento desde o Renascimento). \textit{O nascimento de Vênus}, por exemplo, constitui um exemplo notável. É importante mencionar, nesse sentido, que o Parnasianismo era contemporâneo da arte acadêmica (artes irmãs).

Ao longo da Era Vitoriana, é curioso identificar que, em uma sociedade tão conservadora, produziu-se tantas obras de mulheres nuas; havia uma restrição aos temas clássicos, como a representação de Vênus ou Frineia, em processo de distanciamento histórico e hierárquico com a mulher contemporânea (semelhante, por exemplo, aos próprios parentes dos indivíduos, aproximando a temática para o ambiente doméstico das famílias). A outra condição de produção era a ausência de pelos pubianos, o que remete a um corpo infantil na ideia de algo sensual mas não sexual, como forma de ocultar a realidade do sexo. É importante destacar a diferença com a contemporaneidade, marcada pela ausência de representação, por parte da pornografia, de pelos pubianos, como forma de satisfazer os desejos do público.

Com isso em mente, conseguimos entender o escândalo gerado por \textit{Olympia}. A obra, referida como pornográfica, foi diretamente inspirada pela pintura renascentista \textit{Vênus de Urbino}. O quadro em questão tivera de ser quase que escondido em sua exposição. Isso pois, diferentemente da obra de Ticiano, distanciara-se do repertório de referências mitológicas para representar o contemporâneo, mais sexual e pouco sensual, como observado pela face de desprezo e as pernas cruzadas (no caso, de uma prostituta). Havia um grande problema em representar a mulher real. Um crítico de arte norte-americano identificou, na imprensa da época, mais de sessenta artigos escritos criticando a obra.

Algumas questões sobre a representação das mulheres em sociedades conservadoras merece atenção (o tema fora outrora abordado com o Romantismo). Ao analisar os dados de sites de pornografia, facilmente identificamos que os maiores casos de parafilia estão presentes na região sul do país, uma das mais conservadoras (isso pode ser facilmente observado pelas buscas mais frequentes, nas quais figuram termos como ``novinha'', justamente em uma sociedade que busca proteger as crianças). Freud identifica a presença do recalque nesse processo. Lembre-se de que, frequentemente, os abusadores estão presentes na família e na escola, em especial na figura de professores (por que Educação Física?).

As mulheres que serviam como modelo para as pinturas; em certos casos, os artistas poderiam mesmo contratar prostitutas.

Nesse cenário, \textit{L'Origine du monde}, de Gustave Courbet, não tem o objetivo de se constituir diretamente como pornografia, mas sim representar a realidade da mulher real.

Em diferentes culturas e momentos históricos, a representação da mulher possuía consideráveis diferenças. No Brasil, por exemplo, as indígenas já possuíam hábitos de depilação com cera de abelha. No período contemporâneo, há o surgimento das ninfetas e lolitas.

Existe uma tradição singular da pornografia japonesa, sobretudo por uma sociedade na qual não havia uma transição tão clara entre o período da infância e da idade adulta. Assim, ainda que tenha havido considerável influência ocidental, a figura da mulher infantilizada já era presente nessa realidade.

Nos países orientais, em análise geral, não existia um tabu tão grande acerca da nudez — em especial antes da chegada dos Estados Unidos e a moral ocidental, com o fim da Segunda Guerra Mundial.

Na televisão brasileira, em especial da década de 90, era comum a figura da mulher samambaia, desde o período de Chacrinha (inclusive, havia grande competição por audiência entre Faustão e Gugu Liberato, que organizavam programas como a ``Banheira do Gugu'' e ``Sushi erótico'' para atrair o público). Não havia classificação indicativa expressa, e a presença dessas cenas era algo comum — mesmo em horários mais nobres.

Nesse momento, também é válido mencionar a atuação de Edward Bernays, sobrinho de Freud, na transformação do cigarro como símbolo da libertação feminina, incentivando o consumo desse produto em um grupo que outrora era mal visto por fumar publicamente.

Agora, de volta ao quadro \textit{Les Demoiselles d'Avignon}, veja como Picasso utiliza cores contrastantes, intensas, não com o objetivo de gerar profundidade ou realismo; pelo contrário, não há perspectiva e a pintura é chapada (em especial, observe a bandeja de frutas na porção inferior do quadro). As figuras são inspiradas por nus da arte acadêmica francesa do século anterior, como na própria obra renascentista citada anteriormente. Picasso não desejava representar a realidade, mas criar um conjunto de formas e cores que fossem instigantes. O bairro de Avignon, na cidade de Madrid, era uma famosa zona de prostituição da cidade (o termo ``senhoritas'' era utilizado pois as mulheres ainda não eram casadas).

Veja, por exemplo, o quadro \textit{Retrato da mãe do artista}, e perceba a profundidade das técnicas empregadas por um Picasso de apenas 15 anos.

Em \textit{Campo de trigo com corvos}, veja que tanto o trigo quanto as aves são apenas sugeridos por meio de marcantes pinceladas. A perspectiva não é representada de maneira realista, e notamos formas agitadas, convulsas, imersas em um céu contrastante, cenário (questionável) para o suicídio do artista após duas semanas (representação da angústia). 

Van Gogh é um artista pré-impressionista. \textit{Mont Sainte-Victoire}, por sua vez, é uma pintura pós-impressionista.

\textit{San Giorgio Maggiore au crépuscule'}, de Claude Monet, é um quadro impressionista (São Jorge Maior em referência à igreja). O foco central da obra está na silhueta, sombra da igreja. Busca-se a representação por meio da luminosidade, de extrema importância para o artista impressionista, não pela representação da realidade mas das impressões do artista sobre ela (maneiras pelas quais a sensibilidade do artista era tocada pela cena).

Em relação ao expressionismo, temos \textit{O grito}, de Edvard Munch. É retratado o indivíduo que abafa os seus ouvidos ao ouvir o grito da natureza (figura atordoada pela presença da natureza, tamanha a sensibilidade do indivíduo, impressionado com a riqueza de detalhes). É interessante notar como a figura humana se dissolve na paisagem (o artista, potencialmente, utilizara cogumelos na criação desse e de outros quadros).

O pintor expressionista representa as suas impressões sobre a realidade. O artista expressionista deforma a realidade para atender às suas percepções.

O futurismo, por sua vez, busca a representação mo movimento no meio bidimensional e estático, como no dinamismo de um automóvel. Há um culto à máquina e ao progresso.