\chapterimage{chapter_head_2.pdf}

\chapter{Introdução}

%------------------------------------------------

O termo \textbf{estilo} tem origem no latim \textit{stilus}, instrumento metálico utilizado para escrever ou desenhar. Com o tempo, passou a designar, no campo artístico, a forma pela qual determinada expressão ou modo particularizava-se, marcava a sociedade. De maneira geral, o estilo pode ser \textbf{autoral} — exprime características que descrevem o texto de um determinado autor —, ou de \textbf{época} — presença de características compartilhadas por autores de um mesmo período. 

Como convenção, iniciais maiúsculas serão utilizadas em referência aos estilos de época, como forma de evitar confusões ao longo do texto. Essa distinção ficará mais clara na seção sobre o Realismo, com a análise do realismo presente em algumas obras do Romantismo brasileiro.

O Humanismo situa-se na passagem da Idade Média para a Idade Moderna e, em especial, na relação\footnote{Uma analogia curiosa utilizada para descrever tal processo é a de um indivíduo apoiado sobre o termo \textit{centrismo} e com os pés divididos entre \textit{teo} e \textit{antropo}. Com efeito, a religião não fora completamente abandonada ou invalidada no período, mas passou a compartilhar espaço também para a posição do ser humano no meio.} entre o \textbf{teocentrismo} e o \textbf{antropocentrismo}. Entre os principais autores desse período, destacam-se os italianos Dante Alighieri (1265-1321), autor de \textit{A divina comédia}, e Francesco Petrarca (1304-1374), autor de \textit{Cancioneiro}. Transição de uma visão teocêntrica para uma visão antropocêntrica.

%Pretendo remover a primeira figura — não acho que tenha agregado em muito para a compreensão do texto. Assim, seria interessante acrescentar um pequeno texto descrevendo a ideia por trás do desenho, acrescentando o seguinte texto: O recurso utilizado é chamado de \textbf{alegoria}, uma metáfora prolongada que serve de representação concreta para alguma ideia ou conceito abstrato.
		
\textbf{Carece de mais informações.}