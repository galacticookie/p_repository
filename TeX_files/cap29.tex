\chapterimage{chapter_head_2.pdf}

\chapter{Modernismo no Brasil}

Algum texto aqui.

\section{Pré-Modernismo}

A ideia de um Pré-Modernismo foi um rótulo utilizado pela crítica literária em referência aos autores do final do século XIX e início do século XX (até a década de 1920, aproximadamente, com a morte de Lima Barreto em 1922) que, de alguma forma, anteciparam os modelos do Modernismo. Trata-se de uma categoria arbitrária, que abarca autores atualmente classificados como realistas mas, outrora, pré-modernistas, como os casos de Graça Aranha e Raul Pompeia.

\subsection{Euclides da Cunha}

Uma das figuras mais notáveis desse movimento é Euclides da Cunha, jornalista paulista que realizara a cobertura da última expedição do exército brasileiro contra Canudos. Como justificativa para a expedição, elaborou-se uma narrativa de que o grupo possuía financiamento estrangeiro de monarquistas interessados em instaurar o mesmo regime no país. Veja, nesse sentido, a relação oposta estabelecida entre o exército e Canudos:

\begin{table}[h]
\centering
\begin{tabular}{l l}
\toprule
\textbf{Exército} & \textbf{Canudos} \\
\midrule
república & monarquia \\
progresso & atraso \\
razão & ignorância \\
civilização & barbárie \\
\bottomrule
\end{tabular}
\caption{Legenda aqui.}
\end{table}

Logo na primeira expedição o exército fora completamente rechaçado, uma vez que os sertanejos já possuíam grande conhecimento do clima e relevo da região. Já na segunda investida, uma das maiores figuras do exército brasileiro da época foi morta logo no início do confronto.

Ocorre que Euclides da Cunha encontrou uma realidade completamente diferente daquela divulgada pelo exército brasileiro, então dominado por ideais positivistas. Com isso, passou a subverter o discurso oficial, propondo o exército como o grande promotor da barbárie.

\subsubsection{Os sertões}

A obra foi um grande sucesso editorial. É divida em três partes:

\begin{enumerate}
    \item A terra. Fornece uma descrição, com os conhecimentos da época, de aspectos geológicos da região.
    \item O homem. Apresenta a figura do sertanejo com a força de Hércules, mas o físico do Corcunda de Notre-Dame (perceba os pressupostos da degeneração de um povo mestiço, atribuindo as características negativas dessa população, inclusive o fanatismo religioso, à miscigenação). ``O sertanejo é, antes de tudo, um forte''.
    \item A luta. Nota-se a constante do fator mesológico na terra, e da influência racial no homem. Dessa forma, ainda que Euclides tentasse se distanciar do pensamento promovido pelo exército, acaba por ser influenciado por tais correntes.
\end{enumerate}

\subsection{Monteiro Lobato}

Monteiro Lobato é inserido em nossa classificação como pré-modernista em razão da mistura de elementos regionalistas e um sentimento de certo preciosismo identificado em suas obras.

\textit{Urupês}, a primeira obra do autor, é um romance que trata essencialmente da figura do caboclo. O autor trata desse indivíduo como detentor de uma preguiça crônica, que incendeia as florestas em uma tentativa fácil de limpar os terrenos, e cujas moradias tampouco se assemelham a construções humanas.

Em \textit{Velha praga}, a influência da ideia de miscigenação como fator degenerativo continua presente. É realizada uma comparação com os holandeses, que tiveram de lutar contra a natureza, constituindo, assim, de um povo com iniciativa, empreendedor. O caboclo, em contrapartida, possui mandioca disponível a todo momento.

Em suas obras, é frequente a ideia do fator mesológico e racial como determinantes para a formação de um caboclo preguiçoso e adepto à lei do menor esforço, e que representa um grande obstáculo para a modernização do Brasil.

O caboclo Jeca Tatu é uma das principais personagens de Monteiro Lobato. Uma cena que merece ser citada, e que sintetiza as visões do autor acerca desse grupo, é de quando o homem, ao encontrar uma rachadura na parede de sua casa, cobre-a com um quadro de Nossa Senhora.

Anos mais tarde, o autor se desculpara pela figura de Jeca Tatu, atribuindo as características negativas a efeitos de doenças, em especial, da malária. Monteiro Lobato, então, substitui a imagem do caboclo pela de Senhor Brasil.

Em \textit{Cidades mortas}, o autor analisa a situação das cidades de província e identifica a profunda dependência econômica do Vale do Paraíba.

Em \textit{Negrinha}, Monteiro Lobato trata do negro como uma criatura completamente estupidificada — a personagem sequer possuía um nome, e é frequentemente chamada de ``peste negra''. O narrador utiliza certo sarcasmo que supera as barreiras do humor mórbido.

\textit{O presidente negro} é considerado por alguns estudiosos como a primeira obra de ficção científica da literatura brasileira. Narra a disputa eleitoral entre uma mulher feminista e um homem negro — que possuía um dispositivo que lhe permitia visualizar o futuro. As tensões raciais provenientes da vitória eleitoral do candidato negro gerara uma guerra civil. Como solução, os cientistas desenvolveram um raio capaz de alisar os cabelos dos negros, mas que os tornava estéreis. Monteiro Lobato dedica a obra a um médico e então presidente da Associação Brasileira de Eugenia. Nos Estados Unidos, ironicamente, não conseguira nenhuma editora que desejasse publicar a obra. Ao publicá-la no Brasil, omite a dedicatória, enviando uma carta ao médico em que descreve que o país ainda não estava pronto para tais ideias.

Na década passada, Monteiro Lobato se envolveu em diversas polêmicas. O governo enviara uma lista de livros para as escolas públicas, um dos quais era \textit{Caçadas de Pedrinho}. A primeira história narra a revolta dos animais após Pedrinho matar um animal. A segunda história narra a fuga de um rinoceronte. Tia Anastácia, negra, é representada como figura teimosa, tratada como os animais assim o eram no sítio — atrai em si o efeito do ridículo. Figura da mulher subindo a árvore assim como um macaco — descrito pelo próprio Narrador, e não por Emília que outrora também emitira falas de cunho racista.

O maior problema envolve o apego emocional presente nas obras infantis de Monteiro Lobato. Assim, há a sugestão de um dos bisnetos do autor em adaptar as obras, tornando, por exemplo, Anastácia em vizinha de Dona Benta.

Anastácia representa a figura do bom negro, também presente nos Estados Unidos (lembre-se de \textit{O nascimento de uma nação}). Monteiro Lobato identifica na figura do negro, em posição de poder, um indivíduo arrogante e, em certa medida, submisso.

\subsection{Lima Barreto}

Lima Barreto é contemporâneo de Monteiro Lobato, ainda que tenha morrido precocemente, aos 41 anos, em 1922. O escritor, como intelectual público, posicionara-se na imprensa veementemente contra as ideias racistas que circulavam na época. Era um mestiço de classe média-baixa que vivia nos subúrbios da cidade do Rio de Janeiro, e foi um dos primeiros autores a tratar da questão racial de um posto de vista social, e não cientificista, como era comum no século XIX.

Em \textit{Recordações do escrivão Isaías Caminha}, a personagem, ao ter o atendimento negligenciado em uma estação de trem, nota tal processo como consequência de sua condição como mulato. Em certa medida, é uma obra muito biográfica.

Há uma abordagem da questão racial no Brasil de um ponto de vista sociológico, em oposição a uma abordagem eugenista, no contexto do darwinismo social.

Em \textit{Clara dos anjos}, é retratada a história de uma garota mulata de bom coração, educada, que se apaixona e engravida de um malandro branco. É posteriormente trocada por uma mulher branca e rica. Existem dois finais da história: em um deles, a jovem se suicida, e em outro, torna-se prostituta.

É importante realizar a distinção entre o negro, indivíduo mestiço, de origem africana, e o preto, de maior ancestralidade africana, compartilhada com outras etnias, de acordo com o IBGE.

É interessante notar a dialética centro/subúrbio, em especial, na cidade do Rio de Janeiro.

Como visto em \textit{O cortiço}, no início do século XX, o traçado da cidade ainda possuía características coloniais. Com a prefeitura de Pereira Passos, surge a cidade dos cartões-postais conhecida nos dias de hoje. Com tal processo, os indivíduos de classe média-baixa passam a ocupar regiões mais distantes do centro, os subúrbios, conectados por uma malha ferroviária com o centro. As classes ainda mais baixas, por outro lado, deslocaram-se para as periferias.

Lima Barreto transitava entre o universo da classe trabalhadora e com melhores condições de vida, e da elite branca dos centros. Não se reconhecia em nenhum desses âmbitos (a condição social ambígua de um indivíduo não-branco mas de melhor formação acadêmica em comparação com as classes trabalhadoras).

Possivelmente, seu pai possuía um quadro de esquizofrenia. Em "Diário do hospício", relata parte de sua estadia em um manicômio (era alcoólatra).

Estilo menos formal, próximo ao do jornalismo moderno. Lima Barreto era funcionário público, mas parte considerável de sua renda provinha de seus trabalhos nos jornais. Público heterogêneo desses impressos, e que, portanto, deve apresentar uma linguagem popular (não coloquial ou transgressora da norma-padrão, mas menos rebuscada), acessível a todos os leitores (as obras adultas de Monteiro Lobato, por sua vez, estão muito mais próximas de Euclides da Cunha, Coelho Neto e dos escritores parnasianos, de maneria geral, em distinção do capital social).

Século XIX: jornalismo literário (com contos, sonetos, textos acadêmicos), de caráter iluminista (disseminação do conhecimento), com uma linguagem direcionada para as classes letradas, abastadas, e mais distanciada da descrição dos fatos do cotidiano que requer, por sua vez, um tipo de linguagem muito mais direta.

Obra desaparecida até a década de 50, quando fora revista por um jornalista que contratara escritores que elaborassem introduções para as obras do autor. Em certo aspecto, fora isolado por sua linguagem distinta daquela aplicada na época, caracterizada pelo preciosismo e demais características.

Lima Barreto como inimigo do Modernismo, sobretudo ao criticar o futebol e as vanguardas europeias como imposições da cultura europeia.

\subsubsection{O triste fim de Policarpo Quaresma}

Crítica ao ufanismo (orgulho exagerado da nacionalidade; nacionalismo exacerbado).

Policarpo Quaresma era um funcionário público e um grande nacionalista. Deseja que a cultura brasileira seja a mais pura possível (a moda de viola, por exemplo, como o estilo musical mais brasileiro possível; lembre-se que, na época, o violão era associado à vagabundagem).

Propõe que a língua oficial do Brasil se torne o tupi-guarani. A carta é divulgada na imprensa, e se torna alvo de piadas. É demitido mas, com o dinheiro da indenização recebida, compra uma fazenda. Começa a competir com outros fazendeiros na produção de uvas, mas tem a plantação atacada por uma praga e vai a falência. Encerra seus dias na Ilha das Cobras, junto com outros militares, ao comentar sobre o episódio da Revolta da Chibata.

É incapaz de distinguir a realidade de sua concepção ufanista.

Personagem quixotesca (Dom Quixote era um nobre decadente, no período de decadência econômica da nobreza, que não consegue distinguir a fantasia da realidade e que buscara realizar os feitos da nobreza registrados nos livros).

\href{http://www.dominiopublico.gov.br/download/texto/bn000013.pdf}{Domínio público}

\subsection{Augusto dos Anjos}

Seus sonetos seguem as regras de versificação tradicional, estabelecida pelo Parnasianismo e pelo Simbolismo. Há uma ênfase para os aspectos mais sórdidos, repugnantes da existência humana e da realidade, e um pessimismo materialista, no limite do niilismo.

Para o autor, o ser humano nada mais é do que um ajuntamento de processos químicos, incapaz de transcender para além do aspecto material (nesse sentido, possuía ideias semelhantes às correntes cientificistas e, em especial, deterministas da época)

\poemtitle{Versos íntimos}
\begin{verse}
Vês! Ninguém assistiu ao formidável \\
Enterro de tua última quimera\footnote{Metáfora para o sonho, ilusão, esperança.}. \\
Somente a Ingratidão -- esta pantera -- \\
Foi tua companheira inseparável!

Acostuma-te à lama que te espera!  \\
O Homem, que, nesta terra miserável, \\
Mora, entre feras, sente inevitável \\
Necessidade de também ser fera.

Toma um fósforo. Acende teu cigarro! \\
O beijo, amigo, é a véspera do escarro,\footnote{Em referência ao amor como precedente do desprezo, ideia também presente no verso seguinte.} \\
A mão que afaga é a mesma que apedreja.

Se a alguém causa inda pena a tua chaga, \\
Apedreja essa mão vil que te afaga\footnote{Ideia de que cabe antecipar o mal que potencialmente será causado ao indivíduo, escarrando e apedrejando antecipadamente os indivíduos.}, \\
Escarra nessa boca que te beija! 
\end{verse}

\poemtitle{Psicologia de um vencido}
\begin{verse}
Eu, filho do carbono e do amoníaco,\footnote{Em tom de pessimismo materialista.} \\
Monstro de escuridão e rutilância\footnote{Brilho.}, \\
Sofro, desde a epigênese\footnote{Crescimento da vida humana por meio de diferentes fases.} da infância, \\
A influência má dos signos do zodíaco.

Profundissimamente hipocondríaco\footnote{Nesse caso, em referência à depressão. Note que temos um verso dodecassílabo de apenas duas palavras.}, \\
Este ambiente me causa repugnância... \\
Sobe-me à boca uma ânsia análoga à ânsia \\
Que se escapa da boca de um cardíaco.

Já o verme -- este operário das ruínas -- \\
Que o sangue podre das carnificinas \\
Come, e à vida em geral declara guerra,

Anda a espreitar meus olhos para roê-los, \\
E há de deixar-me apenas os cabelos, \\
Na frialdade inorgânica da terra! 
\end{verse}

\poemtitle{A meu pai morto}
\begin{verse}
Podre meu Pai! A morte o olhar lhe vidra. \\
Em seus lábios que os meus lábios osculam\footnote{Beijam.} \\
Microrganismos fúnebres pululam \\
Numa fermentação gorda de cidra.

Duras leis as que os homens e a hórrida hidra \\
A uma só lei biológica vinculam\footnote{Mesmo os animais mais extraordinários estão submetidos à mesma lei natural da morte.}, \\
E a marcha das moléculas regulam\footnote{A vida nada mais é do que uma marcha de moléculas reguladas pela lei da morte.}, \\
Com a invariabilidade da clepsidra\footnote{Relógio de água. Ideia de que a morte é inexorável como o próprio tempo (inevitável).}!

Podre meu Pai! E a mão que enchi de beijos \\
Roída toda de bichos, como os queijos\footnote{Comparação da mão do cadáver com um queijo suíço (ambos são repletos de protusões).} \\
Sobre a mesa de orgíacos festins!...\footnote{O cadáver serve de banquete para os vermes e outros microrganismos.}

Amo meu Pai na atômica desordem\footnote{Continua a amar o seu pai, mesmo deformado.} \\
Entre as bocas necrófagas que o mordem \\
E a terra infecta que lhe cobre os rins! 
\end{verse}

Esse e outros sonetos foram escritos enquanto seu pai estava doente e em período posterior a sua morte. No soneto reproduzido acima, descreve o processo de decomposição do cadáver de seu pai.

Há uma visão pessimista do autor (em relação à natureza e à existência humana, em que todos os seres humanos são vistos como mesquinhos e egoístas), que reduz a existência humana ao seu aspecto biológico, no limiar com o niilismo, cuja descrição existencial fora desenvolvida por Schopenhauer.

Identificamos um enfoque nos aspectos mais sórdidos e repugnantes da existência humana, e a utilização de um vocabulário científico, algo aparentemente muito incomum.

Veja que os sonetos possuem uma sonoridade muito própria, que em certa medida anteciparam o Modernismo ao romperem com o Parnasianismo e o Simbolismo. Entretanto, deveríamos ignorar as tradições literárias locais do Nordeste e, em especial, de Pernambuco e da Paraíba (lembre-se de que a crítica literária brasileira, de maneria geral, possui grande enfoque na região sudeste; nesse sentido, suas obras parecem desconectadas desse cenário, mas é, em grande parte, semelhante à poesia científica das escolas do Recife, surgida com a disseminação dos ideais cientificistas na região). A Padaria Espiritual é como uma paródia de um cenáculo simbolista, ao misturar o cotidiano (pão) com os temas elevados dos quais o Simbolismo tratava.

\href{http://www.dominiopublico.gov.br/download/texto/bv.00054a.pdf}{Domínio público}

\section{Primeira fase}

A grande questão do Modernismo estava em produzir algo genuinamente nacional, por meio da incorporação dos valores de outras culturas. Nesse momento, surge a ideia da arte \textit{pau-brasileira}, idealizada por Oswald de Andrade e Tarsila do Amaral (as vanguardas europeias como forma de transmissão de conteúdos genuinamente brasileiros). Importantes publicações desse momento incluem \textit{Macunaíma}, de Mário de Andrade, e \textit{Manifesto antropófago}, de Oswald de Andrade (tratando-se de uma arte muito disruptiva, a apresentação dos princípios era comum por meio dos manifestos).

\subsection{Manifesto antropófago}

Oswald de Andrade utiliza de uma metáfora para descrever a identidade cultural brasileira, na ideia de que apenas a antropofagia nos une (em todos os aspectos), e é um modo natural do ser brasileiro, que se constitui na medida em que é antropófago.

Note a diferença entre o ser canibal, que devora outros da mesma espécie, e o antropófago, que devora seres humanos (âmbito biológico). Um leão que devora um ser humano não é canibal, mas é antropófago.

Para a Antropologia, a antropofagia envolve o consumo de carne humana inserida em determinado contexto cultural. Com a chegada dos europeus ao Brasil, houve um grande choque cultural, em especial em relação a esse tipo de comportamento. O principal objetivo dos nativos ao consumir a carne de outros (nunca da mesma comunidade, e apenas de homens, detentores de força, coragem, diferentemente das mulheres) era incorporar as qualidades do indivíduo (relembre a história de Hans Staden, e a primeira descrição minuciosa de uma testemunha de uma cerimônia de antropofagia). Para Oswald de Andrade, a cultura brasileira se forma na devoração de outras culturas, incorporando as qualidades e descartando o que não condiz com a cultura brasileira.

Assim, em um processo dialético (o brasileiro não é o indígena e tampouco o português), surge um terceiro elemento, o brasileiro. Há uma metáfora sobre a formação da cultura brasileira, que se dá na incorporação de elementos de outras culturas (relembre as diferentes ondas de imigração no Brasil). O que define o brasileiro é o ato de devorar (e não outros elementos culturais como gostos e tradições), e elaborar isso de maneira própria.

Para melhor entender essa visão, lembre-se de que o mundo e, em especial, a Europa, vivia no contexto do nacionalismo entre-guerras (lembre-se da questão do atavismo, citada no Romantismo)

É importante mencionar o caráter derivativo da cultura dos povos colonizados. Além disso, em relação ao Brasil, não podemos nos esquecer de grande diversidade étnico-cultural da população, diferentemente de muitos dos povos latino-americanos, aliada a um extenso território.

No contexto da Guerra Fria, houve um processo de aculturação do Brasil (então colônia com Portugal, e contemporâneo com os Estados Unidos), em um processo de apagamento da cultura original e incorporação passiva do povo dominador (Oswald de Andrade propõe que o processo nunca é passivo; pensamento muito ousado para a época e a frente do tempo; indispensáveis intelectuais como Gilberto Freye e Sérgio Buarque de Holanda no pensamento das décadas de 60 e 70).

Nesse momento, surge um nacionalismo crítico, derivado de um esforço em representar o brasileiro a partir de princípios próprios, ainda que não louváveis. É interessante mencionar que o tom irreverente de Oswald de Andrade fizera parecer as ideias como uma grande brincadeira (falta de noção do processo disruptivo para e época). Sobre o Primitivismo no Brasil.

Ao enunciar que ``A alegria é a prova dos nove'', Oswald indica que a alta cultura se reveste de toda uma seriedade (a poesia tratada como algo sério), decorrente, em certa medida, da própria tradição literária (decoro). Assim, o autor propõe um olhar irreverente para a cultura (dessacralização).

O nacionalismo crítico assim se forma com um olhar irreverente para a realidade nacional, levando em consideração as suas condições. Francisco de Oliveria, em \textit{Crítica da razão dualista}, descreve que o modelo narrativo da modernidade é baseado na modernidade europeia (feudalismo, mercantilismo, capitalismo, práticas suplantadas pelo período posterior — o moderno e o arcaico), modelo de explicação que não serve para esses países, e tampouco para o Brasil. Na modernidade brasileira, o que é arcaico na Europa é contemporâneo, moderno na realidade brasileira. O próprio processo de industrialização no Brasil (então na Europa um rompimento com a monarquia e surgimento da democracia liberal). O moderno no Brasil se alimenta do arcaico, como visto na mão de obra escrava. O país possui um \textbf{modelo não-canônico de modernidade}, retroalimentação dos modelos.

Ao escrever ``Tupi, or not tupi that is the question'', em referência à peça de \textit{Hamlet}, de Shakespeare, identificamos uma expressão sobre a possibilidade do suicídio (dimensão existencial). A cultura brasileira é a cultura indígena ou não? Oswald demonstra a antropofagia devorando o verso mais famoso de toda a extensa obra shakespereana, para lidar com os problemas que tratava no momento (utiliza da própria antropofagia para falar da antropofagia).

Existiram diferentes correntes do Modernismo no Brasil. As duas principais eram a do pau-brasil e o movimento ufanista (ufanar-se, orgulhar-se de algo). Plínio Salgado, na mesma época, criara um partido brasileiro nos moldes do fascismo europeu (então muito mais preocupado com a literatura do que com a política brasileira, é sensibilizado pelo movimento modernista).

Falaremos, agora, do Primitivismo, surgido com as vanguardas europeias. Tivera inspiração dos artistas em obras do passado e por exemplo, africanas, em processos de rompimento com a tradição ocidental. Lembre-se de Picasso com o Cubismo (máscaras inspiradas em elementos de matriz cultural fora da Europa) e mesmo a representação de indivíduos com transtornos psicológicos. Processo de etnocentrismo e racismo (senso de superioridade ao se referir às outras culturas como primitivas, por exemplo). Supõe a existência de uma cultura sem historicidade, imutável, parada no tempo.

A incorporação do Primitivismo no Brasil assume um novo significado (no Brasil, que significa olhar para dentro de nossa cultura, e não fora como ocorrera na Europa). Há um reconhecimento em sentido de problematizar (e não ufanista, como em Plínio Salgado). Assim, se na Europa o Primitivismo representa um olhar para fora da cultura europeia, na busca de elementos de ruptura com a tradição artística ocidental, no Brasil, representa um movimento de \textbf{desrecalque} da cultura popular. O que era primitivo para os europeus era pauta do dia para os brasileiros; aquilo que era atraso nos define como cultura (os elementos de atraso constituem a nossa formação cultural).

O Brasil sempre foi um país fortemente miscigenado. Essa característica, também presente na elite brasileira, fomentara um sentimento de desprezo da própria existência, assim como a ideia de uma inferioridade cultural da classe trabalhadora e das outras culturas não-europeias, por exemplo. O Modernismo fez com que a elite brasileira incorporasse a cultura popular nas produções artísticas (ainda que Almeida Júnior, por exemplo, já tivesse um interesse pela vida caipira, por exemplo, ainda que de maneira distanciada e pitoresca). Incorporação dos elementos populares na própria formação artística. Processo referido por Antônio Cândido como o desrecalque da cultura popular. Utilização do termo latino, em referência às origens em Roma (outras culturas como impurezas).

Manuel Bandeira se insere no Modernismo por outros meios.

Em \textit{Casa-grande e senzala}, notamos uma visão edulcorada da escravidão no Brasil. Logo na introdução, Freyre atribui à escravidão no Brasil a razão para o atraso no desenvolvimento do país, discurso outrora cientificista e baseado na inferioridade dessa etnia (condição do escravo, e não do negro). Posteriormente, dedica capítulos para a contribuição, na cultura brasileira, de homens e mulheres negros e indígenas.

O brasileiro nunca está pronto e sempre está se transformando (combate ao essencialismo da cultura nacional).

Gilberto Freye escreve sua magnum opus após o doutorado nos Estados Unidos (sul). Identifica, no Brasil, uma relação mais afetiva entre os negros e os escravagistas (e desconsidera, sensivelmente, a questão do estupro). Democracia racial: o racismo como problema individual (o Brasil não é um país racista, mas possui indivíduos racistas; lembre-se da pesquisa citada em aula), termo cunhado ainda na introdução. Relação menos brutal, violenta, do que aquela presente nos Estados Unidos (veja a presença de dois tipos de violência igualmente perversas). Em certa medida, não houve uma institucionalização da segregação no Brasil.

Para Gilberto Freyre, a poesia negra escrita por Jorge Lima (filho de escravagistas) era superior do que aquela escrita por autores negros dos Estados Unidos (oculta o aspecto estrutural do racismo no Brasil, e a organização do trabalho escravo ao longo dos séculos).

A risada aberta proporcionada pelo Modernismo: somos capazes de rir de nossas próprias contradições, e isso nos torna fortes (não é uma fraqueza).

\href{https://www.ufrgs.br/cdrom/oandrade/oandrade.pdf}{Link}

\subsection{Macunaíma: o herói sem nenhum caráter}

A obra de Mário de Andrade pertente à Primeira fase do Modernismo brasileiro, cujos objetivos eram de atualizar a arte brasileira (no contexto das vanguardas europeias) e produzir uma arte genuinamente nacional (motivos aparentemente contraditórios). Primitivismo e nacionalismo crítico (ideia de valorização das questões nacionais como matéria de representação, sem omitir os aspectos problemáticos da formação histórico-cultural).

Macunaíma é o protagonista da história -- narrada em 3ª pessoa, com um narrador onisciente e não-personagem. É filho de uma índia sem nome. Possui um irmão mais velho, Manaape, ancião e feiticeiro, e um irmão caçador e namorador, Jiguê.

Logos nos primeiros parágrafos do livro, fica claro o comportamento extremamente preguiçoso de Macunaíma, que demorou a falar e, na primeira fala, enunciara algo sobre o tema. É um indivíduo oportunista (lembre-se da caçada a uma anta com o irmão). Ao ficar com as piores partes do animal, chora por todo o dia.

\section{Segunda fase}
Em relação aos romances de 30, identificamos as seguintes tendências:

\begin{itemize}
\item Romance social (esquerda), como em Jorge Amado. Apresentava denúncias sociais e, em especial, na relação/inserção dos indivíduos na sociedade e as implicações desse processo. Mais alinhado com uma literatura coletiva, como visto em \textit{O germinal}, de Émile Zola. Maior preocupação, ênfase com as desigualdades sociais.
\item Romance intimista (direita), como em Octavio de Faria. Apresentava certa prospecção psicológica (proposta de inserir a personalidade em condições materiais próprias), como em \textit{Vidas secas}, e as ferramentas para expressar o drama. Caráter mais individualista na luta interna contra a desumanização, como visto em \textit{Caetés} e \textit{Angústia} de Graciliano Ramos. Maior preocupação, ênfase com a crise moral e espiritual do indivíduo na modernidade.
\end{itemize}

Em um pensamento marxista, identificamos a noção da superestrutura — representada, principalmente, pelas ideias — e da infraestrutura — representada, principalmente, pela economia; existe, assim, certo determinismo do qual os intimistas buscaram se distanciar (os intimistas não queriam ser identificados como socialistas, e vice-versa; note, ainda assim, que muitos autores permanecem no limiar dessas definições, razoavelmente arbitrárias).

Em Graciliano Ramos, notamos uma transição mais social de \textit{Vidas secas} para \textit{Infância} e, posteriormente, para \textit{Memórias do cárcere}. Identificamos uma manifestação na literatura nacional de uma tendência internacional.

Em relação aos romances sociais, analisaremos com maior cuidado a prosa regionalista. O enfoque no campesinato se justifica pois a população do campo representava a maioria da população da época — o Brasil ainda não havia passado pelo processo de proletarização. \textit{Lumpemproletariado} em regiões espasmos de São Paulo, Rio de Janeiro, Belo Horizonte e Porto Alegre.

O movimento regionalista remonta ao que hoje conhecemos por Nordeste do final dos anos 20 (lembre-se de que ainda não haviam as divisões por regiões atuais), e a identificação de características culturais próprias dessa população que não eram atendidas pela literatura e pelo pensamento brasileiro como um todo, sempre muito atrelado ao eixo Rio de Janeiro-São Paulo.

Havia o desejo de uma cultura que expressasse esses valores e tradições. Assim, em 1926, Gilberto Freyre publica o \textit{Manifesto regionalista}, a princípio, contra o Modernismo de 22 (a ideia de continuidade era polêmica para a época).

Note um viés de nacionalismo aparentemente contraditório presente na esquerda brasileira, no contexto da atuação imperialista dos Estados Unidos no Brasil; culminação no Golpe de 1964.

\begin{table}[h]
\centering
\begin{tabular}{l l}
\toprule
\textbf{Romance de 22} & \textbf{Romance de 30} \\
\midrule
cosmopolitismo\footnote{Em relação à influência das vanguardas europeias na literatura nacional.} & localismo\footnote{Em crítica à utilização ostensiva dos modelos vanguardistas, e valorização das condições mais próximas.} \\
nacionalismo\footnote{Em uma ideia que tende a dissolver as peculiaridades regionais. Nesse sentido, lembre-se da figura de Macunaíma, que não representava o povo paulista ou nordestino em especial, mas toda a população brasileira.} & regionalismo\footnote{Na ideia de pontuar as diferentes presentes na própria localidade.} \\
experimentalismo\footnote{Em relação às novas formas propostas para se produzir arte.} & tradicionalismo seletivo\footnote{Como visto na retomada do romance realista moderno, mas agora com certa liberdade. Aqui incluímos a verossimilhança (algo plausível; o \textit{no sense} de \textit{Macunaíma}}, de fato, não é factível) e a linearidade da narrativa ( com começo, meio e fim; narrativas encadeadas por causalidade possuem unidade de ação). \\
ênfase na burguesia & ênfase na classe trabalhadora \\
ambientação urbana & ambientação rural \\
\bottomrule
\end{tabular}
\end{table}

Note que muitos artistas modernistas transitam entre essas definições puramente arbitrárias. Jroge Amado, por exemplo, escreve romances ambientados tanto em Salvador quanto em plantações de cacau. O mesmo ocorrera com Érico Veríssimo, autor de \textit{o tempo e o vento}, e Graciliano Ramos.

Algumas características particulares do Romance de 30 também merecem ser analisadas:

\begin{itemize}
\item Dignidade literária da personagem pobre, descrita por Luís Bueno. Veja que, até então, predominava uma visão pitoresca sobre essa classe, construída com uma personalidade superficial e pouco desenvolvida — frequentemente, servia de alívio cômico parra a narrativa\footnote{Nesse sentido, perceba o tom cômico de \textit{Memórias de um sargento de milícias}, e a animalização das personagens em \textit{O cortiço}.}. Nesse momento, por outro lado, a personagem pobre é utilizada para sustentar uma narrativa, e construída com uma personalidade esférica.
\item Perspectiva interna à região\footnote{Até então predominava o olhar do homem da cidade sobre o campo (e mesmo o olhar do jovem que sai da cidade para estudar e, ao retornar, sente-se em uma realidade deslocada). Veja que a utilização do narrador em 1ª pessoa influencia algum grau de identificação com o leitor; assim, em textos destinados para a classe letrada\footnote{Nessa época, em centros urbanos — que concentravam a maior parte da população alfabetizada —, a taxa de alfabetização era inferior a um terço da população.}, predominava um olhar citadino}. José Lins do Rego, por exemplo, utilizara desse artifício literário para representar sua própria região.
\item Média estilística. Citando novamente o crítico literário Antônio Cândido, trazemos agora o conceito de dualidade estilística (pressuposto perverso de uma variedade urbana culta comparada com uma variedade informal). Busca-se uma média, um português nem tão castiço e nem oralizado, seja no narrador ou nas personagens. Meio termo entre o coloquial e a norma culta (norma padrão com elementos de regionalismo linguístico e de oralidade), tanto no discurso do narrador quanto no dos personagens, a exemplo de Jorge Amado\footnote{Há um processo que humanização desses indivíduos, ainda que recaia sobre o que Drummond criticara em \textit{Operário do mar}.}.
\end{itemize}

Na poesia, os principais autores do período incluem, mas não estão limitados a:
\begin{itemize}
    \item Carlos Drummond de Andrade
    \item Cecília Meireles
    \item Vinícius de Moraes
    \item Jorge de Lima
    \item Murilo Mendes
\end{itemize}

De mesmo modo, na prosa encontramos:
\begin{itemize}
    \item Graciliano Ramos
    \item Jorge Amado\footnote{Antes de Paulo Coelho, foi o autor brasileiro mais lido internacionalmente; é também o autor mais adaptado, em especial, para novelas.}
    \item José Lins do Rego
    \item Érico Veríssimo\footnote{Como informação adicional, podemos comentar que Érico Veríssimo foi um autor regionalista de fora do Nordeste, do Rio Grande do Sul, autor de \textit{O tempo e o vento}, que acompanha as gerações de uma família.}
    \item Rachel de Queiroz
\end{itemize}

\section{Terceira fase}

É interessante observar que, enquanto João Luís Lafetá escrevia sobre o Modernismo, as produções correntes se enquadravam como literatura contemporânea (com o tempo, no entanto, ficaria mais clara a ideia de continuidade desse movimento). O autor utilizara da ideia de ênfases a partir de críticas de Antônio Cândido sobre Graciliano Ramos, que unira diferentes estilos de romance em suas obras, a síntese de uma tese e uma antítese.

Nesse momento, identificamos a retomada do experimentalismo da Primeira fase, mas com algumas diferenças. Havia, antes, uma unidade de propostas em vistas ao processo de renovação literária; agora, temos um corpo mais heterogêneo de projetos estéticos.

Em especial com o Modernismo e suas diferentes fases, a ideia de agrupar projetos tão diferentes pode parecer razoavelmente arbitrária. Veja que, no Brasil, incorporamos o modelo francês dos estilos. Bakhtin, por exemplo, apresentara um modelo baseado na evolução dos gêneros ao longo da história. Com efeito, as classificações são puramente arbitrárias (\textit{Angústia} e \textit{Cacau}, por exemplo, são obras muito diferentes). Perceba, nesse sentido, certa influência do Romantismo e seu viés historicizante.