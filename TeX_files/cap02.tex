\chapterimage{chapter_head_2.pdf}

\chapter{Humanismo em Portugal}

\section{Fernão Lopes}\index{Fernão Lopes}

O marco inicial do Humanismo em Portugal são as crônicas de Fernão Lopes (1380-1460). Vale lembrar que a \textbf{crônica}, na concepção da Idade Média, é uma forma de gênero historiográfico que expressava uma visão \textbf{providencial}\footnote{Providência divina como a vontade de Deus manifesta. Mais adiante, na epopeia classicista \textit{Os lusíadas}, veremos novamente a providência, descrita no último canto, pela forma como Deus conduz a realidade.} da história, na qual Deus interfere no destino humano com o intuito de ensinar (caráter exemplar). É interessante notar que Fernão Lopes introduzira na historiografia ocidental a noção da história como resultado das ações, vontades humanas, em um contexto específico.

\section{Gil Vicente}\index{Gil Vicente}

Dramaturgo (lembre-se de que o drama era um gênero elaborado com o propósito de ser encenado).

\begin{definition}[Auto]
Forma do gênero dramático de origem medieval e ibérica. De maneira geral, possuía propósitos didáticos e moralizantes, expressando uma cosmovisão (visão de mundo) cristã. Ainda nesse sentido, são frequentes as figuras do imaginário cristão e, em especial, católico, como santos, anjos e demônios. Cabe, por fim, realizar uma distinção entre as personagens planas — de personalidade simples e comportamentos previsíveis, que não aprendem com a experiência — e as alegóricas que, nesse contexto, representam pecados e virtudes (lembre-se de que uma alegoria é uma representação concreta de um conceito abstrato). Além disso, é interessante notar que os autos não obedecem à regra de unidade de ação do teatro clássico, e as cenas são frequentemente independentes; também nota-se, na escrita, a presença de rimas e versos medidos em redondilha maior, com sete sílabas poéticas.
\end{definition}

Algumas particularidades dos autos vicentinos merecem maior atenção. Primeiramente, é válido mencionar a crítica aos costumes e comportamentos das pessoas de sua época. Além disso, é frequente a utilização de tipos sociais (as personagens como representação de grupos sociais).

\textbf{Carece de mais informações.}

\subsection{Auto da barca do inferno}
Fidalgo (soberba) é um nobre. Pajem é um servo doméstico. O frade, associado no auto à luxúria, é integrante do clero. Visão de uma sociedade estamental. Parvo como tolo. Johanes.

\textbf{Carece de mais informações.}