\chapterimage{chapter_head_2.pdf}

\chapter{Introdução}

O Parnasianismo se dá na segunda metade do século XIX, por contemporâneos da prosa realista — anteriormente ao Simbolismo. Suas origens remontam à publicação da revista francesa \textit{O parnaso contemporâneo}, do final dos anos de 1860. Parnaso era um monte da Grécia antiga na qual viviam as musas — que inspiravam (entusiasmo) o trabalho poético. O termo \textit{parnaso} acaba por representar um conjunto de poetas.

Apolo abre a fonte do monte.

O Parnasianismo fora muito breve e de importância limitada na França, na qual não houve uma 2ª geração de poetas parnasianos, superados pelo Simbolismo. No Brasil, o Parnasianismo remonta ao ano de 1878.

Trata-se de uma reação crítica ao sentimentalismo presente na poesia romântica. Nesse sentido, uma de suas principais características era o objetivismo: a poesia voltada para o mundo dos fenômenos visíveis, sensíveis (descritivismo), aos elementos que fogem ao mundo cotidiano.

Dois recursos da linguagem foram muito utilizados nesse período:

\begin{description}
\item[Écfrase] Descrição de obras de arte e objetos exóticos, valiosos (ideia do belo e da beleza como universais); pode ser observado, por exemplo, no poema \textit{Vaso chinês}, de Alberto de Oliveira.
\item[Hipotipose] Descrição de uma cena (quadro em movimento).
\end{description}

A ênfase da poesia é voltada ao mundo exterior. O descritivismo e o objetivismo assumem diferentes contornos na poesia parnasiana e na prosa realista. A prosa realista trata do cotidiano e dos dias comuns, o banal, ao passo que a poesia parnasiana trata do momento solene, que se destaca da trivialidade, da vulgaridade e da vida comum.

Ideia da \textit{enargeia}, vivacidade, como visto em Alberto de Oliveira que descreve o leque de uma senhora se distanciando da banalidade, ou como o poema de [monarmet].

Houve um retorno aos temas da Antiguidade e um maior interesse pelas civilizações exóticas. Em paralelo, observou-se um desenvolvimento da filologia, sobretudo na Alemanha, assim como a descoberta e tradução de textos inéditos da cultura hindu, chinesa e mesmo grega. Há o surgimento da arqueologia (como ciência ou técnica) e a descoberta de Troia e Pompeia.

Ideia da reconstituição do passado, ímpeto arqueológico na poesia parnasiana (na poesia classicista e neoclássica, a mitologia possuía um caráter mais simbólico, como visto em \textit{O nascimento de Vênus}). Aspecto alegórico nas obras do Classicismo, em paralelo ao imperativo visual parnasiano.

O ideal (doutrina) da arte pela arte (esteticismo). Théophile Gautier (1811-1872), em seu romance \textit{Mademoiselle de Maupin}, descreve que a única função da arte é gerar beleza (utilidade como aquilo que atende às necessidades básicas do ser humano, que não o distinguem de um animal; note a utilidade baseada pelas classes letradas). Ideia kantiana de um fim em sim mesmo. Comparação com a latrina de uma casa.

A arte não deve ter nenhuma função social (rompimento com a moralidade burguesa de edificação e com a crítica social do Realismo). Max Weber e o fenômeno de ``desencantamento com o mundo'' (substituição da religião pela arte).

Algumas características do Parnasianismo incluem:

\begin{itemize}
    \item Formalismo (busca pela perfeição formal).
    \item Vocabulário preciosista.
    \item Desprezo pelas rimas pobres e afeição pelas rimas raras e preciosas. Rimas pobres são termos fáceis (de associação automática) ou pertencentes à mesma classe gramatical. Rimas preciosas são difíceis de serem encontradas (como cisne e tisne; -yx).
    \item Rigor na métrica — ainda que em detrimento do significado.
    \item Ideia da torre de marfim: o olhar do poeta que se distancia da realidade contemporânea; cabe a ele retratar os temas preciosos (aristocracia do espírito e o fenômeno do dandismo — como Oscar Wilde —, o setor da burguesia que busca se distinguir por meio da sofisticação). Olhar de desprezo pela burguesia, ainda que pertencessem precisamente a essa classe.
\end{itemize}

Transição do dandismo para as vanguardas.

O poeta romântico priorizava a expressão em detrimento da forma. Para o poeta parnasiano, no entanto, a completa expressão se dava precisamente pela perfeição formal. Nota-se também a presença de impassibilidade presente no Estoicismo: o indivíduo que não se perturba pela felicidade ou pelo sofrimento (apatia; \textit{passum}, paixão). Distanciamento emocional do eu lírico (ou apenas narrativo) diante do tema representado. Vivacidade por meio do vocabulário, rimas, ritmo e adjetivos.