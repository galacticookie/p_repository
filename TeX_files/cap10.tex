\chapterimage{chapter_head_2.pdf}

\chapter{Introdução}

Comumente associamos a um indivíduo romântico qualidades que, de um modo geral, permeiam a ideia de amor. É aquele que demonstra as suas paixões, por exemplo. É curioso notar, nesse sentido, que tal associação entre questões amorosas e o romance ocorrera justamente em razão do Romantismo, \textbf{constituído por um universo muito maior do que a temática amorosa}.

O Romantismo surgira na passagem do século XVIII para o século XIX, estabelecendo-se nas primeiras décadas deste. Mais do que um estilo de época ou uma manifestação artística, foi um amplo movimento cultural que causou uma revolução em quase todos os campos do conhecimento humano, um modelo de pensamento que caracterizaria o início da Idade Contemporânea.

Com origens em uma Alemanha ainda não unificada (e, posteriormente, disseminando-se pela Inglaterra), ainda inserido na Filosofia, fora suscitado por uma ruptura dos filósofos desse país com o racionalismo iluminista expresso por Kant. Note, no entanto, que o Romantismo não é uma negação total e, em muitos aspectos, é um desdobramento das ideias contidas no Iluminismo. \textbf{O principal aspecto de crítica é o racionalismo}.

Em \textit{Crítica da razão pura}, Kant afirma que o ser humano constrói o conhecimento por meio da interação entre os sentidos e a razão. Por conseguinte, na Metafísica, nenhuma abstração poderia ser retirada, permanecendo, pois, excluída da Filosofia (não há uma negação da existência, mas a constatação de que seu escopo está além desta). Os filósofos alemães concordam com Kant, mas propõe que novas faculdades humanas sejam trazidas para que a Filosofia possa tratar deste e de outros temas. Em suma, é preciso incorporar a \textbf{sensibilidade} e a \textbf{imaginação} ao processo de apreensão da realidade, de tal forma a não excluir a Metafísica da Filosofia. 

Note as seguintes definições:

\begin{enumerate}
\item \textbf{Sensibilidade}: reação subjetiva aos elementos externos (diferentes de reações sensoriais). Envolve as emoções e sentimentos.
\item \textbf{Subjetividade}: diz respeito à vida interior (Psicologia). Envolve a visão pessoal acerca do mundo. Veja que a sensibilidade forma a subjetividade do indivíduo.
\end{enumerate}

Paralelo com as \textbf{revoluções burguesas}.

O principal argumento de tais pensadores era de que a experiência estética não poderia ser explicada inteiramente pela razão e pela objetividade, pois \textbf{existe algo na beleza que escapa ao geométrico e ao matemático, apreendido pela sensibilidade}. De forma semelhante, também as interações humanas requerem empatia, emoções (sensibilidade). A razão dá acesso às coisas como são; a imaginação, a como poderiam ou mesmo deveriam ser (utopia). Permanece aberta a especulação acerca de Deus e da possibilidade de existência da alma (ênfase no possível, e não o existente). O Romantismo não nega a razão, mas afirma que esta deve trabalhar em conjunto com as demais faculdades humanas, sobretudo a imaginação, considerada a mais importante.

É interessante notar a inversão de valores ocorrida no período: a imaginação, outrora uma fantasia inconsequente, da qual o filósofo deveria se afastar para apreender a realidade, torna-se com o Romantismo parte fundamental da experiência estética e artística. Ainda nesse sentido, é válido mencionar o processo de hierarquização de manifestações, de acordo com o grau de idealidade: a arquitetura, escultura, pintura, música e, por fim, a literatura (puramente imaginativa, e utilizada como base para se pensar em outras artes). O movimento da \textit{Tempestade e Ímpeto} rapidamente se espalharia para outros campos do conhecimento, baseado em uma negação do racionalismo iluminista, e não uma total ruptura (anti-racionalista, mas não irracionalista).

Duas características importantes a serem mencionadas são o \textbf{idealismo}, a realidade não como é, mas como poderia ou deveria ser, assim como o \textbf{sentimentalismo}, pautado na valorização da sensibilidade. É notório, nesse sentido, o caráter subjetivista do Romantismo, para o qual importa mais o indivíduo e sua visão do que o mundo exterior. Assim sendo, em vez de descrever a beleza de uma mulher, o autor se preocupa em descrever como se sente em relação a ela (lembre-se de Bocage e Tomás Antônio Gonzaga, tratados anteriormente como \textit{pré-românticos}). Além disso, elementos comumente presentes nesse período são o \textbf{platonismo} (em paralelo com Iluminismo e seu caráter, em geral, aristotélico) e o \textbf{amor platônico}.

Retornando à ideia introduzida no primeiro parágrafo, ainda nesse período o termo \textit{romântico} era utilizado em sentido pejorativo, indicativo de algo muito fantasioso, falso, assim como contos de fadas distantes da realidade. Tal termo fora retomado e incorporado pelos filósofos alemães no século XVIII, que então priorizavam \textbf{tanto a razão quanto a imaginação e sensibilidade}.

O termo Romantismo deriva de Roma. Lembre-se que, na Antiguidade, o latim era uma língua muito presente, que acabara por originar as chamadas línguas latinas ou românicas (em francês, tais línguas eram referidas como \textit{romances}). Durante muito tempo, a língua de prestígio, empregada nas produções literárias, era o latim. No século XII, no entanto, passaram a surgir expressões literárias em romance (línguas românicas), como histórias de cavalarias. Assim, por um processo de metonímia, tais narrativas ficaram conhecidas como romances de cavalaria, originando também a relação com as narrativas em prosa de longa extensão.

De um modo geral, tais histórias possuíam um caráter fantástico e, com o passar do tempo, o termo passou a designar algo muito fantasioso. Posteriormente, ainda seria utilizado de maneira pejorativa durante o Iluminismo, o qual defendia que a imaginação afastava o indivíduo da realidade, inimiga da razão, \textbf{a louca da casa}. A exceção notável fica por parte de Rousseau, que utilizou o termo de forma neutra ao se referir ao seu jardim. Por essa razão, é considerado, em certo grau, um precursor do Romantismo.

Dentro do movimento romântico, em especial nas produções literárias como nas poesias, teatro e outras narrativas, o tema dileto era o amor. Dessa forma, o público passara a associar o Romantismo com a ideia de amor. Veja, assim, que a literatura romântica não incorporara tal tema ao Romantismo, mas a partir deste começou-se a realizar tal insinuação.

O Romantismo se dá em meio a ascensão da burguesia como classe hegemônica, em processo iniciado ainda na Idade Média, na qual atividades econômicas eram praticas por comerciantes em burgos, ainda que predominasse uma economia de subsistência. Posteriormente, o capital se tornara o símbolo de riqueza. É interessante traçar o paralelo entre as Grandes Navegações, um investimento realizado, sobretudo pelas Coroas, com as novas atividades econômicas industriais do século XVIII, a qual marcavam o início do capitalismo mercantil; estas não foram um empreendimento estatal, mas da burguesia. Nesse momento, esta classe se torna independente do Estado, e a indústria passa a modificar as estruturas econômicas do Antigo Regime. Nesse cenário, a arte já não possuía mais a aristocracia, mas a burguesia como público-alvo.

Em tal transição, é válido mencionar alguns aspectos. Primeiramente, é interessante identificar a libertação de muitos dos costumes do antigo regime. Isso pois, no campo da Música, por exemplo, houve maior liberdade para que o músico pudesse realizar as suas composições. Lembre-se de que, para ouvir música - a dizer, erudita - era necessário participar de festas da aristocracia ou participar das celebrações promovidas pela Igreja, por exemplo. Surgem, nesse cenário, as casas de concerto (a princípio, na Alemanha), destinadas exclusivamente para tal finalidade, e abertas a qualquer um que tivesse dinheiro.

Por fim, cabe mencionar a Primavera dos Povos, ocorrida entre os séculos XVIII e XIX, e que marcara o fim do Antigo Regime. Expressa em movimentos como a independência dos Estados Unidos, a Revolução Francesa e a Revolução do Porto, significara a passagem do poder político e econômico para as mãos da burguesia.

Nesse cenário, se tínhamos de um lado, o Neoclassicismo como expressão de uma aristocracia que desejava se proteger das transformações sociais, por outro temos uma burguesia emergente, cuja expressão cultural de valores e visão de mundo é realizada pelo Romantismo. Quais seriam os valores fundamentais da visão de mundo da burguesia desse período? Segue-se:

\begin{enumerate}
\item \textbf{Nacionalismo}: até então, a ideia de uma comunidade, de maneira geral, sempre esteve intrinsecamente ligada à família real, à Coroa, fato representado na figura do Leviatã de Thomas Hobbes, por exemplo. Em uma democracia, no entanto, não os governantes, mas as estruturas de poder que são fixas. Dessa forma, é preciso estabelecer algum grau de aproximação do indivíduo com tais organizações. Nesse contexto, o nacionalismo é essencial para o arranjo das instituições sociais. Por essa razão, houve nesse período uma grande busca pelas raízes culturais, na ideia de que a cultura popular é a representação de um povo. Sentimento de pertencimento dos indivíduos. A ideia de uma sociedade, e não mais de uma comunidade (a cabeça como o rei, e o corpo como os outros indivíduos). Veja como é mais fácil nos identificarmos em outras pessoas, do que em alguma ideia abstrata como a democracia ou a república. Surgimento da democracia liberal/burguesa/moderna, caracterizada pela representatividade (eleição de indivíduos para representar os ideias no debate púbico), em que os governantes passam, mas as instituições ficam. Identificação com o Estado democrático de direito (algo mais abstrato). O anarquismo também surge nesse período. Max Weber e o fenômeno da liderança carismática. Constituição da noção de povo: comunidade de valores e símbolos que possibilitam uma unidade cultural baseada na ideia de atavismo (o indivíduo nasce carregando não apenas características físicas, mas também morais, de seus antepassados). Existência de um caráter nacional, no qual a cultura é uma espécie de essência. No caso da Alemanha, a noção de povo se relaciona à noção de raça (unificação por meio de um passado remoto em comum). Não possuíam a mesma língua ou religião, e a organização política também era diferente. No processo da unificação, a cultura não era suficiente para servir como fator de estabilidade, por isso surge a ideia de descendência de uma raça comum, o povo ariano (derivado da filologia); não havia fundamentação científica (indivíduos de grupos étnicos distintos podem compartilhar genes mais semelhantes do que outros dentro de uma mesma etnia). \textit{Um povo, uma nação, um líder}, um dos slogans da Alemanha nazista (incorporara a Áustria e outras regiões). Valorização da cultura popular pelas classes letradas, vista como a \textbf{alma} de uma determinada nação (\textit{folk}, povo, e \textit{lore}, conhecimento, sabedoria). Os contos dos irmãos Grim adquiriram grande notoriedade nesse época. De origem germânica, existiam as chamadas \textit{baladas}, contos narrativos em versos, cantados, sobre algum causo sobrenatural, típicas dessa população. Já na Península Ibérica, encontramos os \textbf{romanceiros}, semelhantes às baladas, mas com um esquema métrico diferente (proclamadas). Note a forma como cada país possui suas formas populares. Há, em certa medida, um processo de idealização da Idade Média e, em especial, do medievalismo, também equivocado, pois se tratava de uma construção em que as virtudes guerreiras ainda eram possíveis. O nacionalismo leva a uma valorização das origens (a partir do período medieval). Análise das influências de Grécia e Roma na cultura europeia (heranã greco-latina) incapazes de diferenciar as nações. Por essa razão, busca-se a origem na dissolução do Império Romano e invasões bárbaras (também o momento em que se começa a falar as línguas vernáculas). Inversão do juízo feito sobre a Idade Média (termo pejorativo por si só: intervalo, estagnação desde o fim do Império Romano, até o Renascimento, em que a verdadeira cultura morreu; o outro termo, Idade das Trevas, em oposição ao Iluminismo, que representa a razão e o progresso).
O escritor James Macpherson alegara ter encontrado o poema que narra a formação dos povos da região (a \textit{Ilíada} e a \textit{Odisseia} da Inglaterra), o \textit{Bardo Ossian}, uma fraude aceita quase que por unanimidade. Todos buscam um mito fundador, unificador.
Surgimento do romance histórico, como em Walter Scott, um romance medievalista que, geralmente, tinha como protagonista um cavaleiro (síntese das virtudes nacionais; antepassado, transposto nas melhores qualidades/virtudes da população contemporânea), em uma tentativa d construir modelos de virtudes, o qual chamamos de heróis nacionais.
Revalorização do cristianismo após o Iluminismo, e a Igreja Católica como aspecto universal da Idade Média.
No Romantismo, surge a ideia da cultura de massas, que atravessa diversas classes sociais. Em decorrência da alfabetização (como o sistema público de educação estabelecido por Napoleão). Desenvolvimento de tecnologias para o meio impresso que reduziram os custos de produção. Veja que, até então, um poeta não poderia viver de suas obras, pois o livro era um produto artesanal e caro. Com tal processo, temos uma consolidação dos jornais e folhetins (utilização de ganchos para manutenção do interesse do público). A literatura era a principal forma de entretenimento.
\item \textbf{Individualismo}: comumente associamos tal característica a aspectos negativos. Veja, contudo, que se trata de um valor de papel revolucionário na Europa dos séculos XVIII e XIX, em oposição a visões tradicionalistas. Nesse período, o todo social então determinava o indivíduo: a depender da família em que nascera, por exemplo, todo o seu futuro já estaria determinado. No entanto, com a noção do individualismo o indivíduo pode determinar os seus vínculos sociais (corrosão da ideia de uma sociedade estamental, à medida em que o indivíduo torna-se capaz de determinar a si mesmo e sua existência). Por fim, ainda sobre essa questão, note a visão estabelecida entre os camponeses que residem nas vilas, referidos como vilões (vil, pessoa má). Manifesto Comunista, de Karl Marx. O individualismo se manifesta na literatura como um \textbf{conflito entre o eu e o mundo}, na ideia de que o mundo não foi construído por ele e nem para o indivíduo (o mundo, nesse sentido, representa a sociedade, a natureza, Deus, etc.). Está diretamente relacionado com a ideia da \textbf{autodeterminação}, nas escolhas do indivíduo independentes de relações sociais pré-estabelecidas. Uma ideia comum compartilhada na época era de que o voluntarismo era o que transformaria a realidade. Tome o exemplo de Napoleão Bonaparte, por exemplo, de família de militares de baixa patente (Beethoven, inclusive, dedicou uma de suas sinfonias ao francês, fato do qual se arrependera mais tarde). Há o surgimento da ideia de comunidade e sociedade (associar, sócios, indivíduos que associados determinam a realidade), assim como um desconforto com a realidade pré-existente (lembre-se da representação de tal fenômeno, em relação à natureza, por exemplo, presente me Moby-Dick); outras esferas incluem a própria sociedade, Deus, a religião, o destino e outros indivíduos. Ainda nesse tópico, segue-se:
\begin{enumerate}
\item \textbf{Sentimentalismo}: o indivíduo em contradição com o mundo mergulha na sua vida interior e nos seus sentimentos, em uma forma de evasionismo. O \textit{egotismo} (o eu como tema principal), no qual o indivíduo trata predominantemente de si e de sua vida interior (é importante realizar a distinção com a imagem do artista). O \textbf{subjetivismo}: expressão de sentimentos e emoções, seja por meio do eu lírico, seja pelo foco nas emoções do narrador ou do protagonista na narrativa (menos sobre a realidade, e mais sobre as emoções e sentimentos). O \textbf{evasionismo} e a \textbf{idealização do amor}. Uma das principais características dessas obras era a idealização do amor: on indivíduo só se torna pleno por meio da experiência amorosa, tratada como uma força revolucionária, alma das convenções sociais (note, em especial, a ideia do amor como fruto da escolha de dois indivíduos livres, acima da família, das classes sociais, do casamento - arranjado - e da religião, um uma concepção próxima a atual). O indivíduo não é um produto de suas relações pessoais, e está em conflito com o que não é ele. Assim, se volta para o interior, ignorando o mundo externo (a temática amorosa como um pretexto para falar de si mesmo, como o sofrimento).
\item \textbf{Reformismo}: o indivíduo se propõe a transformar a sociedade para adequá-la aos seus ideais de justiça, igualdade, etc.
\item \textbf{Satanismo}: não como prática religiosa, mas como temática/simbolismo de um indivíduo que busca subverter os valores sociais, que se rebela contra a sociedade, em um processo de subversão dos valores sociais (tentativa de causar choque no público leitor). Temáticas comuns dessa vertente incluem a morte, o mórbido. Uma das figuras mais expressivas dessa linha fora o inglês Lord Byron (lembre-se da história de que, com o crânio de um monge, criara um cálice para uso pessoal), inclusive uma das figuras mais populares da Europa e muito cobiçado entre a juventude (por essa razão, o byronismo é frequentemente associado ao satanismo). O escritor se apropriara de personagens e tradições, como a figura de Dom Juan, a imagem do homem fatal (a figura de Byron é frequentemente associada aos seus personagens). Também é válido mencionar John Polidori e a representação dos vampiros. A figura de Lúcifer se torna símbolo máximo de rebeldia (revolta do indivíduo contra a sociedade). Lembre-se de \textit{O paraíso perdido}, de John Milton, que narra a história de Adão e Eva. Antes, descreve a revolta de Lúcifer e outros anjos contra Deus, contando ainda o processo de criação do inferno. Adão é tratado como um herói trágico, e Lúcifer o vilão. No século XIX é uma releitura por parte dos escritores românticos, que dão maior destaque para Lúcifer, considerado o herói da história. Temática macabra (relativa à morte), com a presença de cadáveres, cemitérios, esqueletos (relembre a taça de Lord Byron), assim como a exploração do sobrenatural, histórias de horror, o que chamamos de literatura gótica (fantasmas, vampiros, monstros, etc.). Note que os primeiros filmes de terror eram adaptações de obras desse período. A literatura gótica era anterior ao Romantismo, mas é neste período em que esta alcança o seu auge, tanto em volume de textos quanto em popularidade.
\begin{enumerate}
\item \textbf{Egotismo}:
\item \textbf{Subjetivismo}:
\item \textbf{Evasionismo}: fuga para a vida interior. Lembre-se do romance epistolar de Goethe, \textit{O sofrimento do jovem Verter}, e o processo de glamorização do suicídio (uma forma de evasionismo), na ideia de que ``viver é um pesadelo, e morrer é acordar desse pesadelo''. Ainda nesse sentido, identificamos grande ênfase na morte jovem, na ideia de que não foram corrompidos pela sociedade, e estavam no auge de suas vidas. Ideia de que o jovem é rebelde e, à medida em que se vive, o indivíduo se acomoda.
\item \textbf{Idealização do amor}: tratado para além das convenções sociais (família, casamento, religião, etc.).
\textbf{Sublimação da figura feminina}: representação das mulheres como indivíduos perfeitos, distanciamento com a carnalidade (dessexualização relacionada com a ideia de virgindade), aproximação de uma imagem idealizada e espiritualizada. O \textit{prestígio romântico da mulher}: a mulher amada é chamada de virgem, anjo, criança, irmã
\end{enumerate}
\end{enumerate}
De maneira geral, buscava-se uma identidade ainda não desenvolvida. Outra característica comum desse período era a romantização/idealização da morte do jovem, presente, por exemplo, em \textit{O sofrimento do jovem Verter}. Um aspecto interessante a ser notado é de que a sensibilidade contemporânea surgira com o Romantismo. As novelas, por exemplo, foram produzidas a partir de folhetins (também por isso, era necessária a utilização de um gancho ao final da história que criasse expectativas acerca da próxima publicação). Também as histórias de terror anteriormente ao surgimento do cinema.
\item \textbf{Liberalismo}: início dos ideais de democracia.
\end{enumerate}

Um formato de obra que se popularizou nesse período foram os \textit{penny dreadfuls}, livros baratos e populares com histórias de terror, amor e aventura. Outra técnica comumente utilizada era a dos \textit{romances epistolares}, escrito no formato de cartas ou diários, e que exaltava, sobretudo, a sensibilidade do indivíduo.

Ao longo da Era Vitoriana (reinado da Rainha Vitória), coincidente como o Romantismo, a Inglaterra já era um país burguês, mesmo com a forte presença da Monarquia. Nesse contexto, essa classe utilizara de sua moralidade como instrumento de justificação do seu domínio, realizando críticas a valores como a promiscuidade da aristocracia. Assim, também se distanciando das classes trabalhadoras, a burguesia atuara como um \textbf{farol moral}, modelo de família e tradição.

É interessante notar a forma como a burguesia trouxe o \textbf{recato do corpo}, com cômodos e aparelhos que dispensassem e ocultassem os rejeitos fisiológicos, por exemplo (algo que não existia no Palácio de Versalhes, por exemplo), e que, aos poucos, construíam a noção de \textbf{privacidade}.

Veja, no entanto, que o estabelecimento de um modelo de uma sociedade obcecada com o controle da homossexualidade, por exemplo, e a sequente repressão a tais comportamentos, gerara em grande parte justamente o efeito contrário. Do termo da Psiquiatria \textit{perversão}, mudança da finalidade, como fora o caso da prática sexual. Tome como exemplo a prostituição e a masturbação, ações atribuídas a uma série de efeitos negativos (exemplo de perversão), com tratamentos médicos específicos (torna-se um assunto de Estado, dada a importância da moralidade vitoriana).

Michel Foucault e a explosão discursiva sobre o sexo.

Freud e o questionamento: a perversão como um fim em si mesmo ou não?

O termo ``homossexual'' utilizado pela Medicina (anteriormente, era aquele que cometeu um pecado). Posteriormente, utilizado antes mesmo do ato sexual. Em um processo de cadeia de controle social, é razoável apontar as semelhanças entre a democracia e o absolutismo. Inviolabilidade do corpo, seguida de tais instâncias (biopolítica).

Segundo {autor desconhecido}, embora houvesse um discurso de moralidade, a prática era diferente (a maior repressão proporcionava um crescimento na realização de tais práticas). Ainda nesse contexto, há o surgimento da pornografia como um comércio, por meio de fotos, produções cinematográfica, etc. Processo conhecido como \textbf{vício inglês} na sociedade mais moral da época (sobretudo em razão de tamanha repressão), o que justifica o fato do mesmo autor passar de uma visão idealizada para outra mais perversa do amor. Descrição de Foucault como o \textit{modus vivendi} sexual como um padrão.

Alguns temas frequentemente presentes no Romantismo merecem atenção. Segue-se:

\begin{enumerate}
\item Idealização da mulher, em especial da figura da virgem adormecida e do adolescente tímido (o eu lírico admira a sua amada enquanto dorme e, mesmo que nãos se apresente como um adolescente, revela características como o comportamento temeroso, tímido, inseguro). Nesse sentido, é interessante avaliar a etimologia da palavra tímido, do latim, \textit{timere}, temer. Uma variação existente da figura da virgem adormecida é a figura da virgem morta, caracterizada pela exaltação da beleza da mulher que, ao morrer virgem, permanece pura, ainda não corrompida, e assim está mais próxima da santidade.
\item O padrão de beleza, de um modo geral, permeava a figura de indivíduos magros,  pálidos e com olheiras (relação com os casos de tuberculose). Um dos principais motivos é que a figura doente, frágil, inspira o sentimento de proteção.
\item A presença da melancolia (\textit{spleen}\footnote{Refere-se ao baço, que acreditava ser o produtor de um fluido chamado de bile negra.}) também era uma constante. A princípio, um dos quatro temperamentos (melancólico, sanguíneo, colérico e fleumático). Paralelamente com a depressão, identifica um indivíduo triste, dado a devaneios (condição tratada como clínica).
\item O tédio existencial, na ideia de que viver é um vale de lágrimas, e nada faz sentido (tom sentimentalista mais tristonho).
\item Idealização da morte, como forma de se livrar do sofrimento da vida (destaque para a morte prematura), como o caso de Castro Alves, que morrera com apenas 24 anos.
\item O conceito do reformismo, na ideia de que o mundo não vai corresponde aos ideias de liberdade, igualdade, fraternidade, etc. Assim, em vez de fugir para si mesmo, como na corrente sentimentalista, o indivíduo se propõe a mudar o mundo e transformar as consciências. Ideia da Literatura como instrumento de transformação das consciências, na busca por uma sociedade mais justa, baseada nos ideais do liberalismo (doutrina econômica e política que diz que uma maior liberdade dos indivíduos para buscar os próprios interesses é benéfico para a sociedade). Os indivíduos, juntos, deliberam para encontrar o melhor lugar comum.
\item Presença da figura do gênio, indivíduo naturalmente dotado de inteligência, imaginação e sensibilidade extraordinários. Serve como um farol da humanidade, capaz de guiar a sociedade para uma situação de maior justiça. O artista é relacionado com a figura do gênio, sobretudo o escritor e o poeta (arte de maior idealização), que atuam como vate, visionários que projetarão o futuro.
\item Por fim, citamos o compromisso com as principais causas políticas do período, a dizer, o republicanismo, a independência nacional, a separação entre Igreja e Estado, assim como a abolição da escravidão.
\end{enumerate}

A ideia da Idade Média como Idade das Trevas fora uma expressão criada pelo Iluminismo. No Romantismo, há uma visão ambígua desse período. É interessante notar, nesse sentido, a forma como a arquitetura gótica era tida como sombria (relembre os gárgulas como forma de levar os fiéis à reflexão; além disso, com o tempo - sobretudo na Revolução Industrial -, as catedrais foram ficando velhas e escuras, remetendo à percepção que se tinha da cultura gótica que é, na realidade, o contrário disso; evolução tecnológica que possibilitava construções mais altas e finas, permitindo maior circulação de ar e luz). A princípio, é uma arquitetura da luz, mas cuja percepção se altera com o decorrer dos séculos (expoente na literatura com o \textit{Drácula}).

A ideia da \textbf{libertinagem} também era muito presente. Trata-se de um estilo de vida amoral voltado para a satisfação dos prazeres, sem qualquer freio moral (inserido em uma sociedade cristã que valoriza a castidade, a temperança e a virgindade da mulher), em uma tentativa de corromper a família. É referida por alguns autores como \textit{dom juanismo}. Dom Juan é uma figura que já existia no imaginário europeu; seu objetivo é obter o maior prazer sexual possível, seduzindo mulheres virgens e casadas. Até então, suas histórias sempre possuíam um final ruim, em uma forma de apresentar uma punição. Com o Romantismo, no entanto, Lord Byron escreve um poema chamado \textit{Dom Juan}, no qual tenta heroificar a figura do libertino, que se torna quase como um anti-herói (a figura de Lord Byron se mistura com a de seus personagens).

Retornando às temas das perversões sexuais, em temas como incesto, necrofilia e sadismo, é válido mencionar algumas de suas origens. O sadismo, por exemplo, possui origem literária com Marquês de Sade, que possui uma literatura pornográfica que explora a violência explícita. Estava preso no período da queda da Bastilha, mas após ser solto, é preso novamente pelo governo revolucionário. O masoquismo, por sua vez, tem origem em Sacher-Masoch e sua literatura gótica, que fugiam do aceitável pelos padrões vitorianos.

Também é igualmente importante diferenciar o erotismo, que provoca a imaginação, sugere, e a pornografia, com o intuito de diretamente excitar sexualmente.

Na ideia do direito natural, a natureza estaria fundada em certas leis naturais. Marquês de Sade, por exemplo, possui uma visão da natureza como mal cósmico, pois o ser humano deve viver de acordo com as leis impostas impostas por esta, que são destrutivas.

Por fim, evidenciamos a heroificação dos criminosos, como indivíduos que se rebelam contra as imposições sociais (identificação dos interesses do público em assassinatos descritos em jornais). Narrativas de assassinatos pela perspectiva do assassino, estuprador, sequestrador, etc.