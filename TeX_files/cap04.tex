\chapterimage{chapter_head_2.pdf}

\chapter{Classicismo em Portugal}

Sá de Miranda (1481-1558) fora o introdutor do Classicismo em Portugal. Utilizara o chamado \textbf{doce estilo novo} (de origem na poesia lírica humanista desenvolvida na Itália), o qual inaugurava uma nova medida - o verso decassílabo, com dez sílabas poéticas. 

Cabe lembrar:

\begin{definition}[Escansão]
Processo de contagem das sílabas poéticas de um verso.
\end{definition}

\begin{definition}[Métrica]
Medida para o número de sílabas poéticas de um verso. Versos em redondilha menor e maior, por exemplo, possuem, respectivamente, cinco e sete sílabas poéticas. Versos com onze sílabas poéticas, por sua vez, são chamados de hendecassílabos.
\end{definition}

\begin{definition}[Icto]
Tônica obrigatória. Para o caso de um verso decassílabo, por exemplo, ictos na quarta e oitava silabas poéticas formam um verso decassílabo sáfico, ao passo que um icto na sexta forma um verso decassílabo heroico.
\end{definition}
	
\section{Luís de Camões}\index{Luís de Camões}

\textbf{Carece de mais informações.}

\subsection{Sonetos}\index{Luís de Camões!Sonetos}

Novamente, há o processo de emulação dos clássicos da Antiguidade, por meio de sonetos de e sobre amor que, de maneira geral, buscam o definir, e não o expressar. Nota-se a sublimação da figura feminina, assim como uma concepção platônica de amor perfeito entre as almas (e paixão entre corpos). Relembre, em relação ao amor platônico, o episódio grego da \textbf{gigantomaquia}.

Lembre-se da distinção entre o plano das ideias e o plano sensível. Há uma abordagem analítica e universalista do amor, uma reflexão racional sobre a natureza do sentimento amoroso (o que é o amor?). O eu lírico encarna a natureza humana (concepção universal, que não aborda um amor particular, mas do amor que seria sentido por qualquer pessoa).

\subsubsection{Vertentes dos sonetos de Camões}\index{Luís de Camões!Sonetos!Vertentes dos sonetos de Camões}

\textbf{Carece de mais informações.}
 
\subsection{Os lusíadas}\index{Luís de Camões!Os lusíadas}

Trata-se de uma \textbf{epopeia}, poema épico\footnote{O gênero épico se baseia na narração dos feitos de um herói, por meio de uma linguagem grandiloquente e tom heroico.} de longa extensão. O herói em questão é Vasco da Gama (1469-1524), navegador português responsável por estabelecer uma rota comercial marítima entre a Europa e as Índias. Com efeito, a obra fora escrita no contexto das \textbf{Grandes Navegações}.

A obra é escrita em versos decassílabos heroicos, i.e., com dez sílabas poéticas e icto na sexta. Possui um esquema de rimas em oitava - sistema ABABACC, e é dividida em um total de dez cantos. A emulação dos clássicos, assim como nos sonetos, também se faz presente. Em especial, emula a \textit{Eneida}, de Virgílio (70 a.C.-19 d.C.). Note, novamente, que a originalidade não era algo fundamental para os autores da época.

A obra é dividia em três partes: introito, narração e epílogo, que serão analisadas adiante.

\subsubsection{Introito}\index{Luís de Camões!Os lusíadas!Introito}

A primeira parte, o introito, divide-se em outras três porções:

\begin{description}
\item[Proposição] O poeta apresenta o assunto do poema. Nessa parte inicial, é interessante notar a forma pela qual as armas, em processo de metonímia, referem-se aos indivíduos que as utilizam. Os guerreiros e barões são abordados como nobres, que se destacam ao partirem das praias de Portugal em direção às Índias (``além da taprobana''), viagem esta que parecia impossível. Nesse cenário, busca-se a construção de um novo império. Aborda também os reis que expandiram a nação (Fé, Império e terras viciosas, sem virtude, pois não são cristãs). O termo devastar é utilizado em referência à conquista militar. Por fim, relembre a limitação da morte expressa pela \textit{kleos}, e a questão do engenho e da arte presente na obra de Horácio (65 a.C.-8 a.C.), e em especial a ideia do \textit{carpe diem}. Epopeia como um canto. Algumas das figuras citadas incluem Odisseu (grego) e Eneias (troiano), assim como Alexandre Magno (356 a.C.-324 a.C.) - grande, em latim - e Trajano (53 d.C.-117 d.C.). Lusitânia. Netuno indicando a vitória dos portugueses sobre o mar, e Marte como indicativo das conquistas. Musa antiga cessar no sentido de esquecer as epopeias do passado, pois surge algo de maior valor (sentimento de competição). Três primeiras estrofes. Arte marcial.
\item[Evocação] O poeta pede ajuda às Musas para compor o seu poema. Tágides refere-se às ninfas do rio Tejo, símbolo nacional de Portugal. É interessante notar como o autor não pede ajuda para as musas obsoletas. Presença de um viés nacionalista. Tejo (lírico) como épico. Grandíloquo e corrente no sentido de fluido. Febo em referência a Apolo, o deus da poesia. Hipocrene como o monte das musas (rio). Para cantar os feitos, é necessário inspiração - a tuba que ora canora, canta, e ora é belicosa (guerra). Avena (pefano), agreste (campo) e ruda (rude). Desperta coragem e estimula o heroísmo. Quarta e quinta estrofes.
\item[Dedicatória] O poeta dedica o seu poema a alguma figura importante, divina ou mortal (no caso de \textit{Os lusíadas}, Camões dedica o texto ao rei de Portugal, Dom Sebastião\footnote{Relembre a Batalha de Alcácer-Quibir, assim como a lenda do Rei Arthur.}).
\end{description}

\subsubsection{Narração}\index{Luís de Camões!Os lusíadas!Narração}

O primeiro dos dez cantos narra a assembleia dos deuses (relembre o conflito entre gregos e troianos, a relação de Atenas e Era, de Afrodite e Áries, assim como o Pomo de Ouro). Lembre-se que Poseidon evitara a chegada de Odisseu à Itaca, ao passo que Atenas o protegera. Era não desejava a viagem de Eneias para Laço, então ajudado por Vênus, sua mãe. Conflito entre a deusa e Baco, deus estrangeiro, e chegada às Índias. 

É interessante notar como a obra emula a aventura de Eneias na fundação de uma nova Troia no contexto dos portugueses, na busca por formar um novo Império Romano que o supere. Dionísio é contrário à renúncia, oposição. Demonstra uma visão cristã de Camões, à medida em que o Deus queria evitar a chegada do cristianismo dos portugueses aos povos bárbaros. Vênus, em paralelo, e Júpiter neutro. 

No segundo canto, Baco manipula os moçambicanos, criando uma emboscada para os navegadores. Vênus, ao descobrir, lança uma tempestade que revela um reino amigável aos viajantes. Na região, é interessante o simbolismo do banquete com o rei de Melinde (técnica conhecida como \textit{In medias res}), assim como a emulação tanto da Odisseia quando da Eneida. 

Do segundo ao sétimo canto, de Vasco da Gama, há a celebração da história de Portugal e a viagem do português em duas ocasiões. O navegador, assim, é representante da história, de outro protagonista. 

O terceiro conto nos apresenta Inês de Castro (1320/1325-1355), ama de Isabel e esposa de Dom Pedro I (1320-1367), ainda na época em que Portugal era um condado da Espanha. Identifica-se certa preocupação com a volta de Portugal à Espanha. Dom Afonso IV, pai de Dom Pedro I, ordenara o assassinato dos quatro netos, filhos de Inês. Dom Pedro I, ao tornar-se rei, executa os assassinos. Há a personificação do Amor por meio do pronome ``tu''. Crua como cruel. Morte sua em relação a Inês (esta causada pelo Amor). Obriga no sentido de controla. Fero como feroz. Aras como altar. Sacrifício humano pois as lágrimas são insuficientes. Lugar-comum do \textbf{amor tirano}. Travessão como a própria Inês de Castro (espaço feminino). Aparência humana ao assassino. De maneira geral, Camões eterniza e romantiza a história. 

O quarto canto descreve a viagem às Índias, e compreende o episódio do \textbf{velho do rastelo}, voz dissonante que revelara os impactos sociais e econômicos proporcionados por essa grande expedição. É interessante destacar, sobre esse indivíduo, a aparência venerável, um sábio e, pois, com diversas experiências.

\begin{table}[h]
\centering
\begin{tabular}{l l}
\toprule
\textbf{Velho do rastelo} & \textbf{Vasco da Gama} \\
\midrule
rural & urbano \\
tradicional & moderno \\
feudalismo & mercantilismo \\
conservador & aventureiro \\
\bottomrule
\end{tabular}
\end{table}
    
Na visão do autor, Portugal deve se modernizar, mas não deve esquecer de seus valores tradicionais. 

O quinto canto narra a \textbf{gigantomaquia}. Adamastor, o gigante de pedra, serve de alegoria para o Cabo das Tormentas, região rochosa e de clima instável. Vasco da Gama fora o primeiro a completar a travessia com sua frota, então protegida por Vênus. Ao fim, retornamos à cronologia da narrativa. 

Avançando para o oitavo canto, temos a chegada em Calicute. Por influência de Baco, Vasco é preso, mas seu discurso impressiona o rei, que muda seu discurso. É notável a analogia com o Evangelho. Sucesso das navegações, e prêmio de Vênus: ilha das ninfas. Lugar-comum da perseguição às ninfas (descrita no penúltimo canto). É curiosa a passagem ``risinhos alegres''. Aprofundando o nono canto, há uma narrativa situada na ilha dos amores. Há uma mudança de tom, do heroico característico do gênero épico para o tom idílico, presente no gênero lírico (vida amorosa feliz em contato com a natureza). A tal processo damos o nome de \textbf{decoro}: estilo e tom adequados para cada gênero. 

O último, e um dos mais célebres cantos, narra o episódio da \textbf{Máquina do Mundo}. A deusa Tetes apresenta um modelo geocêntrico do universo em miniatura, indicando o anseio dos portugueses. Camões se apodera do conhecimento astronômico clássico, utilizando-se de termos literários, e não científicos. Deuses romanos fabulosos, alegorias para a vontade de Deus. Deuses são anjos de nomes falsos. Baco é a representação do anjo mau, ao passo que Vênus representa a Estrela Dalva (Miguel).

\subsubsection{Epílogo}\index{Luís de Camões!Os lusíadas!Epílogo}

O poeta faz uma reflexão com base no que foi narrado, na tentativa de transmitir algum ensinamento. Neste caso, de Portugal antes e depois das navegações. O autor interpreta a decadência dos portugueses como consequência de um coração endurecido. Portugal está surdo e rude pela cobiça, como representado pelo velho do restelo. Após Camões, temos Bom Sebastião e a União Ibérica. O português escreve uma glória de um passado recente, e em tom melancólico (elegíaco) pela decadência de Portugal. Elegia: nada é eterno. Alfinetadas a Dom Sebastião.