\chapterimage{chapter_head_2.pdf}

\chapter{Introdução}

O Simbolismo é contemporâneo ao Parnasianismo. Trata-se de uma reação ao racionalismo cientificista de meados do século XIX. Houve uma revigoração de alguns elementos do Romantismo, assim como a representação de estados alternados de consciência (melancolia, transe místico - êxtase religioso -, álcool, substâncias estupefacientes); absinto, cânhamo, ópio.

Estabelece-se um senso de mistério e resgate de diversas formas de espiritualidade (cristianismo, budismo, ocultismo, satanismo, espiritismo). O símbolo se insere como metáfora para representar de modo cifrado (na ideia dos estados incomuns de consciência).

Ideia da poesia hermética: o poema precisa ser decifrado e interpretado pelo leitor (Hermes, deus da Grécia antiga e o culto ao deus secreto). Subjetividade na ideia de que, em vez de definir o assunto do poema, o poeta deve sugeri-lo. Linguagem sutil e metafórica (similar ao Cultismo; caráter jocoso). Musicalidade que transmite a mensagem sem a necessidade de racionalização. Aliteração, assonância e rimas métricas.

Sinestesia (``roxo doloroso'', ``harmonia de sentidos''); síntese de sentidos indescritíveis.

Imagem sublimada da figura feminina ou mulher fatal.

Perversões sexuais (sadismo, masoquismo, necrofilia, homossexualidade).