\chapterimage{chapter_head_2.pdf}

\chapter{Bons dias!}

Para iniciar o estudo da obra de Machado de Assis, a coletânia de crônicas \textit{Bons dias!} será a primeira obra a ser analisada. O excerto abaixo refere-se à crônica de cerca de seis dias após a promulgação da Lei Áurea.

\begin{corollary}{BONS DIAS!}
\\
\textsc{19 de maio de 1888}

Eu pertenço a uma família de profetas \textit{après coup}, \textit{post factum}, depois do gato morto, ou como melhor nome tenha em holandês. Por isso digo juro se necessário for, que toda a história desta lei de 13 de maio estava por mim prcvista, tanto que na segunda-feira, antes mesmo dos debates, tratei de alforriar um molecote que tinha, pessoa de seus dezoito anos, mais ou menos. Alforriá-lo era nada; entendi que, perdido por mil, perdido por mil e quinhentos, e dei um jantar.

Neste jantar, a que meus amigos deram o nome de banquete, em falta de outro melhor, reuni umas cinco pessoas, conquanto as notícias dissessem trinta e três (anos de Cristo), no intuito de lhe dar um aspecto simbólico.

No golpe do meio (\textit{coupe do milieu}\footnote{Brinde. Note a apresentação de uma falsa humildade no período seguinte.}, mas eu prefiro falar a minha língua) levantei-me eu com a taça de champanha e declarei que acompanhando as idéias pregadas por Cristo, há dezoito séculos restituía a liberdade ao meu escravo Pancrácio; que entendia a que a nação inteira devia acompanhar as mcsmas idéias e imitaar o meu exemplo; finalmente, que a liberdade era um dom de Deus que os homens não podiam roubar sem pecado.

Pancrácio, que estava à espreita, entrou na sala, como um furacão, e veio abraçar-me os pés. Um dos meus amigos (creio que e ainda meu sobrinho\footnote{Representa um indivíduo da confiança do eu lírico.}) pegou de outra taça e pediu à ilustre assembéia que correspondesse ao ato que acabava de publicar brindando ao primeiro dos cariocas Ouvi cabisbaixo: fiz outro discurso agradecendo, e entreguei a carta ao molecote. Todos os lenços comovidos apanharam as lágrimas de admiração. Caí na cadeira e não vi mais nada. De noite, recebi muitos cartões Creio que cstão pintando o meu retrato, e suponho que a óleo\footnote{O Brasil, nesse período, ainda vivia o início dos registros por meio da fotografia.}.

No dia seguinte, chamei o Pancrácio e disse-lhe com rara franqueza:

— Tu és livre, podes ir para onde quiseres. Aqui tens casa amiga, já conhecida e tens mais um ordenado, um ordenado que...

— Oh! meu senhô! fico

— Um ordenado pequeno, mas que há de crescer. Tudo cresce neste mundo: tu cresceste imensamcnte.

Quando nasceste eras um pirralho deste tamanho; hoje estás mais alto que eu. Deixa ver; olha, és mais alto quatro dedos...

— Artura não qué dizê nada, não, senhô...

— Pequeno ordenado, repito, uns seis mil-réis\footnote{O salário de Pancrácio era aproximadamente igual ao valor de uma galinha.}: mas é de grão em grão que a galinha enche o seu papo. Tu vales muito mais que uma galinha.

— Justamente. Pois seis mil-réis. No fim de um ano, se andares bem, conta com oito. Oito ou sete. 

Pancrácio aceitou tudo: aceitou até um peteleco\footnote{A ``folga'' dos trbalhadores livres.} que lhe dei no dia seguinte, por me não cscovar bem as botas; efeitos da liberdade. Mas eu expliquei-lhe que o peteleco, sendo um impulso natural, não podia anular o direito civil adquirido por um título que lhe dei. Ele continuava livre, eu de mau humor; eram dois estados naturais, quase divinos.

Tudo compreendeu o meu bom Pancrácio: daí para cá, tenho-lhe despedido alguns pontapés, um ou outro puxão de orelhas. e chamo-lhe besta quando lhe não chamo filho do diabo; cousas todas que ele recebe humildemente, e (Deus me perdoe!) creio que até alegre.

O meu plano está feito; quero ser deputado, e, na circular que mandarei aos meus eleitores, direi que, antes, muito antes de abolição legal, já eu em casa, na modéstia da familia, libertava um escravo ato que comoveu a toda a gente que dele teve notícia; que esse escravo tendo aprendendo a ler, escrever e contar, (simples suposição) é éntão professor de filosofia no Rio das Cobras: que os homens puros, grandes e verdadeiramente políticos, não são os que obedecem\footnote{Indicação dos privilégios dessa classe.} à lei, mas os que se antecipam a ela, dizendo ao escravo: es livre, antes que o digam os poderes públicos, sempre retardatários, trôpegos e incapazes de restaurar a justiça na terra, para satisfação do céu.

Boas noites
\end{corollary}

A crônica em questão fora publicada na semana da promulgação da Lei Áurea, a qual marcara a abolição da escravatura no Brasil. Nota-se a utilização da 1ª pessoa, e um eu lírico que demonstra ser egoísta, oportunista e cínico — comportamento presente especialmente em suas mentiras deliberadas.

No primeiro parágrafo, o eu lírico insinua que já tinha conhecimento de que esse processo ocorreria, ainda que fosse um debate que permeasse com certa profundidade o imaginário da população. Na última linha, também há uma referência ao valor (patrimônio) que o escravo lhe representava. Mesmo no úlimo parágrafo da crônica, no qual o narrador comenta que ``muito antes da abolição legal, ...'', nota-se que apenas tivera conhecimento do processo na mesma semana.

No segundo parágrafo, é interessante notar as discrepâncias presentes no discurso do eu lírico acerca do jantar, e o que realmente acontecera.

O terceiro parágrago, por sua vez, é marcado por uma justificativa de cunho religioso para a alforria. Note, no entanto, que os trechos seguintes revelam que mesmo Pancrácio, o escravo, não participava do banquete e desconhecia a alforria que lhe seria dada.

Por fim, cabe detalher o conceito de \textit{impulso natural} referido no texto. No contexto do Iluminismo do século XIX, existia a ideia de que as leis humanas não poderiam se opor às leis da natureza, universais — representavam o \textbf{direito natural}. Pour outro lado, eixstiam também as leis criadas pelos homens para o convívio em sociedade, estabelecidas por meio de um consenso social — representavam o \textit{direiro civil}. Com a ideia do direito natural — intrinsecamente humano —, ocorreu, em partes, o combate à escravidão, uma vez que o estado natural do ser humano é de liberdade, em harmonia com seus direitos civis. Note, no entanto, que na crônica descrita, o eu lírico trata do direiro de espancar Pancrácio também como algo natural (pela violência e hostilidade; deturpação).

Para melhor entender a visão do autor e a crônica em questão, diversos pontos devem ser aprofundados. Na juventude, Machado de Assis era republicano e, posteriormente, tornara-se monarquista; notara, contudo, que a estrutura da sociedade permenecera a mesma (Pedro e Paulo e a placa do comércio de café). Outro aspecto importante, e sobre o qual não há consenso na Academia, refere-se à questão se machado era um autor do Realismo (enquantot escola liteária) ou apenas incorporara elementos realistas em sua obra. Nesse sentido, nota-se a ausência do descritivismo, assim como um narrador (e sua subjetividade trazida) central — Machado não é um herdeiro do Realismo de Flaubert — e do objetivismo. Apropriação dos romances pré-românticos ingleses, alternativas ao Realismo francês.

Em relação à crônica, há uma narração sob o ponto de vista da elite escravocrata, em especial no início do trabalho livre na região do Nordeste (anterior deslocamento da produção de café para o Sudeste). São evidentes as contradições presentes no discurso do narrador, pela qual se observa a visão do autor de maneira implícita, e que permitem a pela compreensão do texto (o narrador é alvo das críticas). Note também a presença de uma ironia machadiana, formulada pelo autor e expressa pelo narrador.

John Gledson.

Machado observara que as estruturas da sociedade permaneceram inalteradas semanas seguintes à abolição, como indicado pelo fato de que a vida de Pancrácio não sofrera maiores mudanças. Por fim, no último parágrafo, cabe comentar que a opinião pública era, de maneira geral, contrária à escravidão (mesmo entre a elite).

Machado de Assis atuara no Ministério da Fazenda. Lembre-se de Rui Barbosa, que queimara os registros dos escravos.

Capitalização política da ação de libertar o escravo. 